% Options for packages loaded elsewhere
\PassOptionsToPackage{unicode}{hyperref}
\PassOptionsToPackage{hyphens}{url}
\documentclass[
]{article}
\usepackage{xcolor}
\usepackage[margin=1in]{geometry}
\usepackage{amsmath,amssymb}
\setcounter{secnumdepth}{5}
\usepackage{iftex}
\ifPDFTeX
  \usepackage[T1]{fontenc}
  \usepackage[utf8]{inputenc}
  \usepackage{textcomp} % provide euro and other symbols
\else % if luatex or xetex
  \usepackage{unicode-math} % this also loads fontspec
  \defaultfontfeatures{Scale=MatchLowercase}
  \defaultfontfeatures[\rmfamily]{Ligatures=TeX,Scale=1}
\fi
\usepackage{lmodern}
\ifPDFTeX\else
  % xetex/luatex font selection
\fi
% Use upquote if available, for straight quotes in verbatim environments
\IfFileExists{upquote.sty}{\usepackage{upquote}}{}
\IfFileExists{microtype.sty}{% use microtype if available
  \usepackage[]{microtype}
  \UseMicrotypeSet[protrusion]{basicmath} % disable protrusion for tt fonts
}{}
\makeatletter
\@ifundefined{KOMAClassName}{% if non-KOMA class
  \IfFileExists{parskip.sty}{%
    \usepackage{parskip}
  }{% else
    \setlength{\parindent}{0pt}
    \setlength{\parskip}{6pt plus 2pt minus 1pt}}
}{% if KOMA class
  \KOMAoptions{parskip=half}}
\makeatother
\usepackage{color}
\usepackage{fancyvrb}
\newcommand{\VerbBar}{|}
\newcommand{\VERB}{\Verb[commandchars=\\\{\}]}
\DefineVerbatimEnvironment{Highlighting}{Verbatim}{commandchars=\\\{\}}
% Add ',fontsize=\small' for more characters per line
\usepackage{framed}
\definecolor{shadecolor}{RGB}{248,248,248}
\newenvironment{Shaded}{\begin{snugshade}}{\end{snugshade}}
\newcommand{\AlertTok}[1]{\textcolor[rgb]{0.94,0.16,0.16}{#1}}
\newcommand{\AnnotationTok}[1]{\textcolor[rgb]{0.56,0.35,0.01}{\textbf{\textit{#1}}}}
\newcommand{\AttributeTok}[1]{\textcolor[rgb]{0.13,0.29,0.53}{#1}}
\newcommand{\BaseNTok}[1]{\textcolor[rgb]{0.00,0.00,0.81}{#1}}
\newcommand{\BuiltInTok}[1]{#1}
\newcommand{\CharTok}[1]{\textcolor[rgb]{0.31,0.60,0.02}{#1}}
\newcommand{\CommentTok}[1]{\textcolor[rgb]{0.56,0.35,0.01}{\textit{#1}}}
\newcommand{\CommentVarTok}[1]{\textcolor[rgb]{0.56,0.35,0.01}{\textbf{\textit{#1}}}}
\newcommand{\ConstantTok}[1]{\textcolor[rgb]{0.56,0.35,0.01}{#1}}
\newcommand{\ControlFlowTok}[1]{\textcolor[rgb]{0.13,0.29,0.53}{\textbf{#1}}}
\newcommand{\DataTypeTok}[1]{\textcolor[rgb]{0.13,0.29,0.53}{#1}}
\newcommand{\DecValTok}[1]{\textcolor[rgb]{0.00,0.00,0.81}{#1}}
\newcommand{\DocumentationTok}[1]{\textcolor[rgb]{0.56,0.35,0.01}{\textbf{\textit{#1}}}}
\newcommand{\ErrorTok}[1]{\textcolor[rgb]{0.64,0.00,0.00}{\textbf{#1}}}
\newcommand{\ExtensionTok}[1]{#1}
\newcommand{\FloatTok}[1]{\textcolor[rgb]{0.00,0.00,0.81}{#1}}
\newcommand{\FunctionTok}[1]{\textcolor[rgb]{0.13,0.29,0.53}{\textbf{#1}}}
\newcommand{\ImportTok}[1]{#1}
\newcommand{\InformationTok}[1]{\textcolor[rgb]{0.56,0.35,0.01}{\textbf{\textit{#1}}}}
\newcommand{\KeywordTok}[1]{\textcolor[rgb]{0.13,0.29,0.53}{\textbf{#1}}}
\newcommand{\NormalTok}[1]{#1}
\newcommand{\OperatorTok}[1]{\textcolor[rgb]{0.81,0.36,0.00}{\textbf{#1}}}
\newcommand{\OtherTok}[1]{\textcolor[rgb]{0.56,0.35,0.01}{#1}}
\newcommand{\PreprocessorTok}[1]{\textcolor[rgb]{0.56,0.35,0.01}{\textit{#1}}}
\newcommand{\RegionMarkerTok}[1]{#1}
\newcommand{\SpecialCharTok}[1]{\textcolor[rgb]{0.81,0.36,0.00}{\textbf{#1}}}
\newcommand{\SpecialStringTok}[1]{\textcolor[rgb]{0.31,0.60,0.02}{#1}}
\newcommand{\StringTok}[1]{\textcolor[rgb]{0.31,0.60,0.02}{#1}}
\newcommand{\VariableTok}[1]{\textcolor[rgb]{0.00,0.00,0.00}{#1}}
\newcommand{\VerbatimStringTok}[1]{\textcolor[rgb]{0.31,0.60,0.02}{#1}}
\newcommand{\WarningTok}[1]{\textcolor[rgb]{0.56,0.35,0.01}{\textbf{\textit{#1}}}}
\usepackage{longtable,booktabs,array}
\newcounter{none} % for unnumbered tables
\usepackage{calc} % for calculating minipage widths
% Correct order of tables after \paragraph or \subparagraph
\usepackage{etoolbox}
\makeatletter
\patchcmd\longtable{\par}{\if@noskipsec\mbox{}\fi\par}{}{}
\makeatother
% Allow footnotes in longtable head/foot
\IfFileExists{footnotehyper.sty}{\usepackage{footnotehyper}}{\usepackage{footnote}}
\makesavenoteenv{longtable}
\usepackage{graphicx}
\makeatletter
\newsavebox\pandoc@box
\newcommand*\pandocbounded[1]{% scales image to fit in text height/width
  \sbox\pandoc@box{#1}%
  \Gscale@div\@tempa{\textheight}{\dimexpr\ht\pandoc@box+\dp\pandoc@box\relax}%
  \Gscale@div\@tempb{\linewidth}{\wd\pandoc@box}%
  \ifdim\@tempb\p@<\@tempa\p@\let\@tempa\@tempb\fi% select the smaller of both
  \ifdim\@tempa\p@<\p@\scalebox{\@tempa}{\usebox\pandoc@box}%
  \else\usebox{\pandoc@box}%
  \fi%
}
% Set default figure placement to htbp
\def\fps@figure{htbp}
\makeatother
\setlength{\emergencystretch}{3em} % prevent overfull lines
\providecommand{\tightlist}{%
  \setlength{\itemsep}{0pt}\setlength{\parskip}{0pt}}
\usepackage[]{natbib}
\bibliographystyle{plainnat}
\usepackage{tcolorbox}
\usepackage{xcolor}

% Define custom colors matching style.css
\definecolor{capaianbg}{HTML}{e3f2fd}
\definecolor{capaianfg}{HTML}{1565c0}
\definecolor{capaianborder}{HTML}{90caf9}

% Define the \capaian command using tcbox (tcolorbox for inline content)
\newtcbox{\capaian}{on line, 
  boxsep=1pt, left=2pt, right=2pt, top=1pt, bottom=1pt,
  colframe=capaianborder, colback=capaianbg, text only,
  fontupper=\ttfamily\footnotesize\color{capaianfg},
  boxrule=0.5pt, arc=2pt
}
\usepackage{fontspec}
\usepackage{multirow}
\usepackage{multicol}
\usepackage{colortbl}
\usepackage{hhline}
\newlength\Oldarrayrulewidth
\newlength\Oldtabcolsep
\usepackage{longtable}
\usepackage{array}
\usepackage{hyperref}
\usepackage{float}
\usepackage{wrapfig}
\usepackage{booktabs}
\usepackage{caption}
\usepackage{anyfontsize}
\usepackage{bookmark}
\IfFileExists{xurl.sty}{\usepackage{xurl}}{} % add URL line breaks if available
\urlstyle{same}
\hypersetup{
  pdftitle={Modul Praktik Statistika Untuk Perencanaan},
  hidelinks,
  pdfcreator={LaTeX via pandoc}}

\title{Modul Praktik Statistika Untuk Perencanaan}
\usepackage{etoolbox}
\makeatletter
\providecommand{\subtitle}[1]{% add subtitle to \maketitle
  \apptocmd{\@title}{\par {\large #1 \par}}{}{}
}
\makeatother
\subtitle{Penggunaan R dalam Pengolahan Data Kuantitatif}
\author{}
\date{\vspace{-2.5em}}

\begin{document}
\maketitle

{
\setcounter{tocdepth}{2}
\tableofcontents
}
\section*{Pengantar}\label{pengantar}
\addcontentsline{toc}{section}{Pengantar}

Selamat datang di modul daring Praktikum Statistika untuk Perencanaan. Modul ini digunakan untuk mendampingi kalian mengikuti Praktikum dalam mata kuliah Statistika untuk Perencanaan, mata kuliah di Program Studi Perencanaan Wilayah dan Kota, Institut Teknologi Sumatera.

Modul praktikum ini akan mendampingi Anda mempelajari pengolahan statistik menggunakan bahasa pemrograman R. Untuk itu, modul ini terbagi menjadi \textbf{9 modul} yang disusun secara bertahap, dimulai dari pengenalan dasar hingga analisis multivariat:

{\def\LTcaptype{none} % do not increment counter
\begin{longtable}[]{@{}
  >{\centering\arraybackslash}p{(\linewidth - 4\tabcolsep) * \real{0.0360}}
  >{\centering\arraybackslash}p{(\linewidth - 4\tabcolsep) * \real{0.3022}}
  >{\centering\arraybackslash}p{(\linewidth - 4\tabcolsep) * \real{0.6619}}@{}}
\toprule\noalign{}
\begin{minipage}[b]{\linewidth}\centering
Modul
\end{minipage} & \begin{minipage}[b]{\linewidth}\centering
Topik
\end{minipage} & \begin{minipage}[b]{\linewidth}\centering
Deskripsi Singkat
\end{minipage} \\
\midrule\noalign{}
\endhead
\bottomrule\noalign{}
\endlastfoot
1 & Pengolahan Data Terstruktur & Dasar \emph{data wrangling}: Impor, seleksi, standardisasi, penggabungan, dan ekspor data/dataset \\
2 & Statistika Deskriptif & Perhitungan proporsi, mean, median, standar deviasi, dan IQV \\
3 & Visualisasi Data & Pembuatan grafik dengan \texttt{ggplot2}: bar chart, histogram, boxplot, scatter plot \\
4 & Distribusi Sampling \& Interval Kepercayaan & Simulasi sampling, standard error, dan estimasi parameter \\
5 & Uji Hipotesis & Uji hipotesis 1 dan 2 populasi untuk rata-rata dan proporsi \\
6 & Korelasi Nominal \& Ordinal & Koefisien V Cramer, Lambda, dan Gamma \\
7 & Korelasi Metrik & Koefisien Spearman's ρ dan Pearson's r \\
8 & Regresi Linear & Regresi linear sederhana dan berganda \\
9 & Analisis Komponen Prinsip & PCA dan Analisis Faktor \\
\end{longtable}
}

Setiap modul dilengkapi dengan \textbf{capaian pembelajaran}, \textbf{contoh kode}, dan \textbf{soal latihan} yang harus dikerjakan.

\begin{center}\rule{0.5\linewidth}{0.5pt}\end{center}

\section*{Pengenalan R dan RStudio}\label{pengenalan-r-dan-rstudio}
\addcontentsline{toc}{section}{Pengenalan R dan RStudio}

Selamat datang di dunia pengolahan data yang lebih canggih! Dalam praktikum ini, kalian akan berkenalan dengan teknik pengolahan dan analisis data dengan menggunakan bahasa pemrograman bernama R. R adalah bahasa pemrograman yang dibuat khusus untuk keperluan analisis statistik.

\subsection{Apa itu R?}\label{apa-itu-r}

R adalah lingkungan perangkat lunak (\emph{software environment}) dan bahasa pemrograman yang digunakan untuk komputasi statistik dan grafik. R dikembangkan oleh Ross Ihaka dan Robert Gentleman di Universitas Auckland, Selandia Baru, pada awal tahun 1990-an sebagai evolusi dari bahasa S yang sebelumnya dikembangkan di Bell Laboratories \citep{history_overview_r}. R bersifat \emph{open-source} (berlisensi GNU GPL), yang berarti dapat digunakan secara gratis dan dikembangkan secara kolaboratif oleh komunitas global \citep{r_wikipedia}.

Keunggulan utama R dalam konteks perencanaan dan pemerintahan meliputi:

\begin{enumerate}
\def\labelenumi{\arabic{enumi}.}
\tightlist
\item
  \textbf{Ekosistem Paket yang Luas}: R memiliki puluhan ribu paket (\emph{packages}) di CRAN untuk berbagai keperluan, mulai dari manipulasi data, pemodelan statistik, hingga analisis geospasial menggunakan \texttt{sf} dan \texttt{terra} \citep{spatial_r_sf, spatial_data_science_terra}.
\item
  \textbf{Visualisasi Data}: Dengan paket seperti \texttt{ggplot2} dan \texttt{Shiny}, R memungkinkan pembuatan grafik berkualitas tinggi dan dashboard interaktif yang informatif \citep{shiny_gallery}.
\item
  \textbf{Dukungan Kebijakan Berbasis Data}: R mendukung implementasi kebijakan ``Satu Data Indonesia'' dengan kemampuan standardisasi, interoperabilitas, dan transparansi metode analisis yang dapat diaudit \citep{satu_data_local_govt}. R memfasilitasi proses ETL (\emph{Extract-Transform-Load}) otomatis untuk membersihkan data dari berbagai instansi \citep{r_public_data}.
\end{enumerate}

Dalam bidang perencanaan wilayah dan kota, R digunakan untuk:

\begin{itemize}
\tightlist
\item
  \textbf{Analisis Geospasial}: Mengelola data vektor dan raster untuk pemetaan dan pemodelan lingkungan \citep{spatial_data_science_terra}.
\item
  \textbf{Perencanaan Transportasi}: Menggunakan paket seperti \texttt{stplanr} untuk analisis jaringan, rute, dan pemodelan transportasi berkelanjutan \citep{stplanr_cran, sustainable_transport_stplanr}.
\item
  \textbf{Simulasi Kebijakan}: Melakukan simulasi dampak kebijakan publik, seperti perubahan penggunaan lahan atau intervensi sosial, sebelum diterapkan secara nyata \citep{simulation_public_policies}.
\end{itemize}

\subsection{Mengapa Menggunakan R?}\label{mengapa-menggunakan-r}

R memiliki kelebihan yang signifikan dibandingkan dengan alat analisis berbasis antarmuka grafis (\emph{Graphical User Interface} - GUI) seperti SPSS atau Microsoft Excel. Berikut adalah beberapa alasan utama mengapa R menjadi standar de facto dalam sains data modern:

\begin{enumerate}
\def\labelenumi{\arabic{enumi}.}
\item
  \textbf{Reproduksibilitas (\emph{Reproducibility})}:
  Berbeda dengan perangkat lunak berbasis \emph{point-and-click} di mana langkah-langkah analisis seringkali sulit dilacak kembali, R bekerja berbasis skript (\emph{script-based}). Setiap langkah analisis, mulai dari pembersihan data hingga pembuatan model, terekam secara eksplisit dalam kode. Hal ini memungkinkan analisis untuk diverifikasi, diaudit, dan dijalankan ulang oleh orang lain (atau diri kita sendiri di masa depan) dengan hasil yang identik, sebuah prinsip fundamental dalam penelitian ilmiah yang kredibel \citep{peng2011reproducible, gandrud2013reproducible}.
\item
  \textbf{Fleksibilitas dan Kemampuan Kostumisasi}:
  R tidak membatasi pengguna pada menu yang sudah tersedia. Sebagai bahasa pemrograman, R memungkinkan kita untuk membangun alat analisis sendiri atau memodifikasi metode yang sudah ada sesuai dengan kebutuhan spesifik penelitian, memberikan fleksibilitas yang hampir tak terbatas dibandingkan perangkat lunak tertutup (\emph{proprietary}) \citep{wickham2016r}.
\item
  \textbf{Gratis dan \emph{Open Source}}:
  Sebagai perangkat lunak \emph{open source}, R dapat digunakan sepenuhnya secara gratis. Ini menghilangkan hambatan biaya lisensi yang mahal yang sering ditemui pada perangkat lunak komersial, membuatnya sangat aksesibel bagi mahasiswa, peneliti, dan pemerintah daerah dengan anggaran terbatas.
\item
  \textbf{Komunitas yang Masif}:
  Dukungan komunitas R sangat luar biasa melalui platform seperti StackOverflow dan R-Bloggers. Jika Anda menemui masalah, kemungkinan besar orang lain sudah pernah menghadapinya dan solusinya sudah tersedia secara daring.
\end{enumerate}

\subsection{Apa itu RStudio?}\label{apa-itu-rstudio}

RStudio (yang kini bertransformasi menjadi \textbf{Posit}) adalah \emph{Integrated Development Environment} (IDE) untuk R. Diluncurkan pertama kali pada tahun 2009, RStudio menyediakan antarmuka visual yang lebih ramah pengguna dibandingkan antarmuka baris perintah (\emph{command line}) asli R \citep{about_posit}. Fitur-fitur seperti penyorotan sintaksis (\emph{syntax highlighting}), \emph{auto-completion}, dan manajemen \emph{workspace} membuat pengguna baru lebih mudah mempelajari dan menggunakan R. Transformasi menjadi Posit mencerminkan komitmen terhadap ekosistem \emph{data science} modern yang lebih luas, termasuk dukungan untuk Python \citep{goodbye_rstudio}.

Gambar \ref{fig:jendela-rstudio} memberikan penjelasan singkat tentang tampilan jendela RStudio yang terdiri atas 4 panel.

\begin{figure}

{\centering \includegraphics[width=25.72in]{images/jendela_rstudio} 

}

\caption{Bagian-bagian jendela RStudio}\label{fig:jendela-rstudio}
\end{figure}

\begin{enumerate}
\def\labelenumi{\arabic{enumi}.}
\tightlist
\item
  Panel \emph{Environment} yang menampilkan daftar data atau variabel yang telah kita impor atau buat.
\item
  Panel \emph{Files, Plots, dan Help}, tempat kita melihat daftar \emph{file} yang tersedia, grafik yang dihasilkan, dan menemukan dokumen bantuan untuk berbagai bagian dari R.
\item
  Panel \emph{Console}, digunakan untuk menjalankan kode.
\item
  Panel \emph{Script}, digunakan untuk menulis kode. Di sinilah kita akan menghabiskan sebagian besar waktu kerja kita.
\end{enumerate}

\subsection{Instalasi R dan RStudio}\label{instalasi-r-dan-rstudio}

Sebelum memulai praktikum, kalian perlu menginstal R dan RStudio terlebih dahulu. R harus diinstal \textbf{sebelum} RStudio, karena RStudio memerlukan instalasi R yang berfungsi untuk dapat berjalan.

\textbf{Langkah 1: Instalasi R}

\begin{enumerate}
\def\labelenumi{\arabic{enumi}.}
\tightlist
\item
  Kunjungi situs resmi CRAN di \textbf{\href{https://cran.r-project.org}{cran.r-project.org}}
\item
  Pilih sistem operasi Anda (Windows/macOS/Linux)
\item
  Untuk \textbf{Windows}: Klik ``base'' lalu unduh installer terbaru
\item
  Untuk \textbf{macOS}: Pilih versi sesuai prosesor (Apple Silicon M1/M2/M3/M4/M5 atau Intel)
\item
  Jalankan file installer dan ikuti petunjuk hingga selesai
\end{enumerate}

\textbf{Langkah 2: Instalasi RStudio}

\begin{enumerate}
\def\labelenumi{\arabic{enumi}.}
\tightlist
\item
  Kunjungi situs resmi Posit di \textbf{\href{https://posit.co/downloads}{posit.co/downloads}}
\item
  Unduh \textbf{RStudio Desktop} versi gratis sesuai sistem operasi Anda
\item
  Jalankan installer dan ikuti petunjuk hingga selesai
\end{enumerate}

\textbf{Verifikasi}: Buka RStudio setelah instalasi selesai. RStudio akan otomatis mendeteksi instalasi R yang sudah ada.

\subsection{Cara Menjalankan R}\label{cara-menjalankan-r}

\subsubsection{\texorpdfstring{Langsung di \emph{Console}}{Langsung di Console}}\label{langsung-di-console}

Pertama, silakan buka program \textbf{RGui}. Caranya, buka Start Menu, ketikkan \texttt{R\ x.x.x}. Ganti \texttt{x.x.x} dengan nomor versi yang Anda unduh sebelumnya. Kemudian tekan \emph{Enter}. Tampilannya adalah seperti yang terlihat pada Gambar \ref{fig:jendela-r} berikut.

\begin{figure}

{\centering \includegraphics[width=19.82in]{images/jendela-r} 

}

\caption{Jendela R 4.5.2}\label{fig:jendela-r}
\end{figure}

\textbf{Aktivitas Mandiri: Kode R Pertama}

Cobalah ketik perintah berikut di jendela yang terdapat kursor di dalamnya, lalu tekan Enter setelah mengetikkan kode berikut:

\texttt{print(2\ +\ 2)}

Anda akan melihat angka \texttt{4} sebagai hasilnya.

Selamat! Anda baru saja menjalankan kode R pertama Anda.

Anda akan menyadari bahwa antarmuka RGui tidak bersahabat dengan pengguna \emph{(user friendly)}. Selain itu tidak ada juga alat-alat yang memudahkan kita memanajemen pekerjaan seperti menampilkan lokasi file atau \emph{script} kode sumber.

Oleh karena itu, kita membutuhkan cara kedua, yakni menggunakan \textbf{RStudio} melalui panel \textbf{Console} (jendela nomor 3 pada Gambar \ref{fig:jendela-rstudio}).

\textbf{Aktivitas Mandiri: Kode R pertama di RStudio}

Cobalah ketik perintah berikut di Console, lalu tekan Enter:

\texttt{print(2\ +\ 2)}

Anda akan melihat hasil yang sama seperti yang Anda dapat di RGui. Inilah tempat menjalankan kode R di RStudio.

Selamat! Anda baru saja menjalankan kode R pertama Anda di RStudio.

\subsubsection{\texorpdfstring{Menggunakan \emph{File} Kode Sumber}{Menggunakan File Kode Sumber}}\label{menggunakan-file-kode-sumber}

Namun, mengetik satu per satu di Console akan melelahkan jika perintahnya banyak. Oleh karena itu, kita biasanya menulis kode dalam sebuah file \emph{source code} yang berekstensi (berakhiran) \texttt{.R}.

\paragraph{\texorpdfstring{Membuat \emph{File} Kode Sumber R (\texttt{.R})}{Membuat File Kode Sumber R (.R)}}\label{membuat-file-kode-sumber-r-.r}

File kode sumber R (berekstensi \texttt{.R}) adalah dokumen teks yang berisi rangkaian perintah R yang dapat disimpan, diedit, dan dijalankan berulang kali. Ini adalah cara profesional dalam bekerja dengan R, karena kode kita menjadi \textbf{terdokumentasi}, \textbf{dapat direproduksi}, dan \textbf{mudah dibagikan}.

\textbf{Aktivitas Mandiri: Membuat \emph{file} Kode Sumber R Pertama}

\begin{enumerate}
\def\labelenumi{\arabic{enumi}.}
\tightlist
\item
  Di RStudio, pilih menu \textbf{File → New File → R Script} (atau tekan \textbf{Ctrl + Shift + N} di Windows/Linux, \textbf{Cmd + Shift + N} di Mac)
\item
  Sebuah tab baru akan muncul di panel \emph{Script} (jendela kiri atas)
\item
  Ketikkan kode R berikut di dalam tab tersebut
\end{enumerate}

\begin{Shaded}
\begin{Highlighting}[]
\CommentTok{\# File kode R pertama saya}
\CommentTok{\# Dibuat pada: [isi dengan tanggal hari ini]}

\FunctionTok{print}\NormalTok{(}\StringTok{"Halo dunia!"}\NormalTok{)}
\FunctionTok{print}\NormalTok{(}\StringTok{"Hari ini saya mulai belajar R!"}\NormalTok{)}

\CommentTok{\# Operasi matematika sederhana}
\DecValTok{1} \SpecialCharTok{+} \DecValTok{1}
\DecValTok{2} \SpecialCharTok{*} \DecValTok{3}
\end{Highlighting}
\end{Shaded}

\begin{enumerate}
\def\labelenumi{\arabic{enumi}.}
\setcounter{enumi}{3}
\tightlist
\item
  Simpan file dengan memilih \textbf{File → Save} (atau \textbf{Ctrl + S} / \textbf{Cmd + S}) di direktori/folder pertama muncul di jendela Anda tersebut
\item
  Beri nama file dengan ekstensi \texttt{.R}, misalnya \texttt{kode\_R\_pertama\_saya.R}
\end{enumerate}

\paragraph{\texorpdfstring{Menjalankan kode dari file \texttt{.R}}{Menjalankan kode dari file .R}}\label{menjalankan-kode-dari-file-.r}

Untuk menjalankan kode dari file \texttt{.R} ke Console tanpa mengetik ulang:

\begin{enumerate}
\def\labelenumi{\arabic{enumi}.}
\tightlist
\item
  Anda bisa melakukan salah satu dari sekian cara ini:

  \begin{enumerate}
  \def\labelenumii{\arabic{enumii}.}
  \tightlist
  \item
    Sorot (blok) baris-baris kode yang ingin dijalankan di panel \emph{Script}. Ini akan menjalankan kode yang Anda sorot saja.
  \item
    Klik angka baris di sebelah kiri untuk memilh seluruh kode yang ada dalam satu baris.
  \item
    \textbf{REKOMENDASI}: menempatkan kursor pada baris yang kodenya akan dijalankan. Ini akan secara otomatis meminta R menjalankan baris kode yang dianggap satu paket.
  \end{enumerate}
\item
  Tekan kombinasi tombol \textbf{Ctrl + Enter} (Windows/Linux) atau \textbf{Cmd + Enter} (Mac)
\item
  Kode tersebut akan otomatis dikirim ke \emph{Console} dan dijalankan
\end{enumerate}

\textbf{Aktivitas Mandiri: Menjalankan Kode R dari File Kode Sumber}

\begin{enumerate}
\def\labelenumi{\arabic{enumi}.}
\tightlist
\item
  Masih dari \emph{script} \texttt{kode\_R\_pertama\_saya.R} tadi, tempatkan kursor pada baris berikut
\end{enumerate}

\begin{Shaded}
\begin{Highlighting}[]
\NormalTok{...}
\FunctionTok{print}\NormalTok{(}\StringTok{"Halo dunia!"}\NormalTok{)}
\NormalTok{...}
\end{Highlighting}
\end{Shaded}

\begin{enumerate}
\def\labelenumi{\arabic{enumi}.}
\setcounter{enumi}{1}
\tightlist
\item
  Jalankan baris tersebut dengan menekan tombol kombinasi sesuai penjelasan.
\item
  Kemudian, tempatkan kursor pada baris berikut
\end{enumerate}

\begin{Shaded}
\begin{Highlighting}[]
\NormalTok{...}
\DecValTok{2} \SpecialCharTok{*} \DecValTok{3}
\NormalTok{...}
\end{Highlighting}
\end{Shaded}

\begin{enumerate}
\def\labelenumi{\arabic{enumi}.}
\setcounter{enumi}{3}
\tightlist
\item
  Jalankan baris tersebut kembali.
\end{enumerate}

Ini adalah cara kerja standar yang efisien dalam menggunakan R. Dengan cara ini, kita bisa:

\begin{itemize}
\tightlist
\item
  \textbf{Menyimpan pekerjaan} untuk digunakan di lain waktu
\item
  \textbf{Menjalankan ulang} analisis dengan mudah
\item
  \textbf{Berbagi kode} dengan rekan kerja atau asisten praktikum
\item
  \textbf{Mendokumentasikan} setiap langkah analisis kita
\end{itemize}

\begin{center}\rule{0.5\linewidth}{0.5pt}\end{center}

\section*{Pengenalan Teknis R}\label{pengenalan-teknis-r}
\addcontentsline{toc}{section}{Pengenalan Teknis R}

Sebelum memulai praktikum dengan set data sesungguhnya, ada baiknya kita memahami beberapa konsep teknis dasar dalam bahasa pemrograman R. Bagian ini merangkum keterampilan-keterampilan dasar yang akan kalian temui tersebar di modul-modul praktikum.

\subsection{Pengaturan Direktori}\label{pengaturan-direktori}

Salah satu keterampilan penting dalam bekerja dengan R adalah mengorganisasi \emph{file} dan data dengan baik. Sepanjang semester ini, kalian akan bekerja dengan berbagai dataset, kode R, dan hasil analisis. Pengaturan direktori yang terstruktur akan memudahkan kalian dalam mengelola semua file tersebut dan memastikan kode dapat berjalan dengan lancar.

\subsubsection{Struktur Direktori yang Ideal}\label{struktur-direktori-yang-ideal}

Sebelum memulai praktikum, kalian perlu menyiapkan direktori kerja (\emph{working directory}) yang terorganisir dengan baik. Berikut adalah struktur direktori yang \textbf{wajib} kalian gunakan sepanjang semester:

\begin{Shaded}
\begin{Highlighting}[]
\NormalTok{Praktikum STP 2026/}
\NormalTok{├── Praktikum STP 2026.Rproj}
\NormalTok{├── datasets/}
\NormalTok{│   ├── data.csv}
\NormalTok{│   └── ...}
\NormalTok{├── output/}
\NormalTok{│   ├── output\_1.xlsx}
\NormalTok{│   └── ...}
\NormalTok{├── modul\_1.R}
\NormalTok{└── modul\_2.R}
\end{Highlighting}
\end{Shaded}

\textbf{Komponen-komponen penting:}

\begin{itemize}
\tightlist
\item
  \textbf{\texttt{Praktikum\ STP\ 2026.Rproj}}: File RStudio Project yang mengatur \emph{working directory} secara otomatis dan menyimpan \emph{environment} kerja kalian.
\item
  \textbf{\texttt{datasets/}}: Folder khusus untuk menyimpan \textbf{semua dataset mentah} (file Excel, CSV, atau format lainnya yang kalian terima dari asisten praktikum).
\item
  \textbf{\texttt{output/}}: Folder khusus untuk menyimpan \textbf{hasil-hasil analisis} seperti file Excel yang sudah diolah, file CSV hasil export, atau grafik yang dihasilkan.
\item
  \textbf{\texttt{modul\_1.R}, \texttt{modul\_2.R}, \ldots{}}: File-file kode R (script) yang akan kalian buat untuk setiap modul praktikum.
\end{itemize}

Dengan struktur seperti ini, kode R kalian akan lebih \textbf{ringkas} dan \textbf{portabel}. Sebagai contoh, untuk membaca data, kalian cukup menulis:

\begin{Shaded}
\begin{Highlighting}[]
\NormalTok{data }\OtherTok{\textless{}{-}} \FunctionTok{read.xlsx}\NormalTok{(}\StringTok{"datasets/data\_penduduk.xlsx"}\NormalTok{)}
\end{Highlighting}
\end{Shaded}

Tanpa perlu menulis \emph{path} lengkap seperti \texttt{C:\textbackslash{}Users\textbackslash{}NamaAnda\textbackslash{}Documents\textbackslash{}praktikum\textbackslash{}data\_penduduk.xlsx} yang panjang, rumit, dan berbeda-beda di setiap komputer.

\subsubsection{Membuat dan Mengatur Direktori Praktikum}\label{membuat-dan-mengatur-direktori-praktikum}

Ikuti langkah-langkah berikut untuk menyiapkan direktori praktikum kalian:

\paragraph{Langkah 1: Membuat Folder Utama}\label{langkah-1-membuat-folder-utama}

Buatlah sebuah folder di komputer kalian dengan nama \textbf{\texttt{Praktikum\ STP\ 2026}} (sesuai tahun ajaran yang sedang berjalan). Penamaan yang seragam ini \textbf{sangat penting} agar asisten praktikum dapat dengan mudah membantu kalian jika terjadi masalah atau kesalahan.

Folder ini akan menjadi \textbf{direktori kerja} (\emph{working directory}) kalian selama satu semester penuh.

\paragraph{\texorpdfstring{Langkah 2: Membuat Subfolder \texttt{datasets} dan \texttt{output}}{Langkah 2: Membuat Subfolder datasets dan output}}\label{langkah-2-membuat-subfolder-datasets-dan-output}

Di dalam folder \texttt{Praktikum\ STP\ 2026}, buatlah \textbf{dua subfolder}:

\begin{enumerate}
\def\labelenumi{\arabic{enumi}.}
\tightlist
\item
  Subfolder \textbf{\texttt{datasets}} - untuk menyimpan semua data mentah
\item
  Subfolder \textbf{\texttt{output}} - untuk menyimpan hasil analisis
\end{enumerate}

\textbf{Cara membuat folder:}

\begin{itemize}
\tightlist
\item
  \textbf{Di Windows}: Klik kanan di dalam folder \texttt{Praktikum\ STP\ 2026} → pilih \emph{New} → \emph{Folder} → beri nama \texttt{datasets}, lalu ulangi untuk membuat folder \texttt{output}
\item
  \textbf{Di Mac/Linux}: Klik kanan di dalam folder \texttt{Praktikum\ STP\ 2026} → pilih \emph{New Folder} → beri nama \texttt{datasets}, lalu ulangi untuk membuat folder \texttt{output}
\end{itemize}

Setiap kali kalian mendapatkan dataset baru untuk praktikum (misalnya file Excel atau CSV dari asisten), \textbf{simpanlah di dalam folder \texttt{datasets}}. Setiap kali kalian menghasilkan output analisis, \textbf{simpanlah di dalam folder \texttt{output}}.

\paragraph{Langkah 3: Mengubahnya Menjadi RStudio Project}\label{langkah-3-mengubahnya-menjadi-rstudio-project}

Setelah struktur folder dasar sudah siap, langkah selanjutnya adalah mengubahnya menjadi \textbf{RStudio Project}. RStudio Project adalah fitur yang sangat berguna untuk mengelola sesi kerja R kalian.

\textbf{Keuntungan RStudio Project:}

\begin{enumerate}
\def\labelenumi{\arabic{enumi}.}
\tightlist
\item
  \textbf{Otomatis mengatur \emph{working directory}}: Ketika kalian membuka sebuah Project, RStudio otomatis mengatur \emph{working directory} ke folder tersebut, sehingga kalian tidak perlu menggunakan perintah \texttt{setwd()} lagi.
\end{enumerate}

Perintah \texttt{setwd()} digunakan untuk mengatur \emph{working directory} secara manual. Misalnya, jika kita ingin mengatur \emph{working directory} ke folder \texttt{Praktikum\ STP\ 2026} yang terletak dalam folder \texttt{Documents} pada direktori \texttt{C:/Users/NamaAnda/}, kita bisa menggunakan perintah

\begin{Shaded}
\begin{Highlighting}[]
\FunctionTok{setwd}\NormalTok{(}\StringTok{"C:/Users/NamaAnda/Documents/Praktikum STP 2026"}\NormalTok{)}
\end{Highlighting}
\end{Shaded}

Namun, jika kita sudah membuat direktori Anda sebagai Project, dalam praktikum ini, kita tidak perlu menggunakan perintah ini lagi.

\begin{enumerate}
\def\labelenumi{\arabic{enumi}.}
\setcounter{enumi}{1}
\tightlist
\item
  \textbf{Mempertahankan status kerja}: RStudio menyimpan \emph{environment} (variabel, data yang sudah dimuat) sehingga kalian bisa melanjutkan pekerjaan dari tempat terakhir kalian berhenti.
\item
  \textbf{Memudahkan manajemen path}: Kalian cukup menulis \emph{relative path} (misalnya \texttt{datasets/data.xlsx}) tanpa perlu mengetik \emph{path} lengkap yang panjang dan berisiko error.
\item
  \textbf{Portabilitas}: Project folder bisa dipindahkan ke komputer lain atau dibagikan ke rekan tanpa perlu mengubah kode, karena semua path bersifat relatif.
\end{enumerate}

\textbf{Cara membuat RStudio Project:}

\begin{enumerate}
\def\labelenumi{\arabic{enumi}.}
\tightlist
\item
  Buka RStudio
\item
  Pilih menu \textbf{File → New Project\ldots{}}
\item
  Pilih \textbf{Existing Directory} (karena folder sudah dibuat)
\item
  Klik \textbf{Browse\ldots{}} dan cari folder \textbf{\texttt{Praktikum\ STP\ 2026}} yang sudah kalian buat
\item
  Klik \textbf{Create Project}
\end{enumerate}

Setelah itu, RStudio akan membuat file \textbf{\texttt{Praktikum\ STP\ 2026.Rproj}} di dalam folder kalian. File \texttt{.Rproj} ini adalah ``pintu masuk'' ke Project kalian.

\textbf{Setiap kali ingin bekerja}: Cukup \emph{double-click} file \textbf{\texttt{Praktikum\ STP\ 2026.Rproj}} tersebut, dan RStudio akan terbuka dengan \emph{working directory} yang sudah otomatis diatur ke folder praktikum kalian. Kalian siap untuk mulai bekerja!

\textbf{Catatan Penting:} Selalu buka RStudio melalui file \texttt{.Rproj}, bukan dengan membuka RStudio terlebih dahulu lalu membuka \emph{script}. Ini memastikan \emph{working directory} selalu benar dan kode kalian berjalan sebagaimana mestinya.

\subsubsection{Mengunduh Dataset Praktikum dari GitHub}\label{mengunduh-dataset-praktikum-dari-github}

Setelah struktur direktori dan Project sudah siap, langkah selanjutnya adalah \textbf{mengunduh dataset-dataset praktikum} yang akan digunakan sepanjang semester. Semua dataset praktikum disimpan di \emph{repository} GitHub dan dapat diunduh secara gratis.

\textbf{Mengapa Dataset Disimpan di GitHub? \{.header\}}

Dataset praktikum disimpan di GitHub agar:
- Semua mahasiswa mendapatkan dataset yang \textbf{sama} dan \textbf{terbaru}
- Dataset dapat \textbf{diakses kapan saja} tanpa perlu menunggu distribusi dari asisten
- Jika ada pembaruan dataset, kalian bisa \textbf{mengunduh ulang} dengan mudah
- Dataset \textbf{tidak hilang} meskipun laptop rusak atau file terhapus

\paragraph{Langkah-Langkah Mengunduh Dataset}\label{langkah-langkah-mengunduh-dataset}

Ikuti langkah-langkah berikut dengan seksama untuk mengunduh folder \texttt{datasets/} dari GitHub:

\textbf{1. Buka halaman GitHub repository praktikum}

Klik link berikut atau copy-paste ke browser kalian:

\textbf{\url{https://github.com/abdulmubdibindar/praktikum-r-stp}}

Kalian akan melihat halaman repository yang berisi semua file praktikum.

\textbf{2. Download seluruh repository sebagai file ZIP}

\begin{itemize}
\tightlist
\item
  Di halaman GitHub, cari dan klik tombol hijau \textbf{``Code''} di bagian kanan atas
\item
  Pada menu yang muncul, pilih \textbf{``Download ZIP''}
\item
  Simpan file ZIP di lokasi yang mudah kalian temukan (misalnya folder \textbf{Downloads} atau \textbf{Desktop})
\item
  Tunggu hingga proses download selesai
\end{itemize}

\textbf{3. Ekstrak file ZIP yang sudah didownload}

\begin{itemize}
\tightlist
\item
  Buka folder tempat kalian menyimpan file ZIP (misalnya \textbf{Downloads})
\item
  Klik kanan pada file \textbf{\texttt{praktikum-r-stp-main.zip}}
\item
  Pilih \textbf{``Extract All\ldots{}''} atau \textbf{``Extract Here''} (di Windows)
\item
  Atau untuk Mac/Linux: double-click file ZIP, akan otomatis ter-ekstrak
\item
  Setelah ekstrak selesai, akan muncul folder bernama \textbf{\texttt{praktikum-r-stp-main}}
\end{itemize}

\textbf{4. Copy folder \texttt{datasets} ke dalam Project RStudio kalian}

Ini adalah langkah paling penting! Ikuti dengan teliti:

\begin{itemize}
\tightlist
\item
  Buka folder \textbf{\texttt{praktikum-r-stp-main}} yang baru diekstrak
\item
  Di dalam folder tersebut, cari folder bernama \textbf{\texttt{datasets}}
\item
  \textbf{Klik kanan} pada folder \texttt{datasets}, lalu pilih \textbf{``Copy''} (atau tekan \texttt{Ctrl\ +\ C})
\item
  Buka folder \textbf{\texttt{Praktikum\ STP\ 2026}} (folder Project RStudio kalian)
\item
  \textbf{Klik kanan} di dalam folder tersebut, lalu pilih \textbf{``Paste''} (atau tekan \texttt{Ctrl\ +\ V})
\item
  Folder \texttt{datasets} beserta seluruh isinya sekarang sudah tersalin ke Project kalian
\end{itemize}

\textbf{5. Verifikasi dataset sudah tersimpan dengan benar}

Untuk memastikan dataset sudah tersimpan di lokasi yang benar, buka RStudio (klik file \texttt{.Rproj} kalian), lalu jalankan kode berikut di \textbf{Console}:

Jika hasilnya \texttt{TRUE}, berarti folder \texttt{datasets} sudah berada di lokasi yang benar! ✓

Selanjutnya, cek isi folder \texttt{datasets} dengan kode:

Kalian akan melihat daftar file-file dataset seperti \texttt{Data\ Praktikum\ 01.xlsx}, \texttt{Data\ Praktikum\ 02.xlsx}, \texttt{DataUtama\_mhsUBL.csv}, dan lain-lain.

\textbf{Struktur Folder yang Benar \{.header\}}

Pastikan struktur folder Project kalian sekarang terlihat seperti ini:

\begin{Shaded}
\begin{Highlighting}[]
\NormalTok{Praktikum STP 2026/           ← Folder Project kalian}
\NormalTok{├── Praktikum STP 2026.Rproj  ← File Project RStudio}
\NormalTok{├── datasets/                 ← Folder datasets (HARUS ada di sini!)}
\NormalTok{│   ├── Data Praktikum 01.xlsx}
\NormalTok{│   ├── Data Praktikum 02.xlsx}
\NormalTok{│   ├── Data Praktikum 03.xlsx}
\NormalTok{│   ├── DataUtama\_mhsUBL.csv}
\NormalTok{│   ├── DataUtama\_mhsUINRIL.csv}
\NormalTok{│   ├── DataUtama\_mhsUNILA.csv}
\NormalTok{│   ├── DataUtama\_mhsITERA.csv}
\NormalTok{│   ├── data\_mahasiswa.csv}
\NormalTok{│   └── ... (file{-}file dataset lainnya)}
\NormalTok{├── output/                   ← Folder untuk hasil analisis}
\NormalTok{└── ... (file .R kalian)}
\end{Highlighting}
\end{Shaded}

\textbf{PENTING}: Jika folder \texttt{datasets} berada di lokasi lain (misalnya di \textbf{Desktop} atau \textbf{Downloads}), kode praktikum \textbf{tidak akan bisa} membaca dataset. Pastikan folder \texttt{datasets} berada \textbf{langsung di dalam} folder \texttt{Praktikum\ STP\ 2026}.

\subsection{Sintaks Dasar Bahasa R}\label{sintaks-dasar-bahasa-r}

Dalam subbab ini, kita akan membahas sintaks dasar bahasa R yang akan sering digunakan dalam praktikum ini.

\subsubsection{Komentar}\label{komentar}

Pada dasarnya, semua yang kita tulis di \emph{script} akan dijalankan oleh R. Akan tetapi, kita bisa memberi tahu R untuk tidak menjalankan kode yang kita tulis dengan mengubah formatnya menjadi \textbf{komentar} (\emph{comment}). Kita bisa menuliskan komentar dengan menggunakan tanda pagar (\texttt{\#}) sebelum kode yang ingin kita komentari.

\begin{Shaded}
\begin{Highlighting}[]
\CommentTok{\# Kode ini akan menghitung 1 + 1}
\DecValTok{1} \SpecialCharTok{+} \DecValTok{1} 
\end{Highlighting}
\end{Shaded}

Komentar sangat berperan penting dalam kode kalian. Malahan, ini adalah kekuatan terbesar dari penggunaan R dalam Kita akan sering menggunakan komentar dalam kode kalian. Jadi, jangan ragu untuk menggunakan komentar untuk menjelaskan kode kalian.

\paragraph{\texorpdfstring{Komentar Blok (\emph{Block Comment})}{Komentar Blok (Block Comment)}}\label{komentar-blok-block-comment}

Untuk menulis komentar yang panjang (lebih dari satu baris), kita bisa menggunakan \textbf{komentar blok}. Ada dua cara:

\textbf{Cara 1: Tanda Pagar di Setiap Baris}

\begin{Shaded}
\begin{Highlighting}[]
\CommentTok{\# Ini adalah komentar blok}
\CommentTok{\# yang terdiri dari beberapa baris}
\CommentTok{\# untuk menjelaskan konsep yang kompleks}
\NormalTok{hasil }\OtherTok{\textless{}{-}}\NormalTok{ data.ubl }\SpecialCharTok{|\textgreater{}}
  \FunctionTok{filter}\NormalTok{(Jenis.Kelamin }\SpecialCharTok{==} \StringTok{"Laki{-}Laki"}\NormalTok{) }\SpecialCharTok{|\textgreater{}}
  \FunctionTok{summarize}\NormalTok{(}\AttributeTok{rata\_rata =} \FunctionTok{mean}\NormalTok{(jarak))}
\end{Highlighting}
\end{Shaded}

\textbf{Cara 2: Menggunakan if(FALSE)}

\begin{Shaded}
\begin{Highlighting}[]
\ControlFlowTok{if}\NormalTok{(}\ConstantTok{FALSE}\NormalTok{) \{}
\NormalTok{  Ini adalah komentar blok alternatif.}
\NormalTok{  Semua teks di dalam kurung kurawal tidak akan dijalankan.}
\NormalTok{  Berguna untuk men}\SpecialCharTok{{-}}\NormalTok{disable blok kode sementara saat debugging.}
\NormalTok{\}}
\end{Highlighting}
\end{Shaded}

\textbf{Kapan Menggunakan Komentar Blok?}

\begin{itemize}
\tightlist
\item
  Menulis jawaban analisis yang panjang (misalnya interpretasi hasil statistik)
\item
  Mendokumentasikan proses pemikiran Anda
\item
  Men-disable sementara kode yang tidak ingin dihapus
\item
  Menjawab soal Aktivitas Mandiri dalam bentuk teks
\end{itemize}

\subsubsection{Variabel}\label{variabel}

Variabel adalah tempat menyimpan data dan informasi di R. Variabel terdiri atas nama variabel dan nilainya, Cara mendeklarasikan variabel adalah dengan operator \texttt{\textless{}-}. Operator deklarasi bisa dimasukkan dengan \emph{hotkey} \texttt{Alt} + \texttt{I}.

Contoh:

Meskipun kita bisa menggunakan \texttt{=} untuk meng-\emph{assign} nilai ke variabel, praktik tersebut tidak lazim dipakai dalam komunitas R. Oleh karena itu, tetap gunakan \emph{assignment operator} \texttt{\textless{}-} untuk memberi nilai pada suatu variabel.

Variabel juga bisa dideklarasikan menggunakan operasi matematika dan variabel sebelumnya.

\begin{verbatim}
## [1] "Hasil: 3"
\end{verbatim}

Untuk menampilkan nilai variabel tuliskan kembali nama variabelnya lalu jalankan.

\begin{verbatim}
## [1] 1
\end{verbatim}

\begin{verbatim}
## [1] 2
\end{verbatim}

\begin{verbatim}
## [1] 3
\end{verbatim}

\subsubsection{Konvensi Penamaan dalam R}\label{konvensi-penamaan-dalam-r}

R mendukung beberapa gaya penamaan variabel. Penting untuk diingat bahwa \textbf{R membedakan huruf besar dan kecil} (\emph{case-sensitive}), sehingga \texttt{data} berbeda dengan \texttt{Data}, \texttt{JenisKelamin} berbeda dengan \texttt{jeniskelamin}, dan \texttt{"Laki-Laki"} berbeda dengan \texttt{"Laki-laki"}. Kesalahan dalam kapitalisasi adalah sumber \emph{error} yang sangat umum bagi pemula.

\textbf{Konvensi Penamaan Variabel:}

R mendukung beberapa gaya penamaan variabel:

\begin{enumerate}
\def\labelenumi{\arabic{enumi}.}
\tightlist
\item
  \textbf{snake\_case} (disarankan untuk variabel R):

  \begin{itemize}
  \tightlist
  \item
    \texttt{data\_mahasiswa}, \texttt{jarak\_tempuh}, \texttt{biaya\_sepekan}
  \item
    Menggunakan garis bawah (\texttt{\_}) untuk memisahkan kata
  \end{itemize}
\item
  \textbf{dot.notation} (umum untuk nama kolom):

  \begin{itemize}
  \tightlist
  \item
    \texttt{data.ubl}, \texttt{Jenis.Kelamin}, \texttt{biaya.dalam.sepekan}
  \item
    Menggunakan titik (\texttt{.}) untuk memisahkan kata
  \item
    Sering muncul saat mengimpor data dari Excel
  \end{itemize}
\item
  \textbf{camelCase} (jarang dipakai di R):

  \begin{itemize}
  \tightlist
  \item
    \texttt{dataMahasiswa}, \texttt{jarakTempuh}
  \item
    Huruf besar menandai awal kata baru
  \end{itemize}
\end{enumerate}

\textbf{Rekomendasi:}

\begin{itemize}
\tightlist
\item
  Gunakan \textbf{snake\_case} untuk nama variabel/objek R yang Anda buat sendiri
\item
  Gunakan \textbf{dot.notation} untuk nama kolom dataset (konsisten dengan output dari Excel)
\item
  Hindari spasi dan karakter khusus dalam nama variabel
\item
  Gunakan nama yang deskriptif: \texttt{data\_ubl} lebih baik dari \texttt{d1}
\item
  Konsisten: Pilih satu gaya dan gunakan terus sepanjang kode Anda
\end{itemize}

\textbf{Aktivitas Mandiri: Membuat Variabel Disertai Komentar}

\begin{enumerate}
\def\labelenumi{\arabic{enumi}.}
\tightlist
\item
  Buatlah variabel dengan nama \texttt{nama} yang nilainya adalah nama kalian.
\item
  Buatlah variabel dengan nama \texttt{nim} yang nilainya adalah NIM kalian.
\item
  Buatlah variabel dengan nama \texttt{prodi} yang nilainya adalah program studi kalian.
\item
  Buatlah variabel dengan nama \texttt{angkatan} yang nilainya adalah angkatan kalian.
\item
  Tambahkan komentar pada baris paling atas bahwa variabel-variabel di bawahnya adalah ``Identitas Saya''.
\item
  Tampilkan nilai variabel \texttt{nama}, \texttt{nim}, \texttt{prodi}, dan \texttt{angkatan} dengan perintah berikut.
\end{enumerate}

\begin{Shaded}
\begin{Highlighting}[]
\FunctionTok{cat}\NormalTok{(}\StringTok{"Nama saya adalah"}\NormalTok{,nama,}\StringTok{". NIM saya adalah"}\NormalTok{,nim,}\StringTok{". Saya adalah mahasiswa"}\NormalTok{,prodi,}\StringTok{" angkatan"}\NormalTok{,angkatan,}\StringTok{"."}\NormalTok{)}
\end{Highlighting}
\end{Shaded}

\subsection{\texorpdfstring{Paket (\emph{Packages})}{Paket (Packages)}}\label{paket-packages}

R memiliki ribuan paket tambahan yang menyediakan fungsi-fungsi khusus.

\begin{itemize}
\tightlist
\item
  \textbf{\texttt{install.packages("nama\_paket")}}: Digunakan untuk mengunduh dan menginstal paket dari internet ke komputer kita. Ini cukup dilakukan satu kali saja. \textbf{Perhatikan bahwa nama paket ditulis dengan tanda \texttt{"} di awal dan akhir nama paket yang akan diinstal.}
\item
  \textbf{\texttt{library(nama\_paket)}}: Digunakan untuk memuat paket yang sudah terinstal ke dalam sesi kerja kita agar isinya bisa digunakan. Ini harus dilakukan setiap kali kita membuka RStudio baru. \textbf{Perhatikan bahwa tanda \texttt{"} tidak perlu lagi ditulis ketika kita ingin memuat paket yang sudah kita instal.}
\end{itemize}

\textbf{Aktivitas Mandiri: Instalasi Paket}

Berikut adalah paket-paket yang akan kita gunakan selama praktikum ini. Instalasilah paket-paket berikut di komputer kalian:

\begin{itemize}
\tightlist
\item
  \texttt{tidyverse}
\item
  \texttt{openxlsx}
\item
  \texttt{gtsummary}
\item
  \texttt{flextable}
\item
  \texttt{treemapify}
\item
  \texttt{MKinfer}
\item
  \texttt{rcompanion}
\item
  \texttt{RCPA3}
\item
  \texttt{vcdExtra}
\item
  \texttt{psych}
\end{itemize}

Tunggu hingga masing-masing proses instalasi selesai. Ingat, tahap ini hanya perlu dilakukan satu kali saja.

\subsection{\texorpdfstring{Fungsi (\emph{Functions})}{Fungsi (Functions)}}\label{fungsi-functions}

Fungsi adalah serangkaian instruksi yang dibungkus menjadi satu perintah untuk melakukan tugas tertentu. R memiliki banyak fungsi bawaan yang berguna untuk banyak hal, seperti \texttt{print()} untuk menampilkan output, \texttt{mean()} untuk menghitung rata-rata, atau \texttt{sum()} untuk menghitung jumlah.

\subsubsection{Menggunakan Fungsi}\label{menggunakan-fungsi}

Untuk menggunakan fungsi, kita memanggil namanya diikuti dengan tanda kurung \texttt{()}. Di dalam tanda kurung, kita bisa memasukkan \textbf{argumen} (input).

\begin{Shaded}
\begin{Highlighting}[]
\CommentTok{\# Contoh fungsi bawaan: mean (rata{-}rata)}
\NormalTok{nilai }\OtherTok{\textless{}{-}} \FunctionTok{c}\NormalTok{(}\DecValTok{10}\NormalTok{, }\DecValTok{20}\NormalTok{, }\DecValTok{30}\NormalTok{)}
\NormalTok{rata\_rata }\OtherTok{\textless{}{-}} \FunctionTok{mean}\NormalTok{(nilai) }\CommentTok{\# Hasil: 20}
\end{Highlighting}
\end{Shaded}

\subsubsection{Membuat Fungsi Sendiri}\label{membuat-fungsi-sendiri}

Kita dapat membuat fungsi buatan kita sendiri menggunakan perintah \texttt{function()}. Struktur dasarnya adalah sebagai berikut:

\begin{Shaded}
\begin{Highlighting}[]
\NormalTok{nama\_fungsi }\OtherTok{\textless{}{-}} \ControlFlowTok{function}\NormalTok{(input) \{}
  \CommentTok{\# Proses yang dilakukan}
\NormalTok{  hasil }\OtherTok{\textless{}{-}}\NormalTok{ input }\SpecialCharTok{*} \DecValTok{2}
  \FunctionTok{return}\NormalTok{(hasil)}
\NormalTok{\}}
\end{Highlighting}
\end{Shaded}

Penjelasan:

\begin{itemize}
\tightlist
\item
  \texttt{nama\_fungsi}: Nama yang akan kita gunakan untuk memanggil fungsi ini.
\item
  \texttt{function(input)}: Mendefinisikan bahwa kita membuat sebuah fungsi yang menerima satu argumen bernama \texttt{input}.
\item
  \texttt{\{\ ...\ \}}: Blok kode yang berisi instruksi yang akan dijalankan.
\item
  \texttt{return(hasil)}: Mengembalikan nilai \texttt{hasil} sebagai \emph{output} dari fungsi.
\end{itemize}

\textbf{Aktivitas Mandiri: Membuat Fungsi Sendiri}

\begin{enumerate}
\def\labelenumi{\arabic{enumi}.}
\tightlist
\item
  Buatlah dua variabel dengan nama \texttt{c} dan \texttt{d} dengan nilai 10 dan 20
\item
  Buatlah fungsi dengan nama \texttt{kali\_dua} yang menerima satu argumen \texttt{x} sebagai berikut:
\end{enumerate}

\begin{Shaded}
\begin{Highlighting}[]
\NormalTok{kali\_dua }\OtherTok{\textless{}{-}} \ControlFlowTok{function}\NormalTok{(x) \{}
  \CommentTok{\# Proses yang dilakukan}
\NormalTok{  hasil }\OtherTok{\textless{}{-}}\NormalTok{ x }\SpecialCharTok{*} \DecValTok{2}
  \FunctionTok{return}\NormalTok{(hasil)}
\NormalTok{\}}
\end{Highlighting}
\end{Shaded}

\begin{enumerate}
\def\labelenumi{\arabic{enumi}.}
\setcounter{enumi}{2}
\tightlist
\item
  Jalankan fungsi tersebut dengan argumen nilai \texttt{c} dan \texttt{d} dan simpan nilainya berturut-turut pada variabel \texttt{c\_kali2} dan \texttt{d\_kali2}.
\item
  Tampilkan nilai variabel \texttt{c\_kali2} dan \texttt{d\_kali2}.
\end{enumerate}

Ini akan sangat berguna ketika kita ingin melakukan perhitungan yang sama berulang kali, seperti pada perhitungan IQV di Modul 2.

\subsection{Struktur Data Dasar}\label{struktur-data-dasar}

Kita perlu mengenal jenis-jenis struktur data yang ada di R. Mereka akan sangat sering kita gunakan selama praktikum ini.

\subsubsection{Vektor}\label{vektor}

Vektor adalah struktur data paling dasar di R yang dapat menampung lebih dari satu nilai. Kita membuatnya dengan fungsi \texttt{c()} (\emph{combine}).

Ciri utama vektor adalah semua elemen di dalamnya harus memiliki tipe data yang sama. Jika tidak, R akan mengubah tipe data elemen tersebut menjadi tipe data yang paling umum.

\begin{Shaded}
\begin{Highlighting}[]
\NormalTok{angka }\OtherTok{\textless{}{-}} \FunctionTok{c}\NormalTok{(}\DecValTok{1}\NormalTok{, }\DecValTok{2}\NormalTok{, }\DecValTok{3}\NormalTok{, }\DecValTok{4}\NormalTok{, }\DecValTok{5}\NormalTok{)}
\NormalTok{huruf }\OtherTok{\textless{}{-}} \FunctionTok{c}\NormalTok{(}\StringTok{"a"}\NormalTok{, }\StringTok{"b"}\NormalTok{, }\StringTok{"c"}\NormalTok{)}
\NormalTok{campuran }\OtherTok{\textless{}{-}} \FunctionTok{c}\NormalTok{(}\DecValTok{1}\NormalTok{, }\StringTok{"a"}\NormalTok{, }\ConstantTok{TRUE}\NormalTok{) }\CommentTok{\# R akan mengubah tipe data elemen tersebut menjadi tipe data yang paling umum, yaitu character/string}
\end{Highlighting}
\end{Shaded}

\textbf{Aktivitas Mandiri: Membuat Vektor}

\begin{enumerate}
\def\labelenumi{\arabic{enumi}.}
\tightlist
\item
  Buatlah vektor dengan nama \texttt{nilai\_ujian} yang berisi nilai ujian 5 mahasiswa berikut: 85, 90, 78, 92, 88
\item
  Buatlah vektor dengan nama \texttt{fakultas} yang berisi nama-nama fakultas di ITERA
\item
  Buatlah vektor dengan nama \texttt{apa\_saja} yang berisi nilai \texttt{1} dan \texttt{"a"}
\item
  Tampilkan tipe data dari variabel \texttt{nilai\_ujian}, \texttt{fakultas}, dan \texttt{apa\_saja}. Perhatikan hasil pada bagian \texttt{apa\_saja}. Apa jenis nilai-nilai dalam vektor tersebut?
\end{enumerate}

\subsubsection{List}\label{list}

List adalah struktur data yang dapat menampung banyak nilai yang berbeda tipe datanya.

\begin{Shaded}
\begin{Highlighting}[]
\NormalTok{list\_saya }\OtherTok{\textless{}{-}} \FunctionTok{list}\NormalTok{(}\StringTok{"kucing"}\NormalTok{, }\DecValTok{1}\NormalTok{, }\ConstantTok{TRUE}\NormalTok{)}
\NormalTok{list\_saya }\CommentTok{\# Untuk menampilkan nilai list\_saya}
\end{Highlighting}
\end{Shaded}

List juga bahkan dapat menampung vektor.

\begin{Shaded}
\begin{Highlighting}[]
\NormalTok{list\_saya }\OtherTok{\textless{}{-}} \FunctionTok{list}\NormalTok{(}\FunctionTok{c}\NormalTok{(}\DecValTok{1}\NormalTok{, }\DecValTok{2}\NormalTok{, }\DecValTok{3}\NormalTok{), }\FunctionTok{c}\NormalTok{(}\StringTok{"a"}\NormalTok{, }\StringTok{"b"}\NormalTok{, }\StringTok{"c"}\NormalTok{), }\FunctionTok{c}\NormalTok{(}\ConstantTok{TRUE}\NormalTok{, }\ConstantTok{FALSE}\NormalTok{, }\ConstantTok{TRUE}\NormalTok{))}
\NormalTok{list\_saya}
\end{Highlighting}
\end{Shaded}

Kita dapat memberi nama elemen-elemen dalam list dengan mengetikkan nama elemen dan diikuti tanda \texttt{=} sebelum nilai elemen tersebut.

\begin{Shaded}
\begin{Highlighting}[]
\NormalTok{list\_saya }\OtherTok{\textless{}{-}} \FunctionTok{list}\NormalTok{(}\AttributeTok{nama =} \StringTok{"Budi"}\NormalTok{, }\AttributeTok{nilai\_ujian =} \FunctionTok{c}\NormalTok{(}\DecValTok{85}\NormalTok{, }\DecValTok{90}\NormalTok{, }\DecValTok{78}\NormalTok{), }\AttributeTok{fakultas =} \StringTok{"FTI"}\NormalTok{)}
\end{Highlighting}
\end{Shaded}

\textbf{Aktivitas Mandiri: Membuat List}

\begin{enumerate}
\def\labelenumi{\arabic{enumi}.}
\tightlist
\item
  Buatlah list dengan nama \texttt{list\_saya} yang berisi nama, nilai ujian, dan fakultas
\item
  Tampilkan nilai variabel \texttt{list\_saya}
\end{enumerate}

\subsubsection{Tipe Data}\label{tipe-data}

Saat mengolah data, kalian akan sering menemukan singkatan tipe data berikut:

\begin{itemize}
\tightlist
\item
  \textbf{\texttt{dbl} (Double)}: Angka numerik (bisa desimal).
\item
  \textbf{\texttt{chr} (Character)}: Teks atau \emph{string}. Ditandai dengan tanda kutip.
\item
  \textbf{\texttt{fct} (Factor)}: Data kategorik (bisa nominal atau ordinal). Factor menyimpan nilai sebagai angka di belakang layar namun menampilkan label teks, yang berguna untuk analisis statistik dan plot.
\end{itemize}

\subsubsection{Data Frame dan Tibble}\label{data-frame-dan-tibble}

Data dalam R biasanya disimpan dalam bentuk tabel yang disebut \emph{Data Frame}. Di era modern (tidyverse), kita sering menggunakan versi yang lebih canggih disebut \textbf{Tibble}. Strukturnya mirip spreadsheet: baris mewakili observasi, kolom mewakili variabel.

\subsection{\texorpdfstring{Operator Pipa (\texttt{\textbar{}\textgreater{}})}{Operator Pipa (\textbar\textgreater)}}\label{operator-pipa}

Dalam \texttt{tidyverse}, kita sering menggunakan operator pipa (\texttt{\textbar{}\textgreater{}}) untuk merangkai beberapa perintah sekaligus. Bayangkan ini sebagai penyambung proses:

\begin{Shaded}
\begin{Highlighting}[]
\CommentTok{\# Tanpa pipa}
\NormalTok{hasil }\OtherTok{\textless{}{-}} \FunctionTok{fungsi\_2}\NormalTok{(}\FunctionTok{fungsi\_1}\NormalTok{(data))}

\CommentTok{\# Dengan pipa (lebih mudah dibaca)}
\NormalTok{data }\SpecialCharTok{|\textgreater{}} 
  \FunctionTok{fungsi\_1}\NormalTok{() }\SpecialCharTok{|\textgreater{}} 
  \FunctionTok{fungsi\_2}\NormalTok{()}
\end{Highlighting}
\end{Shaded}

Artinya: ``Ambil \texttt{data}, lalu \texttt{fungsi\_1}, kemudian hasilnya dimasukkan ke \texttt{fungsi\_2}.''

Kita akan mempelajari penggunaan operator pipa (\texttt{\textbar{}\textgreater{}}) secara lebih praktis dalam \ref{modul-1}.

\subsection{Manajemen Data (Input/Output)}\label{manajemen-data-inputoutput}

Ini adalah kemampuan yang penting dalam pengolahan data di R karena seringkali kita perlu mengolah data dari file eksternal (seperti Excel) atau menyimpan data hasil pengolahan ke file eksternal.

Kita akan mempraktikkan kemampuan ini lebih banyak mulai di \ref{modul-1}.

\subsubsection{Membaca Data}\label{membaca-data}

Kita sering menggunakan \texttt{read.xlsx()} dari paket \texttt{openxlsx} untuk membaca file Excel.

\begin{Shaded}
\begin{Highlighting}[]
\NormalTok{data }\OtherTok{\textless{}{-}} \FunctionTok{read.xlsx}\NormalTok{(}\StringTok{"folder/nama\_file.xlsx"}\NormalTok{, }\AttributeTok{sheet =} \StringTok{"nama\_sheet"}\NormalTok{)}
\end{Highlighting}
\end{Shaded}

\subsubsection{Menyimpan Data}\label{menyimpan-data}

Setelah diolah, data bisa disimpan kembali ke Excel atau CSV.

\begin{Shaded}
\begin{Highlighting}[]
\FunctionTok{write.xlsx}\NormalTok{(data, }\StringTok{"folder/nama\_file\_baru.xlsx"}\NormalTok{)}
\FunctionTok{write.csv2}\NormalTok{(data, }\StringTok{"folder/nama\_file\_baru.csv"}\NormalTok{) }\CommentTok{\# Format CSV Excel Indonesia (pemisah titik koma)}
\end{Highlighting}
\end{Shaded}

\section{Modul-1: Pengolahan Data Terstruktur}\label{modul-1}

Setelah mempelajari modul ini Anda diharapkan dapat:

\begin{enumerate}
\def\labelenumi{\arabic{enumi}.}
\tightlist
\item
  menganalisis variabel dan objek dalam sebuah format data terstruktur \capaian{STP-1.3}
\item
  mengelola data terstruktur dengan tepat menggunakan perangkat komputer \capaian{STP-1.4}
\end{enumerate}

\begin{center}\rule{0.5\linewidth}{0.5pt}\end{center}

Kita akan mempelajari cara pengolahan data terstruktur atau \emph{dataset}/set data dari hasil kuesioner kepada mahasiswa-mahasiswa Universitas Islam Negeri Raden Intan Lampung (UINRIL) dan Universitas Bandar Lampung (UBL) tentang pola perjalanan mereka.

Dalam R, pengolahan data terstruktur dipermudah dengan adanya paket \texttt{tidyverse}. Paket ini menyertakan paket-paket lain yang mempunyai kegunaan unik dalam pengolahan data terstruktur.

Seperti yang kalian lihat, paket \texttt{tidyverse} menyertakan paket-paket berikut yang mempunyai kegunaan masing-masing:

\begin{itemize}
\tightlist
\item
  \texttt{dplyr}: menyediakan serangkaian fungsi inti untuk memanipulasi dan mentransformasi data secara efisien.
\item
  \texttt{forcats}: memudahkan pekerjaan dengan data kategorikal atau yang dalam R disebut juga `factors'.
\item
  \texttt{ggplot2}: untuk membuat grafik yang ciamik.
\item
  \texttt{lubridate}: mempermudah pekerjaan terkait format tanggal
\item
  \texttt{purrr}: meningkatkan kemampuan pemrograman fungsi dan vektor
\item
  \texttt{readr}: untuk membaca data tabular seperti \texttt{.csv}
\item
  \texttt{stringr}: menyederhanakan operasi teks/\emph{string}
\item
  \texttt{tibble}: versi modern dari \emph{data.frame} yang lebih rapi dan mudah digunakan.
\item
  \texttt{tidyr}: untuk merapikan data dengan mengubah format atau struktur tata letaknya.
\end{itemize}

Selain paket \texttt{tidyverse}, kita juga akan menggunakan paket \texttt{openxlsx} untuk mengoperasikan \emph{file} Excel.

\subsection{\texorpdfstring{Mengimpor set data \emph{(dataset)} dari Excel}{Mengimpor set data (dataset) dari Excel}}\label{mengimpor-set-data-dataset-dari-excel}

Untuk membuat set data di lingkungan kerja R kita, gunakan fungsi \texttt{read.xlsx()} seperti berikut. Fungsi \texttt{read.xlsx()} mempunyai atribut utama berupa \emph{file path} lokasi file yang ditempatkan di dalam tanda kurung.

⚠️ \textbf{Penting}

Pastikan file Excel yang akan diimpor sudah tertutup, karena mengimpor file yang sedang terbuka akan menghasilkan \emph{error}.

\begin{verbatim}
##   Kampus_PT Nomor.urut Jenis.Kelamin Umur                   Fakultas
## 1    UINRIL          1     Laki-laki   22                    Syariah
## 2    UINRIL          2     Laki-laki   25                    Syariah
## 3    UINRIL          3     Perempuan   24      Tarbiyah dan Keguruan
## 4    UINRIL          4     Laki-laki   19 Dakwah dan Ilmu Komunikasi
## 5    UINRIL          5     Laki-laki   23      Tarbiyah dan Keguruan
## 6    UINRIL          6     Laki-laki   21                       Adab
##                                   Prodi            Tingkat.Semester  Uang.Saku
## 1                        Hukum Keluarga 4 (Semester 7 - Semester 8) 1Jt - 2 Jt
## 2                        Hukum Keluarga 4 (Semester 7 - Semester 8)     < 1 Jt
## 3                Pendidikan Agama Islam 4 (Semester 7 - Semester 8)     < 1 Jt
## 4        Komunikasi dan Penyiaran Islam 1 (Semester 1 - Semester 2)     < 1 Jt
## 5            Manajemen Pendidikan Islam 4 (Semester 7 - Semester 8)     < 1 Jt
## 6 Ilmu Perpustakaan dan Informasi Islam 4 (Semester 7 - Semester 8)     < 1 Jt
##   kepemilikan.mobil kepemilikan.motor kepemilikan.sepedan      kendaraan.utama
## 1                 0                 1                   0 Sepeda Motor Pribadi
## 2                 1                 4                   0 Sepeda Motor Pribadi
## 3                 0                 0                   0        Berjalan Kaki
## 4                 0                 1                   0 Sepeda Motor Pribadi
## 5                 0                 1                   0 Sepeda Motor Pribadi
## 6                 0                 1                   0 Sepeda Motor Pribadi
##      kelurahan          jenis.tempat.tinggal
## 1       Wiyono         Rumah Bersama Saudara
## 2     Sukarame                   Kos Sendiri
## 3     Sukarame                   Kos Sendiri
## 4 Harapan Jaya                   Kos Sendiri
## 5  Korpri Raya Rumah Mengontrak Bersama-sama
## 6    Rajabasa   Rumah Pribadi/Rumah Keluarga
##                 nama.jalan.tempat.tinggal
## 1                            Pasar Wiyono
## 2                     jln endro suratmin 
## 3                         Jl. Pulau ambon
## 4           Jl.pulau pisang, gg way pesai
## 5                              Jl. Durian
## 6 Jln Nunyai Gg. Hi. Ismail Blok B NO 22E
##   alasan.pemilihan.lokasi.tempat.tinggal jarak.(km) biaya.dalam.ribu2
## 1                              Lain-lain      19.27               100
## 2                    Dekat dengan kampus       0.58                80
## 3                    Dekat dengan kampus       0.56                 0
## 4                    Dekat dengan kampus       1.05                50
## 5                    Dekat dengan kampus       1.69                50
## 6                    Dekat dengan kampus       7.91                50
##   Jumlah.Perjalanan.Senin Jumlah.Perjalanan.Selasa Jumlah.Perjalanan.Rabu
## 1                       3                        3                      0
## 2                       3                        3                      0
## 3                       3                        0                      3
## 4                       0                        2                      0
## 5                       4                        4                      0
## 6                       0                        2                      2
##   Jumlah.Perjalanan.Kamis Jumlah.Perjalanan.Jumat Jumlah.Perjalanan.Sabtu
## 1                       0                       0                       0
## 2                       3                       0                       0
## 3                       0                       0                       0
## 4                       2                       2                       0
## 5                       0                       3                       2
## 6                       0                       2                       2
##   Jumlah.Perjalanan.Ahad
## 1                      0
## 2                      0
## 3                      0
## 4                      2
## 5                      0
## 6                      2
\end{verbatim}

Fungsi tersebut akan mengimpor seluruh isi dari file Excel \texttt{Data\ Gabungan.xlsx} pada \emph{sheet} pertama saja.

Apabila kita ingin mengimpor set data dari \emph{sheet} lainnya, kita harus menyertakan detil nama sheet yang kita ingin impor pada atribut \texttt{sheet}.

\emph{Dataset} yang kita impor akan menjadi objek dengan jenis \emph{tibble}.

\begin{verbatim}
##   Timestamp Kampus_PT Nomor.urut Jenis.Kelamin Umur
## 1  45413.83       UBL          1     Perempuan   21
## 2  45413.83       UBL          2     Laki-Laki   20
## 3  45413.85       UBL          3     Laki-Laki   21
## 4  45413.85       UBL          4     Laki-Laki   22
## 5  45413.88       UBL          5     Laki-Laki   21
## 6  45413.92       UBL          6     Perempuan   21
##                           Fakultas           Prodi            Tingkat.Semester
## 1 Fakultas Ilmu Sosial dan Politik Ilmu Komunikasi 4 (Semester 7 - Semester 8)
## 2                   Fakultas Hukum      Ilmu Hukum 3 (Semester 5 - Semester 6)
## 3      Fakultas Ekonomi dan Bisnis       Manajemen 4 (Semester 7 - Semester 8)
## 4      Fakultas Ekonomi dan Bisnis       Akuntansi 4 (Semester 7 - Semester 8)
## 5      Fakultas Ekonomi dan Bisnis       Manajemen 4 (Semester 7 - Semester 8)
## 6      Fakultas Ekonomi dan Bisnis       Akuntansi 4 (Semester 7 - Semester 8)
##       Uang.Saku kepemilikan.mobil kepemilikan.motor kepemilikan.sepeda
## 1        < 1 jt                 1                 2                  1
## 2   1 jt – 2 jt                 2                 1                  1
## 3   1 jt – 2 jt                 1                 2                  2
## 4   1 jt – 2 jt                 1                 4                  1
## 5 2,1 jt – 3 jt                 4                 4                  1
## 6   1 jt – 2 jt                 1                 2                  1
##                                        kendaraan.utama               kelurahan
## 1                                 Sepeda Motor Pribadi       Kalibalau Kencana
## 2                                        Mobil Pribadi            Gunung sulah
## 3                                 Sepeda Motor Pribadi              langkapura
## 4 Kendaraan Bermotor (menumpang dengan keluarga/teman) Bilabong JAYA JAYA JAYA
## 5                                        Mobil Pribadi                sukarame
## 6                                  Transportasi Online               Way halim
##           jenis.tempat.tinggal                       nama.jalan.tempat.tinggal
## 1 Rumah pribadi/rumah keluarga                                    eLBe Loundry
## 2 Rumah pribadi/rumah keluarga                         Jl.Urip Sumoharjo no 88
## 3 Rumah pribadi/rumah keluarga                  JL DARUSSALAM GG LANGGAR LK II
## 4 Rumah pribadi/rumah keluarga Jl Darussalam bilabong bila bolong di jait dong
## 5 Rumah pribadi/rumah keluarga          jl.cendrawasih sukarame bandar lampung
## 6 Rumah pribadi/rumah keluarga                            Jl P tabuan nomor 26
##   jarak.(km)         alasan.pemilihan.lokasi.tempat.tinggal
## 1   4.223797                 Bersama keluarga/saudara/teman
## 2   2.703331 Mudahnya akses berpergian dari tempat tinggal 
## 3   3.237072                 Bersama keluarga/saudara/teman
## 4   3.002336    Lingkungan nyaman karna aman dari kejahatan
## 5   5.961925                    Dekat dengan fasilitas umum
## 6   2.925331                 Bersama keluarga/saudara/teman
##                 biaya.dalam.seminggu Jumlah.perjalanan.Senin
## 1                            25 ribu                       2
## 2                                200                       3
## 3 50rb menggunakan kendaraan pribadi                       1
## 4                      Sminggu 20 rb                       2
## 5                     20 JUTA RUPIAH                       1
## 6                              40 rb                       3
##   Jumlah.Perjalanan.Selasa Jumlah.Perjalanan.Rabu Jumlah.Perjalanan.Kamis
## 1                        3                      2                       3
## 2                        3                      3                       3
## 3                        1                      1                       1
## 4                        2                      2                       2
## 5                        1                      1                       1
## 6                        3                      1                       1
##   Jumlah.Perjalanan.Jumat Jumlah.Perjalanan.Sabtu Jumlah.Perjalanan.Ahad
## 1                       2                       2                      2
## 2                       3                       3                      3
## 3                       1                       1                      1
## 4                       2                       2                      2
## 5                       1                       1                      1
## 6                       1                       1                      1
\end{verbatim}

Kita juga dapat melihat \emph{dataset} kita secara lengkap dengan perintah \texttt{View()}

\subsection{Analisis data terstruktur pada set data}\label{analisis-data-terstruktur-pada-set-data}

Kita sudah mempelajari bahwa data terstruktur terdiri atas \textbf{objek}, \textbf{variabel}, dan \textbf{nilai}. Satu objek diwakili oleh satu baris, variabel oleh kolom, dan nilai berada di dalam sel-sel yang merupakan perpotongan kolom dan baris.

\begin{quote}
⚠️ \textbf{Penting}

Di modul ini kita menggunakan istilah ``variabel'' untuk menyebut dua hal yang berbeda. Makna pertama mengacu pada penyimpan nilai di R, sementara yang kedua adalah elemen data terstruktur yang menjadi kolom-kolom.

Dalam modul ini kita akan menggunakan istilah yang sama untuk kedua makna, jadi pahami konteks kalimatnya terlebih dahulu ketika menemukan kata ini.
\end{quote}

Untuk mengetahui ringkasan struktur data, kita dapat menggunakan fungsi \texttt{glimpse()} yang membutuhkan set data yang kita simpan sebagai argumen.

\begin{verbatim}
## Rows: 400
## Columns: 25
## $ Kampus_PT                              <chr> "UINRIL", "UINRIL", "UINRIL", "~
## $ Nomor.urut                             <dbl> 1, 2, 3, 4, 5, 6, 7, 8, 9, 10, ~
## $ Jenis.Kelamin                          <chr> "Laki-laki", "Laki-laki", "Pere~
## $ Umur                                   <dbl> 22, 25, 24, 19, 23, 21, 22, 21,~
## $ Fakultas                               <chr> "Syariah", "Syariah", "Tarbiyah~
## $ Prodi                                  <chr> "Hukum Keluarga", "Hukum Keluar~
## $ Tingkat.Semester                       <chr> "4 (Semester 7 - Semester 8)", ~
## $ Uang.Saku                              <chr> "1Jt - 2 Jt", "< 1 Jt", "< 1 Jt~
## $ kepemilikan.mobil                      <dbl> 0, 1, 0, 0, 0, 0, 1, 1, 1, 0, 0~
## $ kepemilikan.motor                      <dbl> 1, 4, 0, 1, 1, 1, 1, 4, 4, 1, 0~
## $ kepemilikan.sepedan                    <dbl> 0, 0, 0, 0, 0, 0, 0, 0, 1, 0, 0~
## $ kendaraan.utama                        <chr> "Sepeda Motor Pribadi", "Sepeda~
## $ kelurahan                              <chr> "Wiyono", "Sukarame", "Sukarame~
## $ jenis.tempat.tinggal                   <chr> "Rumah Bersama Saudara", "Kos S~
## $ nama.jalan.tempat.tinggal              <chr> "Pasar Wiyono", "jln endro sura~
## $ alasan.pemilihan.lokasi.tempat.tinggal <chr> "Lain-lain", "Dekat dengan kamp~
## $ `jarak.(km)`                           <dbl> 19.27, 0.58, 0.56, 1.05, 1.69, ~
## $ biaya.dalam.ribu2                      <dbl> 100, 80, 0, 50, 50, 50, 35, 50,~
## $ Jumlah.Perjalanan.Senin                <dbl> 3, 3, 3, 0, 4, 0, 3, 4, 2, 3, 4~
## $ Jumlah.Perjalanan.Selasa               <dbl> 3, 3, 0, 2, 4, 2, 0, 4, 0, 2, 4~
## $ Jumlah.Perjalanan.Rabu                 <dbl> 0, 0, 3, 0, 0, 2, 3, 4, 3, 2, 4~
## $ Jumlah.Perjalanan.Kamis                <dbl> 0, 3, 0, 2, 0, 0, 0, 4, 0, 2, 2~
## $ Jumlah.Perjalanan.Jumat                <dbl> 0, 0, 0, 2, 3, 2, 2, 4, 0, 0, 2~
## $ Jumlah.Perjalanan.Sabtu                <dbl> 0, 0, 0, 0, 2, 2, 2, 2, 2, 0, 2~
## $ Jumlah.Perjalanan.Ahad                 <dbl> 0, 0, 0, 2, 0, 2, 2, 2, 2, 0, 2~
\end{verbatim}

Hasil dari fungsi \texttt{glimpse()} tersebut menunjukkan jumlah baris, jumlah kolom, dan kolom-kolom yang ada dalam set data kita beserta rinciannya. Set data kita memiliki 400 baris yang ditunjukkan oleh \textbf{\texttt{Rows}} dan 25 kolom yang ditunjukkan oleh \textbf{\texttt{Columns}} . Ini artinya, set data kita memiliki 400 objek dan 25 variabel.

Rincian nama-nama variabel ditunjukkan pada baris keluaran yang diawali oleh simbol \textbf{\texttt{\textbackslash{}\$}}. Singkatan-singkatan di dalam kurung sudut \texttt{\textless{}\textgreater{}} bermakna jenis nilai dari variabel tersebut. \href{https://cran.r-project.org/web/packages/tibble/vignettes/types.html}{Laman ini} memberikan penjelasan makna setiap singkatan tersebut.

Terlihat dalam data kita bahwa hanya ada dua tipe nilai variabel: \texttt{dbl} dan \texttt{chr} yang berarti \emph{double} (bilangan desimal) dan \emph{character} (huruf/\emph{string}).

\textbf{Aktivitas Mandiri 1: Menganalisis Struktur Data {[}STP-1.3{]}}

Jalankan \texttt{glimpse(data)} untuk dataset UIN RIL dan jawab:

\begin{enumerate}
\def\labelenumi{\arabic{enumi}.}
\tightlist
\item
  Berapa jumlah objek (baris) dalam dataset ini?
\item
  Berapa jumlah variabel (kolom)?
\item
  Sebutkan 3 variabel dengan tipe \texttt{dbl} dan 3 variabel dengan tipe \texttt{chr}
\item
  Mengapa variabel \texttt{Jenis.Kelamin} bertipe \texttt{chr} bukan \texttt{dbl}? Jelaskan!
\end{enumerate}

\subsection{Mengolah set data}\label{mengolah-set-data}

Dalam bagian ini kita akan mengolah set data kita yang bertujuan untuk memenuhi kebutuhan pengolahan data kita. Sebelum melakukan pengolahan, maka kita perlu menjawab pertanyaan terlebih dahulu ``apa yang kita butuhkan untuk pengolahan data kita?''

Dalam praktikum ini kita sudah mengetahui bahwa terdapat dua jenis set data hasil kuesioner, yakni set data mahasiswa Universitas Islam Negeri Raden Intan Lampung (UINRIL) dan juga Universitas Bandar Lampung (UBL). Kedua jenis set data tersebut disimpan dalam variabel yang berbeda. Kita butuh untuk \emph{menggabungkan kedua set data} tersebut menjadi satu set data yang mencakup kedua kampus agar nanti kita bisa melakukan analisis dengan lebih leluasa.

Pengolahan set data seperti ini punya istilah yang keren: \textbf{\emph{data wrangling}} (bisa dibaca ``data rengling'')

\subsubsection{Merancang Algoritma (Urutan kerja)}\label{merancang-algoritma-urutan-kerja}

Karena R adalah program yang mayoritasnya \emph{character user interface} (CUI) alih-alih \emph{graphical user interface}, maka pekerjaan kita tidak bisa seleluasa GUI yang dapat kita bolak-balik urutannya. Kita sebaiknya memiliki ``rencana kerja'' yang jelas, yang dalam bahasa pemrograman disebut \textbf{algoritma}.

Algoritma yang akan kita lakukan dalam praktikum ini adalah sebagai berikut:

\begin{enumerate}
\def\labelenumi{\arabic{enumi}.}
\tightlist
\item
  Mengidentifikasi nama-nama kolom di set data UIN RIL
\item
  Mengidentifikasi nama-nama kolom di set data UBL
\item
  Menentukan kolom-kolom yang akan diambil di set data baru
\item
  Menyeleksi kolom-kolom yang akan dipakai di set data UIN RIL dan UBL,
\item
  Mengubah nama kolom-kolom di kedua set data menjadi nama-nama yang sudah ditentukan di langkah-3.
\item
  Menyatukan kedua set data.
\end{enumerate}

\begin{quote}
💡 \textbf{Tip}

Sebisa mungkin rancang setiap langkah pada urutan kerja kita dengan rinci agar kita dapat menerjemahkan langkah-langkah tersebut dengan lebih mudah ke baris kode.
\end{quote}

\subsubsection{Mengidentifikasi Nama-nama Kolom}\label{mengidentifikasi-nama-nama-kolom}

Untuk mengidentifikasi nama-nama kolom kita dapat menggunakan fungsi \texttt{colnames()} yang mengambil argumen nama set data.

\begin{verbatim}
##  [1] "Kampus_PT"                             
##  [2] "Nomor.urut"                            
##  [3] "Jenis.Kelamin"                         
##  [4] "Umur"                                  
##  [5] "Fakultas"                              
##  [6] "Prodi"                                 
##  [7] "Tingkat.Semester"                      
##  [8] "Uang.Saku"                             
##  [9] "kepemilikan.mobil"                     
## [10] "kepemilikan.motor"                     
## [11] "kepemilikan.sepedan"                   
## [12] "kendaraan.utama"                       
## [13] "kelurahan"                             
## [14] "jenis.tempat.tinggal"                  
## [15] "nama.jalan.tempat.tinggal"             
## [16] "alasan.pemilihan.lokasi.tempat.tinggal"
## [17] "jarak.(km)"                            
## [18] "biaya.dalam.ribu2"                     
## [19] "Jumlah.Perjalanan.Senin"               
## [20] "Jumlah.Perjalanan.Selasa"              
## [21] "Jumlah.Perjalanan.Rabu"                
## [22] "Jumlah.Perjalanan.Kamis"               
## [23] "Jumlah.Perjalanan.Jumat"               
## [24] "Jumlah.Perjalanan.Sabtu"               
## [25] "Jumlah.Perjalanan.Ahad"
\end{verbatim}

\begin{verbatim}
##  [1] "Timestamp"                             
##  [2] "Kampus_PT"                             
##  [3] "Nomor.urut"                            
##  [4] "Jenis.Kelamin"                         
##  [5] "Umur"                                  
##  [6] "Fakultas"                              
##  [7] "Prodi"                                 
##  [8] "Tingkat.Semester"                      
##  [9] "Uang.Saku"                             
## [10] "kepemilikan.mobil"                     
## [11] "kepemilikan.motor"                     
## [12] "kepemilikan.sepeda"                    
## [13] "kendaraan.utama"                       
## [14] "kelurahan"                             
## [15] "jenis.tempat.tinggal"                  
## [16] "nama.jalan.tempat.tinggal"             
## [17] "jarak.(km)"                            
## [18] "alasan.pemilihan.lokasi.tempat.tinggal"
## [19] "biaya.dalam.seminggu"                  
## [20] "Jumlah.perjalanan.Senin"               
## [21] "Jumlah.Perjalanan.Selasa"              
## [22] "Jumlah.Perjalanan.Rabu"                
## [23] "Jumlah.Perjalanan.Kamis"               
## [24] "Jumlah.Perjalanan.Jumat"               
## [25] "Jumlah.Perjalanan.Sabtu"               
## [26] "Jumlah.Perjalanan.Ahad"
\end{verbatim}

Kita dapat mengetahui jumlah kolom setiap set data secara langsung dengan fungsi \texttt{length()}.

\begin{verbatim}
## [1] 25
\end{verbatim}

\begin{verbatim}
## [1] 26
\end{verbatim}

Hasil ini menunjukkan bahwa kedua dataset memiliki jumlah kolom yang sama (25 kolom). Meskipun jumlahnya sama, kita sudah melihat bahwa ada perbedaan penamaan di antara keduanya yang perlu diseragamkan.

\begin{quote}
⚠️ Penting!

Fungsi \texttt{length()} pada dasarnya adalah fungsi penghitung jumlah elemen dalam suatu vektor atau jenis data lain yang berupa daftar.
\end{quote}

\begin{verbatim}
## [1] 5
\end{verbatim}

Dari kedua fungsi tersebut, kita dapat mengetahui bahwa terdapat perbedaan jumlah kolom dan juga perbedaan penamaan:

\begin{itemize}
\item
  \texttt{kepemilikan.sepedan} di data UIN vs \texttt{kepemilikan.sepeda} di data UBL.
\item
  \texttt{biaya.dalam.ribu2} di data UIN vs \texttt{biaya.dalam.seminggu} di data UBL.
\end{itemize}

Perbedaan-perbedaan inilah yang membuat kita tidak bisa langsung menggabungkan data. Kita harus menyeragamkannya terlebih dahulu,

\subsubsection{Menentukan Kolom-kolom yang Akan Diambil di Set Data Baru}\label{menentukan-kolom-kolom-yang-akan-diambil-di-set-data-baru}

Setelah mengidentifikasi nama-nama variabel dan jumlahnya, kita putuskan untuk mengambil variabel-variabel berikut: (1) \texttt{Nomor.urut}, (2) \texttt{Jenis.Kelamin}, (3) \texttt{Umur}, (4) \texttt{Fakultas}, (5) \texttt{Prodi}, (6) \texttt{Tingkat.Semester}, (7) \texttt{Uang.Saku}, (8) \texttt{kepemilikan.mobil}, (9) \texttt{kepemilikan.motor}, (10) \texttt{kepemilikan.sepeda}, (11) \texttt{kendaraan.utama}, (12) \texttt{jenis.tempat.tinggal}, (13) \texttt{jarak}, (14) \texttt{biaya.dalam.seminggu}, (15) \texttt{Jumlah.perjalanan.Senin}, (16) \texttt{Jumlah.perjalanan.Selasa}, (17) \texttt{Jumlah.perjalanan.Rabu}, (18) \texttt{Jumlah.perjalanan.Kamis}, (19) \texttt{Jumlah.perjalanan.Jumat}, (20) \texttt{Jumlah.perjalanan.Sabtu}, (21) \texttt{Jumlah.perjalanan.Ahad}.

Kita akan menyimpan nama-nama kolom yang sudah kita tentukan tersebut menjadi sebuah vektor.

\subsubsection{Menyeleksi Kolom-kolom Yang Akan Dipakai}\label{menyeleksi-kolom-kolom-yang-akan-dipakai}

Kita akan menggunakan fungsi \texttt{select()} dari \texttt{dplyr} (bagian dari \texttt{tidyverse}) untuk memilih hanya kolom-kolom yang relevan dari masing-masing set data. Perhatikan baik-baik, kita harus menggunakan \textbf{nama kolom asli} saat menyeleksi.

Sekarang kita punya dua \emph{data frame} baru (\texttt{data\_uin\_selected} dan \texttt{data\_ubl\_selected}) yang isinya hanya kolom-kolom yang kita butuhkan.

\subsubsection{Mengubah Nama Kolom Menjadi Nama Standar}\label{mengubah-nama-kolom-menjadi-nama-standar}

Ini adalah langkah kuncinya. Kita cukup menimpa nama-nama kolom di kedua \emph{data frame} yang sudah kita seleksi tadi dengan nama-nama kolom dari vektor \texttt{kolom\_final} yang kita buat di 3.3.3.

Untuk membuktikan bahwa namanya sudah berubah, kita bisa cek lagi.

\begin{verbatim}
##  [1] "Nomor.urut"               "Jenis.Kelamin"           
##  [3] "Umur"                     "Fakultas"                
##  [5] "Prodi"                    "Tingkat.Semester"        
##  [7] "Uang.Saku"                "kepemilikan.mobil"       
##  [9] "kepemilikan.motor"        "kepemilikan.sepeda"      
## [11] "kendaraan.utama"          "jenis.tempat.tinggal"    
## [13] "jarak"                    "biaya.dalam.sepekan"     
## [15] "Jumlah.perjalanan.Senin"  "Jumlah.perjalanan.Selasa"
## [17] "Jumlah.perjalanan.Rabu"   "Jumlah.perjalanan.Kamis" 
## [19] "Jumlah.perjalanan.Jumat"  "Jumlah.perjalanan.Sabtu" 
## [21] "Jumlah.perjalanan.Ahad"
\end{verbatim}

\begin{verbatim}
## [1] "============================================================"
\end{verbatim}

\begin{verbatim}
##  [1] "Nomor.urut"               "Jenis.Kelamin"           
##  [3] "Umur"                     "Fakultas"                
##  [5] "Prodi"                    "Tingkat.Semester"        
##  [7] "Uang.Saku"                "kepemilikan.mobil"       
##  [9] "kepemilikan.motor"        "kepemilikan.sepeda"      
## [11] "kendaraan.utama"          "jenis.tempat.tinggal"    
## [13] "jarak"                    "biaya.dalam.sepekan"     
## [15] "Jumlah.perjalanan.Senin"  "Jumlah.perjalanan.Selasa"
## [17] "Jumlah.perjalanan.Rabu"   "Jumlah.perjalanan.Kamis" 
## [19] "Jumlah.perjalanan.Jumat"  "Jumlah.perjalanan.Sabtu" 
## [21] "Jumlah.perjalanan.Ahad"
\end{verbatim}

Hasil dari kedua perintah di atas seharusnya adalah daftar nama yang identik.

\textbf{Aktivitas Mandiri 2: Seleksi dan Rename Kolom {[}STP-1.4{]}}

Gunakan dataset \texttt{data\_ubl} yang sudah diimpor:

\begin{enumerate}
\def\labelenumi{\arabic{enumi}.}
\tightlist
\item
  Pilih hanya kolom: \texttt{Nomor.urut}, \texttt{Jenis.Kelamin}, \texttt{Umur}, \texttt{Fakultas}, \texttt{Prodi}
\item
  Rename kolom \texttt{Jenis.Kelamin} menjadi \texttt{JK} (gunakan fungsi \texttt{rename()})
\item
  Simpan hasilnya ke variabel \texttt{data\_ubl\_subset}
\item
  Gunakan \texttt{glimpse(data\_ubl\_subset)} untuk memverifikasi hanya ada 5 kolom dan nama \texttt{JK} sudah benar
\end{enumerate}

\subsubsection{Menyatukan Kedua Set Data}\label{menyatukan-kedua-set-data}

Ini adalah langkah terakhir kita. Karena kedua set data kini memiliki struktur kolom yang sama persis, kita bisa menyatukannya menjadi satu \emph{data frame} besar dengan fungsi \texttt{bind\_rows()}.

Hasil dari penggabungan tersebut menyebabkan \emph{error} pada kolom \texttt{biaya.dalam.sepekan} . Ini disebabkan oleh \texttt{biaya.dalam.sepekan} dalam set data UINRIL (\texttt{data}) berbeda jenisnya dengan set data UBL (\texttt{data\_ubl}). Begitu pula \texttt{Jumlah.perjalanan.Senin}

\begin{verbatim}
## Rows: 400
## Columns: 21
## $ Nomor.urut               <dbl> 1, 2, 3, 4, 5, 6, 7, 8, 9, 10, 11, 12, 13, 14~
## $ Jenis.Kelamin            <chr> "Laki-laki", "Laki-laki", "Perempuan", "Laki-~
## $ Umur                     <dbl> 22, 25, 24, 19, 23, 21, 22, 21, 21, 21, 19, 1~
## $ Fakultas                 <chr> "Syariah", "Syariah", "Tarbiyah dan Keguruan"~
## $ Prodi                    <chr> "Hukum Keluarga", "Hukum Keluarga", "Pendidik~
## $ Tingkat.Semester         <chr> "4 (Semester 7 - Semester 8)", "4 (Semester 7~
## $ Uang.Saku                <chr> "1Jt - 2 Jt", "< 1 Jt", "< 1 Jt", "< 1 Jt", "~
## $ kepemilikan.mobil        <dbl> 0, 1, 0, 0, 0, 0, 1, 1, 1, 0, 0, 0, 0, 0, 0, ~
## $ kepemilikan.motor        <dbl> 1, 4, 0, 1, 1, 1, 1, 4, 4, 1, 0, 1, 3, 1, 0, ~
## $ kepemilikan.sepeda       <dbl> 0, 0, 0, 0, 0, 0, 0, 0, 1, 0, 0, 0, 1, 0, 0, ~
## $ kendaraan.utama          <chr> "Sepeda Motor Pribadi", "Sepeda Motor Pribadi~
## $ jenis.tempat.tinggal     <chr> "Rumah Bersama Saudara", "Kos Sendiri", "Kos ~
## $ jarak                    <dbl> 19.27, 0.58, 0.56, 1.05, 1.69, 7.91, 2.58, 2.~
## $ biaya.dalam.sepekan      <dbl> 100, 80, 0, 50, 50, 50, 35, 50, 40, 50, 0, 50~
## $ Jumlah.perjalanan.Senin  <dbl> 3, 3, 3, 0, 4, 0, 3, 4, 2, 3, 4, 2, 3, 2, 3, ~
## $ Jumlah.perjalanan.Selasa <dbl> 3, 3, 0, 2, 4, 2, 0, 4, 0, 2, 4, 2, 2, 2, 3, ~
## $ Jumlah.perjalanan.Rabu   <dbl> 0, 0, 3, 0, 0, 2, 3, 4, 3, 2, 4, 2, 2, 2, 2, ~
## $ Jumlah.perjalanan.Kamis  <dbl> 0, 3, 0, 2, 0, 0, 0, 4, 0, 2, 2, 2, 2, 2, 0, ~
## $ Jumlah.perjalanan.Jumat  <dbl> 0, 0, 0, 2, 3, 2, 2, 4, 0, 0, 2, 2, 2, 2, 2, ~
## $ Jumlah.perjalanan.Sabtu  <dbl> 0, 0, 0, 0, 2, 2, 2, 2, 2, 0, 2, 2, 0, 0, 0, ~
## $ Jumlah.perjalanan.Ahad   <dbl> 0, 0, 0, 2, 0, 2, 2, 2, 2, 0, 2, 2, 0, 2, 2, ~
\end{verbatim}

\begin{verbatim}
## [1] "================================================================================="
\end{verbatim}

\begin{verbatim}
## Rows: 379
## Columns: 21
## $ Nomor.urut               <dbl> 1, 2, 3, 4, 5, 6, 7, 8, 9, 10, 11, 12, 13, 14~
## $ Jenis.Kelamin            <chr> "Perempuan", "Laki-Laki", "Laki-Laki", "Laki-~
## $ Umur                     <dbl> 21, 20, 21, 22, 21, 21, 22, 22, 22, 22, 21, 2~
## $ Fakultas                 <chr> "Fakultas Ilmu Sosial dan Politik", "Fakultas~
## $ Prodi                    <chr> "Ilmu Komunikasi", "Ilmu Hukum", "Manajemen",~
## $ Tingkat.Semester         <chr> "4 (Semester 7 - Semester 8)", "3 (Semester 5~
## $ Uang.Saku                <chr> "< 1 jt", "1 jt – 2 jt", "1 jt – 2 jt", "1 jt~
## $ kepemilikan.mobil        <dbl> 1, 2, 1, 1, 4, 1, 1, 1, 1, 1, NA, 0, 0, 1, 0,~
## $ kepemilikan.motor        <dbl> 2, 1, 2, 4, 4, 2, 1, 1, 1, 1, 1, 2, 0, 1, 1, ~
## $ kepemilikan.sepeda       <dbl> 1, 1, 2, 1, 1, 1, 1, 1, 1, 1, NA, 0, 0, 0, 0,~
## $ kendaraan.utama          <chr> "Sepeda Motor Pribadi", "Mobil Pribadi", "Sep~
## $ jenis.tempat.tinggal     <chr> "Rumah pribadi/rumah keluarga", "Rumah pribad~
## $ jarak                    <dbl> 4.2237967, 2.7033310, 3.2370722, 3.0023362, 5~
## $ biaya.dalam.sepekan      <chr> "25 ribu", "200", "50rb menggunakan kendaraan~
## $ Jumlah.perjalanan.Senin  <chr> "2", "3", "1", "2", "1", "3", "2", "4", "4", ~
## $ Jumlah.perjalanan.Selasa <dbl> 3, 3, 1, 2, 1, 3, 2, 4, 4, 3, 3, 1, 1, 4, 1, ~
## $ Jumlah.perjalanan.Rabu   <dbl> 2, 3, 1, 2, 1, 1, 2, 4, 4, 3, 1, 2, 1, 4, 3, ~
## $ Jumlah.perjalanan.Kamis  <dbl> 3, 3, 1, 2, 1, 1, 2, 4, 4, 3, 1, 1, 1, 4, 3, ~
## $ Jumlah.perjalanan.Jumat  <dbl> 2, 3, 1, 2, 1, 1, 2, 4, 3, 3, 1, 2, 1, 4, 3, ~
## $ Jumlah.perjalanan.Sabtu  <dbl> 2, 3, 1, 2, 1, 1, 2, 3, 4, 2, 1, 1, 1, 1, 1, ~
## $ Jumlah.perjalanan.Ahad   <dbl> 2, 3, 1, 2, 1, 1, 2, 3, 4, 1, 1, 1, 1, 1, 1, ~
\end{verbatim}

Ini adalah kasus yang tepat untuk menggambarkan bahwa dalam data terstruktur, jenis nilai harus sama persis. Untuk kesederhanaan praktikum, kita akan melewati dulu variabel \texttt{biaya.dalam.sepekan} dan \texttt{Jumlah.perjalanan.Senin} tersebut.

\subsubsection{Memperbaiki Set Data yang Akan Digabungkan}\label{memperbaiki-set-data-yang-akan-digabungkan}

Kita akan menghapus \texttt{biaya.dalam.sepekan} dari kedua set data kita. Untuk menghapus suatu kolom, kita dapat menggunakan fungsi \texttt{select()} juga, akan tetapi kita tambahkan tanda negatif (\texttt{-}) di depan nama variabelnya.

\begin{verbatim}
## Rows: 400
## Columns: 19
## $ Nomor.urut               <dbl> 1, 2, 3, 4, 5, 6, 7, 8, 9, 10, 11, 12, 13, 14~
## $ Jenis.Kelamin            <chr> "Laki-laki", "Laki-laki", "Perempuan", "Laki-~
## $ Umur                     <dbl> 22, 25, 24, 19, 23, 21, 22, 21, 21, 21, 19, 1~
## $ Fakultas                 <chr> "Syariah", "Syariah", "Tarbiyah dan Keguruan"~
## $ Prodi                    <chr> "Hukum Keluarga", "Hukum Keluarga", "Pendidik~
## $ Tingkat.Semester         <chr> "4 (Semester 7 - Semester 8)", "4 (Semester 7~
## $ Uang.Saku                <chr> "1Jt - 2 Jt", "< 1 Jt", "< 1 Jt", "< 1 Jt", "~
## $ kepemilikan.mobil        <dbl> 0, 1, 0, 0, 0, 0, 1, 1, 1, 0, 0, 0, 0, 0, 0, ~
## $ kepemilikan.motor        <dbl> 1, 4, 0, 1, 1, 1, 1, 4, 4, 1, 0, 1, 3, 1, 0, ~
## $ kepemilikan.sepeda       <dbl> 0, 0, 0, 0, 0, 0, 0, 0, 1, 0, 0, 0, 1, 0, 0, ~
## $ kendaraan.utama          <chr> "Sepeda Motor Pribadi", "Sepeda Motor Pribadi~
## $ jenis.tempat.tinggal     <chr> "Rumah Bersama Saudara", "Kos Sendiri", "Kos ~
## $ jarak                    <dbl> 19.27, 0.58, 0.56, 1.05, 1.69, 7.91, 2.58, 2.~
## $ Jumlah.perjalanan.Selasa <dbl> 3, 3, 0, 2, 4, 2, 0, 4, 0, 2, 4, 2, 2, 2, 3, ~
## $ Jumlah.perjalanan.Rabu   <dbl> 0, 0, 3, 0, 0, 2, 3, 4, 3, 2, 4, 2, 2, 2, 2, ~
## $ Jumlah.perjalanan.Kamis  <dbl> 0, 3, 0, 2, 0, 0, 0, 4, 0, 2, 2, 2, 2, 2, 0, ~
## $ Jumlah.perjalanan.Jumat  <dbl> 0, 0, 0, 2, 3, 2, 2, 4, 0, 0, 2, 2, 2, 2, 2, ~
## $ Jumlah.perjalanan.Sabtu  <dbl> 0, 0, 0, 0, 2, 2, 2, 2, 2, 0, 2, 2, 0, 0, 0, ~
## $ Jumlah.perjalanan.Ahad   <dbl> 0, 0, 0, 2, 0, 2, 2, 2, 2, 0, 2, 2, 0, 2, 2, ~
\end{verbatim}

\begin{verbatim}
##  [1] "Nomor.urut"               "Jenis.Kelamin"           
##  [3] "Umur"                     "Fakultas"                
##  [5] "Prodi"                    "Tingkat.Semester"        
##  [7] "Uang.Saku"                "kepemilikan.mobil"       
##  [9] "kepemilikan.motor"        "kepemilikan.sepeda"      
## [11] "kendaraan.utama"          "jenis.tempat.tinggal"    
## [13] "jarak"                    "Jumlah.perjalanan.Selasa"
## [15] "Jumlah.perjalanan.Rabu"   "Jumlah.perjalanan.Kamis" 
## [17] "Jumlah.perjalanan.Jumat"  "Jumlah.perjalanan.Sabtu" 
## [19] "Jumlah.perjalanan.Ahad"
\end{verbatim}

\begin{verbatim}
## Rows: 379
## Columns: 19
## $ Nomor.urut               <dbl> 1, 2, 3, 4, 5, 6, 7, 8, 9, 10, 11, 12, 13, 14~
## $ Jenis.Kelamin            <chr> "Perempuan", "Laki-Laki", "Laki-Laki", "Laki-~
## $ Umur                     <dbl> 21, 20, 21, 22, 21, 21, 22, 22, 22, 22, 21, 2~
## $ Fakultas                 <chr> "Fakultas Ilmu Sosial dan Politik", "Fakultas~
## $ Prodi                    <chr> "Ilmu Komunikasi", "Ilmu Hukum", "Manajemen",~
## $ Tingkat.Semester         <chr> "4 (Semester 7 - Semester 8)", "3 (Semester 5~
## $ Uang.Saku                <chr> "< 1 jt", "1 jt – 2 jt", "1 jt – 2 jt", "1 jt~
## $ kepemilikan.mobil        <dbl> 1, 2, 1, 1, 4, 1, 1, 1, 1, 1, NA, 0, 0, 1, 0,~
## $ kepemilikan.motor        <dbl> 2, 1, 2, 4, 4, 2, 1, 1, 1, 1, 1, 2, 0, 1, 1, ~
## $ kepemilikan.sepeda       <dbl> 1, 1, 2, 1, 1, 1, 1, 1, 1, 1, NA, 0, 0, 0, 0,~
## $ kendaraan.utama          <chr> "Sepeda Motor Pribadi", "Mobil Pribadi", "Sep~
## $ jenis.tempat.tinggal     <chr> "Rumah pribadi/rumah keluarga", "Rumah pribad~
## $ jarak                    <dbl> 4.2237967, 2.7033310, 3.2370722, 3.0023362, 5~
## $ Jumlah.perjalanan.Selasa <dbl> 3, 3, 1, 2, 1, 3, 2, 4, 4, 3, 3, 1, 1, 4, 1, ~
## $ Jumlah.perjalanan.Rabu   <dbl> 2, 3, 1, 2, 1, 1, 2, 4, 4, 3, 1, 2, 1, 4, 3, ~
## $ Jumlah.perjalanan.Kamis  <dbl> 3, 3, 1, 2, 1, 1, 2, 4, 4, 3, 1, 1, 1, 4, 3, ~
## $ Jumlah.perjalanan.Jumat  <dbl> 2, 3, 1, 2, 1, 1, 2, 4, 3, 3, 1, 2, 1, 4, 3, ~
## $ Jumlah.perjalanan.Sabtu  <dbl> 2, 3, 1, 2, 1, 1, 2, 3, 4, 2, 1, 1, 1, 1, 1, ~
## $ Jumlah.perjalanan.Ahad   <dbl> 2, 3, 1, 2, 1, 1, 2, 3, 4, 1, 1, 1, 1, 1, 1, ~
\end{verbatim}

\subsubsection{Menyatukan Kedua Set Data Setelah Diperbaiki}\label{menyatukan-kedua-set-data-setelah-diperbaiki}

Kita akan coba menggabungkan kedua set data tadi.

Selesai! 🥳 Mari kita lihat hasil akhir dari kerja keras kita dengan \texttt{glimpse()}. Hasil akan menunjukkan bahwa kita mempunyai 779 objek yang merupakan gabungan responden mahasiswa UIN RIL dan UBL.

\begin{verbatim}
## Rows: 779
## Columns: 19
## $ Nomor.urut               <dbl> 1, 2, 3, 4, 5, 6, 7, 8, 9, 10, 11, 12, 13, 14~
## $ Jenis.Kelamin            <chr> "Laki-laki", "Laki-laki", "Perempuan", "Laki-~
## $ Umur                     <dbl> 22, 25, 24, 19, 23, 21, 22, 21, 21, 21, 19, 1~
## $ Fakultas                 <chr> "Syariah", "Syariah", "Tarbiyah dan Keguruan"~
## $ Prodi                    <chr> "Hukum Keluarga", "Hukum Keluarga", "Pendidik~
## $ Tingkat.Semester         <chr> "4 (Semester 7 - Semester 8)", "4 (Semester 7~
## $ Uang.Saku                <chr> "1Jt - 2 Jt", "< 1 Jt", "< 1 Jt", "< 1 Jt", "~
## $ kepemilikan.mobil        <dbl> 0, 1, 0, 0, 0, 0, 1, 1, 1, 0, 0, 0, 0, 0, 0, ~
## $ kepemilikan.motor        <dbl> 1, 4, 0, 1, 1, 1, 1, 4, 4, 1, 0, 1, 3, 1, 0, ~
## $ kepemilikan.sepeda       <dbl> 0, 0, 0, 0, 0, 0, 0, 0, 1, 0, 0, 0, 1, 0, 0, ~
## $ kendaraan.utama          <chr> "Sepeda Motor Pribadi", "Sepeda Motor Pribadi~
## $ jenis.tempat.tinggal     <chr> "Rumah Bersama Saudara", "Kos Sendiri", "Kos ~
## $ jarak                    <dbl> 19.27, 0.58, 0.56, 1.05, 1.69, 7.91, 2.58, 2.~
## $ Jumlah.perjalanan.Selasa <dbl> 3, 3, 0, 2, 4, 2, 0, 4, 0, 2, 4, 2, 2, 2, 3, ~
## $ Jumlah.perjalanan.Rabu   <dbl> 0, 0, 3, 0, 0, 2, 3, 4, 3, 2, 4, 2, 2, 2, 2, ~
## $ Jumlah.perjalanan.Kamis  <dbl> 0, 3, 0, 2, 0, 0, 0, 4, 0, 2, 2, 2, 2, 2, 0, ~
## $ Jumlah.perjalanan.Jumat  <dbl> 0, 0, 0, 2, 3, 2, 2, 4, 0, 0, 2, 2, 2, 2, 2, ~
## $ Jumlah.perjalanan.Sabtu  <dbl> 0, 0, 0, 0, 2, 2, 2, 2, 2, 0, 2, 2, 0, 0, 0, ~
## $ Jumlah.perjalanan.Ahad   <dbl> 0, 0, 0, 2, 0, 2, 2, 2, 2, 0, 2, 2, 0, 2, 2, ~
\end{verbatim}

🎉 \textbf{Selamat!} Kita telah berhasil menggabungkan dataset dari dua kampus berbeda. Dataset gabungan kita sekarang memiliki:

\begin{itemize}
\tightlist
\item
  \textbf{779 baris} (400 dari UIN RIL + 379 dari UBL)
\item
  \textbf{19 kolom} yang sudah diseragamkan namanya
\item
  Sudah siap untuk analisis lebih lanjut!
\end{itemize}

\subsubsection{\texorpdfstring{Mengekspor Set Data Hasil Pengolahan ke Fail \emph{(File)} Terpisah}{Mengekspor Set Data Hasil Pengolahan ke Fail (File) Terpisah}}\label{mengekspor-set-data-hasil-pengolahan-ke-fail-file-terpisah}

Dalam dunia data sains dikenal jenis fail \texttt{.csv} yang merupakan singkatan dari \emph{``comma-separated values.''} Fail ini menyimpan nilai-nilai sehingga dapat dibaca sebagai tabel seperti contoh berikut:

\begin{verbatim}
Nama; umur
Ifna; 20
Salbina; 21
\end{verbatim}

Hasilnya adalah:

{\def\LTcaptype{none} % do not increment counter
\begin{longtable}[]{@{}ll@{}}
\toprule\noalign{}
Nama & umur \\
\midrule\noalign{}
\endhead
\bottomrule\noalign{}
\endlastfoot
Ifna & 20 \\
Salbina & 21 \\
\end{longtable}
}

Kita dapat mengekspor set data hasil pengolahan menjadi fail dengan format data terstruktur seperti \texttt{.csv} atau Excel (\texttt{.xls}, \texttt{.xlsx}) dengan fungsi \texttt{write.csv2()} atau \texttt{write.xlsx()}.

Format perintah fungsi-fungsi ini adalah:

\begin{Shaded}
\begin{Highlighting}[]
\FunctionTok{write.csv2}\NormalTok{(variabel\_dataset, }\StringTok{"nama\_file\_ekspor.csv"}\NormalTok{)}

\FunctionTok{write.xlsx}\NormalTok{(variabel\_dataset, }\StringTok{"nama\_file\_ekspor.xlsx"}\NormalTok{)}
\end{Highlighting}
\end{Shaded}

Seluruh fail yang diekspor lokasi direktorinya \textbf{sama seperti lokasi file .Rproj} kita.

\paragraph{\texorpdfstring{Mengekspor set data ke fail \texttt{.csv}}{Mengekspor set data ke fail .csv}}\label{mengekspor-set-data-ke-fail-.csv}

Terdapat dua jenis fungsi untuk mengekspor set data ke fail \texttt{.csv}: \texttt{write.csv()} dan \texttt{write.csv2()}. Perbedaan antara keduanya adalah \textbf{pemisah nilainya}.

Fungsi \texttt{write.csv()} digunakan untuk set data yang menggunakan \textbf{tanda titik} sebagai pemisah desimal, sehingga pemisahnya adalah \textbf{tanda koma (,)}. Ini adalah format yang biasa dipakai di Amerika Utara.

Sementara itu, \texttt{write.csv2()} menggunakan pemisah \textbf{titik-koma (;)} \emph{(semicolon)}. Ini digunakan untuk set data yang pemisah desimalnya adalah \textbf{tanda koma}.

\paragraph{\texorpdfstring{Mengekspor set data ke fail \texttt{.xlsx}}{Mengekspor set data ke fail .xlsx}}\label{mengekspor-set-data-ke-fail-.xlsx}

Sama seperti write.csv2(), fungsi write.xlsx() juga mengikuti format yang sama.

\textbf{Aktivitas Mandiri 3: Menggabungkan Data ITERA dan UNILA {[}STP-1.3, STP-1.4{]}}

Dari file \texttt{Data\ Praktikum\ 01.xlsx}, terdapat dua sheet tambahan:
- \texttt{DataUtama\_mhsITERA}
- \texttt{DataUtama\_mhsUNILA}

\textbf{Langkah-langkah:}

\begin{enumerate}
\def\labelenumi{\arabic{enumi}.}
\tightlist
\item
  \textbf{Impor data dari kedua sheet} menggunakan \texttt{read.xlsx()}

  \begin{itemize}
  \tightlist
  \item
    Simpan sheet ITERA ke variabel \texttt{data\_itera}
  \item
    Simpan sheet UNILA ke variabel \texttt{data\_unila}
  \end{itemize}
\item
  \textbf{Identifikasi kolom yang bisa digabungkan}

  \begin{itemize}
  \tightlist
  \item
    Gunakan \texttt{colnames()} untuk melihat nama kolom di kedua dataset
  \item
    Gunakan \texttt{glimpse()} untuk melihat tipe data setiap kolom
  \item
    Tentukan kolom mana yang tipenya sama dan bisa digabungkan
  \end{itemize}
\item
  \textbf{Seleksi kolom yang sesuai} menggunakan \texttt{select()}

  \begin{itemize}
  \tightlist
  \item
    Pilih hanya kolom yang ada di kedua dataset dan tipenya sama
  \item
    Ingat menggunakan backtick (`) untuk kolom dengan karakter khusus
  \end{itemize}
\item
  \textbf{Seragamkan nama kolom}

  \begin{itemize}
  \tightlist
  \item
    Buat vektor \texttt{kolom\_final} berisi nama standar untuk semua kolom
  \item
    Gunakan \texttt{colnames(data\_itera\_selected)\ \textless{}-\ kolom\_final}
  \item
    Lakukan hal sama untuk \texttt{data\_unila\_selected}
  \end{itemize}
\item
  \textbf{Gabungkan dengan \texttt{bind\_rows()}}

  \begin{itemize}
  \tightlist
  \item
    Jika ada error tipe data, periksa dengan \texttt{glimpse()} dan perbaiki
  \item
    Simpan hasil gabungan ke variabel \texttt{data\_gabungan\_iteraunila}
  \end{itemize}
\item
  \textbf{Simpan ke file}

  \begin{itemize}
  \tightlist
  \item
    Ekspor hasil gabungan dengan \texttt{write.xlsx()}
  \item
    Nama file: \texttt{Prak1\_{[}NIM1{]}\_{[}NIM2{]}.xlsx}
  \end{itemize}
\end{enumerate}

\textbf{Analisis {[}STP-1.3{]}:}

\begin{enumerate}
\def\labelenumi{\arabic{enumi}.}
\setcounter{enumi}{6}
\tightlist
\item
  Berapa jumlah \textbf{variabel} (kolom) dalam data gabungan?
\item
  Berapa jumlah \textbf{objek} (baris) dalam data gabungan?
\item
  Tampilkan bukti dengan \texttt{glimpse(data\_gabungan\_iteraunila)} dan sertakan screenshot
\end{enumerate}

\textbf{Refleksi:}
- Apa tantangan terbesar yang Anda hadapi dalam menggabungkan data?
- Mengapa penting untuk mengecek tipe data sebelum menggabungkan dataset?

\begin{center}\rule{0.5\linewidth}{0.5pt}\end{center}

\section{Modul-2: Analisis Statistika Deskriptif}\label{modul-2-analisis-statistika-deskriptif}

Setelah mempelajari modul ini, Anda diharapkan dapat mengoperasikan perhitungan persentase/proporsi, rasio, laju, ukuran pemusatan, dan ukuran penyebaran untuk suatu data kuantitatif dengan menggunakan perangkat lunak \capaian{STP-2.4}

\begin{center}\rule{0.5\linewidth}{0.5pt}\end{center}

Dalam praktikum ini kita akan melanjutkan analisis data terstruktur kita dengan analisis statistika deskriptif untuk variabel-variabel \textbf{kategoris} dan \textbf{numerik}. Kedua jenis variabel ini berkaitan juga dengan tingkat pengukurannya yang terdiri atas \textbf{nominal}, \textbf{ordinal}, dan \textbf{interval/rasio (metrik).}

Analisis statistik deskriptif yang akan kita lakukan mencakup \textbf{persentase/proporsi}, \textbf{rata-rata \emph{(mean)}}, \textbf{median}, dan \textbf{standar deviasi}. Ada beberapa ukuran statistik yang tidak kita pelajari di kelas tetapi dikenalkan pada praktikum ini, seperti \textbf{persentil} dan \textbf{kuartil}. Perhitungan IQV akan dilakukan dengan sekaligus mengenalkan mekanisme \texttt{function} di R.

\subsection{Persiapan}\label{persiapan}

Seperti biasa, kita perlu memuat \emph{library} andalan kita dalam mengolah data, \texttt{tidyverse}, dan juga untuk membaca file Excel, \texttt{openxlsx}.

\subsection{Mengimpor Dataset}\label{mengimpor-dataset}

Kita menggunakan data mahasiswa UBL yang sudah lengkap karena sudah diolah kembali.

Sebelumnya kita sudah mengenal \texttt{glimpse()} yang berasal dari \emph{library} \texttt{dplyr}, sekarang kita akan mengenal perintah untuk menampilkan ringkasan dataset bawaan R (disebut juga \emph{`base R'}) bernama \texttt{summary()}.

\begin{verbatim}
##    Timestamp      Kampus_PT           Nomor.urut    Jenis.Kelamin     
##  Min.   :45414   Length:379         Min.   :  1.0   Length:379        
##  1st Qu.:45416   Class :character   1st Qu.: 95.5   Class :character  
##  Median :45423   Mode  :character   Median :190.0   Mode  :character  
##  Mean   :45420                      Mean   :190.0                     
##  3rd Qu.:45423                      3rd Qu.:284.5                     
##  Max.   :45424                      Max.   :379.0                     
##                                                                       
##       Umur         Fakultas            Prodi           Tingkat.Semester  
##  Min.   :18.00   Length:379         Length:379         Length:379        
##  1st Qu.:20.00   Class :character   Class :character   Class :character  
##  Median :21.00   Mode  :character   Mode  :character   Mode  :character  
##  Mean   :20.78                                                           
##  3rd Qu.:22.00                                                           
##  Max.   :23.00                                                           
##                                                                          
##   Uang.Saku         kepemilikan.mobil kepemilikan.motor kepemilikan.sepeda
##  Length:379         Min.   :0.0000    Min.   :0.000     Min.   :0.0000    
##  Class :character   1st Qu.:0.0000    1st Qu.:1.000     1st Qu.:0.0000    
##  Mode  :character   Median :1.0000    Median :1.000     Median :0.0000    
##                     Mean   :0.9654    Mean   :1.517     Mean   :0.2857    
##                     3rd Qu.:2.0000    3rd Qu.:2.000     3rd Qu.:0.0000    
##                     Max.   :4.0000    Max.   :5.000     Max.   :3.0000    
##                     NA's   :3                           NA's   :1         
##  kendaraan.utama     kelurahan         jenis.tempat.tinggal
##  Length:379         Length:379         Length:379          
##  Class :character   Class :character   Class :character    
##  Mode  :character   Mode  :character   Mode  :character    
##                                                            
##                                                            
##                                                            
##                                                            
##  nama.jalan.tempat.tinggal   jarak.(km)      
##  Length:379                Min.   : 0.06352  
##  Class :character          1st Qu.: 1.31127  
##  Mode  :character          Median : 2.69037  
##                            Mean   : 3.37914  
##                            3rd Qu.: 4.69466  
##                            Max.   :43.54519  
##                                              
##  alasan.pemilihan.lokasi.tempat.tinggal biaya.dalam.seminggu
##  Length:379                             Min.   :  0.00      
##  Class :character                       1st Qu.: 50.00      
##  Mode  :character                       Median : 70.00      
##                                         Mean   : 93.57      
##                                         3rd Qu.:110.00      
##                                         Max.   :600.00      
##                                         NA's   :2           
##  Jumlah.perjalanan.Senin Jumlah.Perjalanan.Selasa Jumlah.Perjalanan.Rabu
##  Min.   :1.000           Min.   :1.000            Min.   :1.000         
##  1st Qu.:2.000           1st Qu.:1.000            1st Qu.:1.000         
##  Median :2.000           Median :2.000            Median :2.000         
##  Mean   :2.679           Mean   :2.264            Mean   :2.092         
##  3rd Qu.:4.000           3rd Qu.:3.000            3rd Qu.:3.000         
##  Max.   :8.000           Max.   :6.000            Max.   :6.000         
##  NA's   :43                                                             
##  Jumlah.Perjalanan.Kamis Jumlah.Perjalanan.Jumat Jumlah.Perjalanan.Sabtu
##  Min.   :1.000           Min.   :1.000           Min.   :1.000          
##  1st Qu.:1.000           1st Qu.:1.000           1st Qu.:1.000          
##  Median :2.000           Median :2.000           Median :2.000          
##  Mean   :2.092           Mean   :2.032           Mean   :1.749          
##  3rd Qu.:3.000           3rd Qu.:3.000           3rd Qu.:2.000          
##  Max.   :7.000           Max.   :9.000           Max.   :5.000          
##                                                                         
##  Jumlah.Perjalanan.Ahad
##  Min.   :1.00          
##  1st Qu.:1.00          
##  Median :1.00          
##  Mean   :1.52          
##  3rd Qu.:2.00          
##  Max.   :4.00          
## 
\end{verbatim}

\subsection{\texorpdfstring{Merapikan Dataset \emph{(Data Wrangling)}}{Merapikan Dataset (Data Wrangling)}}\label{merapikan-dataset-data-wrangling}

Sering dikatakan bahwa 50\% pekerjaan ilmuwan data adalah \textbf{merapikan dataset}. Ini diistilahkan dengan \emph{wrangling}. \emph{Data wrangling} mencakup penggantian nama variabel, pengecekan nilai yang hilang \emph{(missing values)}, mengubah/memanipulasi variabel, dan masih banyak lagi.

\paragraph{Mengganti Nama Variabel}\label{mengganti-nama-variabel}

Pada praktikum sebelumnya kita mengubah nama kolom/variabel dalam \emph{dataset} secara \textbf{keseluruhan}, yakni dengan menimpakan vektor nama-nama baru variabel kita dengan fungsi \texttt{colnames()} .

Kita pastinya tidak akan selalu mengubah semua kolom secara sekaligus, karena itu kita harus belajar bagaimana cara mengganti nama beberapa kolom saja, yakni dengan perintah \texttt{rename()}.

Perintah \texttt{rename()} memiliki sintaks seperti berikut

\begin{Shaded}
\begin{Highlighting}[]
\FunctionTok{rename}\NormalTok{(dataset\_kita, }\AttributeTok{nama\_kolom\_baru =}\NormalTok{ nama\_kolom\_lama) }
\end{Highlighting}
\end{Shaded}

\begin{verbatim}
##  [1] "Timestamp"                             
##  [2] "Kampus_PT"                             
##  [3] "Nomor.urut"                            
##  [4] "Jenis.Kelamin"                         
##  [5] "Umur"                                  
##  [6] "Fakultas"                              
##  [7] "Prodi"                                 
##  [8] "Tingkat.Semester"                      
##  [9] "Uang.Saku"                             
## [10] "kepemilikan.mobil"                     
## [11] "kepemilikan.motor"                     
## [12] "kepemilikan.sepeda"                    
## [13] "kendaraan.utama"                       
## [14] "kelurahan"                             
## [15] "jenis.tempat.tinggal"                  
## [16] "nama.jalan.tempat.tinggal"             
## [17] "jarak.(km)"                            
## [18] "alasan.pemilihan.lokasi.tempat.tinggal"
## [19] "biaya.dalam.seminggu"                  
## [20] "Jumlah.perjalanan.Senin"               
## [21] "Jumlah.Perjalanan.Selasa"              
## [22] "Jumlah.Perjalanan.Rabu"                
## [23] "Jumlah.Perjalanan.Kamis"               
## [24] "Jumlah.Perjalanan.Jumat"               
## [25] "Jumlah.Perjalanan.Sabtu"               
## [26] "Jumlah.Perjalanan.Ahad"
\end{verbatim}

\textbf{Pengenalan Teknik Pengolahan dalam R: \emph{Pipe Operator} (\texttt{\textbar{}\textgreater{}})}

Dalam pengolahan data dengan \texttt{tidyverse} penting untuk kita kuasai penggunaan operator yang satu ini. Perhatikan dua cara berikut untuk mengganti nama variabel-variabel kita.

\begin{verbatim}
##  [1] "Timestamp"                             
##  [2] "Kampus_PT"                             
##  [3] "Nomor.urut"                            
##  [4] "Jenis.Kelamin"                         
##  [5] "Umur"                                  
##  [6] "Fakultas"                              
##  [7] "Prodi"                                 
##  [8] "Tingkat.Semester"                      
##  [9] "Uang.Saku"                             
## [10] "kepemilikan.mobil"                     
## [11] "kepemilikan.motor"                     
## [12] "kepemilikan.sepeda"                    
## [13] "kendaraan.utama"                       
## [14] "kelurahan"                             
## [15] "jenis.tempat.tinggal"                  
## [16] "nama.jalan.tempat.tinggal"             
## [17] "jarak.(km)"                            
## [18] "alasan.pemilihan.lokasi.tempat.tinggal"
## [19] "biaya.dalam.seminggu"                  
## [20] "Jumlah.perjalanan.Senin"               
## [21] "Jumlah.Perjalanan.Selasa"              
## [22] "Jumlah.Perjalanan.Rabu"                
## [23] "Jumlah.Perjalanan.Kamis"               
## [24] "Jumlah.Perjalanan.Jumat"               
## [25] "Jumlah.Perjalanan.Sabtu"               
## [26] "Jumlah.Perjalanan.Ahad"
\end{verbatim}

Dalam cara 1, kita harus menulis \texttt{data.ubl} berkali-kali untuk setiap perintah. Ini mungkin tidak akan terasa jika jumlah kolom yang kita miliki sedikit. Tetapi, akan lain ceritanya jika jumlahnya banyak. Selain itu, data kita tidak bisa secara intuitif seperti biasanya kita membaca suatu teks dari kiri ke kanan.

Operator pipa (\texttt{\textbar{}\textgreater{}}) memungkinkan hal tersebut. Selain kita tidak perlu menuliskan dataset kita berkali-kali, kita dengan lebih mudah memahami baris-baris kode kita tersebut sebagai proses dari atas ke bawah, dari kiri ke kanan, seperti pipa.

Berikut adalah penjelasan cara-2:

\begin{itemize}
\item
  \texttt{data.ubl\ \textbar{}\textgreater{}}: perintah ``Ambil dataset \texttt{data.ubl}.''
\item
  Kirimkan dataset itu ke fungsi berikutnya (\texttt{rename}).
\item
  Hasil dari fungsi \texttt{rename} pertama dikirimkan lagi ke fungsi \texttt{rename} kedua, dan seterusnya.
\end{itemize}

\begin{quote}
⚠️ \textbf{Penting}

Pipe operator dapat dimasukkan dengan kombinasi tombol \texttt{Ctrl} + \texttt{Shift} + \texttt{M} di \emph{keyboard}.
\end{quote}

\textbf{Aktivitas Mandiri 1: Menguasai Pipe Operator {[}STP-2.4{]}}

Dengan satu blok kode menggunakan pipe operator \texttt{\textbar{}\textgreater{}}:
1. Baca dataset UBL dari file dan sheet yang sudah ditentukan
2. Ubah nama variabel \texttt{biaya.dalam.seminggu} menjadi \texttt{biaya.dalam.sepekan}
3. Ubah nama variabel \texttt{Jumlah.perjalanan.Senin} menjadi \texttt{Jumlah.Perjalanan.Senin}
4. Buang baris yang mengandung missing values dengan fungsi \texttt{drop\_na()}
5. Simpan hasilnya ke variabel \texttt{data\_ubl\_bersih}
6. Verifikasi dengan \texttt{glimpse(data\_ubl\_bersih)} - pastikan tidak ada NA

\paragraph{\texorpdfstring{Mengecek \emph{Missing Values}}{Mengecek Missing Values}}\label{mengecek-missing-values}

Pengecekan \emph{missing values} dilakukan dengan perintah \texttt{is.na()} yang menghasilkan tabel berisi nilai \emph{boolean} \texttt{FALSE} dan \texttt{TRUE} yang berarti `tidak kosong' dan `kosong' secara berturut-turut.

\texttt{colSums()} menjumlahkan nilai-nilai \texttt{TRUE} pada setiap kolom, menandakan berapa jumlah objek yang kosong nilainya.

\begin{verbatim}
##   Timestamp Kampus_PT Nomor.urut Jenis.Kelamin  Umur Fakultas Prodi
## 1     FALSE     FALSE      FALSE         FALSE FALSE    FALSE FALSE
## 2     FALSE     FALSE      FALSE         FALSE FALSE    FALSE FALSE
## 3     FALSE     FALSE      FALSE         FALSE FALSE    FALSE FALSE
## 4     FALSE     FALSE      FALSE         FALSE FALSE    FALSE FALSE
## 5     FALSE     FALSE      FALSE         FALSE FALSE    FALSE FALSE
## 6     FALSE     FALSE      FALSE         FALSE FALSE    FALSE FALSE
##   Tingkat.Semester Uang.Saku kepemilikan.mobil kepemilikan.motor
## 1            FALSE     FALSE             FALSE             FALSE
## 2            FALSE     FALSE             FALSE             FALSE
## 3            FALSE     FALSE             FALSE             FALSE
## 4            FALSE     FALSE             FALSE             FALSE
## 5            FALSE     FALSE             FALSE             FALSE
## 6            FALSE     FALSE             FALSE             FALSE
##   kepemilikan.sepeda kendaraan.utama kelurahan jenis.tempat.tinggal
## 1              FALSE           FALSE     FALSE                FALSE
## 2              FALSE           FALSE     FALSE                FALSE
## 3              FALSE           FALSE     FALSE                FALSE
## 4              FALSE           FALSE     FALSE                FALSE
## 5              FALSE           FALSE     FALSE                FALSE
## 6              FALSE           FALSE     FALSE                FALSE
##   nama.jalan.tempat.tinggal jarak alasan.pemilihan.lokasi.tempat.tinggal
## 1                     FALSE FALSE                                  FALSE
## 2                     FALSE FALSE                                  FALSE
## 3                     FALSE FALSE                                  FALSE
## 4                     FALSE FALSE                                  FALSE
## 5                     FALSE FALSE                                  FALSE
## 6                     FALSE FALSE                                  FALSE
##   biaya.dalam.sepekan Jumlah.Perjalanan.Senin Jumlah.Perjalanan.Selasa
## 1               FALSE                   FALSE                    FALSE
## 2               FALSE                   FALSE                    FALSE
## 3               FALSE                   FALSE                    FALSE
## 4               FALSE                   FALSE                    FALSE
## 5                TRUE                   FALSE                    FALSE
## 6               FALSE                   FALSE                    FALSE
##   Jumlah.Perjalanan.Rabu Jumlah.Perjalanan.Kamis Jumlah.Perjalanan.Jumat
## 1                  FALSE                   FALSE                   FALSE
## 2                  FALSE                   FALSE                   FALSE
## 3                  FALSE                   FALSE                   FALSE
## 4                  FALSE                   FALSE                   FALSE
## 5                  FALSE                   FALSE                   FALSE
## 6                  FALSE                   FALSE                   FALSE
##   Jumlah.Perjalanan.Sabtu Jumlah.Perjalanan.Ahad
## 1                   FALSE                  FALSE
## 2                   FALSE                  FALSE
## 3                   FALSE                  FALSE
## 4                   FALSE                  FALSE
## 5                   FALSE                  FALSE
## 6                   FALSE                  FALSE
\end{verbatim}

\begin{verbatim}
##                              Timestamp                              Kampus_PT 
##                                      0                                      0 
##                             Nomor.urut                          Jenis.Kelamin 
##                                      0                                      0 
##                                   Umur                               Fakultas 
##                                      0                                      0 
##                                  Prodi                       Tingkat.Semester 
##                                      0                                      0 
##                              Uang.Saku                      kepemilikan.mobil 
##                                      0                                      3 
##                      kepemilikan.motor                     kepemilikan.sepeda 
##                                      0                                      1 
##                        kendaraan.utama                              kelurahan 
##                                      0                                      0 
##                   jenis.tempat.tinggal              nama.jalan.tempat.tinggal 
##                                      0                                      4 
##                                  jarak alasan.pemilihan.lokasi.tempat.tinggal 
##                                      0                                      0 
##                    biaya.dalam.sepekan                Jumlah.Perjalanan.Senin 
##                                      2                                     43 
##               Jumlah.Perjalanan.Selasa                 Jumlah.Perjalanan.Rabu 
##                                      0                                      0 
##                Jumlah.Perjalanan.Kamis                Jumlah.Perjalanan.Jumat 
##                                      0                                      0 
##                Jumlah.Perjalanan.Sabtu                 Jumlah.Perjalanan.Ahad 
##                                      0                                      0
\end{verbatim}

\paragraph{\texorpdfstring{Menghapus \emph{Missing Values}}{Menghapus Missing Values}}\label{menghapus-missing-values}

Penghapusan \emph{missing values} juga dapat digunakan dengan \emph{pipe operator.} Adapun perintah yang dapat kita gunakan adalah \texttt{drop\_na()} dari paket \texttt{tidyr}.

\begin{verbatim}
##                              Timestamp                              Kampus_PT 
##                                      0                                      0 
##                             Nomor.urut                          Jenis.Kelamin 
##                                      0                                      0 
##                                   Umur                               Fakultas 
##                                      0                                      0 
##                                  Prodi                       Tingkat.Semester 
##                                      0                                      0 
##                              Uang.Saku                      kepemilikan.mobil 
##                                      0                                      0 
##                      kepemilikan.motor                     kepemilikan.sepeda 
##                                      0                                      0 
##                        kendaraan.utama                              kelurahan 
##                                      0                                      0 
##                   jenis.tempat.tinggal              nama.jalan.tempat.tinggal 
##                                      0                                      0 
##                                  jarak alasan.pemilihan.lokasi.tempat.tinggal 
##                                      0                                      0 
##                    biaya.dalam.sepekan                Jumlah.Perjalanan.Senin 
##                                      0                                      0 
##               Jumlah.Perjalanan.Selasa                 Jumlah.Perjalanan.Rabu 
##                                      0                                      0 
##                Jumlah.Perjalanan.Kamis                Jumlah.Perjalanan.Jumat 
##                                      0                                      0 
##                Jumlah.Perjalanan.Sabtu                 Jumlah.Perjalanan.Ahad 
##                                      0                                      0
\end{verbatim}

\paragraph{\texorpdfstring{Menyesuaikan Variabel Nominal dan Ordinal dengan \texttt{factor()}}{Menyesuaikan Variabel Nominal dan Ordinal dengan factor()}}\label{menyesuaikan-variabel-nominal-dan-ordinal-dengan-factor}

Apabila kita mengimpor dataset dengan variabel nonangka, bentuknya akan terbaca sebagai \emph{character} (\texttt{chr}). Perintah \texttt{factor()} dari paket \texttt{forcats} berfungsi untuk mendefinisikan nilai-nilai yang ada dalam variabel kategoris tersebut sehingga kita mendefinisikan urutannya dan menetapkannya sebagai variabel ordinal.

\begin{verbatim}
## Rows: 330
## Columns: 26
## $ Timestamp                              <dbl> 45413.83, 45413.83, 45413.85, 4~
## $ Kampus_PT                              <chr> "UBL", "UBL", "UBL", "UBL", "UB~
## $ Nomor.urut                             <dbl> 1, 2, 3, 4, 6, 7, 8, 9, 10, 12,~
## $ Jenis.Kelamin                          <chr> "Perempuan", "Laki-Laki", "Laki~
## $ Umur                                   <dbl> 21, 20, 21, 22, 21, 22, 22, 22,~
## $ Fakultas                               <chr> "Fakultas Ilmu Sosial dan Polit~
## $ Prodi                                  <chr> "Ilmu Komunikasi", "Ilmu Hukum"~
## $ Tingkat.Semester                       <chr> "4 (Semester 7 - Semester 8)", ~
## $ Uang.Saku                              <chr> "< 1 jt", "1 jt – 2 jt", "1 jt ~
## $ kepemilikan.mobil                      <dbl> 1, 2, 1, 1, 1, 1, 1, 1, 1, 0, 1~
## $ kepemilikan.motor                      <dbl> 2, 1, 2, 4, 2, 1, 1, 1, 1, 2, 1~
## $ kepemilikan.sepeda                     <dbl> 1, 1, 2, 1, 1, 1, 1, 1, 1, 0, 0~
## $ kendaraan.utama                        <chr> "Sepeda Motor Pribadi", "Mobil ~
## $ kelurahan                              <chr> "Kalibalau Kencana", "Gunung su~
## $ jenis.tempat.tinggal                   <chr> "Rumah pribadi/rumah keluarga",~
## $ nama.jalan.tempat.tinggal              <chr> "eLBe Loundry", "Jl.Urip Sumoha~
## $ jarak                                  <dbl> 4.2237967, 2.7033310, 3.2370722~
## $ alasan.pemilihan.lokasi.tempat.tinggal <chr> "Bersama keluarga/saudara/teman~
## $ biaya.dalam.sepekan                    <dbl> 25, 200, 50, 20, 40, 70, 35, 50~
## $ Jumlah.Perjalanan.Senin                <dbl> 2, 3, 1, 2, 3, 2, 4, 4, 3, 2, 4~
## $ Jumlah.Perjalanan.Selasa               <dbl> 3, 3, 1, 2, 3, 2, 4, 4, 3, 1, 4~
## $ Jumlah.Perjalanan.Rabu                 <dbl> 2, 3, 1, 2, 1, 2, 4, 4, 3, 2, 4~
## $ Jumlah.Perjalanan.Kamis                <dbl> 3, 3, 1, 2, 1, 2, 4, 4, 3, 1, 4~
## $ Jumlah.Perjalanan.Jumat                <dbl> 2, 3, 1, 2, 1, 2, 4, 3, 3, 2, 4~
## $ Jumlah.Perjalanan.Sabtu                <dbl> 2, 3, 1, 2, 1, 2, 3, 4, 2, 1, 1~
## $ Jumlah.Perjalanan.Ahad                 <dbl> 2, 3, 1, 2, 1, 2, 3, 4, 1, 1, 1~
\end{verbatim}

Contoh pertama kita adalah variabel yang paling mudah: \texttt{Jenis.Kelamin}.

\begin{verbatim}
## Rows: 330
## Columns: 26
## $ Timestamp                              <dbl> 45413.83, 45413.83, 45413.85, 4~
## $ Kampus_PT                              <chr> "UBL", "UBL", "UBL", "UBL", "UB~
## $ Nomor.urut                             <dbl> 1, 2, 3, 4, 6, 7, 8, 9, 10, 12,~
## $ Jenis.Kelamin                          <fct> Perempuan, NA, NA, NA, Perempua~
## $ Umur                                   <dbl> 21, 20, 21, 22, 21, 22, 22, 22,~
## $ Fakultas                               <chr> "Fakultas Ilmu Sosial dan Polit~
## $ Prodi                                  <chr> "Ilmu Komunikasi", "Ilmu Hukum"~
## $ Tingkat.Semester                       <chr> "4 (Semester 7 - Semester 8)", ~
## $ Uang.Saku                              <chr> "< 1 jt", "1 jt – 2 jt", "1 jt ~
## $ kepemilikan.mobil                      <dbl> 1, 2, 1, 1, 1, 1, 1, 1, 1, 0, 1~
## $ kepemilikan.motor                      <dbl> 2, 1, 2, 4, 2, 1, 1, 1, 1, 2, 1~
## $ kepemilikan.sepeda                     <dbl> 1, 1, 2, 1, 1, 1, 1, 1, 1, 0, 0~
## $ kendaraan.utama                        <chr> "Sepeda Motor Pribadi", "Mobil ~
## $ kelurahan                              <chr> "Kalibalau Kencana", "Gunung su~
## $ jenis.tempat.tinggal                   <chr> "Rumah pribadi/rumah keluarga",~
## $ nama.jalan.tempat.tinggal              <chr> "eLBe Loundry", "Jl.Urip Sumoha~
## $ jarak                                  <dbl> 4.2237967, 2.7033310, 3.2370722~
## $ alasan.pemilihan.lokasi.tempat.tinggal <chr> "Bersama keluarga/saudara/teman~
## $ biaya.dalam.sepekan                    <dbl> 25, 200, 50, 20, 40, 70, 35, 50~
## $ Jumlah.Perjalanan.Senin                <dbl> 2, 3, 1, 2, 3, 2, 4, 4, 3, 2, 4~
## $ Jumlah.Perjalanan.Selasa               <dbl> 3, 3, 1, 2, 3, 2, 4, 4, 3, 1, 4~
## $ Jumlah.Perjalanan.Rabu                 <dbl> 2, 3, 1, 2, 1, 2, 4, 4, 3, 2, 4~
## $ Jumlah.Perjalanan.Kamis                <dbl> 3, 3, 1, 2, 1, 2, 4, 4, 3, 1, 4~
## $ Jumlah.Perjalanan.Jumat                <dbl> 2, 3, 1, 2, 1, 2, 4, 3, 3, 2, 4~
## $ Jumlah.Perjalanan.Sabtu                <dbl> 2, 3, 1, 2, 1, 2, 3, 4, 2, 1, 1~
## $ Jumlah.Perjalanan.Ahad                 <dbl> 2, 3, 1, 2, 1, 2, 3, 4, 1, 1, 1~
\end{verbatim}

Ternyata, hasil pendefinisian \texttt{factor} ke dalam data kita mengandung \texttt{NA} yang berarti data kosong atau \emph{missing value}. Hal tersebut terjadi karena kita mendefinisikan vektor \texttt{jk} berisi \texttt{c("Laki-laki",\ "Perempuan")}.

Sementara itu, jika kita cermat, sebelumnya nilai jenis kelamin ditulis dengan \texttt{"Laki-Laki"} . Maka, kita harus memperbaiki vektor kita yang berisi nilai-nilai yang sesuai dengan dataset terlebih dahulu.

\begin{verbatim}
## Rows: 330
## Columns: 26
## $ Timestamp                              <dbl> 45413.83, 45413.83, 45413.85, 4~
## $ Kampus_PT                              <chr> "UBL", "UBL", "UBL", "UBL", "UB~
## $ Nomor.urut                             <dbl> 1, 2, 3, 4, 6, 7, 8, 9, 10, 12,~
## $ Jenis.Kelamin                          <fct> Perempuan, Laki-Laki, Laki-Laki~
## $ Umur                                   <dbl> 21, 20, 21, 22, 21, 22, 22, 22,~
## $ Fakultas                               <chr> "Fakultas Ilmu Sosial dan Polit~
## $ Prodi                                  <chr> "Ilmu Komunikasi", "Ilmu Hukum"~
## $ Tingkat.Semester                       <chr> "4 (Semester 7 - Semester 8)", ~
## $ Uang.Saku                              <chr> "< 1 jt", "1 jt – 2 jt", "1 jt ~
## $ kepemilikan.mobil                      <dbl> 1, 2, 1, 1, 1, 1, 1, 1, 1, 0, 1~
## $ kepemilikan.motor                      <dbl> 2, 1, 2, 4, 2, 1, 1, 1, 1, 2, 1~
## $ kepemilikan.sepeda                     <dbl> 1, 1, 2, 1, 1, 1, 1, 1, 1, 0, 0~
## $ kendaraan.utama                        <chr> "Sepeda Motor Pribadi", "Mobil ~
## $ kelurahan                              <chr> "Kalibalau Kencana", "Gunung su~
## $ jenis.tempat.tinggal                   <chr> "Rumah pribadi/rumah keluarga",~
## $ nama.jalan.tempat.tinggal              <chr> "eLBe Loundry", "Jl.Urip Sumoha~
## $ jarak                                  <dbl> 4.2237967, 2.7033310, 3.2370722~
## $ alasan.pemilihan.lokasi.tempat.tinggal <chr> "Bersama keluarga/saudara/teman~
## $ biaya.dalam.sepekan                    <dbl> 25, 200, 50, 20, 40, 70, 35, 50~
## $ Jumlah.Perjalanan.Senin                <dbl> 2, 3, 1, 2, 3, 2, 4, 4, 3, 2, 4~
## $ Jumlah.Perjalanan.Selasa               <dbl> 3, 3, 1, 2, 3, 2, 4, 4, 3, 1, 4~
## $ Jumlah.Perjalanan.Rabu                 <dbl> 2, 3, 1, 2, 1, 2, 4, 4, 3, 2, 4~
## $ Jumlah.Perjalanan.Kamis                <dbl> 3, 3, 1, 2, 1, 2, 4, 4, 3, 1, 4~
## $ Jumlah.Perjalanan.Jumat                <dbl> 2, 3, 1, 2, 1, 2, 4, 3, 3, 2, 4~
## $ Jumlah.Perjalanan.Sabtu                <dbl> 2, 3, 1, 2, 1, 2, 3, 4, 2, 1, 1~
## $ Jumlah.Perjalanan.Ahad                 <dbl> 2, 3, 1, 2, 1, 2, 3, 4, 1, 1, 1~
\end{verbatim}

Kita sudah berhasil menjadikan variabel \texttt{Jenis.Kelamin} sebagai factor. Selanjutnya, kita akan mengubah variabel-variabel kategoris lain menjadi factor juga. Variabel-variabel tersebut di antaranya adalah \texttt{Prodi}, \texttt{Tingkat.Semester}, \texttt{Uang.Saku}, dan \texttt{jenis.tempat.tinggal}.

Sekarang kita akan mengecek nilai variabel-variabel kategoris tersebut dengan dua cara berikut:

\begin{verbatim}
## 
##           Fakultas Ekonomi dan Bisnis                        Fakultas Hukum 
##                                   104                                    98 
##                Fakultas Ilmu Komputer      Fakultas Ilmu Sosial dan Politik 
##                                    29                                    52 
## Fakultas Keguruan dan Ilmu Pendidikan                       Fakultas Teknik 
##                                     6                                    41
\end{verbatim}

\begin{verbatim}
##                        Prodi  n
## 1        Administrasi Bisnis 15
## 2        Administrasi Publik 21
## 3                  Akuntansi 23
## 4                 Arsitektur  8
## 5                 Ilmu Hukum 98
## 6            Ilmu Komunikasi 20
## 7                Informatika 18
## 8                  Manajemen 77
## 9  Pendidikan Bahasa Inggris  6
## 10          Sistem Informasi 13
## 11              Teknik Mesin  7
## 12              Teknik Sipil 24
\end{verbatim}

\begin{verbatim}
##              Tingkat.Semester   n
## 1 1 (Semester 1 – Semester 2)  34
## 2 2 (Semester 3 – Semester 4)  50
## 3 3 (Semester 5 - Semester 6)  63
## 4 4 (Semester 7 - Semester 8) 183
\end{verbatim}

\begin{verbatim}
##       Uang.Saku   n
## 1   1 jt – 2 jt 167
## 2 2,1 jt – 3 jt 116
## 3 3,1 jt – 4 jt  18
## 4        < 1 jt  28
## 5        > 4 jt   1
\end{verbatim}

\begin{verbatim}
## 
##              Kos bersama-sama                   Kos Sendiri 
##                            14                            52 
##         Rumah bersama saudara Rumah mengontrak bersama-sama 
##                            23                             5 
##      Rumah mengontrak pribadi  Rumah pribadi/rumah keluarga 
##                             3                           233
\end{verbatim}

Kemudian kita buat vektor-vektor yang menyimpan nilai-nilai yang mungkinnya.

Perhatikan bahwa untuk \textbf{variabel ordinal} kita harus menuliskan vektor nilainya dengan urutan yang sesuai dengan tingkatannya.

Sekarang kita sudah bisa mengubah nilai-nilai di variabel kategoris tersebut dari \emph{character} menjadi \emph{factor}. Variabel ordinal menggunakan atribut \texttt{ordered} yang bernilai \texttt{TRUE}, artinya urutan dipentingkan dalam \texttt{levels}.

\begin{verbatim}
## Rows: 330
## Columns: 26
## $ Timestamp                              <dbl> 45413.83, 45413.83, 45413.85, 4~
## $ Kampus_PT                              <chr> "UBL", "UBL", "UBL", "UBL", "UB~
## $ Nomor.urut                             <dbl> 1, 2, 3, 4, 6, 7, 8, 9, 10, 12,~
## $ Jenis.Kelamin                          <fct> Perempuan, Laki-Laki, Laki-Laki~
## $ Umur                                   <dbl> 21, 20, 21, 22, 21, 22, 22, 22,~
## $ Fakultas                               <fct> Fakultas Ilmu Sosial dan Politi~
## $ Prodi                                  <fct> Ilmu Komunikasi, Ilmu Hukum, Ma~
## $ Tingkat.Semester                       <ord> 4 (Semester 7 - Semester 8), 3 ~
## $ Uang.Saku                              <ord> "< 1 jt", "1 jt – 2 jt", "1 jt ~
## $ kepemilikan.mobil                      <dbl> 1, 2, 1, 1, 1, 1, 1, 1, 1, 0, 1~
## $ kepemilikan.motor                      <dbl> 2, 1, 2, 4, 2, 1, 1, 1, 1, 2, 1~
## $ kepemilikan.sepeda                     <dbl> 1, 1, 2, 1, 1, 1, 1, 1, 1, 0, 0~
## $ kendaraan.utama                        <chr> "Sepeda Motor Pribadi", "Mobil ~
## $ kelurahan                              <chr> "Kalibalau Kencana", "Gunung su~
## $ jenis.tempat.tinggal                   <chr> "Rumah pribadi/rumah keluarga",~
## $ nama.jalan.tempat.tinggal              <chr> "eLBe Loundry", "Jl.Urip Sumoha~
## $ jarak                                  <dbl> 4.2237967, 2.7033310, 3.2370722~
## $ alasan.pemilihan.lokasi.tempat.tinggal <chr> "Bersama keluarga/saudara/teman~
## $ biaya.dalam.sepekan                    <dbl> 25, 200, 50, 20, 40, 70, 35, 50~
## $ Jumlah.Perjalanan.Senin                <dbl> 2, 3, 1, 2, 3, 2, 4, 4, 3, 2, 4~
## $ Jumlah.Perjalanan.Selasa               <dbl> 3, 3, 1, 2, 3, 2, 4, 4, 3, 1, 4~
## $ Jumlah.Perjalanan.Rabu                 <dbl> 2, 3, 1, 2, 1, 2, 4, 4, 3, 2, 4~
## $ Jumlah.Perjalanan.Kamis                <dbl> 3, 3, 1, 2, 1, 2, 4, 4, 3, 1, 4~
## $ Jumlah.Perjalanan.Jumat                <dbl> 2, 3, 1, 2, 1, 2, 4, 3, 3, 2, 4~
## $ Jumlah.Perjalanan.Sabtu                <dbl> 2, 3, 1, 2, 1, 2, 3, 4, 2, 1, 1~
## $ Jumlah.Perjalanan.Ahad                 <dbl> 2, 3, 1, 2, 1, 2, 3, 4, 1, 1, 1~
\end{verbatim}

Sekarang, tipe data untuk variabel-variabel kategoris kita tadi sudah menampilkan \texttt{\textless{}fct\textgreater{}} yang berarti ``factor''. Variabel ordinal diidentifikasi dengan tipe \texttt{\textless{}ord\textgreater{}}.

Kita dapat mengecek urutan factor kita dengan perintah \texttt{levels()} berikut.

\begin{verbatim}
## [1] "1 (Semester 1 – Semester 2)" "2 (Semester 3 – Semester 4)"
## [3] "3 (Semester 5 - Semester 6)" "4 (Semester 7 - Semester 8)"
\end{verbatim}

\begin{verbatim}
## [1] "< 1 jt"        "1 jt – 2 jt"   "2,1 jt – 3 jt" "3,1 jt – 4 jt"
## [5] "> 4 jt"
\end{verbatim}

\textbf{Aktivitas Mandiri 2: Membuat Factor untuk Variabel Kategoris {[}STP-2.4{]}}

Untuk dataset UBL yang sudah dibersihkan:
1. Buat vektor \texttt{tempat\_tinggal} dengan nilai-nilai jenis tempat tinggal yang ada di data
(Gunakan \texttt{table(data.ubl\$jenis.tempat.tinggal)} untuk melihat nilai-nilainya)
2. Gunakan \texttt{mutate()} untuk mengubah \texttt{jenis.tempat.tinggal} menjadi factor
3. Verifikasi dengan \texttt{glimpse()} - tipe harus \texttt{\textless{}fct\textgreater{}}
4. Cek level dengan \texttt{levels(data.ubl\$jenis.tempat.tinggal)}

\subsection{Analisis Statistika Deskriptif}\label{analisis-statistika-deskriptif}

Kita akan menggunakan paket \texttt{gtsummary} dan \texttt{flextable} untuk melakukan analisis statistik deskriptif dengan ringkas. Paket ini memiliki perintah \texttt{tbl\_summary()} yang dapat mengeluarkan analisis statistika deskriptif secara intuitif berdasarkan jenis data yang kita masukkan.

Untuk meng-install kedua paket tersebut, hapus tanda pagar di depan baris pertama dalam \emph{chunk} di bawah dan muat paket \texttt{gtsummary} saja.

\subsubsection{Variabel Kategoris}\label{variabel-kategoris}

Kita masih akan menggunakan \emph{pipe operator} dalam mengoperasikan analisis ini. Perintah yang kita gunakan adalah \texttt{tbl\_summary()} yang kita teruskan ke \texttt{as\_flex\_table()} sebagai keluarannya.

Untuk menampilkan persentase/proporsi, kita cukup memasukkan variabel-variabel kategoris ke dalam tabel dengan atribut \texttt{include}. Atribut tersebut menerima masukan berupa vektor nama-nama variabel kategoris dalam dataset kita. Fungsi \texttt{tbl\_summary} ini akan menampilkan secara otomatis tabel distribusi frekuensinya disertai dengan nilai persentasenya di samping nilai frekuensi tersebut.

\global\setlength{\Oldarrayrulewidth}{\arrayrulewidth}

\global\setlength{\Oldtabcolsep}{\tabcolsep}

\setlength{\tabcolsep}{2pt}

\renewcommand*{\arraystretch}{1.5}



\providecommand{\ascline}[3]{\noalign{\global\arrayrulewidth #1}\arrayrulecolor[HTML]{#2}\cline{#3}}

\begin{longtable}[c]{|p{3.18in}|p{0.99in}}



\ascline{1pt}{000000}{1-2}

\multicolumn{1}{>{\raggedright}m{\dimexpr 3.18in+0\tabcolsep}}{\textcolor[HTML]{000000}{\fontsize{11}{11}\selectfont{\global\setmainfont{Arial}{\textbf{Characteristic}}}}} & \multicolumn{1}{>{\centering}m{\dimexpr 0.99in+0\tabcolsep}}{\textcolor[HTML]{000000}{\fontsize{11}{11}\selectfont{\global\setmainfont{Arial}{\textbf{N\ =\ 330}}}}\textcolor[HTML]{000000}{\fontsize{11}{11}\selectfont{\global\setmainfont{Arial}{\textsuperscript{1}}}}} \\

\ascline{1pt}{000000}{1-2}\endfirsthead 

\ascline{1pt}{000000}{1-2}

\multicolumn{1}{>{\raggedright}m{\dimexpr 3.18in+0\tabcolsep}}{\textcolor[HTML]{000000}{\fontsize{11}{11}\selectfont{\global\setmainfont{Arial}{\textbf{Characteristic}}}}} & \multicolumn{1}{>{\centering}m{\dimexpr 0.99in+0\tabcolsep}}{\textcolor[HTML]{000000}{\fontsize{11}{11}\selectfont{\global\setmainfont{Arial}{\textbf{N\ =\ 330}}}}\textcolor[HTML]{000000}{\fontsize{11}{11}\selectfont{\global\setmainfont{Arial}{\textsuperscript{1}}}}} \\

\ascline{1pt}{000000}{1-2}\endhead



\multicolumn{2}{>{\raggedright}m{\dimexpr 4.17in+2\tabcolsep}}{\textcolor[HTML]{000000}{\fontsize{11}{11}\selectfont{\global\setmainfont{Arial}{\textsuperscript{1}}}}\textcolor[HTML]{000000}{\fontsize{11}{11}\selectfont{\global\setmainfont{Arial}{n\ (\%)}}}} \\

\endlastfoot



\multicolumn{1}{>{\raggedright}p{\dimexpr 3.18in+0\tabcolsep}}{\textcolor[HTML]{000000}{\fontsize{11}{11}\selectfont{\global\setmainfont{Arial}{Jenis.Kelamin}}}} & \multicolumn{1}{>{\centering}p{\dimexpr 0.99in+0\tabcolsep}}{\textcolor[HTML]{000000}{\fontsize{11}{11}\selectfont{\global\setmainfont{Arial}{}}}} \\





\multicolumn{1}{>{\raggedright}p{\dimexpr 3.18in+0\tabcolsep}}{\textcolor[HTML]{000000}{\fontsize{11}{11}\selectfont{\global\setmainfont{Arial}{Laki-Laki}}}} & \multicolumn{1}{>{\centering}p{\dimexpr 0.99in+0\tabcolsep}}{\textcolor[HTML]{000000}{\fontsize{11}{11}\selectfont{\global\setmainfont{Arial}{165\ (50\%)}}}} \\





\multicolumn{1}{>{\raggedright}p{\dimexpr 3.18in+0\tabcolsep}}{\textcolor[HTML]{000000}{\fontsize{11}{11}\selectfont{\global\setmainfont{Arial}{Perempuan}}}} & \multicolumn{1}{>{\centering}p{\dimexpr 0.99in+0\tabcolsep}}{\textcolor[HTML]{000000}{\fontsize{11}{11}\selectfont{\global\setmainfont{Arial}{165\ (50\%)}}}} \\





\multicolumn{1}{>{\raggedright}p{\dimexpr 3.18in+0\tabcolsep}}{\textcolor[HTML]{000000}{\fontsize{11}{11}\selectfont{\global\setmainfont{Arial}{Fakultas}}}} & \multicolumn{1}{>{\centering}p{\dimexpr 0.99in+0\tabcolsep}}{\textcolor[HTML]{000000}{\fontsize{11}{11}\selectfont{\global\setmainfont{Arial}{}}}} \\





\multicolumn{1}{>{\raggedright}p{\dimexpr 3.18in+0\tabcolsep}}{\textcolor[HTML]{000000}{\fontsize{11}{11}\selectfont{\global\setmainfont{Arial}{Fakultas\ Ekonomi\ dan\ Bisnis}}}} & \multicolumn{1}{>{\centering}p{\dimexpr 0.99in+0\tabcolsep}}{\textcolor[HTML]{000000}{\fontsize{11}{11}\selectfont{\global\setmainfont{Arial}{104\ (32\%)}}}} \\





\multicolumn{1}{>{\raggedright}p{\dimexpr 3.18in+0\tabcolsep}}{\textcolor[HTML]{000000}{\fontsize{11}{11}\selectfont{\global\setmainfont{Arial}{Fakultas\ Hukum}}}} & \multicolumn{1}{>{\centering}p{\dimexpr 0.99in+0\tabcolsep}}{\textcolor[HTML]{000000}{\fontsize{11}{11}\selectfont{\global\setmainfont{Arial}{98\ (30\%)}}}} \\





\multicolumn{1}{>{\raggedright}p{\dimexpr 3.18in+0\tabcolsep}}{\textcolor[HTML]{000000}{\fontsize{11}{11}\selectfont{\global\setmainfont{Arial}{Fakultas\ Ilmu\ Komputer}}}} & \multicolumn{1}{>{\centering}p{\dimexpr 0.99in+0\tabcolsep}}{\textcolor[HTML]{000000}{\fontsize{11}{11}\selectfont{\global\setmainfont{Arial}{29\ (8.8\%)}}}} \\





\multicolumn{1}{>{\raggedright}p{\dimexpr 3.18in+0\tabcolsep}}{\textcolor[HTML]{000000}{\fontsize{11}{11}\selectfont{\global\setmainfont{Arial}{Fakultas\ Ilmu\ Sosial\ dan\ Politik}}}} & \multicolumn{1}{>{\centering}p{\dimexpr 0.99in+0\tabcolsep}}{\textcolor[HTML]{000000}{\fontsize{11}{11}\selectfont{\global\setmainfont{Arial}{52\ (16\%)}}}} \\





\multicolumn{1}{>{\raggedright}p{\dimexpr 3.18in+0\tabcolsep}}{\textcolor[HTML]{000000}{\fontsize{11}{11}\selectfont{\global\setmainfont{Arial}{Fakultas\ Keguruan\ dan\ Ilmu\ Pendidikan}}}} & \multicolumn{1}{>{\centering}p{\dimexpr 0.99in+0\tabcolsep}}{\textcolor[HTML]{000000}{\fontsize{11}{11}\selectfont{\global\setmainfont{Arial}{6\ (1.8\%)}}}} \\





\multicolumn{1}{>{\raggedright}p{\dimexpr 3.18in+0\tabcolsep}}{\textcolor[HTML]{000000}{\fontsize{11}{11}\selectfont{\global\setmainfont{Arial}{Fakultas\ Teknik}}}} & \multicolumn{1}{>{\centering}p{\dimexpr 0.99in+0\tabcolsep}}{\textcolor[HTML]{000000}{\fontsize{11}{11}\selectfont{\global\setmainfont{Arial}{41\ (12\%)}}}} \\





\multicolumn{1}{>{\raggedright}p{\dimexpr 3.18in+0\tabcolsep}}{\textcolor[HTML]{000000}{\fontsize{11}{11}\selectfont{\global\setmainfont{Arial}{Tingkat.Semester}}}} & \multicolumn{1}{>{\centering}p{\dimexpr 0.99in+0\tabcolsep}}{\textcolor[HTML]{000000}{\fontsize{11}{11}\selectfont{\global\setmainfont{Arial}{}}}} \\





\multicolumn{1}{>{\raggedright}p{\dimexpr 3.18in+0\tabcolsep}}{\textcolor[HTML]{000000}{\fontsize{11}{11}\selectfont{\global\setmainfont{Arial}{1\ (Semester\ 1\ –\ Semester\ 2)}}}} & \multicolumn{1}{>{\centering}p{\dimexpr 0.99in+0\tabcolsep}}{\textcolor[HTML]{000000}{\fontsize{11}{11}\selectfont{\global\setmainfont{Arial}{34\ (10\%)}}}} \\





\multicolumn{1}{>{\raggedright}p{\dimexpr 3.18in+0\tabcolsep}}{\textcolor[HTML]{000000}{\fontsize{11}{11}\selectfont{\global\setmainfont{Arial}{2\ (Semester\ 3\ –\ Semester\ 4)}}}} & \multicolumn{1}{>{\centering}p{\dimexpr 0.99in+0\tabcolsep}}{\textcolor[HTML]{000000}{\fontsize{11}{11}\selectfont{\global\setmainfont{Arial}{50\ (15\%)}}}} \\





\multicolumn{1}{>{\raggedright}p{\dimexpr 3.18in+0\tabcolsep}}{\textcolor[HTML]{000000}{\fontsize{11}{11}\selectfont{\global\setmainfont{Arial}{3\ (Semester\ 5\ -\ Semester\ 6)}}}} & \multicolumn{1}{>{\centering}p{\dimexpr 0.99in+0\tabcolsep}}{\textcolor[HTML]{000000}{\fontsize{11}{11}\selectfont{\global\setmainfont{Arial}{63\ (19\%)}}}} \\





\multicolumn{1}{>{\raggedright}p{\dimexpr 3.18in+0\tabcolsep}}{\textcolor[HTML]{000000}{\fontsize{11}{11}\selectfont{\global\setmainfont{Arial}{4\ (Semester\ 7\ -\ Semester\ 8)}}}} & \multicolumn{1}{>{\centering}p{\dimexpr 0.99in+0\tabcolsep}}{\textcolor[HTML]{000000}{\fontsize{11}{11}\selectfont{\global\setmainfont{Arial}{183\ (55\%)}}}} \\





\multicolumn{1}{>{\raggedright}p{\dimexpr 3.18in+0\tabcolsep}}{\textcolor[HTML]{000000}{\fontsize{11}{11}\selectfont{\global\setmainfont{Arial}{Uang.Saku}}}} & \multicolumn{1}{>{\centering}p{\dimexpr 0.99in+0\tabcolsep}}{\textcolor[HTML]{000000}{\fontsize{11}{11}\selectfont{\global\setmainfont{Arial}{}}}} \\





\multicolumn{1}{>{\raggedright}p{\dimexpr 3.18in+0\tabcolsep}}{\textcolor[HTML]{000000}{\fontsize{11}{11}\selectfont{\global\setmainfont{Arial}{<\ 1\ jt}}}} & \multicolumn{1}{>{\centering}p{\dimexpr 0.99in+0\tabcolsep}}{\textcolor[HTML]{000000}{\fontsize{11}{11}\selectfont{\global\setmainfont{Arial}{28\ (8.5\%)}}}} \\





\multicolumn{1}{>{\raggedright}p{\dimexpr 3.18in+0\tabcolsep}}{\textcolor[HTML]{000000}{\fontsize{11}{11}\selectfont{\global\setmainfont{Arial}{1\ jt\ –\ 2\ jt}}}} & \multicolumn{1}{>{\centering}p{\dimexpr 0.99in+0\tabcolsep}}{\textcolor[HTML]{000000}{\fontsize{11}{11}\selectfont{\global\setmainfont{Arial}{167\ (51\%)}}}} \\





\multicolumn{1}{>{\raggedright}p{\dimexpr 3.18in+0\tabcolsep}}{\textcolor[HTML]{000000}{\fontsize{11}{11}\selectfont{\global\setmainfont{Arial}{2,1\ jt\ –\ 3\ jt}}}} & \multicolumn{1}{>{\centering}p{\dimexpr 0.99in+0\tabcolsep}}{\textcolor[HTML]{000000}{\fontsize{11}{11}\selectfont{\global\setmainfont{Arial}{116\ (35\%)}}}} \\





\multicolumn{1}{>{\raggedright}p{\dimexpr 3.18in+0\tabcolsep}}{\textcolor[HTML]{000000}{\fontsize{11}{11}\selectfont{\global\setmainfont{Arial}{3,1\ jt\ –\ 4\ jt}}}} & \multicolumn{1}{>{\centering}p{\dimexpr 0.99in+0\tabcolsep}}{\textcolor[HTML]{000000}{\fontsize{11}{11}\selectfont{\global\setmainfont{Arial}{18\ (5.5\%)}}}} \\





\multicolumn{1}{>{\raggedright}p{\dimexpr 3.18in+0\tabcolsep}}{\textcolor[HTML]{000000}{\fontsize{11}{11}\selectfont{\global\setmainfont{Arial}{>\ 4\ jt}}}} & \multicolumn{1}{>{\centering}p{\dimexpr 0.99in+0\tabcolsep}}{\textcolor[HTML]{000000}{\fontsize{11}{11}\selectfont{\global\setmainfont{Arial}{1\ (0.3\%)}}}} \\





\multicolumn{1}{>{\raggedright}p{\dimexpr 3.18in+0\tabcolsep}}{\textcolor[HTML]{000000}{\fontsize{11}{11}\selectfont{\global\setmainfont{Arial}{jenis.tempat.tinggal}}}} & \multicolumn{1}{>{\centering}p{\dimexpr 0.99in+0\tabcolsep}}{\textcolor[HTML]{000000}{\fontsize{11}{11}\selectfont{\global\setmainfont{Arial}{}}}} \\





\multicolumn{1}{>{\raggedright}p{\dimexpr 3.18in+0\tabcolsep}}{\textcolor[HTML]{000000}{\fontsize{11}{11}\selectfont{\global\setmainfont{Arial}{Kos\ bersama-sama}}}} & \multicolumn{1}{>{\centering}p{\dimexpr 0.99in+0\tabcolsep}}{\textcolor[HTML]{000000}{\fontsize{11}{11}\selectfont{\global\setmainfont{Arial}{14\ (4.2\%)}}}} \\





\multicolumn{1}{>{\raggedright}p{\dimexpr 3.18in+0\tabcolsep}}{\textcolor[HTML]{000000}{\fontsize{11}{11}\selectfont{\global\setmainfont{Arial}{Kos\ Sendiri}}}} & \multicolumn{1}{>{\centering}p{\dimexpr 0.99in+0\tabcolsep}}{\textcolor[HTML]{000000}{\fontsize{11}{11}\selectfont{\global\setmainfont{Arial}{52\ (16\%)}}}} \\





\multicolumn{1}{>{\raggedright}p{\dimexpr 3.18in+0\tabcolsep}}{\textcolor[HTML]{000000}{\fontsize{11}{11}\selectfont{\global\setmainfont{Arial}{Rumah\ bersama\ saudara}}}} & \multicolumn{1}{>{\centering}p{\dimexpr 0.99in+0\tabcolsep}}{\textcolor[HTML]{000000}{\fontsize{11}{11}\selectfont{\global\setmainfont{Arial}{23\ (7.0\%)}}}} \\





\multicolumn{1}{>{\raggedright}p{\dimexpr 3.18in+0\tabcolsep}}{\textcolor[HTML]{000000}{\fontsize{11}{11}\selectfont{\global\setmainfont{Arial}{Rumah\ mengontrak\ bersama-sama}}}} & \multicolumn{1}{>{\centering}p{\dimexpr 0.99in+0\tabcolsep}}{\textcolor[HTML]{000000}{\fontsize{11}{11}\selectfont{\global\setmainfont{Arial}{5\ (1.5\%)}}}} \\





\multicolumn{1}{>{\raggedright}p{\dimexpr 3.18in+0\tabcolsep}}{\textcolor[HTML]{000000}{\fontsize{11}{11}\selectfont{\global\setmainfont{Arial}{Rumah\ mengontrak\ pribadi}}}} & \multicolumn{1}{>{\centering}p{\dimexpr 0.99in+0\tabcolsep}}{\textcolor[HTML]{000000}{\fontsize{11}{11}\selectfont{\global\setmainfont{Arial}{3\ (0.9\%)}}}} \\





\multicolumn{1}{>{\raggedright}p{\dimexpr 3.18in+0\tabcolsep}}{\textcolor[HTML]{000000}{\fontsize{11}{11}\selectfont{\global\setmainfont{Arial}{Rumah\ pribadi/rumah\ keluarga}}}} & \multicolumn{1}{>{\centering}p{\dimexpr 0.99in+0\tabcolsep}}{\textcolor[HTML]{000000}{\fontsize{11}{11}\selectfont{\global\setmainfont{Arial}{233\ (71\%)}}}} \\

\ascline{1pt}{000000}{1-2}



\end{longtable}



\arrayrulecolor[HTML]{000000}

\global\setlength{\arrayrulewidth}{\Oldarrayrulewidth}

\global\setlength{\tabcolsep}{\Oldtabcolsep}

\renewcommand*{\arraystretch}{1}

Untuk menghitung IQV, kita bisa memanfaatkan perintah \texttt{function} yang mengambil masukan nilai-nilai yang ada di tanda kurungnya dan memprosesnya di dalam kurung kurawal \texttt{\{\}}. Perintah function selalu diakhiri dengan \texttt{return()} di dalamnya untuk menunjukkan hasil yang akan menjadi keluaran fungsi tersebut.

Untuk menghasilkan nilai IQV suatu variabel, kita perlu memasukkan vektor yang berisi nilai-nilai dalam variabel dataset kita. Ini dapat dilakukan dengan sintaks \texttt{dataset\$nama\_variabel}. Tanda \$ berfungsi memberi tahu R untuk memilih variabel yang ada dalam dataset yang digunakan.

\begin{verbatim}
## [1] 0.5636143
\end{verbatim}

\subsubsection{Variabel Numerik}\label{variabel-numerik}

Perintah \texttt{tbl\_summary} akan menentukan jenis statistik deskriptif yang ditampilkan secara intuitif berdasarkan dataset yang dibacanya. Perhatikan hasil dari \texttt{tbl\_summary} berikut untuk variabel \texttt{kepemilikan.mobil} dan \texttt{jarak}.

\global\setlength{\Oldarrayrulewidth}{\arrayrulewidth}

\global\setlength{\Oldtabcolsep}{\tabcolsep}

\setlength{\tabcolsep}{2pt}

\renewcommand*{\arraystretch}{1.5}



\providecommand{\ascline}[3]{\noalign{\global\arrayrulewidth #1}\arrayrulecolor[HTML]{#2}\cline{#3}}

\begin{longtable}[c]{|p{1.50in}|p{1.41in}}



\ascline{1pt}{000000}{1-2}

\multicolumn{1}{>{\raggedright}m{\dimexpr 1.5in+0\tabcolsep}}{\textcolor[HTML]{000000}{\fontsize{11}{11}\selectfont{\global\setmainfont{Arial}{\textbf{Characteristic}}}}} & \multicolumn{1}{>{\centering}m{\dimexpr 1.41in+0\tabcolsep}}{\textcolor[HTML]{000000}{\fontsize{11}{11}\selectfont{\global\setmainfont{Arial}{\textbf{N\ =\ 330}}}}\textcolor[HTML]{000000}{\fontsize{11}{11}\selectfont{\global\setmainfont{Arial}{\textsuperscript{1}}}}} \\

\ascline{1pt}{000000}{1-2}\endfirsthead 

\ascline{1pt}{000000}{1-2}

\multicolumn{1}{>{\raggedright}m{\dimexpr 1.5in+0\tabcolsep}}{\textcolor[HTML]{000000}{\fontsize{11}{11}\selectfont{\global\setmainfont{Arial}{\textbf{Characteristic}}}}} & \multicolumn{1}{>{\centering}m{\dimexpr 1.41in+0\tabcolsep}}{\textcolor[HTML]{000000}{\fontsize{11}{11}\selectfont{\global\setmainfont{Arial}{\textbf{N\ =\ 330}}}}\textcolor[HTML]{000000}{\fontsize{11}{11}\selectfont{\global\setmainfont{Arial}{\textsuperscript{1}}}}} \\

\ascline{1pt}{000000}{1-2}\endhead



\multicolumn{2}{>{\raggedright}m{\dimexpr 2.91in+2\tabcolsep}}{\textcolor[HTML]{000000}{\fontsize{11}{11}\selectfont{\global\setmainfont{Arial}{\textsuperscript{1}}}}\textcolor[HTML]{000000}{\fontsize{11}{11}\selectfont{\global\setmainfont{Arial}{n\ (\%);\ Median\ (Q1,\ Q3)}}}} \\

\endlastfoot



\multicolumn{1}{>{\raggedright}p{\dimexpr 1.5in+0\tabcolsep}}{\textcolor[HTML]{000000}{\fontsize{11}{11}\selectfont{\global\setmainfont{Arial}{kepemilikan.mobil}}}} & \multicolumn{1}{>{\centering}p{\dimexpr 1.41in+0\tabcolsep}}{\textcolor[HTML]{000000}{\fontsize{11}{11}\selectfont{\global\setmainfont{Arial}{}}}} \\





\multicolumn{1}{>{\raggedright}p{\dimexpr 1.5in+0\tabcolsep}}{\textcolor[HTML]{000000}{\fontsize{11}{11}\selectfont{\global\setmainfont{Arial}{0}}}} & \multicolumn{1}{>{\centering}p{\dimexpr 1.41in+0\tabcolsep}}{\textcolor[HTML]{000000}{\fontsize{11}{11}\selectfont{\global\setmainfont{Arial}{107\ (32\%)}}}} \\





\multicolumn{1}{>{\raggedright}p{\dimexpr 1.5in+0\tabcolsep}}{\textcolor[HTML]{000000}{\fontsize{11}{11}\selectfont{\global\setmainfont{Arial}{1}}}} & \multicolumn{1}{>{\centering}p{\dimexpr 1.41in+0\tabcolsep}}{\textcolor[HTML]{000000}{\fontsize{11}{11}\selectfont{\global\setmainfont{Arial}{126\ (38\%)}}}} \\





\multicolumn{1}{>{\raggedright}p{\dimexpr 1.5in+0\tabcolsep}}{\textcolor[HTML]{000000}{\fontsize{11}{11}\selectfont{\global\setmainfont{Arial}{2}}}} & \multicolumn{1}{>{\centering}p{\dimexpr 1.41in+0\tabcolsep}}{\textcolor[HTML]{000000}{\fontsize{11}{11}\selectfont{\global\setmainfont{Arial}{80\ (24\%)}}}} \\





\multicolumn{1}{>{\raggedright}p{\dimexpr 1.5in+0\tabcolsep}}{\textcolor[HTML]{000000}{\fontsize{11}{11}\selectfont{\global\setmainfont{Arial}{3}}}} & \multicolumn{1}{>{\centering}p{\dimexpr 1.41in+0\tabcolsep}}{\textcolor[HTML]{000000}{\fontsize{11}{11}\selectfont{\global\setmainfont{Arial}{15\ (4.5\%)}}}} \\





\multicolumn{1}{>{\raggedright}p{\dimexpr 1.5in+0\tabcolsep}}{\textcolor[HTML]{000000}{\fontsize{11}{11}\selectfont{\global\setmainfont{Arial}{4}}}} & \multicolumn{1}{>{\centering}p{\dimexpr 1.41in+0\tabcolsep}}{\textcolor[HTML]{000000}{\fontsize{11}{11}\selectfont{\global\setmainfont{Arial}{2\ (0.6\%)}}}} \\





\multicolumn{1}{>{\raggedright}p{\dimexpr 1.5in+0\tabcolsep}}{\textcolor[HTML]{000000}{\fontsize{11}{11}\selectfont{\global\setmainfont{Arial}{jarak}}}} & \multicolumn{1}{>{\centering}p{\dimexpr 1.41in+0\tabcolsep}}{\textcolor[HTML]{000000}{\fontsize{11}{11}\selectfont{\global\setmainfont{Arial}{2.71\ (1.31,\ 4.74)}}}} \\

\ascline{1pt}{000000}{1-2}



\end{longtable}



\arrayrulecolor[HTML]{000000}

\global\setlength{\arrayrulewidth}{\Oldarrayrulewidth}

\global\setlength{\tabcolsep}{\Oldtabcolsep}

\renewcommand*{\arraystretch}{1}

Seperti yang kalian lihat, \texttt{tbl\_summary} langsung menampilkan persentase tiap nilai yang ada di kepemilikan.mobil seolah-olah \texttt{kepemilikan.mobil} adalah variabel diskret. Hal ini terjadi karena nilai variabel \texttt{kepemilikan.mobil} adalah nilai numerik diskret, sehingga masing-masing nilai dikenali sebagai kategori.

Hal ini berbeda lagi dengan variabel \texttt{jarak} yang menampilkan statistik deskriptif untuk variabel numerik, yakni median, kuartil bawah dan kuartil atas (Q1 \& Q3).

Untuk meminta \texttt{tbl\_summary} menampilkan analisis statistik deskriptif untuk variabel numerik seperti mean, median, kuartil, dan standar deviasi (simpangan baku), kita dapat menyatakannya dalam atribut-atribut berikut.

\global\setlength{\Oldarrayrulewidth}{\arrayrulewidth}

\global\setlength{\Oldtabcolsep}{\tabcolsep}

\setlength{\tabcolsep}{2pt}

\renewcommand*{\arraystretch}{1.5}



\providecommand{\ascline}[3]{\noalign{\global\arrayrulewidth #1}\arrayrulecolor[HTML]{#2}\cline{#3}}

\begin{longtable}[c]{|p{1.52in}|p{1.02in}}



\ascline{1pt}{000000}{1-2}

\multicolumn{1}{>{\raggedright}m{\dimexpr 1.52in+0\tabcolsep}}{\textcolor[HTML]{000000}{\fontsize{11}{11}\selectfont{\global\setmainfont{Arial}{\textbf{Characteristic}}}}} & \multicolumn{1}{>{\centering}m{\dimexpr 1.02in+0\tabcolsep}}{\textcolor[HTML]{000000}{\fontsize{11}{11}\selectfont{\global\setmainfont{Arial}{\textbf{N\ =\ 330}}}}\textcolor[HTML]{000000}{\fontsize{11}{11}\selectfont{\global\setmainfont{Arial}{\textsuperscript{1}}}}} \\

\ascline{1pt}{000000}{1-2}\endfirsthead 

\ascline{1pt}{000000}{1-2}

\multicolumn{1}{>{\raggedright}m{\dimexpr 1.52in+0\tabcolsep}}{\textcolor[HTML]{000000}{\fontsize{11}{11}\selectfont{\global\setmainfont{Arial}{\textbf{Characteristic}}}}} & \multicolumn{1}{>{\centering}m{\dimexpr 1.02in+0\tabcolsep}}{\textcolor[HTML]{000000}{\fontsize{11}{11}\selectfont{\global\setmainfont{Arial}{\textbf{N\ =\ 330}}}}\textcolor[HTML]{000000}{\fontsize{11}{11}\selectfont{\global\setmainfont{Arial}{\textsuperscript{1}}}}} \\

\ascline{1pt}{000000}{1-2}\endhead



\multicolumn{2}{>{\raggedright}m{\dimexpr 2.55in+2\tabcolsep}}{\textcolor[HTML]{000000}{\fontsize{11}{11}\selectfont{\global\setmainfont{Arial}{\textsuperscript{1}}}}\textcolor[HTML]{000000}{\fontsize{11}{11}\selectfont{\global\setmainfont{Arial}{Median\ (Q1,\ Q3);\ Median;\ Mean\ (SD)}}}} \\

\endlastfoot



\multicolumn{1}{>{\raggedright}p{\dimexpr 1.52in+0\tabcolsep}}{\textcolor[HTML]{000000}{\fontsize{11}{11}\selectfont{\global\setmainfont{Arial}{kepemilikan.mobil}}}} & \multicolumn{1}{>{\centering}p{\dimexpr 1.02in+0\tabcolsep}}{\textcolor[HTML]{000000}{\fontsize{11}{11}\selectfont{\global\setmainfont{Arial}{1\ (0,\ 2)}}}} \\





\multicolumn{1}{>{\raggedright}p{\dimexpr 1.52in+0\tabcolsep}}{\textcolor[HTML]{000000}{\fontsize{11}{11}\selectfont{\global\setmainfont{Arial}{kepemilikan.motor}}}} & \multicolumn{1}{>{\centering}p{\dimexpr 1.02in+0\tabcolsep}}{\textcolor[HTML]{000000}{\fontsize{11}{11}\selectfont{\global\setmainfont{Arial}{1}}}} \\





\multicolumn{1}{>{\raggedright}p{\dimexpr 1.52in+0\tabcolsep}}{\textcolor[HTML]{000000}{\fontsize{11}{11}\selectfont{\global\setmainfont{Arial}{jarak}}}} & \multicolumn{1}{>{\centering}p{\dimexpr 1.02in+0\tabcolsep}}{\textcolor[HTML]{000000}{\fontsize{11}{11}\selectfont{\global\setmainfont{Arial}{3.35\ (3.54)}}}} \\

\ascline{1pt}{000000}{1-2}



\end{longtable}



\arrayrulecolor[HTML]{000000}

\global\setlength{\arrayrulewidth}{\Oldarrayrulewidth}

\global\setlength{\tabcolsep}{\Oldtabcolsep}

\renewcommand*{\arraystretch}{1}

Untuk mengatur jenis statistik deskriptif yang ditampilkan, ubah bagian kanan tanda \texttt{\textasciitilde{}} pada atribut \texttt{statistic} sehingga apa yang ditulis di antara tanda petik menjadi template untuk ditampilkan di tabel. Adapun teknik statistik yang kita pakai harus kita tuliskan dalam kurung kurawalnya. Teknik-teknik statistik yang bisa kita gunakan di antaranya:

\begin{itemize}
\item
  \texttt{\{mean\}} : rata-rata
\item
  \texttt{\{median\}} : median
\item
  \texttt{\{min\}}, \texttt{\{max\}}: nilai minimum, maksimum
\item
  \texttt{\{p\#\#\}} : persentil \#\#. Persentil adalah nilai yang membagi data menjadi seratus bagian. Karena kuartil membagi data menjadi empat bagian, maka untuk menampilkan kuartil bawah (Q1), yang berarti nilai yang membagi data menjadi 1/4 terbawah (25\%), kita menuliskan \texttt{\{p25\}}. Begitu juga dengan Q3 yang membagi data menjadi 3/4 terbawah (75\%), kita menuliskan \texttt{\{p75\}}
\item
  \texttt{\{sd\}} : \emph{standard deviation}/simpangan baku
\item
  \texttt{\{n\}} : frekuensi kategori
\item
  \texttt{\{N\}} : jumlah seluruhnya
\item
  \texttt{\{p\}} : persentase
\end{itemize}

\textbf{Pengenalan Tipe Data Baru R: List (\texttt{list)}}

\emph{List} pada dasarnya adalah tipe data nontunggal seperti vektor, hanya saja perbedaannya terletak pada tipe nilai yang dikandung. Vektor mewajibkan setiap elemen bernilai sama: string saja, angka saja, atau \emph{boolean} saja misalnya.

Di sisi lain, \emph{list} bisa memuat lebih dari satu jenis nilai. Ini sangat berguna ketika kita harus mendefinisikan sesuatu yang membutuhkan lebih dari satu nilai, seperti cara menampilkan statistik deskriptif untuk tiap-tiap variabel kita di fungsi \texttt{tbl\_summary} di atas:

\begin{Shaded}
\begin{Highlighting}[]
\NormalTok{type }\OtherTok{=} \FunctionTok{list}\NormalTok{(kepemilikan.motor }\SpecialCharTok{\textasciitilde{}} \StringTok{"continuous"}\NormalTok{, kepemilikan.mobil }\SpecialCharTok{\textasciitilde{}} \StringTok{"continuous"}\NormalTok{)}

\NormalTok{digits }\OtherTok{=} \FunctionTok{list}\NormalTok{(kepemilikan.motor }\SpecialCharTok{\textasciitilde{}} \DecValTok{0}\NormalTok{, kepemilikan.mobil }\SpecialCharTok{\textasciitilde{}} \DecValTok{0}\NormalTok{)}
\end{Highlighting}
\end{Shaded}

\begin{itemize}
\item
  \emph{list} dalam pengaturan atribut \texttt{type} berarti ``anggap tipe variabel \texttt{kepemilikan.motor} sebagai \emph{continuous\textbf{,}} begitu juga untuk variabel \texttt{kepemilikan.mobil}
\item
  \emph{list} dalam pengaturan atribut \texttt{digits} berarti ``atur agar nilai desimal untuk \texttt{kepemilikan.motor} adalah 0, begitu juga untuk variabel \texttt{kepemilikan.mobil}
\end{itemize}

\textbf{Aktivitas Mandiri 3: Analisis Statistik Deskriptif Komprehensif {[}STP-2.4{]}}

\textbf{Persiapan:}
Impor dataset UIN RIL dari sheet \texttt{DataUtama\_mhsUINRIL} dan bersihkan seperti yang sudah dipraktikkan:
- Rename variabel yang perlu disesuaikan
- Gunakan \texttt{drop\_na()} untuk menghapus missing values
- Buat factor untuk variabel kategoris (minimal: \texttt{Jenis.Kelamin}, \texttt{Tingkat.Semester}, \texttt{Uang.Saku})

\textbf{Analisis yang Diminta:}

\begin{enumerate}
\def\labelenumi{\alph{enumi}.}
\tightlist
\item
  \textbf{Persentase/Proporsi:}

  \begin{itemize}
  \tightlist
  \item
    Hitung persentase mahasiswa berdasarkan \texttt{Uang.Saku}
  \item
    Gunakan \texttt{tbl\_summary(include\ =\ Uang.Saku)} dan \texttt{as\_flex\_table()}
  \end{itemize}
\item
  \textbf{Ukuran Pemusatan (Median dan Kuartil):}

  \begin{itemize}
  \tightlist
  \item
    Untuk variabel \texttt{Jumlah.Perjalanan.Senin} (atau hari lain jika variabel ini tidak ada)
  \item
    Tampilkan Median, Q1 (p25), dan Q3 (p75)
  \item
    Gunakan \texttt{statistic\ =\ "\{median\}\ (\{p25\},\ \{p75\})"}
  \end{itemize}
\item
  \textbf{Ukuran Pemusatan dan Penyebaran (Mean dan SD):}

  \begin{itemize}
  \tightlist
  \item
    Untuk variabel biaya perjalanan per pekan
  \item
    Tampilkan Mean dan Standar Deviasi
  \item
    Gunakan \texttt{statistic\ =\ "\{mean\}\ (\{sd\})"}
  \end{itemize}
\item
  \textbf{Ukuran Variasi Nominal (IQV):}

  \begin{itemize}
  \tightlist
  \item
    Hitung IQV untuk variabel \texttt{alasan.pemilihan.lokasi.tempat.tinggal}
  \item
    Gunakan fungsi \texttt{iqv()} yang sudah dipelajari
  \item
    Interpretasi: seberapa bervariasi alasan mahasiswa memilih tempat tinggal?
  \end{itemize}
\end{enumerate}

\textbf{Dokumentasi:}
- Kumpulkan semua hasil dalam file R Markdown Anda
- Sertakan interpretasi singkat untuk setiap hasil analisis

\begin{center}\rule{0.5\linewidth}{0.5pt}\end{center}

\section{Modul-3: Visualisasi Data Kuantitatif}\label{modul-3-visualisasi-data-kuantitatif}

Setelah mempelajari modul ini, Anda diharapkan dapat:

\begin{enumerate}
\def\labelenumi{\arabic{enumi}.}
\tightlist
\item
  memilih visualiasi yang tepat sesuai dengan variabel yang akan disajikan dan informasi yang ingin disampaikan \capaian{STP-3.1}
\item
  menginterpretasikan suatu visualiasi data kuantitatif secara mendalam \capaian{STP-3.2}
\item
  menjelaskan pentingnya menentukan tingkat pengukuran untuk sebuah variabel dari kaitannya dengan analisis statistik deskriptif dan diagram yang dipilih untuk menyajikan informasi \capaian{STP-3.4}
\item
  menghasilkan grafik yang tepat sesuai variabel yang akan disajikan \capaian{STP-3.3}
\end{enumerate}

\begin{center}\rule{0.5\linewidth}{0.5pt}\end{center}

\subsection{\texorpdfstring{Visualisasi Data dengan \texttt{ggplot2}}{Visualisasi Data dengan ggplot2}}\label{visualisasi-data-dengan-ggplot2}

\texttt{ggplot2} adalah sebuah paket R yang dibuat oleh Hadley Wickham untuk membuat grafik dan visualisasi data. Paket ini didasarkan pada ``Grammar of Graphics'', sebuah kerangka kerja yang memecah visualisasi menjadi komponen-komponen terpisah seperti data, sistem koordinat, dan elemen-elemen visual (geometries). Dengan pendekatan ini, Anda dapat membangun grafik lapis demi lapis \emph{(layer by layer)}.

\begin{figure}
\centering
\pandocbounded{\includegraphics[keepaspectratio,alt={Konsep ggplot2}]{images/basics_ggplot2.png}}
\caption{Konsep \protect\texttt{ggplot2}}
\end{figure}

\subsubsection{\texorpdfstring{Mengimpor \emph{Library} \& Mengatur Dataset}{Mengimpor Library \& Mengatur Dataset}}\label{mengimpor-library-mengatur-dataset}

Pertama, kita perlu memuat paket \texttt{tidyverse} yang sudah mencakup \texttt{ggplot2} untuk visualisasi dan \texttt{dplyr} serta \texttt{readr} untuk manipulasi data.

Tak lupa, kita juga akan menyertakan \texttt{openxlsx} dan \texttt{gtsummary} untuk mengolah data secara tabular jika sekiranya diperlukan dalam menunjang alur kerja kita.

Selanjutnya, kita akan mengimpor \emph{dataset} kita, yakni hasil kuesioner kepada mahasiswa UBL., seperti halnya praktikum-praktikum sebelumnya.

\begin{verbatim}
##   Timestamp Kampus_PT Nomor.urut Jenis.Kelamin Umur
## 1  45413.83       UBL          1     Perempuan   21
## 2  45413.83       UBL          2     Laki-Laki   20
## 3  45413.85       UBL          3     Laki-Laki   21
## 4  45413.85       UBL          4     Laki-Laki   22
## 5  45413.88       UBL          5     Laki-Laki   21
## 6  45413.92       UBL          6     Perempuan   21
##                           Fakultas           Prodi            Tingkat.Semester
## 1 Fakultas Ilmu Sosial dan Politik Ilmu Komunikasi 4 (Semester 7 - Semester 8)
## 2                   Fakultas Hukum      Ilmu Hukum 3 (Semester 5 - Semester 6)
## 3      Fakultas Ekonomi dan Bisnis       Manajemen 4 (Semester 7 - Semester 8)
## 4      Fakultas Ekonomi dan Bisnis       Akuntansi 4 (Semester 7 - Semester 8)
## 5      Fakultas Ekonomi dan Bisnis       Manajemen 4 (Semester 7 - Semester 8)
## 6      Fakultas Ekonomi dan Bisnis       Akuntansi 4 (Semester 7 - Semester 8)
##       Uang.Saku kepemilikan.mobil kepemilikan.motor kepemilikan.sepeda
## 1        < 1 jt                 1                 2                  1
## 2   1 jt – 2 jt                 2                 1                  1
## 3   1 jt – 2 jt                 1                 2                  2
## 4   1 jt – 2 jt                 1                 4                  1
## 5 2,1 jt – 3 jt                 4                 4                  1
## 6   1 jt – 2 jt                 1                 2                  1
##                                        kendaraan.utama               kelurahan
## 1                                 Sepeda Motor Pribadi       Kalibalau Kencana
## 2                                        Mobil Pribadi            Gunung sulah
## 3                                 Sepeda Motor Pribadi              langkapura
## 4 Kendaraan Bermotor (menumpang dengan keluarga/teman) Bilabong JAYA JAYA JAYA
## 5                                        Mobil Pribadi                sukarame
## 6                                  Transportasi Online               Way halim
##           jenis.tempat.tinggal                       nama.jalan.tempat.tinggal
## 1 Rumah pribadi/rumah keluarga                                    eLBe Loundry
## 2 Rumah pribadi/rumah keluarga                         Jl.Urip Sumoharjo no 88
## 3 Rumah pribadi/rumah keluarga                  JL DARUSSALAM GG LANGGAR LK II
## 4 Rumah pribadi/rumah keluarga Jl Darussalam bilabong bila bolong di jait dong
## 5 Rumah pribadi/rumah keluarga          jl.cendrawasih sukarame bandar lampung
## 6 Rumah pribadi/rumah keluarga                            Jl P tabuan nomor 26
##   jarak.(km)         alasan.pemilihan.lokasi.tempat.tinggal
## 1   4.223797                 Bersama keluarga/saudara/teman
## 2   2.703331 Mudahnya akses berpergian dari tempat tinggal 
## 3   3.237072                 Bersama keluarga/saudara/teman
## 4   3.002336    Lingkungan nyaman karna aman dari kejahatan
## 5   5.961925                    Dekat dengan fasilitas umum
## 6   2.925331                 Bersama keluarga/saudara/teman
##   biaya.dalam.seminggu Jumlah.perjalanan.Senin Jumlah.Perjalanan.Selasa
## 1                   25                       2                        3
## 2                  200                       3                        3
## 3                   50                       1                        1
## 4                   20                       2                        2
## 5                   NA                       1                        1
## 6                   40                       3                        3
##   Jumlah.Perjalanan.Rabu Jumlah.Perjalanan.Kamis Jumlah.Perjalanan.Jumat
## 1                      2                       3                       2
## 2                      3                       3                       3
## 3                      1                       1                       1
## 4                      2                       2                       2
## 5                      1                       1                       1
## 6                      1                       1                       1
##   Jumlah.Perjalanan.Sabtu Jumlah.Perjalanan.Ahad
## 1                       2                      2
## 2                       3                      3
## 3                       1                      1
## 4                       2                      2
## 5                       1                      1
## 6                       1                      1
\end{verbatim}

Kemudian kita perlu menetapkan \texttt{factor} untuk variabel-variabel kategoris kita agar data kita lebih `bersih.'

Kita akan mengubah kategori \texttt{factor} pada variabel \texttt{Tingkat.Semester} dan \texttt{Uang.Saku} agar lebih mudah dibaca saat divisualisasikan. Kita akan menggunakan fungsi dari \texttt{dplyr} (bagian dari \texttt{tidyverse}) untuk membersihkan dan mengubah data.

\begin{verbatim}
## Rows: 330
## Columns: 26
## $ Timestamp                              <dbl> 45413.83, 45413.83, 45413.85, 4~
## $ Kampus_PT                              <chr> "UBL", "UBL", "UBL", "UBL", "UB~
## $ Nomor.urut                             <dbl> 1, 2, 3, 4, 6, 7, 8, 9, 10, 12,~
## $ Jenis.Kelamin                          <fct> Perempuan, Laki-Laki, Laki-Laki~
## $ Umur                                   <dbl> 21, 20, 21, 22, 21, 22, 22, 22,~
## $ Fakultas                               <fct> Fakultas Ilmu Sosial dan Politi~
## $ Prodi                                  <fct> Ilmu Komunikasi, Ilmu Hukum, Ma~
## $ Tingkat.Semester                       <ord> Semester 7 & 8, Semester 5 & 6,~
## $ Uang.Saku                              <ord> "< 1 jt", "1 jt – 2 jt", "1 jt ~
## $ kepemilikan.mobil                      <dbl> 1, 2, 1, 1, 1, 1, 1, 1, 1, 0, 1~
## $ kepemilikan.motor                      <dbl> 2, 1, 2, 4, 2, 1, 1, 1, 1, 2, 1~
## $ kepemilikan.sepeda                     <dbl> 1, 1, 2, 1, 1, 1, 1, 1, 1, 0, 0~
## $ kendaraan.utama                        <chr> "Sepeda Motor Pribadi", "Mobil ~
## $ kelurahan                              <chr> "Kalibalau Kencana", "Gunung su~
## $ jenis.tempat.tinggal                   <chr> "Rumah pribadi/rumah keluarga",~
## $ nama.jalan.tempat.tinggal              <chr> "eLBe Loundry", "Jl.Urip Sumoha~
## $ jarak                                  <dbl> 4.2237967, 2.7033310, 3.2370722~
## $ alasan.pemilihan.lokasi.tempat.tinggal <chr> "Bersama keluarga/saudara/teman~
## $ biaya.dalam.sepekan                    <dbl> 25, 200, 50, 20, 40, 70, 35, 50~
## $ Jumlah.Perjalanan.Senin                <dbl> 2, 3, 1, 2, 3, 2, 4, 4, 3, 2, 4~
## $ Jumlah.Perjalanan.Selasa               <dbl> 3, 3, 1, 2, 3, 2, 4, 4, 3, 1, 4~
## $ Jumlah.Perjalanan.Rabu                 <dbl> 2, 3, 1, 2, 1, 2, 4, 4, 3, 2, 4~
## $ Jumlah.Perjalanan.Kamis                <dbl> 3, 3, 1, 2, 1, 2, 4, 4, 3, 1, 4~
## $ Jumlah.Perjalanan.Jumat                <dbl> 2, 3, 1, 2, 1, 2, 4, 3, 3, 2, 4~
## $ Jumlah.Perjalanan.Sabtu                <dbl> 2, 3, 1, 2, 1, 2, 3, 4, 2, 1, 1~
## $ Jumlah.Perjalanan.Ahad                 <dbl> 2, 3, 1, 2, 1, 2, 3, 4, 1, 1, 1~
\end{verbatim}

Sekarang kita siap memvisualkan data kita.

\subsubsection{\texorpdfstring{Tata Tulis Grafik \emph{(Grammar of Graphics)}}{Tata Tulis Grafik (Grammar of Graphics)}}\label{tata-tulis-grafik-grammar-of-graphics}

Setiap grafik \texttt{ggplot2} terdiri dari beberapa komponen kunci:

\begin{itemize}
\tightlist
\item
  \textbf{DATA}: Dataset yang ingin Anda visualisasikan.
\item
  \textbf{MAPPING}: \texttt{aes()} (aesthetics), yang menghubungkan variabel dari data Anda ke properti visual dari grafik (misalnya, sumbu x, sumbu y, warna, ukuran).
\item
  \textbf{GEOM\_FUNCTION}: Objek geometris yang merepresentasikan data (misalnya, \texttt{geom\_point()} untuk scatter plot, \texttt{geom\_bar()} untuk diagram batang).
\item
  \textbf{STAT}: Transformasi statistik. Setiap \texttt{geom} memiliki statistik default (misalnya, \texttt{geom\_bar} secara default menggunakan \texttt{stat\_count}), tetapi Anda bisa menentukannya secara manual.
\item
  \textbf{POSITION}: Penyesuaian posisi untuk \texttt{geom} yang tumpang tindih (misalnya, \texttt{position\_dodge()} atau \texttt{position\_stack()}).
\item
  \textbf{COORDINATE\_FUNCTION}: Sistem koordinat yang digunakan (\texttt{coord\_cartesian}, \texttt{coord\_flip}, dll.).
\item
  \textbf{FACET\_FUNCTION}: Membagi plot menjadi beberapa sub-plot berdasarkan variabel kategori (\texttt{facet\_wrap} atau \texttt{facet\_grid}).
\end{itemize}

\begin{Shaded}
\begin{Highlighting}[]
\FunctionTok{ggplot}\NormalTok{(}\SpecialCharTok{\textless{}}\NormalTok{DATA}\SpecialCharTok{\textgreater{}}\NormalTok{) }\SpecialCharTok{+} 
  \ErrorTok{\textless{}}\NormalTok{GEOM\_FUNCTION}\SpecialCharTok{\textgreater{}}\NormalTok{(}\AttributeTok{mapping =} \FunctionTok{aes}\NormalTok{(}\SpecialCharTok{\textless{}}\NormalTok{MAPPING}\SpecialCharTok{\textgreater{}}\NormalTok{),}
             \AttributeTok{stat =} \SpecialCharTok{\textless{}}\NormalTok{STAT}\SpecialCharTok{\textgreater{}}\NormalTok{,}
             \AttributeTok{position =} \SpecialCharTok{\textless{}}\NormalTok{POSITION}\SpecialCharTok{\textgreater{}}\NormalTok{) }\SpecialCharTok{+}
  \ErrorTok{\textless{}}\NormalTok{COORDINATE\_FUNCTION}\SpecialCharTok{\textgreater{}} \SpecialCharTok{+}
  \ErrorTok{\textless{}}\NormalTok{FACET\_FUNCTION}\SpecialCharTok{\textgreater{}} \SpecialCharTok{+}
  \ErrorTok{\textless{}}\NormalTok{SCALE\_FUNCTION}\SpecialCharTok{\textgreater{}} \SpecialCharTok{+}  \CommentTok{\# opsional}
  \ErrorTok{\textless{}}\NormalTok{THEME\_FUNCTION}\SpecialCharTok{\textgreater{}}    \CommentTok{\# opsional}
\end{Highlighting}
\end{Shaded}

\begin{figure}
\centering
\pandocbounded{\includegraphics[keepaspectratio,alt={``Grammar of Graphics''}]{images/grammar_of_graphics.png}}
\caption{``Grammar of Graphics''}
\end{figure}

\subsubsection{Praktik Visualisasi Data}\label{praktik-visualisasi-data}

\paragraph{\texorpdfstring{Diagram Batang \emph{(Column/Bar Chart)}}{Diagram Batang (Column/Bar Chart)}}\label{diagram-batang-columnbar-chart}

Diagram batang sangat baik untuk menampilkan distribusi atau perbandingan data \textbf{kategoris}.

\subparagraph{Diagram Batang Tunggal}\label{diagram-batang-tunggal}

Mari kita lihat distribusi mahasiswa berdasarkan tingkat semester. \texttt{geom\_bar()} secara otomatis menghitung jumlah observasi untuk setiap kategori di sumbu x.

\pandocbounded{\includegraphics[keepaspectratio]{figures/bar-chart-single-1.pdf}}

\textbf{Interpretasi:} Grafik di atas menunjukkan bahwa mayoritas responden mahasiswa berasal dari tingkat semester 5 \& 6, diikuti oleh semester 7 \& 8.

\textbf{Penjelasan Sintaks (Grammar of Graphics):}

\begin{itemize}
\tightlist
\item
  \textbf{DATA}: \texttt{ggplot(data.ubl.cleaned)} mendefinisikan dataset yang digunakan.
\item
  \textbf{GEOM}: \texttt{geom\_bar(...)} menentukan bentuk geometris yang digunakan, yaitu batang.
\item
  \textbf{MAPPING}: \texttt{mapping\ =\ aes(x\ =\ Tingkat.Semester)} memetakan variabel \texttt{Tingkat.Semester} dari data ke sumbu x pada grafik.
\item
  \textbf{STAT}: \texttt{geom\_bar()} secara default menggunakan \texttt{stat\ =\ "count"}, yang berarti ia secara otomatis melakukan transformasi statistik dengan menghitung jumlah baris untuk setiap kategori \texttt{Tingkat.Semester} dan menampilkannya sebagai ketinggian batang di sumbu y.
\item
  \texttt{fill\ =\ "skyblue",\ color\ =\ "black"}: Ini adalah pengaturan properti visual, bukan pemetaan. Kita mengatur semua batang agar memiliki warna isian ``skyblue'' dan garis tepi ``black''.
\item
  \texttt{labs(...)}, \texttt{theme\_minimal()}, \texttt{theme(...)}: Ini adalah lapisan tambahan untuk kustomisasi label dan tema, bukan bagian inti dari ``grammar''.
\end{itemize}

\subparagraph{\texorpdfstring{Diagram Batang Bertumpuk \emph{(Stacked)}}{Diagram Batang Bertumpuk (Stacked)}}\label{diagram-batang-bertumpuk-stacked}

Kita bisa menambahkan variabel lain, misalnya \texttt{Uang.Saku}, ke dalam \texttt{aes()} dengan properti \texttt{fill} untuk membuat diagram batang bertumpuk. Ini menunjukkan proporsi uang saku di setiap tingkat semester.

\pandocbounded{\includegraphics[keepaspectratio]{figures/bar-chart-stacked-1.pdf}}

\textbf{Interpretasi:} Dari grafik ini, kita bisa melihat komposisi uang saku di setiap angkatan. Misalnya, pada tingkat ``Semester 5 \& 6'', sebagian besar mahasiswa memiliki uang saku antara 1 juta hingga 2 juta. Hal ini kita ketahui dari perbandingan relatif tinggi porsi warna-warna dalam masing-masing batang.

\textbf{Penjelasan Sintaks (Grammar of Graphics):}

\begin{itemize}
\tightlist
\item
  \textbf{MAPPING}: \texttt{mapping\ =\ aes(x\ =\ Tingkat.Semester,\ fill\ =\ Uang.Saku)} kini memiliki pemetaan tambahan. Selain sumbu x, kita juga memetakan variabel \texttt{Uang.Saku} ke properti visual \texttt{fill} (warna isian). \texttt{ggplot} akan membuat segmen berwarna berbeda di dalam setiap batang sesuai kategori uang saku.
\item
  \textbf{POSITION}: Secara default, \texttt{geom\_bar()} menggunakan \texttt{position\ =\ "stack"} ketika \texttt{fill} dipetakan ke sebuah variabel. Inilah yang menyebabkan segmen-segmen tersebut ditumpuk di atas satu sama lain.
\item
  \texttt{labs(fill\ =\ "Uang\ Saku\ per\ Bulan")}: Argumen \texttt{fill} di dalam \texttt{labs()} berfungsi untuk mengubah judul dari legenda yang secara otomatis dibuat dari pemetaan \texttt{fill}.
\end{itemize}

Untuk membandingkan jumlah absolut antar kategori uang saku, diagram batang berkelompok lebih efektif. Kita gunakan \texttt{position\ =\ "dodge"}.

\pandocbounded{\includegraphics[keepaspectratio]{figures/bar-chart-grouped-1.pdf}}

\textbf{Interpretasi:} Grafik ini mempermudah perbandingan langsung. Terlihat jelas bahwa kategori uang saku ``1 jt -- 2 jt'' mendominasi di hampir semua tingkat semester.

\begin{itemize}
\tightlist
\item
  \textbf{POSITION}: Komponen \texttt{position} diubah secara eksplisit menjadi \texttt{position\ =\ "dodge"}. Ini menginstruksikan \texttt{ggplot} untuk menempatkan batang-batang yang memiliki kategori x yang sama (misalnya, ``Semester 5 \& 6'') bersebelahan, bukan menumpuknya. Ini memungkinkan perbandingan langsung antar kategori \texttt{Uang.Saku}.
\end{itemize}

Terkadang kita lebih tertarik pada perbandingan proporsi antar grup daripada jumlah absolutnya. Dengan mengubah \texttt{position} menjadi \texttt{"fill"}, kita dapat membuat setiap batang memiliki tinggi yang sama (100\%) dan menunjukkan persentase relatif dari setiap subgrup.

\pandocbounded{\includegraphics[keepaspectratio]{figures/bar-chart-fill-1.pdf}}

\textbf{Interpretasi:} Grafik ini menunjukkan bahwa secara proporsional, mahasiswa dengan kategori uang saku tertinggi (``\textgreater4jt'') yang paling dominan terdapat pada tingkat 3 (``Semester 5 \& 6''). Sementara itu, mahasiswa kategori uang saku terrendah paling banyak porsinya pada tingkat 2 (``Semester 3 \& 4''). Ini adalah wawasan yang mungkin tidak terlihat jelas pada grafik jumlah absolut.

\textbf{Penjelasan Sintaks (Grammar of Graphics):}

\begin{itemize}
\item
  \textbf{POSITION}: Komponen \texttt{position} diubah menjadi \texttt{position\ =\ "fill"}. Pengaturan ini secara otomatis melakukan transformasi \textbf{STAT} yang berbeda: ia menghitung proporsi dari setiap subgrup (\texttt{Uang.Saku}) dalam setiap grup (\texttt{Tingkat.Semester}). Hasilnya adalah setiap batang dinormalisasi menjadi setinggi 1 (atau 100\%).
\item
  \textbf{SCALE}: \texttt{scale\_y\_continuous(labels\ =\ scales::percent)} adalah lapisan tambahan yang mengontrol \textbf{SKALA} pada sumbu y. Fungsi \texttt{scales::percent} digunakan untuk memformat label sumbu dari angka desimal (misal: 0.5) menjadi format persentase (misal: 50\%) agar lebih mudah dibaca.
\end{itemize}

\textbf{Aktivitas Mandiri 1: Membuat dan Menginterpretasikan Diagram Batang {[}STP-3.2, STP-3.3{]}}

Buatlah diagram batang untuk variabel \texttt{Fakultas}:

\begin{enumerate}
\def\labelenumi{\arabic{enumi}.}
\tightlist
\item
  \textbf{Diagram batang sederhana} - gunakan \texttt{geom\_bar()}
\item
  \textbf{Diagram batang bertumpuk} dengan \texttt{Uang.Saku} sebagai fill
\item
  \textbf{Diagram batang berkelompok} dengan \texttt{position\ =\ "dodge"} dan interpretasikan hasilnya:

  \begin{itemize}
  \tightlist
  \item
    Fakultas mana yang paling banyak mahasiswa dengan uang saku \textgreater2jt?
  \item
    Bagaimana pola distribusi uang saku antar fakultas?
  \item
    Apakah ada fakultas yang dominan memiliki mahasiswa dengan uang saku tertentu?
  \end{itemize}
\end{enumerate}

\subsubsection{Diagram Lollipop}\label{diagram-lollipop}

Diagram lolipop adalah alternatif dari diagram batang yang dapat mengurangi tinta visual dan memberikan penekanan lebih pada nilai data. Grafik ini menggunakan segmen garis dan titik untuk merepresentasikan nilai. Ini sangat efektif untuk menampilkan data kategoris yang banyak kategorinya.

\pandocbounded{\includegraphics[keepaspectratio]{figures/lollipop-chart-1.pdf}}

\textbf{Interpretasi:} Grafik ini menunjukkan bahwa kebanyakan responden berasal dari program studi Ilmu Hukum dan Manajemen, dengan perbandingan yang cukup timpang dengan prodi-prodi lain. Selain itu, ternyata cukup dominan mahasiswa yang tidak memberikan data prodi mereka (NA) di antara prodi-prodi selain dua yang tertinggi tadi.

\textbf{Penjelasan Sintaks (Grammar of Graphics):}

\begin{itemize}
\tightlist
\item
  \textbf{DATA \& STAT}: Sama seperti contoh sebelumnya, kita melakukan pra-pemrosesan data menggunakan \texttt{dplyr}. Kita mengelompokkan data berdasarkan \texttt{Prodi}, menghitung (\texttt{count}) jumlahnya, mengurutkan (\texttt{arrange}), dan yang terpenting, mengubah \texttt{Prodi} menjadi variabel faktor yang terurut (\texttt{fct\_inorder}) agar plot ditampilkan sesuai urutan yang kita inginkan.
\item
  \textbf{GEOM \& MAPPING (Layering)}: Di sinilah keunikan diagram lolipop. Kita menggunakan \textbf{dua lapisan \texttt{geom}}:

  \begin{itemize}
  \tightlist
  \item
    \texttt{geom\_segment()}: Digunakan untuk membuat ``batang'' atau segmen garis. \textbf{Mapping}-nya membutuhkan empat estetika: \texttt{x} dan \texttt{xend} (yang sama untuk garis vertikal) serta \texttt{y} (titik awal, yaitu 0) dan \texttt{yend} (titik akhir, yaitu \texttt{jumlah}).
  \item
    \texttt{geom\_point()}: Digunakan untuk membuat ``permen'' atau titik di ujung segmen. \textbf{Mapping}-nya lebih sederhana, hanya membutuhkan \texttt{x} dan \texttt{y}.
  \end{itemize}
\item
  \textbf{KOORDINAT}: \texttt{coord\_flip()} digunakan untuk membalik sumbu, membuat diagram lolipop horizontal yang seringkali lebih mudah dibaca label kategorinya.
\item
  \textbf{THEME}: Lapisan tema digunakan untuk membersihkan tampilan, seperti menghilangkan beberapa garis grid dan batas panel untuk menonjolkan data itu sendiri.
\end{itemize}

\subsubsection{\texorpdfstring{Diagram Pai/Donat \emph{(Pie/Donut Chart)}}{Diagram Pai/Donat (Pie/Donut Chart)}}\label{diagram-paidonat-piedonut-chart}

Diagram pai (dan variasinya, diagram donat) digunakan untuk menunjukkan proporsi dari sebuah keseluruhan. Meskipun populer, diagram pai seringkali sulit untuk dibaca secara akurat, terutama ketika ada banyak irisan atau ukurannya mirip. Diagram donat sedikit lebih baik karena mengurangi penekanan pada sudut dan lebih fokus pada panjang busur.

Di \texttt{ggplot2}, diagram pai dibuat dengan memulai dari diagram batang bertumpuk, lalu mengubah sistem \textbf{KOORDINAT}-nya menjadi koordinat polar. Akan tetapi, dalam \texttt{ggplot2} kita tidak bisa membuat diagram pai langsung dari datasetnya, tetapi kita harus membentuk \textbf{tabel distribusi frekuensinya} terlebih dahulu.

\begin{verbatim}
##                                Fakultas jumlah     persen
## 1           Fakultas Ekonomi dan Bisnis    104 0.31515152
## 2                        Fakultas Hukum     98 0.29696970
## 3                Fakultas Ilmu Komputer     29 0.08787879
## 4      Fakultas Ilmu Sosial dan Politik     52 0.15757576
## 5 Fakultas Keguruan dan Ilmu Pendidikan      6 0.01818182
## 6                       Fakultas Teknik     41 0.12424242
\end{verbatim}

Baru kemudian kita bisa menghasilkan perintah \texttt{ggplot} untuk

\pandocbounded{\includegraphics[keepaspectratio]{figures/membuat-diagram-pai-1.pdf}}

\textbf{Interpretasi:} Diagram donat ini menunjukkan proporsi dari setiap fakultas. Terlihat jelas bahwa irisan ``Fakultas Ekonomi dan Bisnis'' dan ``Fakultas Hukum'' mendominasi porsi responden.

\textbf{Penjelasan Sintaks (Grammar of Graphics):}

\begin{itemize}
\tightlist
\item
  \textbf{DATA \& STAT}: Kita menggunakan data \texttt{alasan\_counts} yang sudah diagregasi, lalu menambahkan kolom baru untuk persentase (\texttt{persen}) dan posisi vertikal untuk label (\texttt{posisi\_y\_label}).
\item
  \textbf{GEOM \& MAPPING}: Kita mulai dengan \texttt{geom\_bar(stat\ =\ "identity")} yang membuat diagram batang di mana tinggi batang (\texttt{y}) adalah nilai persentase itu sendiri. \texttt{x=2} adalah trik untuk membuat satu batang tunggal yang akan kita ``lilit''.
\item
  \textbf{KOORDINAT}: \texttt{coord\_polar(theta\ =\ "y")} adalah komponen kunci. Ini mengambil diagram batang dan mengubah sistem koordinatnya dari Kartesius (x,y) menjadi Polar. Sumbu y ``dibengkokkan'' menjadi sebuah lingkaran.
\item
  \texttt{theme\_void()}: Menghilangkan semua elemen tema seperti sumbu, label sumbu, dan latar belakang, yang tidak relevan untuk diagram pai/donat.
\item
  kita bisa menambahkan lubang di tengah pai dengan menambahkan \texttt{xlim(0.5,\ 2.5)} yang merupakan batas dalam dan batas luar dari radius si diagram seperti berikut.
\end{itemize}

\pandocbounded{\includegraphics[keepaspectratio]{figures/membuat-diagram-donat-1.pdf}}

\begin{quote}
⚠️\textbf{Penting}

Komunitas perupa data pada dasarnya menganjurkan kita untuk `menghindari' diagram lingkaran. Hal ini bisa kalian baca di \href{https://www.data-to-viz.com/caveat/pie.html}{laman ini}. Alternatifnya, mereka lebih menyarankan kita untuk menggunakan diagram batang atau diagram lollipop saja
\end{quote}

\subsubsection{\texorpdfstring{Diagram \emph{Treemap}}{Diagram Treemap}}\label{diagram-treemap}

Treemap adalah alternatif lain untuk diagram pai, terutama efektif ketika Anda memiliki banyak kategori. Treemap menampilkan data hierarkis atau bagian-ke-keseluruhan sebagai satu set persegi panjang bersarang. Ukuran setiap persegi panjang sebanding dengan nilainya.

Untuk membuat treemap, kita perlu paket tambahan yaitu \texttt{treemapify}.

Kita akan membuat \emph{treemap} dari alasan mahasiswa memilih tempat tinggal mereka

\begin{verbatim}
##             alasan_singkat jumlah
## 1         Bersama Keluarga    247
## 2                    Murah     16
## 3 Dekat Kampus/lokasi lain     35
## 4        Fasilitas Lengkap      1
## 5               Lain-lain       2
## 6                     Aman     29
\end{verbatim}

\pandocbounded{\includegraphics[keepaspectratio]{figures/treemap-1.pdf}}

\textbf{Penjelasan Sintaks (Grammar of Graphics):}

\begin{itemize}
\tightlist
\item
  \textbf{DATA}: Kita menggunakan data \texttt{alasan\_counts} yang sudah diagregasi.
\item
  \textbf{GEOM \& MAPPING}: Paket \texttt{treemapify} menyediakan \texttt{geom} baru yang terintegrasi dengan \texttt{ggplot2}.

  \begin{enumerate}
  \def\labelenumi{\arabic{enumi}.}
  \tightlist
  \item
    \texttt{geom\_treemap()}: Ini adalah \texttt{geom} utama. Alih-alih \texttt{x} dan \texttt{y}, \textbf{MAPPING} utamanya adalah \texttt{aes(area\ =\ jumlah)}. \texttt{ggplot} akan secara otomatis menghitung tata letak persegi panjang berdasarkan nilai \texttt{jumlah}. Kita juga memetakan \texttt{alasan\_singkat} ke \texttt{fill} untuk warna dan \texttt{label} untuk teks.
  \item
    \texttt{geom\_treemap\_text()}: Ini adalah \texttt{geom} tambahan khusus untuk menempatkan teks di dalam setiap area treemap.
  \end{enumerate}
\item
  \textbf{THEME}: Kita bisa menyembunyikan legenda karena setiap area sudah diberi label secara langsung, sehingga legenda menjadi berlebihan. Caranya adalah mengatur nilai \texttt{bottom\ =} menjadi \texttt{"none"}. Akan tetapi, untuk kasus kita, kita punya area yang terlalu kecil untuk diberi label, sehingga kita tetap tampilkan legenda.
\end{itemize}

\subsubsection{Histogram}\label{histogram}

Histogram digunakan untuk melihat distribusi dari variabel numerik/kontinu, seperti \texttt{Umur}.

\pandocbounded{\includegraphics[keepaspectratio]{figures/histogram-1.pdf}}

\textbf{Penjelasan Sintaks (Grammar of Graphics):}

\begin{itemize}
\tightlist
\item
  \textbf{DATA}: \texttt{ggplot(data.ubl.cleaned)} mendefinisikan dataset.
\item
  \textbf{GEOM}: \texttt{geom\_histogram(...)} menentukan bentuk geometris berupa histogram.
\item
  \textbf{MAPPING}: \texttt{mapping\ =\ aes(x\ =\ Umur)} memetakan variabel numerik \texttt{Umur} ke sumbu x.
\item
  \textbf{STAT}: \texttt{geom\_histogram} memiliki \texttt{stat\ =\ "bin"} sebagai defaultnya. Transformasi statistik ini akan membagi data \texttt{Umur} ke dalam beberapa rentang (bins) yang lebarnya diatur oleh \texttt{binwidth\ =\ 1}, lalu menghitung frekuensi data di setiap rentang tersebut untuk ditampilkan di sumbu y.
\end{itemize}

\subsubsection{\texorpdfstring{\emph{Boxplot}}{Boxplot}}\label{boxplot}

Box plot berguna untuk membandingkan distribusi variabel numerik di antara beberapa grup/kategori. Mari kita bandingkan distribusi jarak tempat tinggal (\texttt{jarak}) untuk setiap jenis kendaraan utama.

\pandocbounded{\includegraphics[keepaspectratio]{figures/box-plot-1.pdf}}

\textbf{Interpretasi:} Box plot ini menunjukkan bahwa mahasiswa yang menggunakan mobil pribadi cenderung memiliki rentang jarak tempat tinggal yang lebih bervariasi dan median yang sedikit lebih tinggi dibandingkan pengguna sepeda motor. Pengguna ojek online memiliki median jarak yang paling rendah di antara kategori lainnya.

\textbf{Penjelasan Sintaks (Grammar of Graphics):}

\begin{itemize}
\item
  \textbf{DATA}: \texttt{ggplot(data.ubl.cleaned)} mendefinisikan dataset.
\item
  \textbf{GEOM}: \texttt{geom\_boxplot(...)} menentukan bentuk geometris berupa diagram kotak.
\item
  \textbf{MAPPING}: \texttt{mapping\ =\ aes(x\ =\ kendaraan.utama,\ y\ =\ \textbackslash{}}jarak.(km)`, fill = kendaraan.utama)\texttt{memetakan\ tiga\ hal:\ variabel\ kategori}kendaraan.utama\texttt{ke\ sumbu\ x,\ variabel\ numerik}jarak.(km)\texttt{ke\ sumbu\ y,\ dan}kendaraan.utama\texttt{ke\ warna\ isian}fill`.
\item
  \textbf{STAT}: \texttt{geom\_boxplot} secara default menggunakan \texttt{stat\_boxplot}, yang menghitung ringkasan lima angka (minimum, Q1, median, Q3, maksimum) untuk setiap grup di sumbu x.
\item
  \textbf{KOORDINAT}: \texttt{coord\_flip()} secara eksplisit mengubah sistem koordinat dengan membalik sumbu x dan y. Ini adalah komponen terpisah yang diterapkan setelah komponen lainnya.
\item
  \texttt{theme(legend.position\ =\ "none")}: Kustomisasi lapisan tema untuk menyembunyikan legenda.
\end{itemize}

\subsubsection{\texorpdfstring{Grafik Pencar \emph{(Scatter Plot)}}{Grafik Pencar (Scatter Plot)}}\label{grafik-pencar-scatter-plot}

Scatter plot ideal untuk melihat hubungan antara dua variabel numerik. Mari kita lihat hubungan antara \texttt{Umur} dan \texttt{jarak} tempat tinggal.

\pandocbounded{\includegraphics[keepaspectratio]{figures/scatter-plot-1.pdf}}

\textbf{Interpretasi:} Grafik ini tidak menunjukkan adanya pola atau hubungan yang jelas antara umur mahasiswa dan jarak tempat tinggal mereka dari kampus. Titik-titik tersebar secara acak.

\textbf{Penjelasan Sintaks (Grammar of Graphics):}

\begin{itemize}
\tightlist
\item
  \textbf{DATA}: \texttt{ggplot(data.ubl.cleaned)} mendefinisikan dataset.
\item
  \textbf{GEOM}: \texttt{geom\_point(...)} menentukan bentuk geometris berupa titik.
\item
  \textbf{MAPPING}: \texttt{mapping\ =\ aes(x\ =\ Umur,\ y\ =\ \textbackslash{}}jarak.(km)`)\texttt{memetakan\ dua\ variabel,}Umur\texttt{ke\ sumbu\ x\ dan}jarak.(km)` ke sumbu y. Setiap baris data akan menjadi satu titik pada plot.
\item
  \texttt{alpha\ =\ 0.6} dan \texttt{color\ =\ "darkblue"} adalah pengaturan properti visual untuk semua titik.
\end{itemize}

Penting untuk menjaga konteks dalam interpretasi. Scatter plot kita menunjukkan bahwa mahasiswa yang lebih tua cenderung tinggal lebih jauh dari kampus. Namun, untuk benar-benar memahami kekuatan hubungan ini, kita perlu menggunakan statistik korelasi (akan dipelajari di modul selanjutnya).

\textbf{Aktivitas Mandiri 2: Histogram dan Boxplot untuk Variabel Numerik {[}STP-3.2, STP-3.3{]}}

Gunakan variabel \texttt{jarak} (jarak tempat tinggal dari kampus):

\begin{enumerate}
\def\labelenumi{\arabic{enumi}.}
\tightlist
\item
  \textbf{Buat histogram} dengan \texttt{binwidth\ =\ 2} dan interpretasikan:

  \begin{itemize}
  \tightlist
  \item
    Apakah distribusi jarak simetris atau menceng (skewed)?
  \item
    Pada rentang jarak berapa frekuensi mahasiswa paling tinggi?
  \item
    Apakah ada pola tertentu dalam distribusi jarak?
  \end{itemize}
\item
  \textbf{Buat boxplot} yang membandingkan \texttt{jarak} berdasarkan \texttt{Tingkat.Semester} dan interpretasikan:

  \begin{itemize}
  \tightlist
  \item
    Tingkat semester mana yang rata-rata (median) tinggal paling jauh dari kampus?
  \item
    Apakah ada outlier? Jika ada, pada tingkat semester mana?
  \item
    Bagaimana variabilitas jarak pada setiap tingkat semester?
  \end{itemize}
\end{enumerate}

\textbf{Aktivitas Mandiri 3: Visualisasi Komprehensif untuk \texttt{biaya.dalam.sepekan} {[}STP-3.1, STP-3.2, STP-3.3, STP-3.4{]}}

\textbf{A. Menghasilkan Grafik {[}STP-3.3{]}:}

\begin{enumerate}
\def\labelenumi{\arabic{enumi}.}
\tightlist
\item
  Buat \textbf{histogram} untuk \texttt{biaya.dalam.sepekan}

  \begin{itemize}
  \tightlist
  \item
    Coba beberapa nilai \texttt{binwidth} (misal: 20, 50, atau 100) dan pilih yang paling informatif
  \item
    Tambahkan judul dan label sumbu yang jelas
  \end{itemize}
\item
  Buat \textbf{boxplot} untuk \texttt{biaya.dalam.sepekan} berdasarkan \texttt{kendaraan.utama}

  \begin{itemize}
  \tightlist
  \item
    Gunakan \texttt{geom\_boxplot()} dengan \texttt{fill} berdasarkan kendaraan
  \item
    Tambahkan label yang jelas
  \end{itemize}
\item
  Buat \textbf{scatter plot} untuk \texttt{Umur} vs \texttt{biaya.dalam.sepekan}

  \begin{itemize}
  \tightlist
  \item
    Tambahkan \texttt{geom\_point()} dengan \texttt{alpha\ =\ 0.5} untuk transparansi
  \item
    Pertimbangkan menambahkan \texttt{geom\_smooth(method\ =\ "lm")} untuk melihat trend
  \end{itemize}
\end{enumerate}

\textbf{B. Pemilihan Visualisasi dan Tingkat Pengukuran {[}STP-3.1, STP-3.4{]}:}

\begin{enumerate}
\def\labelenumi{\arabic{enumi}.}
\setcounter{enumi}{3}
\tightlist
\item
  \textbf{Mengapa histogram cocok untuk \texttt{biaya.dalam.sepekan}?}

  \begin{itemize}
  \tightlist
  \item
    Jelaskan kaitannya dengan tingkat pengukuran variabel (metrik/rasio)
  \item
    Informasi apa yang bisa diperoleh dari histogram? (distribusi, spread, outlier)
  \end{itemize}
\item
  \textbf{Apakah scatter plot cocok untuk \texttt{Jenis.Kelamin} vs \texttt{Umur}? Mengapa tidak?}

  \begin{itemize}
  \tightlist
  \item
    Jelaskan kaitannya dengan tingkat pengukuran variabel
  \item
    Diagram apa yang lebih sesuai untuk membandingkan umur berdasarkan jenis kelamin?
  \end{itemize}
\end{enumerate}

\textbf{C. Interpretasi Mendalam {[}STP-3.2{]}:}

\begin{enumerate}
\def\labelenumi{\arabic{enumi}.}
\setcounter{enumi}{5}
\tightlist
\item
  \textbf{Dari histogram no.1:}

  \begin{itemize}
  \tightlist
  \item
    Berapa rentang biaya yang paling sering muncul (modus)?
  \item
    Apakah distribusinya simetris, menceng kanan, atau menceng kiri?
  \item
    Apakah ada outlier (nilai ekstrem)?
  \end{itemize}
\item
  \textbf{Dari boxplot no.2:}

  \begin{itemize}
  \tightlist
  \item
    Kendaraan apa yang memiliki median biaya tertinggi?
  \item
    Kendaraan apa yang paling bervariasi biayanya (IQR terbesar)?
  \item
    Apakah ada outlier? Pada jenis kendaraan apa?
  \end{itemize}
\item
  \textbf{Dari scatter plot no.3:}

  \begin{itemize}
  \tightlist
  \item
    Apakah ada pola hubungan antara umur dan biaya transportasi?
  \item
    Jika ada trend line, apakah slopenya positif atau negatif?
  \item
    Apa interpretasi kontekstualnya?
  \end{itemize}
\end{enumerate}

\textbf{D. Dokumentasi {[}STP-3.3{]}:}

\begin{enumerate}
\def\labelenumi{\arabic{enumi}.}
\setcounter{enumi}{8}
\tightlist
\item
  Kumpulkan file modul ini dengan:

  \begin{itemize}
  \tightlist
  \item
    Seluruh kode diagram yang sudah Anda buat di modul ini
  \item
    Seluruh interpretasi dan analisis untuk setiap diagram
  \item
    Screenshot atau output grafik yang sudah dihasilkan
  \end{itemize}
\end{enumerate}

\begin{center}\rule{0.5\linewidth}{0.5pt}\end{center}

\section{Modul-4: Distribusi Sampling dan Interval Kepercayaan}\label{modul-4-distribusi-sampling-dan-interval-kepercayaan}

Setelah mempelajari modul ini, Anda diharapkan dapat:

\begin{enumerate}
\def\labelenumi{\arabic{enumi}.}
\tightlist
\item
  menghasilkan distribusi statistik sampel dan menghitung standard error-nya \capaian{STP-4.3}
\item
  menghasilkan interval kepercayaan dengan menggunaan perangkat lunak komputer \capaian{STP-5.2}
\end{enumerate}

\begin{center}\rule{0.5\linewidth}{0.5pt}\end{center}

\subsection{Pendahuluan}\label{pendahuluan}

Dalam statistika inferensial, kita seringkali ingin mengetahui karakteristik dari sebuah populasi (misalnya, rata-rata uang saku seluruh mahasiswa di Indonesia). Namun, mengumpulkan data dari seluruh populasi seringkali tidak memungkinkan. Sebagai solusinya, kita mengambil sampel acak dari populasi tersebut dan menggunakan statistik dari sampel (seperti rata-rata sampel) untuk menduga parameter populasi.

Modul ini akan membahas dua konsep fundamental dalam statistika inferensial: \textbf{Distribusi Sampling} dan \textbf{Interval Kepercayaan} untuk rata-rata dan proporsi. Kita akan menggunakan studi kasus data jarak tempuh mahasiswa dari empat universitas di Bandar Lampung dan sekitarnya untuk memahami bagaimana konsep ini diterapkan dalam praktik.

\subsection{Perangkat Lunak dan Pustaka (Libraries)}\label{perangkat-lunak-dan-pustaka-libraries}

Pastikan Anda telah menginstal pustaka \texttt{tidyverse} yang akan sangat membantu dalam proses manipulasi dan visualisasi data.

\subsection{Memuat dan Mempersiapkan Data}\label{memuat-dan-mempersiapkan-data}

Langkah pertama adalah memuat data survei mahasiswa dari empat universitas. Kita akan menggabungkan dan membersihkan data tersebut untuk analisis. Variabel yang akan menjadi fokus kita adalah \textbf{jarak tempuh (km)} dan \textbf{jenis tempat tinggal}.

\begin{verbatim}
## # A tibble: 6 x 4
##   kampus jarak_km jenis_tinggal                 tipe_tinggal_baku     
##   <chr>     <dbl> <chr>                         <chr>                 
## 1 UINRIL    19.3  Rumah bersama saudara         Rumah Keluarga/Pribadi
## 2 UINRIL     0.58 Kos sendiri                   Kos/Asrama            
## 3 UINRIL     0.56 Kos sendiri                   Kos/Asrama            
## 4 UINRIL     1.05 Kos sendiri                   Kos/Asrama            
## 5 UINRIL     1.69 Rumah mengontrak bersama-sama Rumah Keluarga/Pribadi
## 6 UINRIL     7.91 Rumah pribadi/rumah keluarga  Rumah Keluarga/Pribadi
\end{verbatim}

\begin{verbatim}
##     kampus             jarak_km        jenis_tinggal      tipe_tinggal_baku 
##  Length:1600        Min.   : 0.06352   Length:1600        Length:1600       
##  Class :character   1st Qu.: 1.31127   Class :character   Class :character  
##  Mode  :character   Median : 2.90884   Mode  :character   Mode  :character  
##                     Mean   : 3.62552                                        
##                     3rd Qu.: 5.14000                                        
##                     Max.   :43.54519
\end{verbatim}

\begin{verbatim}
## 
##             Kos/Asrama Rumah Keluarga/Pribadi 
##                    639                    961
\end{verbatim}

\begin{quote}
\textbf{Penjelasan \texttt{case\_when()}}:

Fungsi \texttt{case\_when()} (bagian dari \texttt{dplyr}) sangat berguna untuk membuat variabel baru berdasarkan serangkaian aturan kondisional, mirip seperti pernyataan \texttt{IF-ELSE\ IF-ELSE}. Di dalam \texttt{mutate()}, kita membuat kolom baru bernama \texttt{tipe\_tinggal\_baku}.

\begin{itemize}
\item
  \textbf{Aturan pertama}: \texttt{str\_detect(tolower(jenis\_tinggal),\ "kos\textbar{}asrama\textbar{}rusunawa")\ \textasciitilde{}\ "Kos/Asrama"}

  \begin{itemize}
  \item
    \texttt{tolower(jenis\_tinggal)}: Mengubah semua teks di kolom \texttt{jenis\_tinggal} menjadi huruf kecil agar pencarian tidak sensitif terhadap huruf besar/kecil.
  \item
    \texttt{str\_detect(...)}: Fungsi ini memeriksa apakah sebuah teks mengandung pola tertentu.
  \item
    \texttt{"kos\textbar{}asrama\textbar{}rusunawa"}: Ini adalah polanya. Tanda \texttt{\textbar{}} berarti ``ATAU''. Jadi, fungsi ini mencari kata ``kos'' ATAU ``asrama'' ATAU ``rusunawa''.
  \item
    \texttt{\textasciitilde{}\ "Kos/Asrama"}: Jika salah satu kata kunci ditemukan, maka nilai untuk kolom \texttt{tipe\_tinggal\_baku} adalah ``Kos/Asrama''.
  \end{itemize}
\item
  \textbf{Aturan kedua}: \texttt{TRUE\ \textasciitilde{}\ "Rumah\ Keluarga/Pribadi"}

  \begin{itemize}
  \item
    \texttt{TRUE}: Ini adalah kondisi ``penampung'' atau \emph{default}. Jika tidak ada kondisi sebelumnya yang terpenuhi, aturan ini akan dijalankan.
  \item
    \texttt{\textasciitilde{}\ "Rumah\ Keluarga/Pribadi"}: Memberi nilai ``Rumah Keluarga/Pribadi'' untuk semua baris yang tidak cocok dengan aturan pertama.
  \end{itemize}
\end{itemize}

Singkatnya, kode ini membakukan data \texttt{jenis\_tinggal} yang bervariasi menjadi dua kategori yang bersih dan konsisten untuk analisis proporsi.
\end{quote}

Pada tahap ini, kita menganggap data gabungan dari 1.206 mahasiswa sebagai \textbf{``populasi''} kita untuk tujuan simulasi.

\subsection{\texorpdfstring{Distribusi \emph{Sampling} dan \emph{Standard Error}}{Distribusi Sampling dan Standard Error}}\label{distribusi-sampling-dan-standard-error}

Distribusi \emph{sampling} adalah distribusi dari suatu statistik (misalnya, rata-rata) yang dihitung dari semua kemungkinan sampel dengan ukuran yang sama yang diambil dari sebuah populasi. Ini adalah istilah lain dari \textbf{distribusi statistik sampel}.

Teorema Limit Pusat (\emph{Central Limit Theorem}) menyatakan bahwa jika kita mengambil sampel yang cukup besar, distribusi \emph{sampling} dari rata-rata akan mendekati distribusi normal.

\subsubsection{Mendefinisikan Parameter Populasi}\label{mendefinisikan-parameter-populasi}

Pertama, mari kita hitung rata-rata dan standar deviasi ``populasi'' kita sebagai acuan.

\begin{verbatim}
## [1] "Rata-rata Jarak Populasi (μ): 3.63 km"
\end{verbatim}

\begin{verbatim}
## [1] "Standar Deviasi Populasi (σ): 3.35 km"
\end{verbatim}

\subsubsection{Simulasi Pengambilan Sampel}\label{simulasi-pengambilan-sampel}

Sekarang, kita akan mensimulasikan proses pengambilan sampel secara berulang. Kita akan mengambil 1000 sampel acak, masing-masing berukuran 50 mahasiswa (\texttt{n=50}), lalu menghitung rata-rata jarak untuk setiap sampel.

\begin{verbatim}
##   rataan_sampel
## 1      3.393546
## 2      4.115982
## 3      3.263581
## 4      4.171112
## 5      3.345113
## 6      3.373455
\end{verbatim}

\textbf{Aktivitas Mandiri 1: Simulasi dengan Parameter Berbeda {[}STP-4.3{]}}

Modifikasi kode simulasi di atas:
1. Ubah \texttt{ukuran\_sampel} menjadi 30 (dari 50)
2. Ubah \texttt{jumlah\_simulasi} menjadi 500 (dari 1000)
3. Hitung ulang SE empiris dan teoritis
4. Buat histogram distribusi sampling yang baru
5. Bandingkan:
- Bagaimana bentuk distribusi berubah dengan n=30 vs n=50?
- Apakah SE empiris dan teoritis masih dekat?

\subsubsection{Visualisasi Distribusi Sampling}\label{visualisasi-distribusi-sampling}

Kumpulan dari 1000 rata-rata sampel inilah yang membentuk \textbf{distribusi sampling}. Mari kita visualisasikan dalam bentuk histogram.

\pandocbounded{\includegraphics[keepaspectratio]{figures/mod04-visualize-sampling-distribution-1.pdf}}

Perhatikan bagaimana distribusi dari rata-rata sampel berbentuk seperti lonceng (mendekati normal) dan berpusat di sekitar rata-rata populasi (garis biru putus-putus).

\subsubsection{\texorpdfstring{Menghitung \emph{Standard Error} (SE)}{Menghitung Standard Error (SE)}}\label{menghitung-standard-error-se}

\emph{Standard Error} (SE) adalah standar deviasi dari distribusi sampling. Ini mengukur seberapa besar variasi rata-rata sampel di sekitar rata-rata populasi. SE dapat dipahami dari dua perspektif: satu dari sisi teori dalam simulasi, dan satu lagi dari sisi praktik ketika kita hanya memiliki satu sampel.

\begin{enumerate}
\def\labelenumi{\arabic{enumi}.}
\item
  \textbf{Empiris (dari simulasi)}: Menghitung standar deviasi dari 1000 rata-rata sampel yang kita hasilkan.
\item
  \textbf{Teoritis (dari populasi)}: Menggunakan rumus \(SE=\frac{\sigma}{\sqrt{n}}\), di mana \(σ\) adalah standar deviasi populasi dan \(n\) adalah ukuran sampel.
\item
  \textbf{Estimasi (dari sampel)}: Dalam praktik statistika inferensial, kita hampir tidak pernah mengetahui . Oleh karena itu, kita mengestimasinya menggunakan standar deviasi dari sampel kita sendiri (s). Rumusnya menjadi \(SE=\frac{s}{\sqrt{n}}\). \textbf{Inilah nilai yang sebenarnya digunakan saat kita membuat interval kepercayaan atau melakukan uji hipotesis dari data sampel nyata.}
\end{enumerate}

\begin{verbatim}
## [1] "Standard Error (Empiris, dari simulasi): 0.452"
\end{verbatim}

\begin{verbatim}
## [1] "Standard Error (Teoritis, dari populasi): 0.474"
\end{verbatim}

Mari kita hitung estimasi standard error dari satu sampel acak yang akan kita gunakan nanti di Bagian 4.5.

\begin{verbatim}
## [1] "Estimasi Standard Error (dari satu sampel): 0.307"
\end{verbatim}

Perhatikan bahwa nilai SE Empiris dan Teoritis sangat dekat karena berasal dari simulasi di mana parameter populasi diketahui. Nilai SE Estimasi akan bervariasi tergantung sampel mana yang kita ambil, namun nilai inilah yang paling realistis untuk digunakan dalam analisis nyata.

\subsection{Estimasi Parameter-1: Interval Kepercayaan Rata-rata}\label{estimasi-parameter-1-interval-kepercayaan-rata-rata}

Untuk bagian ini kita akan menggunakan \emph{library} bernama \texttt{MKinfer}yang berguna untuk menghasilkan interval kepercayaan sebagai estimasi rentang.

Lakukan instalasi \emph{library} ini dengan perintah berikut.

Kemudian muat paket \texttt{MKinfer} tersebut.

\subsubsection{Mengambil Satu Sampel}\label{mengambil-satu-sampel}

\begin{verbatim}
## Rata-rata Sampel Tunggal: 3.46 km
\end{verbatim}

\begin{verbatim}
## Standar Deviasi Sampel Tunggal: 2.17 km
\end{verbatim}

\subsubsection{Menghasilkan Interval Kepercayaan}\label{menghasilkan-interval-kepercayaan}

\begin{verbatim}
## 
##  Exact confidence interval(s)
## 
## 95 percent confidence interval:
##         2.5 %   97.5 %
## mean 2.845641 4.081422
## 
## sample estimates:
##     mean       sd 
## 3.463532 2.174163 
## 
## additional information:
## SE of mean 
##   0.307473
\end{verbatim}

Hasil dari fungsi \texttt{meanCI} tersebut langsung menunjukkan rentang kepercayaan yang limit bawahnya ditunjukkan oleh angka di bawah \texttt{2.5\%}dan limit atasnya oleh angka di bawah \texttt{97.5\%}. Nilai-nilai ini adalah nilai \(\alpha/2\) yang dibagi dua ke kiri dan kanan grafik.

\textbf{Pertanyaan}

Berapakah rentang kepercayaan untuk rata-rata jarak tempat tinggal dari kampus dari sampel kita? Tuliskan interpretasinya yang tepat. Bandingkan dengan rata-rata parameter

\begin{verbatim}
## [1] 3.625515
\end{verbatim}

\subsection{Estimasi Parameter-2: Interval Kepercayaan Proporsi}\label{estimasi-parameter-2-interval-kepercayaan-proporsi}

Selain rata-rata, kita juga sering tertarik untuk mengestimasi proporsi dari suatu kategori dalam populasi. Contohnya, berapa persentase mahasiswa yang tinggal di kos/asrama? Prosesnya mirip: kita mengambil sampel, menghitung proporsi sampel, lalu membangun interval kepercayaan di sekitar proporsi tersebut.

\subsubsection{Parameter Proporsi}\label{parameter-proporsi}

Pertama, mari kita hitung proporsi ``populasi'' yang sebenarnya dari data kita. Kita akan mencari proporsi mahasiswa yang tinggal di ``Kos/Asrama''.

\begin{verbatim}
## Tabel Frekuensi Jenis Tinggal di Populasi:
\end{verbatim}

\begin{verbatim}
## 
##             Kos/Asrama Rumah Keluarga/Pribadi 
##                    639                    961
\end{verbatim}

\begin{verbatim}
## 
## Proporsi Mahasiswa di Kos/Asrama (p): 0.3994
\end{verbatim}

\subsubsection{Mengambil Satu Sampel dan Menghitung Proporsinya}\label{mengambil-satu-sampel-dan-menghitung-proporsinya}

Sekarang, kita ambil satu sampel acak (misalnya, \texttt{n=100}) dan hitung proporsi sampel \(\hat{p}\) mahasiswa yang tinggal di kos/asrama.

\begin{verbatim}
## Tabel Frekuensi Jenis Tinggal di Sampel:
\end{verbatim}

\begin{verbatim}
## sampel_prop
##             Kos/Asrama Rumah Keluarga/Pribadi 
##                     36                     64
\end{verbatim}

\begin{verbatim}
## 
## Proporsi Sampel Mahasiswa di Kos/Asrama (p-hat): 0.36
\end{verbatim}

\subsubsection{Menghasilkan Interval Kepercayaan untuk Proporsi}\label{menghasilkan-interval-kepercayaan-untuk-proporsi}

Kita akan menggunakan fungsi \texttt{binomCI()} di R. Fungsi ini memerlukan jumlah ``sukses'', yakni jumlah mahasiswa di kos/asrama, dan total ukuran sampel.

\begin{verbatim}
## 
##  wilson confidence interval
## 
## 95 percent confidence interval:
##          2.5 %   97.5 %
## prob 0.2727122 0.457646
## 
## sample estimate:
## prob 
## 0.36 
## 
## additional information:
## standard error of prob 
##             0.04717785
\end{verbatim}

\textbf{Pertanyaan}

Berapakah rentang kepercayaan untuk proporsi mahasiswa yang tinggal di kos/asrama? Tuliskan interpretasinya yang tepat. Bandingkan dengan proporsi parameter.

\textbf{Aktivitas Mandiri 2: Interval Kepercayaan 99\% {[}STP-5.2{]}}

Untuk proporsi mahasiswa di Kos/Asrama:
1. Hitung interval kepercayaan 99\% (\texttt{conf.level\ =\ 0.99})
2. Bandingkan dengan hasil 95\%
3. Mana yang lebih lebar? Mengapa?

Untuk rata-rata jarak:
1. Ambil sampel baru dengan \texttt{ukuran\_sampel\ =\ 75}
2. Hitung CI 90\%
3. Interpretasikan hasilnya

\textbf{Aktivitas Mandiri 3: Interval Kepercayaan 97\% {[}STP-4.3, STP-5.2{]}}

\textbf{Petunjuk:} Gunakan \texttt{conf.level\ =\ 0.97}

\begin{enumerate}
\def\labelenumi{\arabic{enumi}.}
\tightlist
\item
  \textbf{Rata-rata Jarak:}

  \begin{itemize}
  \tightlist
  \item
    Hitung CI 97\% untuk rata-rata jarak
  \item
    Interpretasi: ``Kita 97\% yakin bahwa rata-rata jarak populasi berada di rentang\ldots{}''
  \end{itemize}
\item
  \textbf{Proporsi Penghuni Kos/Asrama:}

  \begin{itemize}
  \tightlist
  \item
    Hitung CI 97\% untuk proporsi
  \end{itemize}
\item
  \textbf{Analisis Perbandingan:}

  \begin{itemize}
  \tightlist
  \item
    Bandingkan CI 97\%, 95\%, dan 90\%
  \item
    Jelaskan trade-off antara lebar interval dan tingkat kepercayaan
  \end{itemize}
\end{enumerate}

\begin{center}\rule{0.5\linewidth}{0.5pt}\end{center}

\section{Modul-5: Uji Hipotesis}\label{modul-5-uji-hipotesis}

Setelah mempelajari modul ini, Anda diharapkan dapat:

\begin{enumerate}
\def\labelenumi{\arabic{enumi}.}
\tightlist
\item
  mampu menghasilkan uji hipotesis satu populasi dengan menggunakan perangkat lunak komputer \capaian{STP-6.2}
\item
  mampu menghasilkan uji hipotesis dua atau lebih populasi pada suatu kasus menggunakan perangkat lunak komputer \capaian{STP-7.2}
\end{enumerate}

\begin{center}\rule{0.5\linewidth}{0.5pt}\end{center}

\subsection{Pendahuluan}\label{pendahuluan-1}

Uji hipotesis adalah salah satu teknik dalam analisis statistik inferensial yang memperkirakan parameter melalui pernyataan-pernyataan dugaan atau \textbf{hipotesis}. Hipotesis di sini adalah pernyataan yang mengandung dugaan bahwa nilai parameter adalah sama dengan suatu nilai atau berbeda. Hipotesis yang sama dengan suatu nilai kita sebut dengan \textbf{hipotesis kosong \emph{(null hypothesis)}}, sementara yang berbeda disebut \textbf{hipotesis alternatif \emph{(alternative hypothesis)}}

\subsection{Perangkat Lunak dan Pustaka (Libraries)}\label{perangkat-lunak-dan-pustaka-libraries-1}

Seperti biasa, kita perlu memuat pustaka \emph{(libraries)} yang diperlukan dalam pengolahan data kita.

Dalam modul ini kita akan menggunakan sebuah pustaka bernama \texttt{stats} yang merupakan pustaka khusus untuk perhitungan-perhitungan statistik dan juga penghasil angka acak.

Biasanya pustaka \texttt{stats} sudah termuat secara bawaan \emph{(default)} saat kita menjalankan R melalui RStudio. Untuk mengecek apakah suatu pustaka sudah termuat ketika kita menjalankan R, tulis perintah berikut.

\begin{Shaded}
\begin{Highlighting}[]
\StringTok{"package:\{nama pustaka\}"} \SpecialCharTok{\%in\%} \FunctionTok{search}\NormalTok{() }\CommentTok{\# ganti \{nama pustaka\} dengan nama pustaka yang ingin dicari}

\StringTok{"package: stats"} \SpecialCharTok{\%in\%} \FunctionTok{search}\NormalTok{() }\CommentTok{\# untuk mengecek apakah \textquotesingle{}stats\textquotesingle{} sudah dimuat}
\end{Highlighting}
\end{Shaded}

\subsection{Memuat Data}\label{memuat-data}

\begin{Shaded}
\begin{Highlighting}[]
\NormalTok{data\_mahasiswa }\OtherTok{\textless{}{-}} \FunctionTok{read\_csv2}\NormalTok{(}\StringTok{"data\_mahasiswa.csv"}\NormalTok{)}
\CommentTok{\# read\_csv2() → untuk file CSV dgn pemisah titik koma (;)}
\CommentTok{\# read\_csv() → untuk file CSV dgn pemisah koma (,).}
\end{Highlighting}
\end{Shaded}

\subsection{Perkenalan Markdown: LaTeX}\label{perkenalan-markdown-latex}

Dalam file Markdown terdapat fitur bahasa LaTeX. Bahasa ini digunakan untuk menulis simbol matematis.

Untuk menyisipkan persamaan matematis menggunakan LaTeX, kita dapat menggunakan perintah Insert \textgreater{} LaTeX Math \textgreater{} Inline Math/Display Math. Inline Math akan menyisipkan simbol matematis di dalam paragraf, sementara Display Math membuat persamaan di bagian terpisah dari paragraf.

Dalam mode Source, Inline Math disisipkan dengan mengetikkan \texttt{\$\$} terlebih dahulu dan menuliskan simbolnya di antara dua \texttt{\$} tersebut. Sementara itu, untuk Display Math kita perlu mengetikkan empat buah tanda \texttt{\$} sehingga terlihat seperti berikut.

\begin{Shaded}
\begin{Highlighting}[]
\NormalTok{$$}
\NormalTok{H\_0 : \textbackslash{}mu\_0 = 4}
\NormalTok{\textbackslash{}newline H\_1 : \textbackslash{}mu\_0 \textbackslash{}ne 4}
\NormalTok{$$}
\end{Highlighting}
\end{Shaded}

Hasilnya adalah seperti berikut:

\[
H_0 : \mu_0 = 4\newline H_1 : \mu_0 \ne 4
\]

Anda dapat mempelajari penulisan LaTeX selengkapnya \href{https://www.overleaf.com/learn/latex/Mathematical_expressions\#Further_reading}{di sini}.

\subsection{Uji Hipotesis Satu Populasi}\label{uji-hipotesis-satu-populasi}

Pengujian hipotesis satu populasi bermakna pengujian hipotesis untuk suatu parameter yang berasal dari satu buah populasi saja. Misalnya, dalam kasus modul ini, kita memiliki populasi berupa seluruh mahasiswa yang berkuliah di Kota Bandar Lampung dan sekitarnya.

Dalam uji hipotesis satu populasi, kita hanya berfokus pada satu kelompok data, dan ingin mengetahui apakah parameter populasi sama dengan nilai tertentu atau tidak. Nilai acuan ini biasanya berasal dari teori, standar, atau dugaan awal. Ada dua jenis parameter yang bisa diuji:

\textbf{Rata-rata} (µ) → misalnya, apakah rata-rata jarak mahasiswa ke kampus = 4 km?

\textbf{Proporsi} (p) → misalnya, apakah proporsi mahasiswa yang tinggal di kos/asrama = 50\%?

Prinsipnya sama: kita rumuskan hipotesis nol (H0), tentukan hipotesis alternatif (H1), hitung statistik uji (Z atau t), tentukan daerah kritis, lalu disimpulkan.

Penjelasan lebih lanjut akan dibahas dalam subbab-subbab berikut.

\subsubsection{Rata-rata}\label{rata-rata}

Dalam kasus ini, kita akan menguji hipotesis parameter rata-rata berupa jarak mahasiswa yang berkuliah di universitas-universitas di Kota Bandar Lampung dan sekitarnya. Berdasarkan \texttt{data\_mahasiswa} , kita dapat menghitung rata-rata statistik jarak mahasiswa yang menjadi responden. Rata-rata statistik jarak tersebut kita gunakan untuk menguji hipotesis parameter rata-rata dengan suatu nilai.

Jika dimisalkan suatu nilai tersebut adalah 4 km, maka hipotesis yang diujinya menjadi ``rata-rata jarak tempat tinggal mahasiswa ke kampusnya masing-masing di Kota Bandar Lampung dan sekitarnya adalah 4 km.'' Ini disebut \textbf{hipotesis kosong}, karena memuat kesamaan terhadap suatu nilai. Sementara itu, lawannya, yakni \textbf{hipotesis alternatif}, akan dibahas di masing-masing bagian.

\textbf{Membuat fungsi perhitungan statistik uji}

Pertama, mari kita buat sebuah fungsi di R untuk menghitung nilai statistik uji. Dengan adanya fungsi ini, kita tidak perlu lagi melakukan perhitungan manual setiap kali menguji hipotesis. Fungsi ini akan menerima masukan berupa rata-rata sampel, rata-rata hipotesis, simpangan baku, dan ukuran sampel, kemudian menghasilkan nilai statistik uji Z.

\textbf{Mendeklarasikan variabel uji}

Setelah fungsi dibuat, kita deklarasikan variabel-variabel yang akan digunakan. Variabel-variabel ini akan dipakai langsung sebagai input fungsi uji hipotesis. Dengan mendeklarasikannya lebih dulu, kita memastikan semua nilai yang diperlukan untuk perhitungan sudah tersedia dengan jelas dan proses penghitungan menjadi lebih rapi.

\begin{verbatim}
## Rata-rata jarak sampel adalah 3.90177 km. 
## Parameter rata-rata jarak yang diuji adalah 4 km. 
## Simpangan baku jarak sampel adalah 3.534776 km. 
## Ukuran sampel adalah 1206 orang.
\end{verbatim}

\subsubsection{Memahami Distribusi Normal dan Area Kritis}\label{memahami-distribusi-normal-dan-area-kritis}

Dalam uji hipotesis, kita menggunakan distribusi normal standar (kurva Z) untuk menentukan apakah statistik uji kita jatuh pada area ``tidak biasa'' (area kritis/penolakan) atau area ``biasa'' (area penerimaan H₀).

\paragraph{Visualisasi Two-Tailed Test (α = 0.05)}\label{visualisasi-two-tailed-test-ux3b1-0.05}

Untuk memahami konsep area kritis, perhatikan Gambar \ref{fig:dist-normal-two-tailed}. Gambar tersebut menunjukkan bahwa untuk uji two-tailed dengan α = 0.05, area kritis terbagi menjadi dua bagian (masing-masing 2.5\%) di kedua ekor distribusi.

\begin{figure}

{\centering \includegraphics{figures/mod05-visualize-two-tailed-1} 

}

\caption{Distribusi Normal dengan Area Kritis Two-Tailed (α = 0.05)}\label{fig:mod05-visualize-two-tailed}
\end{figure}

\textbf{Penjelasan:}

\begin{itemize}
\tightlist
\item
  Area biru (tengah): Area penerimaan H₀ sebesar 95\%
\item
  Area merah (kiri \& kanan): Area kritis total 5\% (2.5\% + 2.5\%)
\item
  Jika \textbar Z hitung\textbar{} \textgreater{} 1.96 → Tolak H₀ (jatuh di area merah)
\item
  Fungsi \texttt{qnorm(0.025)} = -1.96 dan \texttt{qnorm(0.975)} = +1.96
\end{itemize}

\paragraph{Visualisasi Left-Tailed Test (α = 0.05)}\label{visualisasi-left-tailed-test-ux3b1-0.05}

\begin{figure}

{\centering \includegraphics{figures/mod05-visualize-left-tailed-1} 

}

\caption{Distribusi Normal dengan Area Kritis Left-Tailed (α = 0.05)}\label{fig:mod05-visualize-left-tailed}
\end{figure}

\textbf{Catatan:} Untuk left-tailed test, seluruh α (5\%) berada di ekor kiri. Z kritis = \texttt{qnorm(0.05)} = -1.64

\paragraph{Visualisasi Right-Tailed Test (α = 0.05)}\label{visualisasi-right-tailed-test-ux3b1-0.05}

\begin{figure}

{\centering \includegraphics{figures/mod05-visualize-right-tailed-1} 

}

\caption{Distribusi Normal dengan Area Kritis Right-Tailed (α = 0.05)}\label{fig:mod05-visualize-right-tailed}
\end{figure}

\textbf{Catatan:} Untuk right-tailed test, seluruh α (5\%) berada di ekor kanan. Z kritis = \texttt{qnorm(0.95)} = 1.64

\paragraph{\texorpdfstring{Hubungan dengan Fungsi \texttt{qnorm()}}{Hubungan dengan Fungsi qnorm()}}\label{hubungan-dengan-fungsi-qnorm}

Tiga visualisasi di atas menunjukkan bagaimana fungsi \texttt{qnorm()} bekerja:

\begin{itemize}
\tightlist
\item
  \textbf{Two-tailed:} \texttt{qnorm(alpha/2)} dan \texttt{qnorm(1\ -\ alpha/2)} → ±1.96
\item
  \textbf{Left-tailed:} \texttt{qnorm(alpha)} → -1.64
\item
  \textbf{Right-tailed:} \texttt{qnorm(1\ -\ alpha)} → +1.64
\end{itemize}

\textbf{Interpretasi Keputusan:}
- Jika Z hitung jatuh di area merah → \textbf{Tolak H₀}
- Jika Z hitung jatuh di area biru → \textbf{Gagal tolak H₀}

\paragraph{\texorpdfstring{\emph{Two-tailed Test} (Tidak Berarah)}{Two-tailed Test (Tidak Berarah)}}\label{two-tailed-test-tidak-berarah}

Uji dua arah (two-tailed test) adalah salah satu bentuk pengujian hipotesis statistik yang digunakan ketika peneliti ingin mengetahui apakah nilai rata-rata populasi berbeda secara signifikan dari suatu nilai tertentu (nilai hipotesis, biasanya dilambangkan μ₀), tanpa memandang arah perbedaan tersebut. Artinya, kita tidak hanya mempertanyakan apakah rata-rata lebih besar atau lebih kecil, tetapi fokus pada apakah terdapat perbedaan yang cukup signifikan ke salah satu arah. Dalam uji ini, kita ingin mengetahui apakah rata-rata jarak tempuh mahasiswa menuju kampus berbeda secara signifikan dari 4 km, tanpa menentukan apakah lebih besar atau lebih kecil.

\subparagraph{Merumuskan hipotesis}\label{merumuskan-hipotesis}

Langkah pertama adalah merumuskan hipotesis. Di sini kita mendefinisikan dua kemungkinan: hipotesis nol (H0) dan hipotesis alternatif (H1). H0 biasanya menyatakan bahwa ``tidak ada perbedaan'' atau ``nilai parameter sama dengan nilai tertentu''. Sebaliknya, H1 menyatakan adanya perbedaan yang signifikan. Pada uji ini, Hipotesis nol menyatakan bahwa rata-rata sama dengan 4 km, sedangkan hipotesis alternatif menyatakan bahwa rata-rata tidak sama dengan 4 km. Dengan demikian, kita sedang menguji apakah terdapat perbedaan nyata dari nilai acuan tersebut.

\begin{verbatim}
## H0 : rata-rata jarak = 4 km 
## 
## H1 : rata-rata jarak != 4 km
\end{verbatim}

\subparagraph{\texorpdfstring{Memilih distribusi \emph{sampling}, wilayah dan titik kritis}{Memilih distribusi sampling, wilayah dan titik kritis}}\label{memilih-distribusi-sampling-wilayah-dan-titik-kritis}

Setelah hipotesis dirumuskan, kita perlu menentukan distribusi sampling dan titik kritis. Distribusi sampling adalah distribusi dari nilai rata-rata sampel jika kita mengambil banyak sampel dari populasi. Titik kritis adalah nilai ambang yang menentukan wilayah penolakan hipotesis nol. Dalam uji dua arah, daerah kritis terletak di kedua ujung distribusi normal dengan luas masing-masing sebesar α/2. Misalnya, dengan α = 0,05, maka masing-masing sisi memiliki 0,025. Artinya, hanya jika nilai uji berada jauh di salah satu sisi, kita akan menolak H0.

\begin{verbatim}
## Titik kritis uji two-tailed: 1.959964
\end{verbatim}

\subparagraph{Menghitung statistik uji}\label{menghitung-statistik-uji}

Setelah menentukan titik kritis, kita menghitung nilai statistik uji menggunakan data yang kita miliki. Nilai ini menunjukkan seberapa jauh rata-rata sampel menyimpang dari nilai hipotesis populasi, dalam satuan standar error. Jika nilai ini berada di luar batas kritis, maka penyimpangan tersebut dianggap terlalu besar untuk hanya terjadi karena kebetulan.

\begin{verbatim}
## Hasil statistik uji rata-rata: 0.9650591
\end{verbatim}

\subparagraph{Menarik kesimpulan}\label{menarik-kesimpulan}

Langkah terakhir adalah menarik kesimpulan berdasarkan perbandingan antara nilai statistik uji dan titik kritis. Jika nilai statistik uji (absolut) lebih kecil dari titik kritis, maka kita gagal menolak H₀, artinya perbedaan yang terlihat kemungkinan hanya kebetulan sampel. Jika lebih besar atau sama, kita menolak H₀, artinya ada bukti yang cukup bahwa rata-rata jarak tidak sama dengan 4.

\begin{verbatim}
## Hipotesis kosong gagal ditolak. Perbedaan rata-rata sampel 3.9 dengan 4 hanyalah kebetulan
\end{verbatim}

\paragraph{\texorpdfstring{\emph{One-tailed Test} (Berarah)}{One-tailed Test (Berarah)}}\label{one-tailed-test-berarah}

Uji satu arah (one-tailed test) digunakan ketika kita memiliki dugaan arah perbedaan. Misalnya, kita ingin tahu apakah rata-rata sampel lebih kecil dari 4 km, bukan sekadar berbeda. Dalam konteks ini, daerah kritis hanya berada di salah satu sisi distribusi. Uji satu arah biasanya lebih sensitif karena semua probabilitas α ditempatkan di satu sisi, sehingga lebih mudah untuk mendeteksi perbedaan yang sesuai dengan arah hipotesis alternatif.

\subparagraph{Merumuskan Hipotesis}\label{merumuskan-hipotesis-1}

Untuk uji satu arah, hipotesis alternatif ditentukan dengan arah tertentu. Misalnya, jika kita menduga rata-rata jarak lebih kecil dari 4 km, maka:

Dengan cara ini, kita hanya akan menolak H0 jika bukti menunjukkan bahwa rata-rata memang lebih kecil. Jika ternyata rata-rata lebih besar, maka hal itu tidak mendukung H1.

\begin{verbatim}
## H0 : rata-rata jarak = 4 km 
## 
## H1 : rata-rata jarak < 4 km
\end{verbatim}

\subparagraph{Memilih distribusi sampling, wilayah, serta titik kritis}\label{memilih-distribusi-sampling-wilayah-serta-titik-kritis}

Karena hipotesis alternatif hanya mengarah ke satu sisi, maka daerah kritis hanya ditetapkan di sisi kiri atau kanan distribusi normal. Jika H1 menyatakan rata-rata lebih kecil, maka titik kritis ditentukan pada sisi kiri. Sebaliknya, jika H1 menyatakan rata-rata lebih besar, maka titik kritis ada di sisi kanan. Dengan α = 0,05, seluruh probabilitas kesalahan ditempatkan di satu sisi distribusi.

\begin{verbatim}
## Titik kritis uji one-tailed: -1.644854
\end{verbatim}

\subparagraph{Menghitung statistik uji}\label{menghitung-statistik-uji-1}

Setelah titik kritis ditentukan, kita menghitung nilai statistik uji. Pada uji satu arah, tanda dari nilai Z menjadi sangat penting. Hal ini karena nilai negatif atau positif menunjukkan arah penyimpangan rata-rata sampel dari nilai populasi.

\begin{verbatim}
## Hasil statistik uji rata-rata: -0.9650591
\end{verbatim}

\subparagraph{Menarik kesimpulan}\label{menarik-kesimpulan-1}

Langkah terakhir adalah membandingkan nilai statistik uji dengan titik kritis. Jika nilai uji lebih kecil dari titik kritis atau jatuh ke dalam daerah kritis, maka H0 ditolak, artinya perbedaan rata-rata kemungkinan hanyalah kebetulan. Sebaliknya, jika nilai uji lebih besar, maka H0 gagal ditolak yang berarti tidak ada bukti cukup untuk menyatakan adanya perbedaan.

\begin{verbatim}
## Hipotesis kosong gagal ditolak. Perbedaan rata-rata sampel dengan populasi hanyalah kebetulan
\end{verbatim}

\subsubsection{Proporsi}\label{proporsi}

Proporsi adalah perbandingan jumlah elemen yang memenuhi suatu kriteria terhadap jumlah total. Misalnya, dalam kasus mahasiswa, kita bisa melihat proporsi mahasiswa yang tinggal di kos/asrama dibandingkan dengan seluruh mahasiswa dalam sampel.

Pengujian proporsi bermanfaat ketika kita ingin menguji apakah proporsi sampel yang kita peroleh berbeda secara signifikan dari nilai proporsi populasi yang diasumsikan. Contoh kasusnya adalah: ``Apakah benar 50\% mahasiswa di Bandar Lampung tinggal di kos/asrama?'' Jika hasil sampel kita menunjukkan 47\% atau 55\%, kita perlu uji hipotesis untuk menentukan apakah perbedaan tersebut kebetulan atau memang signifikan.

\subparagraph{Membuat fungsi}\label{membuat-fungsi}

Untuk memudahkan proses, kita membuat sebuah fungsi di R yang dapat menghitung statistik uji Z untuk proporsi satu populasi. Fungsi ini akan menerima input berupa:

\subparagraph{Meringkas Data}\label{meringkas-data}

Sebelum melakukan uji hipotesis, kita perlu mengetahui berapa proporsi mahasiswa yang tinggal di masing-masing tipe tempat tinggal. Kita bisa gunakan fungsi group\_by() dan summarize() untuk menghitung jumlah mahasiswa berdasarkan tipe tempat tinggal, lalu menambahkan kolom proporsi. Dengan cara ini, kita mendapatkan ringkasan data yang memperlihatkan distribusi tempat tinggal mahasiswa.

\begin{verbatim}
## # A tibble: 2 x 3
##   `Tipe tinggal`         Jumlah proporsi
##   <chr>                   <int>    <dbl>
## 1 Kos/Asrama                504    0.418
## 2 Rumah Keluarga/Pribadi    702    0.582
\end{verbatim}

\subparagraph{Menyimpan Nilai Proporsi}\label{menyimpan-nilai-proporsi}

Setelah kita memperoleh ringkasan data, kita simpan nilai proporsi mahasiswa yang tinggal di kos atau asrama ke dalam sebuah variabel. Dengan cara ini, nilai tersebut dapat digunakan dengan mudah dalam perhitungan selanjutnya tanpa harus menghitung ulang.

\begin{verbatim}
## [1] 0.4179104
\end{verbatim}

\subparagraph{Mendeklarasikan variabel pengujian}\label{mendeklarasikan-variabel-pengujian}

Pada tahap ini, kita menentukan variabel-variabel utama yang akan digunakan dalam uji hipotesis. Semua variabel inilah yang nantinya akan dimasukkan ke dalam fungsi uji hipotesis yang telah dibuat.

\begin{verbatim}
## Proporsi penghuni Kos/Asrama sampel adalah 0.4179104 
## Parameter proporsi penghuni Kos/Asrama yang diuji adalah 0.5 
## Ukuran sampel adalah 1206 orang.
\end{verbatim}

\paragraph{\texorpdfstring{\emph{Two-tailed Test}}{Two-tailed Test}}\label{two-tailed-test}

Pada pengujian hipotesis proporsi dengan two-tailed test, kita ingin mengetahui apakah proporsi sampel berbeda secara signifikan dari proporsi populasi yang dihipotesiskan. Artinya, kita hanya fokus pada perbedaan, bukan pada arah perbedaan. Misalnya, kita ingin menguji apakah proporsi mahasiswa yang tinggal di kos/asrama benar-benar sama dengan 50\% atau tidak. Jika hasilnya jauh lebih besar atau jauh lebih kecil, keduanya dianggap sebagai bukti menolak hipotesis nol.

\subparagraph{Merumuskan hipotesis}\label{merumuskan-hipotesis-2}

Sama seperti uji dua arah pada rata-rata, uji dua arah pada proporsi diawali dengan merumuskan hipotesis. Di sini kita mendefinisikan dua kemungkinan: hipotesis nol (H0) dan hipotesis alternatif (H1). H0 biasanya menyatakan bahwa ``tidak ada perbedaan'' atau ``nilai parameter sama dengan nilai tertentu''. Sebaliknya, H1 menyatakan adanya perbedaan yang signifikan. Dalam uji dua arah, rumus hipotesis ditulis sebagai:

\begin{verbatim}
## H0 : prop. Kos/Asrama = 0,5 
## 
## H1 : prop. Kos/Asrama != 0,5
\end{verbatim}

\subparagraph{Memilih distribusi sampling, wilayah dan titik kritis}\label{memilih-distribusi-sampling-wilayah-dan-titik-kritis-1}

Sama seperti sebelumnya, setelah hipotesis ditetapkan, kita perlu menentukan distribusi yang digunakan. Untuk kasus proporsi dengan ukuran sampel besar, digunakan distribusi normal. Kemudian, karena kita menggunakan uji dua arah, wilayah kritis akan terbagi dua, masing-masing di ekor kiri dan kanan distribusi. Kali ini, kita menggunakan taraf signifikansi α=0.10.

\begin{verbatim}
## Titik kritis uji two-tailed: 1.644854
\end{verbatim}

\subparagraph{Menghitung Statistik Uji}\label{menghitung-statistik-uji-2}

Setelah titik kritis ditentukan, kita menghitung nilai statistik uji Z berdasarkan proporsi sampel dan proporsi populasi yang diuji.

\begin{verbatim}
## [1] 5.779963
\end{verbatim}

\subparagraph{Menarik kesimpulan}\label{menarik-kesimpulan-2}

Langkah terakhir adalah membandingkan nilai statistik uji dengan titik kritis. Dari sini kita bisa menarik kesimpulan apakah hipotesis nol ditolak atau gagal ditolak. Jika nilai statistik uji (absolut) lebih kecil dari titik kritis, maka H₀ gagal ditolak, artinya perbedaan yang terlihat kemungkinan hanya kebetulan sampel. Jika lebih besar atau sama, maka H₀ ditolak, artinya ada bukti yang cukup bahwa proporsi tidak sama dengan 0,5.

\begin{verbatim}
## Hipotesis kosong ditolak. Perbedaan proporsi penghuni Kos/Asrama sebanyak 0.418 dari 0.5 adalah signifikan
\end{verbatim}

\paragraph{\texorpdfstring{\emph{One-tailed Test}}{One-tailed Test}}\label{one-tailed-test}

Pada uji hipotesis proporsi satu arah, kita hanya berfokus dengan satu sisi distribusi, yakni apakah proporsi lebih besar atau lebih kecil dari nilai yang dihipotesiskan (kiri untuk dugaan ``\textless'', kanan untuk dugaan ``\textgreater''). Uji ini lebih spesifik karena hipotesis alternatif sudah menetapkan arah. Misalnya, kita ingin tahu apakah proporsi mahasiswa kos/asrama lebih kecil dari 50\%. Dalam uji ini, hanya sisi kiri distribusi yang menjadi perhatian.

\subparagraph{Merumuskan hipotesis}\label{merumuskan-hipotesis-3}

Hipotesis dirumuskan sebagai berikut:

\begin{verbatim}
## H0 : prop. Kos/Asrama = 0,5 
## 
## H1 : prop. Kos/Asrama < 0,5
\end{verbatim}

\subparagraph{Memilih distribusi Sampling, wilayah dan titik kritis}\label{memilih-distribusi-sampling-wilayah-dan-titik-kritis-2}

Distribusi sampling yang digunakan tetap distribusi normal. Namun, karena ini uji satu arah, maka seluruh α ditempatkan pada salah satu sisi. Jika hipotesis alternatifnya menyatakan lebih besar, maka daerah kritis ada di sebelah kanan. Jika menyatakan lebih kecil, maka daerah kritis ada di sebelah kiri. Dengan taraf signifikansi α, titik kritis dihitung dengan qnorm(1 - alpha) untuk sisi kanan atau qnorm(alpha) untuk sisi kiri.

\begin{verbatim}
## Titik kritis uji one-tailed: -1.281552
\end{verbatim}

\subparagraph{Menghitung statistik uji}\label{menghitung-statistik-uji-3}

Sama seperti pada uji dua arah, kita menghitung nilai statistik uji Z dengan fungsi yang sudah dibuat sebelumnya tetapi kali ini kita tidak mengambil nilai absolut karena tanda statistik memberi informasi arah.

\begin{verbatim}
## [1] -5.779963
\end{verbatim}

\subparagraph{Menarik kesimpulan}\label{menarik-kesimpulan-3}

Kesimpulan diambil dengan membandingkan nilai statistik uji (Z\_propKos) dengan titik kritis (Z.crit\_1tail). Untuk uji satu arah (kiri), kita menolak H₀ jika nilai statistik uji lebih kecil dari titik kritis. Jika tidak, kita gagal menolak H₀. Kesimpulan ini menjelaskan apakah ada cukup bukti untuk menyatakan proporsi memang lebih kecil dari nilai acuan.

\begin{verbatim}
## Hipotesis kosong ditolak. Perbedaan proporsi penghuni Kos/Asrama sebanyak 0.418 dari 0.5 adalah signifikan (lebih kecil)
\end{verbatim}

\subsection{Uji Hipotesis 2 Populasi Independen}\label{uji-hipotesis-2-populasi-independen}

Dua populasi dikatakan independen apabila tidak ada pengaruh antara populasi 1 dengan populasi lain. Artinya, data dari populasi pertama dan kedua berdiri sendiri, tidak saling berkaitan. Contohnya adalah membandingkan rata-rata jarak tempat tinggal mahasiswa ITERA dengan mahasiswa UINRIL, di mana setiap kelompok mahasiswa berasal dari populasi yang berbeda.

\subsubsection{Rata-rata}\label{rata-rata-1}

\subparagraph{Merumuskan fungsi}\label{merumuskan-fungsi}

Pada langkah ini, kita akan membuat fungsi uji hipotesis rata-rata untuk dua populasi independen. Fungsi ini diperlukan karena kita ingin menghitung nilai statistik uji (Z) berdasarkan data ringkasan dari dua sampel Dengan fungsi ini, kita bisa langsung memasukkan nilai-nilai statistik dari kedua populasi dan mendapatkan nilai Z yang akan dibandingkan dengan titik kritis.

\subparagraph{Menghitung mean 2 populasi}\label{menghitung-mean-2-populasi}

Sebelum menjalankan uji hipotesis, kita perlu mendapatkan data ringkasan berupa rata-rata, simpangan baku, dan jumlah sampel dari masing-masing populasi. Pada tahap ini, kita akan membuat tabel ringkasan jarak mahasiswa dari rumah ke kampus berdasarkan kelompok kampus. Data ini nantinya akan menjadi input untuk fungsi yang sudah kita buat di langkah sebelumnya. Dengan cara ini, kita bisa lebih mudah membandingkan dua populasi secara langsung.

\begin{verbatim}
## # A tibble: 3 x 4
##   Kampus `Rata-rata` `Simpangan baku` Jumlah
##   <chr>        <dbl>            <dbl>  <int>
## 1 ITERA         5.01             2.82    427
## 2 UBL           3.38             3.74    379
## 3 UINRIL        3.21             3.74    400
\end{verbatim}

\subparagraph{Mendeklarasikan variabel uji}\label{mendeklarasikan-variabel-uji}

Setelah mendapatkan ringkasan data, kita harus menyimpan nilai rata-rata, simpangan baku, dan ukuran sampel ke dalam variabel khusus. Variabel-variabel ini akan dipakai langsung sebagai input fungsi uji hipotesis. Dengan mendeklarasikannya lebih dulu, proses penghitungan menjadi lebih rapi dan mudah diikuti.

\paragraph{\texorpdfstring{\emph{Two-tailed Test}}{Two-tailed Test}}\label{two-tailed-test-1}

\subparagraph{Merumuskan hipotesis}\label{merumuskan-hipotesis-4}

Sama seerti sebelumnya, langkah pertama dalam uji hipotesis adalah menyusun hipotesis nol (H0) dan hipotesis alternatif (H1). Untuk kasus two-tailed test, hipotesis alternatif menyatakan adanya perbedaan rata-rata tanpa menyebutkan arah (lebih besar atau lebih kecil). Artinya, kita hanya ingin tahu apakah terdapat perbedaan signifikan atau tidak, tanpa peduli siapa yang lebih besar. Hipotesis nol menyatakan tidak ada perbedaan, atau selisih rata-rata sama dengan nol.

\begin{verbatim}
## H0: mu_jarakUINRIL - mu_jarakITERA = 0 
## 
## H1: mu_jarakUINRIL - mu_jarakITERA != 0
\end{verbatim}

\subparagraph{Memilih distribusi sampling, wilayah, dan titik kritis}\label{memilih-distribusi-sampling-wilayah-dan-titik-kritis-3}

Setelah hipotesis dirumuskan, kita harus menentukan distribusi sampling yang dipakai serta wilayah kritisnya. Karena ukuran sampel cukup besar, distribusi normal standar (Z) digunakan sebagai acuan. Untuk two-tailed test, wilayah kritis terbagi dua sama besar di kedua sisi kurva distribusi. Nilai kritis diperoleh dari tabel distribusi normal pada α/2, lalu dibuat dalam bentuk absolut karena titik kritis simetris di kedua sisi.

\begin{verbatim}
## Titik kritis uji two-tailed: 3.290527
\end{verbatim}

\subparagraph{Menghitung statistik uji}\label{menghitung-statistik-uji-4}

Setelah mengetahui titik kritis, kita hitung nilai statistik uji (Z) dari data yang dimiliki. Nilai ini diperoleh dengan memasukkan rata-rata, simpangan baku, dan ukuran sampel ke fungsi ht\_mean\_2popIn. Karena pengujian dilakukan dua sisi, nilai Z hasil perhitungan diabsolutkan agar bisa dibandingkan langsung dengan titik kritis.

\begin{verbatim}
## [1] 7.743731
\end{verbatim}

\subparagraph{Menarik kesimpulan}\label{menarik-kesimpulan-4}

Langkah terakhir adalah membandingkan nilai statistik uji dengan titik kritis. Jika Z hitung lebih kecil dari Z kritis, maka H0 gagal ditolak, artinya perbedaan rata-rata jarak antara dua populasi kemungkinan hanyalah kebetulan. Jika Z hitung lebih besar, maka H0 ditolak, yang berarti terdapat cukup bukti untuk menyatakan adanya perbedaan.

\begin{verbatim}
## Hipotesis kosong ditolak. Perbedaan rata-rata jarak mahasiswa UINRIL dengan ITERA dari sampel signifikan.
\end{verbatim}

\paragraph{\texorpdfstring{\emph{One-tailed Test}}{One-tailed Test}}\label{one-tailed-test-1}

\subparagraph{Merumuskan hipotesis}\label{merumuskan-hipotesis-5}

Untuk one-tailed test, hipotesis alternatif (H1) menyebutkan arah perbedaan yang spesifik, misalnya rata-rata populasi 1 lebih kecil dari populasi 2. Berbeda dengan two-tailed test yang hanya memeriksa ``ada perbedaan'', one-tailed test lebih fokus untuk mengetahui apakah perbedaan itu terjadi sesuai arah yang dihipotesiskan. Hipotesis nol (H0) tetap menyatakan tidak ada perbedaan, tetapi pengujian difokuskan hanya pada satu sisi distribusi.

\begin{verbatim}
## H0: mu_jarakUINRIL - mu_jarakITERA = 0 
## 
## H1: mu_jarakUINRIL - mu_jarakITERA < 0
\end{verbatim}

\subparagraph{Memilih distribusi sampling, wilayah, dan titik kritis}\label{memilih-distribusi-sampling-wilayah-dan-titik-kritis-4}

Dalam pengujian one-tailed, kita perlu menentukan sisi mana yang akan digunakan sebagai wilayah kritis. Jika hipotesis alternatif menyebutkan ``lebih kecil'', maka wilayah kritis berada di sisi kiri distribusi. Sebaliknya, jika hipotesis menyebutkan ``lebih besar'', wilayah kritis diletakkan di sisi kanan.

\begin{verbatim}
## Titik kritis uji one-tailed: -3.090232
\end{verbatim}

\subparagraph{Menghitung Statistik uji}\label{menghitung-statistik-uji-5}

Selanjutnya, kita menghitung nilai statistik uji menggunakan fungsi ht\_mean\_2popIn. Berbeda dengan two-tailed test, pada one-tailed test nilai Z tidak diabsolutkan karena arah perbedaan menjadi penting.

\begin{verbatim}
## [1] -7.743731
\end{verbatim}

\subparagraph{Menarik kesimpulan}\label{menarik-kesimpulan-5}

Tahap terakhir adalah menarik kesimpulan dengan cara membandingkan nilai statistik uji terhadap titik kritis. Jika nilai Z hitung jatuh pada wilayah kritis (misalnya lebih kecil dari Z kritis kiri), maka hipotesis nol ditolak. Artinya, ada bukti cukup untuk menyatakan rata-rata jarak mahasiswa ITERA memang lebih kecil dari mahasiswa UINRIL sesuai hipotesis arah. Jika tidak, maka hipotesis nol gagal ditolak, dan kita simpulkan bahwa arah perbedaan yang diduga tidak terbukti secara signifikan.

\begin{verbatim}
## Hipotesis kosong ditolak. Mahasiswa ITERA secara rata-rata tinggal lebih jauh dari kampus dibandingkan mahasiswa UINRIL significanly.
\end{verbatim}

\subsubsection{Proporsi}\label{proporsi-1}

Pada bagian ini kita akan melakukan uji hipotesis terhadap proporsi dua populasi yang independen. Berbeda dengan rata-rata, proporsi digunakan ketika data berbentuk kategori (misalnya pilihan tinggal di kos/asrama atau tinggal bersama keluarga). Prinsip pengujian tetap sama, yaitu membandingkan apakah perbedaan proporsi antar populasi signifikan atau hanya karena kebetulan sampel.

\subparagraph{Membuat fungsi}\label{membuat-fungsi-1}

Sama seperti sebelumnya, langkah pertama adalah membuat fungsi khusus untuk menghitung nilai statistik uji pada uji hipotesis proporsi dua populasi independen. Fungsi ini menerima input berupa proporsi sampel masing-masing populasi (p\_hat1 dan p\_hat2) serta ukuran sampelnya (n1 dan n2). Di dalam fungsi, dihitung terlebih dahulu proporsi gabungan (p\_gab) yang mewakili keseluruhan sampel. Kemudian dihitung standard error (SE), dan akhirnya didapatkan nilai Z hitung.

\subparagraph{Mendeklarasikan variabel uji}\label{mendeklarasikan-variabel-uji-1}

Setelah fungsi dibuat, kita perlu menyiapkan data yang akan diuji. Pertama, data mahasiswa dikelompokkan berdasarkan kampus dan tipe\_tinggal\_baku. Selanjutnya dihitung proporsi tiap kategori (misalnya proporsi mahasiswa yang tinggal di kos/asrama). Data kemudian diubah dari format long ke wide sehingga tiap kategori menjadi variabel. Dari hasil ini kita bisa mengambil proporsi masing-masing populasi, serta ukuran sampel total pada tiap kampus.

\begin{verbatim}
## # A tibble: 3 x 3
## # Groups:   kampus [3]
##   kampus `Kos/Asrama` `Rumah Keluarga/Pribadi`
##   <chr>         <dbl>                    <dbl>
## 1 ITERA         0.550                    0.450
## 2 UBL           0.224                    0.776
## 3 UINRIL        0.46                     0.54
\end{verbatim}

\#\#\#\#\emph{Two-tailed Test}

\begin{verbatim}
## Titik Kritis: 1.811911 
## Z Hitung: 2.597132
\end{verbatim}

\begin{verbatim}
## Hipotesis nol ditolak.
\end{verbatim}

\paragraph{\texorpdfstring{\emph{One-tailed Test}}{One-tailed Test}}\label{one-tailed-test-2}

\begin{verbatim}
## Titik Kritis: -1.750686 
## Z Hitung: -2.597132
\end{verbatim}

\begin{verbatim}
## Hipotesis nol ditolak (Signifikan).
\end{verbatim}

\subsection{Uji Hipotesis 2 Populasi Dependen}\label{uji-hipotesis-2-populasi-dependen}

Menguji hipotesis 2 populasi yang dependen pada dasarnya sama persis dengan uji hipotesis 1 populasi, karena statistik yang diuji adalah \textbf{perbedaan nilai sampel}. Biasanya berbentuk ``sebelum-sesudah''. Contoh: mengukur nilai mahasiswa sebelum dan sesudah diberi pelatihan. Bedanya dengan independen : kita tidak bandingkan dua kelompok berbeda, tapi pasangan data dari kelompok yang sama dalam dua kondisi.

Perbedaan nilai sampel jumlah nilainya hanya 1, sehingga bisa diperlakukan seperti halnya perhitungan uji hipotesis 1 populasi. Simpangan baku yang diketahui pun adalah simpangan baku \textbf{perbedaan nilainya, sehingga jumlah nilainya hanya 1 juga}.

\subsection{Soal Latihan}\label{soal-latihan}

\section{Modul-6: Analisis Hubungan Korelatif: Korelasi Variabel Nominal dan Ordinal}\label{modul-6-analisis-hubungan-korelatif-korelasi-variabel-nominal-dan-ordinal}

Setelah mempelajari modul ini, Anda diharapkan dapat:

\begin{enumerate}
\def\labelenumi{\arabic{enumi}.}
\tightlist
\item
  mampu menghasilkan koefisien korelasi variabel di tingkat nominal dengan perangkat lunak komputer \capaian{STP-9.2}
\item
  mampu menghasilkan koefisien korelasi variabel di tingkat ordinal dengan perangkat lunak komputer \capaian{STP-10.2}
\end{enumerate}

\begin{center}\rule{0.5\linewidth}{0.5pt}\end{center}

\subsection{Pendahuluan}\label{pendahuluan-2}

Analisis korelasi adalah salah satu teknik analisis yang termasuk ke dalam lingkup statistika bivariat, statistika yang analisisnya melibatkan dua variabel \textbf{sekaligus}. Pada praktikum-praktikum sebelumnya kita hanya melibatkan satu variabel saja, misalnya:

\begin{itemize}
\item
  pada praktikum ke-2, kita mengaplikasikan persentase hanya pada variabel \texttt{Jenis.Kelamin}, \texttt{Fakultas}, \texttt{Tingkat.Semester}, \texttt{Uang.Saku}, atau\texttt{jenis.tempat.tinggal} saja, tetapi tidak menghubungkannya satu sama lain,

\begin{Shaded}
\begin{Highlighting}[]
\NormalTok{data.ubl }\SpecialCharTok{|\textgreater{}} 
  \FunctionTok{tbl\_summary}\NormalTok{(}\AttributeTok{include =} \FunctionTok{c}\NormalTok{(Jenis.Kelamin,}
\NormalTok{                          Fakultas,}
\NormalTok{                          Tingkat.Semester,}
\NormalTok{                          Uang.Saku,}
\NormalTok{                          jenis.tempat.tinggal)) }\SpecialCharTok{|\textgreater{}} 
  \FunctionTok{as\_flex\_table}\NormalTok{()}
\end{Highlighting}
\end{Shaded}
\item
  pada praktikum ke-3, kita membuat histogram untuk variabel \texttt{Umur} saja

\begin{Shaded}
\begin{Highlighting}[]
\NormalTok{histogram }\OtherTok{\textless{}{-}} \FunctionTok{ggplot}\NormalTok{(data.ubl.cleaned) }\SpecialCharTok{+}
  \FunctionTok{geom\_histogram}\NormalTok{(}\AttributeTok{mapping =} \FunctionTok{aes}\NormalTok{(}\AttributeTok{x =}\NormalTok{ Umur), }\AttributeTok{binwidth =} \DecValTok{1}\NormalTok{, }\AttributeTok{fill =} \StringTok{"darkseagreen"}\NormalTok{, }\AttributeTok{color =} \StringTok{"white"}\NormalTok{) }\SpecialCharTok{+}
  \FunctionTok{labs}\NormalTok{(}
    \AttributeTok{title =} \StringTok{"Distribusi Umur Mahasiswa"}\NormalTok{,}
    \AttributeTok{x =} \StringTok{"Umur (Tahun)"}\NormalTok{,}
    \AttributeTok{y =} \StringTok{"Frekuensi"}
\NormalTok{  ) }\SpecialCharTok{+}
  \FunctionTok{theme\_minimal}\NormalTok{()}

\NormalTok{histogram}
\end{Highlighting}
\end{Shaded}
\item
  pada praktikum-praktikum tentang analisis statistik inferensial, kita hanya tertarik pada proporsi atau rata-rata suatu variabel saja, yang pastinya tunggal.
\end{itemize}

Dalam praktikum ini kita akan mempraktikkan analisis korelasi pada dua variabel \textbf{secara bersamaan.} Ketika kita tertarik pada analisis untuk dua variabel secara bersamaan, artinya kita sedang menganalisis sebuah \textbf{keterkaitan} atau sebuah \textbf{hubungan}. Dalam praktikum ini, hubungan itu disebut \textbf{korelasi}.

Korelasi yang akan kita pelajari adalah korelasi pada pasangan variabel dengan tingkat pengukuran nominal dan juga ordinal.

\subsection{\texorpdfstring{Pustaka \emph{(Libraries)} yang Diperlukan dan Memuat Data}{Pustaka (Libraries) yang Diperlukan dan Memuat Data}}\label{pustaka-libraries-yang-diperlukan-dan-memuat-data}

Seperti biasa, kita perlu memuat pustaka \emph{(libraries)} yang diperlukan dalam pengolahan data kita. Untuk memuat tabel silang kita akan gunakan pustaka \texttt{gtsummary} kembali seperti pada praktikum ke-2

\textbf{Memuat dataset}

Kita akan menggunakan dataset keempat kampus di Kota Bandar Lampung dan sekitarnya sebagai bahan. Tulis ulang dan jalankan baris perintah berikut untuk mengolah data keempat kampus

\textbf{Mengatur Faktor untuk Variabel Kategoris}

Setelah membaca data, kita perlu mengatur variabel-variabel nominal dan ordinal dengan mengatur nilai-nilainya berupa \emph{factor}.

Namun sebelum itu, kita akan membuat vektor-vektor untuk kita aplikasikan menjadi \emph{factor} dengan memanfaatkan \texttt{group\_by} dan \texttt{summarize} yang berasal dari pustaka \texttt{dplyr}.

Barulah selanjutnya kita bisa mengatur \emph{factor} untuk setiap variabel kategoris

Kita dapat memeriksa hasil pekerjaan kita dengan perintah \texttt{glimpse}

\begin{verbatim}
## Rows: 1,557
## Columns: 22
## $ Kampus_PT                  <fct> ITERA, ITERA, ITERA, ITERA, ITERA, ITERA, I~
## $ `Nomor urut`               <dbl> 17, 211, 342, 42, 56, 254, 267, 289, 330, 2~
## $ `Jenis Kelamin`            <fct> Perempuan, Perempuan, Perempuan, Perempuan,~
## $ Umur                       <dbl> 20, 20, 22, 21, 22, 22, 21, 21, 20, 23, 19,~
## $ Fakultas                   <fct> FTIK, FTIK, FTI, FTIK, FTIK, FTIK, FTIK, FT~
## $ Prodi                      <fct> Perencanaan Wilayah dan Kota, Perencanaan W~
## $ `Tingkat Semester`         <ord> Semester 3-4, Semester 5-6, Semester 7-8, S~
## $ `Uang Saku`                <ord> "1 - 2 Jt", "2,1 - 3 Jt", "2,1 - 3 Jt", "2,~
## $ `jumlah mobil`             <dbl> 1, 0, 3, 3, 1, 3, 3, 3, 3, 2, 1, 0, 3, 2, 2~
## $ `jumlah motor`             <dbl> 0, 1, 2, 3, 0, 2, 2, 3, 2, 2, 0, 1, 1, 2, 1~
## $ `jumlah sepeda`            <dbl> 1, 1, 3, 3, 0, 2, 3, 4, 2, 2, 1, 0, 1, 2, 1~
## $ `kendaraan utama`          <fct> Sepeda motor pribadi, Mobil pribadi, Mobil ~
## $ `jenis tempat tinggal`     <fct> Kos sendiri, Rumah pribadi, Rumah pribadi, ~
## $ jarak                      <dbl> 3.47, 4.88, 3.09, 5.46, 3.70, 6.18, 4.87, 5~
## $ `biaya sepekan`            <dbl> 400, 280, 225, 200, 200, 200, 200, 200, 200~
## $ `Jumlah Perjalanan Senin`  <dbl> 2, 4, 2, 0, 4, 0, 3, 4, 2, 0, 1, 4, 2, 0, 3~
## $ `Jumlah Perjalanan Selasa` <dbl> 3, 4, 2, 3, 3, 0, 4, 4, 2, 2, 1, 3, 2, 2, 2~
## $ `Jumlah Perjalanan Rabu`   <dbl> 3, 5, 0, 4, 3, 3, 4, 4, 0, 2, 1, 2, 2, 2, 4~
## $ `Jumlah Perjalanan Kamis`  <dbl> 0, 4, 2, 3, 2, 0, 4, 4, 2, 2, 1, 2, 2, 2, 2~
## $ `Jumlah Perjalanan Jumat`  <dbl> 0, 5, 0, 3, 2, 2, 4, 5, 2, 2, 1, 3, 3, 4, 4~
## $ `Jumlah Perjalanan Sabtu`  <dbl> 2, 0, 3, 0, 3, 3, 0, 2, 2, 0, 1, 2, 2, 0, 0~
## $ `Jumlah Perjalanan Ahad`   <dbl> 4, 3, 0, 2, 2, 0, 3, 2, 2, 0, 1, 2, 0, 0, 2~
\end{verbatim}

\subsection{Pembuatan Tabel Silang}\label{pembuatan-tabel-silang}

Setelah kita memastikan data kita bersih, kita siap untuk menganalisis korelasi antarvariabel kategoris.

Korelasi antarvariabel kategoris memerlukan ``bahan baku'' berupa \textbf{tabel silang} atau \textbf{\emph{cross table}}. Kebanyakan literatur juga menggunakan istilah \textbf{\emph{contingency table}}. Tabel silang adalah tabel dua arah, yakni yang baik baris maupun kolomnya adalah \textbf{nilai-nilai dari dua variabel} yang disilangkan.

\emph{Chunk} berikut memperlihatkan bagaimana membuat tabel silang antara variabel \texttt{jenis\ tempat\ tinggal} dengan \texttt{kendaraan\ utama}.

\global\setlength{\Oldarrayrulewidth}{\arrayrulewidth}

\global\setlength{\Oldtabcolsep}{\tabcolsep}

\setlength{\tabcolsep}{2pt}

\renewcommand*{\arraystretch}{1.5}



\providecommand{\ascline}[3]{\noalign{\global\arrayrulewidth #1}\arrayrulecolor[HTML]{#2}\cline{#3}}

\begin{longtable}[c]{|p{2.97in}|p{0.93in}|p{1.77in}|p{1.20in}|p{2.48in}|p{2.36in}|p{1.44in}|p{1.52in}}



\ascline{1pt}{000000}{1-8}

\multicolumn{1}{>{\raggedright}m{\dimexpr 2.97in+0\tabcolsep}}{\textcolor[HTML]{000000}{\fontsize{11}{11}\selectfont{\global\setmainfont{Arial}{\textbf{Characteristic}}}}} & \multicolumn{1}{>{\centering}m{\dimexpr 0.93in+0\tabcolsep}}{\textcolor[HTML]{000000}{\fontsize{11}{11}\selectfont{\global\setmainfont{Arial}{\textbf{Asrama}}}}\textcolor[HTML]{000000}{\fontsize{11}{11}\selectfont{\global\setmainfont{Arial}{\ \ }}}\textcolor[HTML]{000000}{\fontsize{11}{11}\selectfont{\global\setmainfont{Arial}{\linebreak }}}\textcolor[HTML]{000000}{\fontsize{11}{11}\selectfont{\global\setmainfont{Arial}{N\ =\ 25}}}\textcolor[HTML]{000000}{\fontsize{11}{11}\selectfont{\global\setmainfont{Arial}{\textsuperscript{1}}}}} & \multicolumn{1}{>{\centering}m{\dimexpr 1.77in+0\tabcolsep}}{\textcolor[HTML]{000000}{\fontsize{11}{11}\selectfont{\global\setmainfont{Arial}{\textbf{Kos\ bersama-sama}}}}\textcolor[HTML]{000000}{\fontsize{11}{11}\selectfont{\global\setmainfont{Arial}{\ \ }}}\textcolor[HTML]{000000}{\fontsize{11}{11}\selectfont{\global\setmainfont{Arial}{\linebreak }}}\textcolor[HTML]{000000}{\fontsize{11}{11}\selectfont{\global\setmainfont{Arial}{N\ =\ 120}}}\textcolor[HTML]{000000}{\fontsize{11}{11}\selectfont{\global\setmainfont{Arial}{\textsuperscript{1}}}}} & \multicolumn{1}{>{\centering}m{\dimexpr 1.2in+0\tabcolsep}}{\textcolor[HTML]{000000}{\fontsize{11}{11}\selectfont{\global\setmainfont{Arial}{\textbf{Kos\ sendiri}}}}\textcolor[HTML]{000000}{\fontsize{11}{11}\selectfont{\global\setmainfont{Arial}{\ \ }}}\textcolor[HTML]{000000}{\fontsize{11}{11}\selectfont{\global\setmainfont{Arial}{\linebreak }}}\textcolor[HTML]{000000}{\fontsize{11}{11}\selectfont{\global\setmainfont{Arial}{N\ =\ 468}}}\textcolor[HTML]{000000}{\fontsize{11}{11}\selectfont{\global\setmainfont{Arial}{\textsuperscript{1}}}}} & \multicolumn{1}{>{\centering}m{\dimexpr 2.48in+0\tabcolsep}}{\textcolor[HTML]{000000}{\fontsize{11}{11}\selectfont{\global\setmainfont{Arial}{\textbf{Rumah\ mengontrak\ bersama}}}}\textcolor[HTML]{000000}{\fontsize{11}{11}\selectfont{\global\setmainfont{Arial}{\ \ }}}\textcolor[HTML]{000000}{\fontsize{11}{11}\selectfont{\global\setmainfont{Arial}{\linebreak }}}\textcolor[HTML]{000000}{\fontsize{11}{11}\selectfont{\global\setmainfont{Arial}{N\ =\ 223}}}\textcolor[HTML]{000000}{\fontsize{11}{11}\selectfont{\global\setmainfont{Arial}{\textsuperscript{1}}}}} & \multicolumn{1}{>{\centering}m{\dimexpr 2.36in+0\tabcolsep}}{\textcolor[HTML]{000000}{\fontsize{11}{11}\selectfont{\global\setmainfont{Arial}{\textbf{Rumah\ mengontrak\ pribadi}}}}\textcolor[HTML]{000000}{\fontsize{11}{11}\selectfont{\global\setmainfont{Arial}{\ \ }}}\textcolor[HTML]{000000}{\fontsize{11}{11}\selectfont{\global\setmainfont{Arial}{\linebreak }}}\textcolor[HTML]{000000}{\fontsize{11}{11}\selectfont{\global\setmainfont{Arial}{N\ =\ 43}}}\textcolor[HTML]{000000}{\fontsize{11}{11}\selectfont{\global\setmainfont{Arial}{\textsuperscript{1}}}}} & \multicolumn{1}{>{\centering}m{\dimexpr 1.44in+0\tabcolsep}}{\textcolor[HTML]{000000}{\fontsize{11}{11}\selectfont{\global\setmainfont{Arial}{\textbf{Rumah\ pribadi}}}}\textcolor[HTML]{000000}{\fontsize{11}{11}\selectfont{\global\setmainfont{Arial}{\ \ }}}\textcolor[HTML]{000000}{\fontsize{11}{11}\selectfont{\global\setmainfont{Arial}{\linebreak }}}\textcolor[HTML]{000000}{\fontsize{11}{11}\selectfont{\global\setmainfont{Arial}{N\ =\ 553}}}\textcolor[HTML]{000000}{\fontsize{11}{11}\selectfont{\global\setmainfont{Arial}{\textsuperscript{1}}}}} & \multicolumn{1}{>{\centering}m{\dimexpr 1.52in+0\tabcolsep}}{\textcolor[HTML]{000000}{\fontsize{11}{11}\selectfont{\global\setmainfont{Arial}{\textbf{Rumah\ saudara}}}}\textcolor[HTML]{000000}{\fontsize{11}{11}\selectfont{\global\setmainfont{Arial}{\ \ }}}\textcolor[HTML]{000000}{\fontsize{11}{11}\selectfont{\global\setmainfont{Arial}{\linebreak }}}\textcolor[HTML]{000000}{\fontsize{11}{11}\selectfont{\global\setmainfont{Arial}{N\ =\ 125}}}\textcolor[HTML]{000000}{\fontsize{11}{11}\selectfont{\global\setmainfont{Arial}{\textsuperscript{1}}}}} \\

\ascline{1pt}{000000}{1-8}\endfirsthead 

\ascline{1pt}{000000}{1-8}

\multicolumn{1}{>{\raggedright}m{\dimexpr 2.97in+0\tabcolsep}}{\textcolor[HTML]{000000}{\fontsize{11}{11}\selectfont{\global\setmainfont{Arial}{\textbf{Characteristic}}}}} & \multicolumn{1}{>{\centering}m{\dimexpr 0.93in+0\tabcolsep}}{\textcolor[HTML]{000000}{\fontsize{11}{11}\selectfont{\global\setmainfont{Arial}{\textbf{Asrama}}}}\textcolor[HTML]{000000}{\fontsize{11}{11}\selectfont{\global\setmainfont{Arial}{\ \ }}}\textcolor[HTML]{000000}{\fontsize{11}{11}\selectfont{\global\setmainfont{Arial}{\linebreak }}}\textcolor[HTML]{000000}{\fontsize{11}{11}\selectfont{\global\setmainfont{Arial}{N\ =\ 25}}}\textcolor[HTML]{000000}{\fontsize{11}{11}\selectfont{\global\setmainfont{Arial}{\textsuperscript{1}}}}} & \multicolumn{1}{>{\centering}m{\dimexpr 1.77in+0\tabcolsep}}{\textcolor[HTML]{000000}{\fontsize{11}{11}\selectfont{\global\setmainfont{Arial}{\textbf{Kos\ bersama-sama}}}}\textcolor[HTML]{000000}{\fontsize{11}{11}\selectfont{\global\setmainfont{Arial}{\ \ }}}\textcolor[HTML]{000000}{\fontsize{11}{11}\selectfont{\global\setmainfont{Arial}{\linebreak }}}\textcolor[HTML]{000000}{\fontsize{11}{11}\selectfont{\global\setmainfont{Arial}{N\ =\ 120}}}\textcolor[HTML]{000000}{\fontsize{11}{11}\selectfont{\global\setmainfont{Arial}{\textsuperscript{1}}}}} & \multicolumn{1}{>{\centering}m{\dimexpr 1.2in+0\tabcolsep}}{\textcolor[HTML]{000000}{\fontsize{11}{11}\selectfont{\global\setmainfont{Arial}{\textbf{Kos\ sendiri}}}}\textcolor[HTML]{000000}{\fontsize{11}{11}\selectfont{\global\setmainfont{Arial}{\ \ }}}\textcolor[HTML]{000000}{\fontsize{11}{11}\selectfont{\global\setmainfont{Arial}{\linebreak }}}\textcolor[HTML]{000000}{\fontsize{11}{11}\selectfont{\global\setmainfont{Arial}{N\ =\ 468}}}\textcolor[HTML]{000000}{\fontsize{11}{11}\selectfont{\global\setmainfont{Arial}{\textsuperscript{1}}}}} & \multicolumn{1}{>{\centering}m{\dimexpr 2.48in+0\tabcolsep}}{\textcolor[HTML]{000000}{\fontsize{11}{11}\selectfont{\global\setmainfont{Arial}{\textbf{Rumah\ mengontrak\ bersama}}}}\textcolor[HTML]{000000}{\fontsize{11}{11}\selectfont{\global\setmainfont{Arial}{\ \ }}}\textcolor[HTML]{000000}{\fontsize{11}{11}\selectfont{\global\setmainfont{Arial}{\linebreak }}}\textcolor[HTML]{000000}{\fontsize{11}{11}\selectfont{\global\setmainfont{Arial}{N\ =\ 223}}}\textcolor[HTML]{000000}{\fontsize{11}{11}\selectfont{\global\setmainfont{Arial}{\textsuperscript{1}}}}} & \multicolumn{1}{>{\centering}m{\dimexpr 2.36in+0\tabcolsep}}{\textcolor[HTML]{000000}{\fontsize{11}{11}\selectfont{\global\setmainfont{Arial}{\textbf{Rumah\ mengontrak\ pribadi}}}}\textcolor[HTML]{000000}{\fontsize{11}{11}\selectfont{\global\setmainfont{Arial}{\ \ }}}\textcolor[HTML]{000000}{\fontsize{11}{11}\selectfont{\global\setmainfont{Arial}{\linebreak }}}\textcolor[HTML]{000000}{\fontsize{11}{11}\selectfont{\global\setmainfont{Arial}{N\ =\ 43}}}\textcolor[HTML]{000000}{\fontsize{11}{11}\selectfont{\global\setmainfont{Arial}{\textsuperscript{1}}}}} & \multicolumn{1}{>{\centering}m{\dimexpr 1.44in+0\tabcolsep}}{\textcolor[HTML]{000000}{\fontsize{11}{11}\selectfont{\global\setmainfont{Arial}{\textbf{Rumah\ pribadi}}}}\textcolor[HTML]{000000}{\fontsize{11}{11}\selectfont{\global\setmainfont{Arial}{\ \ }}}\textcolor[HTML]{000000}{\fontsize{11}{11}\selectfont{\global\setmainfont{Arial}{\linebreak }}}\textcolor[HTML]{000000}{\fontsize{11}{11}\selectfont{\global\setmainfont{Arial}{N\ =\ 553}}}\textcolor[HTML]{000000}{\fontsize{11}{11}\selectfont{\global\setmainfont{Arial}{\textsuperscript{1}}}}} & \multicolumn{1}{>{\centering}m{\dimexpr 1.52in+0\tabcolsep}}{\textcolor[HTML]{000000}{\fontsize{11}{11}\selectfont{\global\setmainfont{Arial}{\textbf{Rumah\ saudara}}}}\textcolor[HTML]{000000}{\fontsize{11}{11}\selectfont{\global\setmainfont{Arial}{\ \ }}}\textcolor[HTML]{000000}{\fontsize{11}{11}\selectfont{\global\setmainfont{Arial}{\linebreak }}}\textcolor[HTML]{000000}{\fontsize{11}{11}\selectfont{\global\setmainfont{Arial}{N\ =\ 125}}}\textcolor[HTML]{000000}{\fontsize{11}{11}\selectfont{\global\setmainfont{Arial}{\textsuperscript{1}}}}} \\

\ascline{1pt}{000000}{1-8}\endhead



\multicolumn{8}{>{\raggedright}m{\dimexpr 14.66in+14\tabcolsep}}{\textcolor[HTML]{000000}{\fontsize{11}{11}\selectfont{\global\setmainfont{Arial}{\textsuperscript{1}}}}\textcolor[HTML]{000000}{\fontsize{11}{11}\selectfont{\global\setmainfont{Arial}{n\ (\%)}}}} \\

\endlastfoot



\multicolumn{1}{>{\raggedright}p{\dimexpr 2.97in+0\tabcolsep}}{\textcolor[HTML]{000000}{\fontsize{11}{11}\selectfont{\global\setmainfont{Arial}{kendaraan\ utama}}}} & \multicolumn{1}{>{\centering}p{\dimexpr 0.93in+0\tabcolsep}}{\textcolor[HTML]{000000}{\fontsize{11}{11}\selectfont{\global\setmainfont{Arial}{}}}} & \multicolumn{1}{>{\centering}p{\dimexpr 1.77in+0\tabcolsep}}{\textcolor[HTML]{000000}{\fontsize{11}{11}\selectfont{\global\setmainfont{Arial}{}}}} & \multicolumn{1}{>{\centering}p{\dimexpr 1.2in+0\tabcolsep}}{\textcolor[HTML]{000000}{\fontsize{11}{11}\selectfont{\global\setmainfont{Arial}{}}}} & \multicolumn{1}{>{\centering}p{\dimexpr 2.48in+0\tabcolsep}}{\textcolor[HTML]{000000}{\fontsize{11}{11}\selectfont{\global\setmainfont{Arial}{}}}} & \multicolumn{1}{>{\centering}p{\dimexpr 2.36in+0\tabcolsep}}{\textcolor[HTML]{000000}{\fontsize{11}{11}\selectfont{\global\setmainfont{Arial}{}}}} & \multicolumn{1}{>{\centering}p{\dimexpr 1.44in+0\tabcolsep}}{\textcolor[HTML]{000000}{\fontsize{11}{11}\selectfont{\global\setmainfont{Arial}{}}}} & \multicolumn{1}{>{\centering}p{\dimexpr 1.52in+0\tabcolsep}}{\textcolor[HTML]{000000}{\fontsize{11}{11}\selectfont{\global\setmainfont{Arial}{}}}} \\





\multicolumn{1}{>{\raggedright}p{\dimexpr 2.97in+0\tabcolsep}}{\textcolor[HTML]{000000}{\fontsize{11}{11}\selectfont{\global\setmainfont{Arial}{Berjalan\ kaki}}}} & \multicolumn{1}{>{\centering}p{\dimexpr 0.93in+0\tabcolsep}}{\textcolor[HTML]{000000}{\fontsize{11}{11}\selectfont{\global\setmainfont{Arial}{23\ (92\%)}}}} & \multicolumn{1}{>{\centering}p{\dimexpr 1.77in+0\tabcolsep}}{\textcolor[HTML]{000000}{\fontsize{11}{11}\selectfont{\global\setmainfont{Arial}{20\ (17\%)}}}} & \multicolumn{1}{>{\centering}p{\dimexpr 1.2in+0\tabcolsep}}{\textcolor[HTML]{000000}{\fontsize{11}{11}\selectfont{\global\setmainfont{Arial}{39\ (8.3\%)}}}} & \multicolumn{1}{>{\centering}p{\dimexpr 2.48in+0\tabcolsep}}{\textcolor[HTML]{000000}{\fontsize{11}{11}\selectfont{\global\setmainfont{Arial}{17\ (7.6\%)}}}} & \multicolumn{1}{>{\centering}p{\dimexpr 2.36in+0\tabcolsep}}{\textcolor[HTML]{000000}{\fontsize{11}{11}\selectfont{\global\setmainfont{Arial}{11\ (26\%)}}}} & \multicolumn{1}{>{\centering}p{\dimexpr 1.44in+0\tabcolsep}}{\textcolor[HTML]{000000}{\fontsize{11}{11}\selectfont{\global\setmainfont{Arial}{0\ (0\%)}}}} & \multicolumn{1}{>{\centering}p{\dimexpr 1.52in+0\tabcolsep}}{\textcolor[HTML]{000000}{\fontsize{11}{11}\selectfont{\global\setmainfont{Arial}{1\ (0.8\%)}}}} \\





\multicolumn{1}{>{\raggedright}p{\dimexpr 2.97in+0\tabcolsep}}{\textcolor[HTML]{000000}{\fontsize{11}{11}\selectfont{\global\setmainfont{Arial}{Layanan\ online}}}} & \multicolumn{1}{>{\centering}p{\dimexpr 0.93in+0\tabcolsep}}{\textcolor[HTML]{000000}{\fontsize{11}{11}\selectfont{\global\setmainfont{Arial}{0\ (0\%)}}}} & \multicolumn{1}{>{\centering}p{\dimexpr 1.77in+0\tabcolsep}}{\textcolor[HTML]{000000}{\fontsize{11}{11}\selectfont{\global\setmainfont{Arial}{19\ (16\%)}}}} & \multicolumn{1}{>{\centering}p{\dimexpr 1.2in+0\tabcolsep}}{\textcolor[HTML]{000000}{\fontsize{11}{11}\selectfont{\global\setmainfont{Arial}{86\ (18\%)}}}} & \multicolumn{1}{>{\centering}p{\dimexpr 2.48in+0\tabcolsep}}{\textcolor[HTML]{000000}{\fontsize{11}{11}\selectfont{\global\setmainfont{Arial}{28\ (13\%)}}}} & \multicolumn{1}{>{\centering}p{\dimexpr 2.36in+0\tabcolsep}}{\textcolor[HTML]{000000}{\fontsize{11}{11}\selectfont{\global\setmainfont{Arial}{6\ (14\%)}}}} & \multicolumn{1}{>{\centering}p{\dimexpr 1.44in+0\tabcolsep}}{\textcolor[HTML]{000000}{\fontsize{11}{11}\selectfont{\global\setmainfont{Arial}{37\ (6.7\%)}}}} & \multicolumn{1}{>{\centering}p{\dimexpr 1.52in+0\tabcolsep}}{\textcolor[HTML]{000000}{\fontsize{11}{11}\selectfont{\global\setmainfont{Arial}{12\ (9.6\%)}}}} \\





\multicolumn{1}{>{\raggedright}p{\dimexpr 2.97in+0\tabcolsep}}{\textcolor[HTML]{000000}{\fontsize{11}{11}\selectfont{\global\setmainfont{Arial}{Menumpang\ dengan\ teman/keluarga}}}} & \multicolumn{1}{>{\centering}p{\dimexpr 0.93in+0\tabcolsep}}{\textcolor[HTML]{000000}{\fontsize{11}{11}\selectfont{\global\setmainfont{Arial}{0\ (0\%)}}}} & \multicolumn{1}{>{\centering}p{\dimexpr 1.77in+0\tabcolsep}}{\textcolor[HTML]{000000}{\fontsize{11}{11}\selectfont{\global\setmainfont{Arial}{18\ (15\%)}}}} & \multicolumn{1}{>{\centering}p{\dimexpr 1.2in+0\tabcolsep}}{\textcolor[HTML]{000000}{\fontsize{11}{11}\selectfont{\global\setmainfont{Arial}{24\ (5.1\%)}}}} & \multicolumn{1}{>{\centering}p{\dimexpr 2.48in+0\tabcolsep}}{\textcolor[HTML]{000000}{\fontsize{11}{11}\selectfont{\global\setmainfont{Arial}{40\ (18\%)}}}} & \multicolumn{1}{>{\centering}p{\dimexpr 2.36in+0\tabcolsep}}{\textcolor[HTML]{000000}{\fontsize{11}{11}\selectfont{\global\setmainfont{Arial}{0\ (0\%)}}}} & \multicolumn{1}{>{\centering}p{\dimexpr 1.44in+0\tabcolsep}}{\textcolor[HTML]{000000}{\fontsize{11}{11}\selectfont{\global\setmainfont{Arial}{11\ (2.0\%)}}}} & \multicolumn{1}{>{\centering}p{\dimexpr 1.52in+0\tabcolsep}}{\textcolor[HTML]{000000}{\fontsize{11}{11}\selectfont{\global\setmainfont{Arial}{4\ (3.2\%)}}}} \\





\multicolumn{1}{>{\raggedright}p{\dimexpr 2.97in+0\tabcolsep}}{\textcolor[HTML]{000000}{\fontsize{11}{11}\selectfont{\global\setmainfont{Arial}{Mobil\ pribadi}}}} & \multicolumn{1}{>{\centering}p{\dimexpr 0.93in+0\tabcolsep}}{\textcolor[HTML]{000000}{\fontsize{11}{11}\selectfont{\global\setmainfont{Arial}{0\ (0\%)}}}} & \multicolumn{1}{>{\centering}p{\dimexpr 1.77in+0\tabcolsep}}{\textcolor[HTML]{000000}{\fontsize{11}{11}\selectfont{\global\setmainfont{Arial}{6\ (5.0\%)}}}} & \multicolumn{1}{>{\centering}p{\dimexpr 1.2in+0\tabcolsep}}{\textcolor[HTML]{000000}{\fontsize{11}{11}\selectfont{\global\setmainfont{Arial}{15\ (3.2\%)}}}} & \multicolumn{1}{>{\centering}p{\dimexpr 2.48in+0\tabcolsep}}{\textcolor[HTML]{000000}{\fontsize{11}{11}\selectfont{\global\setmainfont{Arial}{5\ (2.2\%)}}}} & \multicolumn{1}{>{\centering}p{\dimexpr 2.36in+0\tabcolsep}}{\textcolor[HTML]{000000}{\fontsize{11}{11}\selectfont{\global\setmainfont{Arial}{3\ (7.0\%)}}}} & \multicolumn{1}{>{\centering}p{\dimexpr 1.44in+0\tabcolsep}}{\textcolor[HTML]{000000}{\fontsize{11}{11}\selectfont{\global\setmainfont{Arial}{110\ (20\%)}}}} & \multicolumn{1}{>{\centering}p{\dimexpr 1.52in+0\tabcolsep}}{\textcolor[HTML]{000000}{\fontsize{11}{11}\selectfont{\global\setmainfont{Arial}{13\ (10\%)}}}} \\





\multicolumn{1}{>{\raggedright}p{\dimexpr 2.97in+0\tabcolsep}}{\textcolor[HTML]{000000}{\fontsize{11}{11}\selectfont{\global\setmainfont{Arial}{Sepeda}}}} & \multicolumn{1}{>{\centering}p{\dimexpr 0.93in+0\tabcolsep}}{\textcolor[HTML]{000000}{\fontsize{11}{11}\selectfont{\global\setmainfont{Arial}{0\ (0\%)}}}} & \multicolumn{1}{>{\centering}p{\dimexpr 1.77in+0\tabcolsep}}{\textcolor[HTML]{000000}{\fontsize{11}{11}\selectfont{\global\setmainfont{Arial}{0\ (0\%)}}}} & \multicolumn{1}{>{\centering}p{\dimexpr 1.2in+0\tabcolsep}}{\textcolor[HTML]{000000}{\fontsize{11}{11}\selectfont{\global\setmainfont{Arial}{8\ (1.7\%)}}}} & \multicolumn{1}{>{\centering}p{\dimexpr 2.48in+0\tabcolsep}}{\textcolor[HTML]{000000}{\fontsize{11}{11}\selectfont{\global\setmainfont{Arial}{8\ (3.6\%)}}}} & \multicolumn{1}{>{\centering}p{\dimexpr 2.36in+0\tabcolsep}}{\textcolor[HTML]{000000}{\fontsize{11}{11}\selectfont{\global\setmainfont{Arial}{0\ (0\%)}}}} & \multicolumn{1}{>{\centering}p{\dimexpr 1.44in+0\tabcolsep}}{\textcolor[HTML]{000000}{\fontsize{11}{11}\selectfont{\global\setmainfont{Arial}{1\ (0.2\%)}}}} & \multicolumn{1}{>{\centering}p{\dimexpr 1.52in+0\tabcolsep}}{\textcolor[HTML]{000000}{\fontsize{11}{11}\selectfont{\global\setmainfont{Arial}{1\ (0.8\%)}}}} \\





\multicolumn{1}{>{\raggedright}p{\dimexpr 2.97in+0\tabcolsep}}{\textcolor[HTML]{000000}{\fontsize{11}{11}\selectfont{\global\setmainfont{Arial}{Sepeda\ motor\ pribadi}}}} & \multicolumn{1}{>{\centering}p{\dimexpr 0.93in+0\tabcolsep}}{\textcolor[HTML]{000000}{\fontsize{11}{11}\selectfont{\global\setmainfont{Arial}{2\ (8.0\%)}}}} & \multicolumn{1}{>{\centering}p{\dimexpr 1.77in+0\tabcolsep}}{\textcolor[HTML]{000000}{\fontsize{11}{11}\selectfont{\global\setmainfont{Arial}{57\ (48\%)}}}} & \multicolumn{1}{>{\centering}p{\dimexpr 1.2in+0\tabcolsep}}{\textcolor[HTML]{000000}{\fontsize{11}{11}\selectfont{\global\setmainfont{Arial}{296\ (63\%)}}}} & \multicolumn{1}{>{\centering}p{\dimexpr 2.48in+0\tabcolsep}}{\textcolor[HTML]{000000}{\fontsize{11}{11}\selectfont{\global\setmainfont{Arial}{125\ (56\%)}}}} & \multicolumn{1}{>{\centering}p{\dimexpr 2.36in+0\tabcolsep}}{\textcolor[HTML]{000000}{\fontsize{11}{11}\selectfont{\global\setmainfont{Arial}{23\ (53\%)}}}} & \multicolumn{1}{>{\centering}p{\dimexpr 1.44in+0\tabcolsep}}{\textcolor[HTML]{000000}{\fontsize{11}{11}\selectfont{\global\setmainfont{Arial}{394\ (71\%)}}}} & \multicolumn{1}{>{\centering}p{\dimexpr 1.52in+0\tabcolsep}}{\textcolor[HTML]{000000}{\fontsize{11}{11}\selectfont{\global\setmainfont{Arial}{94\ (75\%)}}}} \\

\ascline{1pt}{000000}{1-8}



\end{longtable}



\arrayrulecolor[HTML]{000000}

\global\setlength{\arrayrulewidth}{\Oldarrayrulewidth}

\global\setlength{\tabcolsep}{\Oldtabcolsep}

\renewcommand*{\arraystretch}{1}

Tabel silang yang merupakan hasil dari \emph{script} di atas adalah seperti berikut. Baris menunjukkan kategori dalam variabel \texttt{kendaraan\ utama}, sementara kolom menunjukkan kategori dalam variabel \texttt{jenis\ tempat\ tinggal}. Nilai-nilai dalam sel-selnya adalah frekuensi/jumlah data yang termasuk ke dalam kondisi berupa kategori kedua variabel.

Perhatikan cara menginterpretasi informasi dari tabel tersebut:

\begin{itemize}
\item
  jumlah pengguna layanan online yang tinggal di kos bersama-sama adalah 19 orang, yang setara dengan 16\%.
\item
  tidak ada pengguna sepeda yang tinggal di Asrama, Kos bersama-sama, atau mengontrak rumah secara pribadi. Keberadaannya juga langka pada kalangan mahasiswa yang tinggal di rumah pribadi atau di saudaranya.
\end{itemize}

\begin{figure}
\centering
\includegraphics[width=6.25in,height=\textheight,keepaspectratio,alt={Tabel Silang antara variabel jenis kendaraan utama dengan jenis tempat tinggal}]{images/hasil tabel silang.png}
\caption{Tabel Silang antara variabel jenis kendaraan utama dengan jenis tempat tinggal}
\end{figure}

Selain dengan \texttt{tbl\_summary()} yang gunanya cenderung untuk presentasi, kita dapat menggunakan perintah \texttt{table()} untuk menghasilkan tabel silang yang siap dioperasikan ke dalam perhitungan koefisien korelasi di tahap selanjutnya. Perintah \texttt{table()} mengambil dua masukan, yakni vektor variabel-variabel yang kita akan analisis, dipisahkan dengan tanda koma (\texttt{,}).

\begin{verbatim}
##                           
##                            Berjalan kaki Layanan online
##   Asrama                              23              0
##   Kos bersama-sama                    20             19
##   Kos sendiri                         39             86
##   Rumah mengontrak bersama            17             28
##   Rumah mengontrak pribadi            11              6
##   Rumah pribadi                        0             37
##   Rumah saudara                        1             12
##                           
##                            Menumpang dengan teman/keluarga Mobil pribadi Sepeda
##   Asrama                                                 0             0      0
##   Kos bersama-sama                                      18             6      0
##   Kos sendiri                                           24            15      8
##   Rumah mengontrak bersama                              40             5      8
##   Rumah mengontrak pribadi                               0             3      0
##   Rumah pribadi                                         11           110      1
##   Rumah saudara                                          4            13      1
##                           
##                            Sepeda motor pribadi
##   Asrama                                      2
##   Kos bersama-sama                           57
##   Kos sendiri                               296
##   Rumah mengontrak bersama                  125
##   Rumah mengontrak pribadi                   23
##   Rumah pribadi                             394
##   Rumah saudara                              94
\end{verbatim}

\textbf{Latihan: buatlah tabel silang antara variabel \texttt{Tingkat\ Semester} dan \texttt{Uang\ Saku} baik dengan \texttt{gt\_summary} maupun dengan \texttt{table()}!}

\subsection{Analisis Korelasi Antarvariabel Nominal}\label{analisis-korelasi-antarvariabel-nominal}

Dalam analisis korelasi variabel nominal, kita mengukur \textbf{keberadaan} dan juga \textbf{kekuatan hubungan}. Ukuran yang kita pakai untuk menyatakan hal tersebut adalah \textbf{koefisien korelasi variabel nominal}, yang di antaranya adalah koefisien \(\phi\) \textbf{(phi)}, \textbf{V Cramer} dan \textbf{koefisien C}. Ketiga koefisien tersebut dihitung dari nilai \(\chi^2\) (``chi kuadrat'').

Selain itu, terdapat juga \(\lambda\) \textbf{(lambda)} yang merupakan koefisien korelasi variabel nominal berbasis \emph{error,} atau disebut juga PRE \emph{(proportional reduction of error).}

Ukuran dari keempat koefisien tersebut adalah besarnya yang berkisar \textbf{antara 0 dan 1} dengan \textbf{0 berarti tidak ada hubungan sama sekali} dan \textbf{1 berarti hubungan yang ada sangat kuat.}

\subsubsection{\texorpdfstring{Koefisien Korelasi Variabel Nominal Berbasis \(\chi^2\)}{Koefisien Korelasi Variabel Nominal Berbasis \textbackslash chi\^{}2}}\label{koefisien-korelasi-variabel-nominal-berbasis-chi2}

Yang pertama adalah koefisien korelasi berbasis nilai \(\chi^2\) . Karena berbasis nilai \(\chi^2\), maka kita harus menghitung terlebih dahulu nilai \(\chi^2\) antara dua variabel kita.

Untuk menghitung \(\chi^2\) kita menggunakan dua perintah utama dalam R, yakni \texttt{table()} dan \texttt{chisq.test()}. Perintah \texttt{table()} berfungsi membuat tabel silang yang berguna sebagai ``bahan baku'' perhitungan \(\chi^2\) kita, kemudian perintah \texttt{chisq.test()} menghitung nilai \(\chi^2\) tersebut.

Seperti yang bisa kita lihat, hasil perhitungan berupa variabel R \texttt{chisq\_t4tinggal\_kend} berada di bagian data, dan jika kita perluas, maka di dalamnya terdapat banyak variabel-variabel lain. Variabel-variabel lain tersebut menyimpan berbagai nilai yang menjadi bagian dari hasil perhitungan. Nilai-nilai yang penting untuk

\begin{itemize}
\item
  \texttt{.\$statistic} menyimpan nilai \(\chi^2\) dari tabel. Ini digunakan untuk menghitung kekuatan korelasi
\item
  \texttt{.\$parameter} menyimpan nilai \emph{degree of freedom} (df)
\item
  \texttt{.\$p.value} menyimpan nilai probabilitas untuk menerima \(H_0\) yang berarti tidak ada hubungan antara dua variabel yang dianalisis
\item
  \texttt{.\$expected} menyimpan tabel frekuensi yang diharapkan dari data kita
\end{itemize}

Tampilkan seluruh nilai tersebut

\begin{verbatim}
## X-squared 
##  600.8622
\end{verbatim}

\begin{verbatim}
## df 
## 30
\end{verbatim}

\begin{verbatim}
## [1] 1.963807e-107
\end{verbatim}

\begin{verbatim}
##                                  
##                                       Asrama Kos bersama-sama Kos sendiri
##   Berjalan kaki                    1.7822736         8.554913   33.364162
##   Layanan online                   3.0186256        14.489403   56.508671
##   Menumpang dengan teman/keluarga  1.5574823         7.475915   29.156069
##   Mobil pribadi                    2.4405909        11.714836   45.687861
##   Sepeda                           0.2890173         1.387283    5.410405
##   Sepeda motor pribadi            15.9120103        76.377649  297.872832
##                                  
##                                   Rumah mengontrak bersama
##   Berjalan kaki                                  15.897881
##   Layanan online                                 26.926140
##   Menumpang dengan teman/keluarga                13.892742
##   Mobil pribadi                                  21.770071
##   Sepeda                                          2.578035
##   Sepeda motor pribadi                          141.935132
##                                  
##                                   Rumah mengontrak pribadi Rumah pribadi
##   Berjalan kaki                                  3.0655106     39.423892
##   Layanan online                                 5.1920360     66.771997
##   Menumpang dengan teman/keluarga                2.6788696     34.451509
##   Mobil pribadi                                  4.1978163     53.985870
##   Sepeda                                         0.4971098      6.393064
##   Sepeda motor pribadi                          27.3686577    351.973667
##                                  
##                                   Rumah saudara
##   Berjalan kaki                        8.911368
##   Layanan online                      15.093128
##   Menumpang dengan teman/keluarga      7.787412
##   Mobil pribadi                       12.202954
##   Sepeda                               1.445087
##   Sepeda motor pribadi                79.560051
\end{verbatim}

Kita akan menghitung nilai koefisien V Cramer untuk menyatakan korelasi antara jenis kendaraan yang digunakan sebagai moda ke kampus dan jenis tempat tinggal mahasiswa menggunakan rumus berikut dan berdasarkan nilai \(\chi^2\) yang kita dapatkan dari tes sebelumnya.

\[
V = \sqrt{\frac{\chi^2}{n \times min(c−1,r−1)}}​​
\]

\begin{verbatim}
## [1] 0.2778166
\end{verbatim}

\textbf{Kesimpulan:} Hasil dari perhitungan koefisien V adalah \textbf{0,253}. Angka ini menunjukkan hubungan yang sedikit lemah, karena berada di bawah nilai tengah-tengah yaitu 0,5. Artinya, hubungan antara jenis tempat tinggal dengan jenis kendaraan yang sering digunakan ke kampus tidak begitu kuat.

\textbf{Cara lain:} kita dapat menggunakan pustaka \texttt{rcompanion} untuk menghitung koefisien V Cramer, yaitu dengan perintah \texttt{cramerV()}. Perintah ini mengambil langsung vektor kolom variabel yang kita akan analisis, sehingga menggunakan notasi \texttt{\$} untuk mengakses kolom \texttt{\textasciigrave{}jenis\ tempat\ tinggal\textasciigrave{}} dan \texttt{\textasciigrave{}kendaraan\ utama\textasciigrave{}}.

Tentunya kita harus menginstal pustaka tersebut jika belum memilikinya dengan perintah \texttt{install.packages(\textquotesingle{}rcompanion\textquotesingle{})}.

\begin{verbatim}
## Cramer V 
##   0.2778
\end{verbatim}

\subsubsection{\texorpdfstring{Koefisien Korelasi Variabel Nominal Berbasis \emph{Error}}{Koefisien Korelasi Variabel Nominal Berbasis Error}}\label{koefisien-korelasi-variabel-nominal-berbasis-error}

Untuk analisis korelasi variabel nominal berbasis \emph{error} yang menggunakan koefisien \(\lambda\) kita dapat menggunakan perintah \texttt{lambda()} yang ada di dalam pustaka \texttt{RCPA3}. Terlebih dahulu tentunya lakukan instalasi pustaka tersebut dengan perintah \texttt{install.packages(\textquotesingle{}RCPA3\textquotesingle{})}.

Perintah \texttt{lambda()} mengambil masukan berupa tabel silang seperti yang kita buat pada perhitungan \(\chi^2\), yakni perintah \texttt{table()} .

\subsection{Analisis Korelasi Antarvariabel Ordinal}\label{analisis-korelasi-antarvariabel-ordinal}

Dalam analisis korelasi variabel ordinal, kita tidak hanya dapat mengukur keberadaan dan juga kekuatan hubungan, tetapi juga \textbf{arah hubungan}. Yang dimaksud dengan `arah hubungan' adalah \textbf{searah} atau \textbf{berlawanan}. Hubungan searah berarti semakin tinggi nilai variabel \#1, maka semakin tinggi juga nilai variabel \#2, begitu pula sebaliknya. Akan tetapi, jika arah hubungan berlawanan, semakin tinggi nilai variabel \#1 maka semakin rendah nilai variabel \#2, begitu pula sebaliknya.

Arah hubungan ditunjukkan oleh \textbf{tanda koefisien} dan kekuatannya ditunjukkan oleh \textbf{besar angka koefisien.} Oleh karena itu, nilai korelasi pada variabel ordinal berkisar antara \(-1\) hingga \(+1\) dengan tanda \(-\) untuk menandakan hubungan berlawanan dan \(+\) untuk hubungan searah.

Kita dapat menggunakan perintah \texttt{GKgamma()} dari pustaka \texttt{vcdExtra}. Lakukan instalasi pustaka terlebih dahulu dengan perintah \texttt{install.packages(\textquotesingle{}vcdExtra\textquotesingle{})}.

Perintah \texttt{GKgamma()} mengambil masukan sama seperti \texttt{chisq.test()}, yakni tabel silang yang dihasilkan dari perintah \texttt{table()}.

Dalam praktikum ini, kita akan menganalisis kekuatan dan arah hubungan antara Uang Saku dan Tingkat Semester. Apakah mahasiswa dengan tingkat semester makin tinggi uang sakunya makin tinggi pula, atau sebaliknya?

\begin{verbatim}
## gamma        : -0.009 
## std. error   : 0.033 
## CI           : -0.073 0.056
\end{verbatim}

Seperti halnya nilai variabel hasil perhitungan \texttt{chisq.test()}, hasil perhitungan \texttt{GKgamma()} juga terdiri atas banyak variabel. Nilai gamma sendiri dapat kita akses dengan \texttt{.\$gamma}.

Nilai lain yang penting bagi kita adalah nilai \texttt{.\$C} dan \texttt{.\$D} . Nilai-nilai tersebut adalah nilai operasi dalam sel tabel silang yang merupakan penyusun dalam perhitungan koefisien tersebut mengikuti rumus berikut:

\[
G = \frac{C-D}{C+D}
\]

\begin{verbatim}
## [1] 283397
\end{verbatim}

\begin{verbatim}
## [1] 288282
\end{verbatim}

\textbf{Aktivitas Mandiri 3: Analisis Korelasi Komprehensif {[}STP-9.2, STP-10.2{]}}

\textbf{A. Korelasi Nominal {[}STP-9.2{]}:}

\begin{enumerate}
\def\labelenumi{\arabic{enumi}.}
\tightlist
\item
  \textbf{Fakultas vs Prodi:}

  \begin{itemize}
  \tightlist
  \item
    Hitung V Cramer dan Lambda
  \item
    Interpretasikan kekuatan hubungan
  \end{itemize}
\item
  \textbf{Jenis Kelamin vs Kendaraan Utama:}

  \begin{itemize}
  \tightlist
  \item
    Hitung kedua koefisien
  \item
    Bandingkan dengan hubungan Fak ultas-Prodi
  \end{itemize}
\end{enumerate}

\textbf{B. Korelasi Ordinal {[}STP-10.2{]}:}

\begin{enumerate}
\def\labelenumi{\arabic{enumi}.}
\setcounter{enumi}{2}
\tightlist
\item
  \textbf{Tingkat Semester vs Uang Saku:}

  \begin{itemize}
  \tightlist
  \item
    Hitung koefisien Gamma
  \item
    Interpretasikan arah dan kekuatan hubungan
  \end{itemize}
\end{enumerate}

\begin{center}\rule{0.5\linewidth}{0.5pt}\end{center}

\section{Modul-7: Analisis Hubungan Korelatif: Korelasi Variabel Metrik}\label{modul-7-analisis-hubungan-korelatif-korelasi-variabel-metrik}

Setelah mempelajari modul ini, Anda diharapkan mampu menghasilkan koefisien korelasi variabel di tingkat metrik dengan perangkat lunak komputer. \capaian{STP-11.2}

\begin{center}\rule{0.5\linewidth}{0.5pt}\end{center}

\subsection{Pendahuluan}\label{pendahuluan-3}

Analisis korelasi adalah salah satu teknik analisis yang termasuk ke dalam lingkup statistika bivariat, statistika yang analisisnya melibatkan dua variabel \textbf{sekaligus}. Pada praktikum sebelumnya kita sudah mengenal koefisien-koefisien berikut untuk menganalisis korelasi antara dua variabel nominal dan ordinal:

\begin{itemize}
\tightlist
\item
  V Cramer dan \(\lambda\) (lambda) untuk dua variabel nominal
\item
  Gamma (G) untuk dua variabel ordinal
\end{itemize}

Dalam praktikum ini kita akan mempraktikkan analisis korelasi pada \textbf{dua variabel} \textbf{metrik.} Pada praktikum sebelumnya kita mengetahui bahwa dari koefisien-koefisien korelasi kita dapat mengetahui:

\begin{itemize}
\item
  \textbf{kekuatan} untuk dua variabel nominal
\item
  \textbf{kekuatan} dan \textbf{arah} untuk dua variabel ordinal
\end{itemize}

Untuk dua variabel metrik, selain \textbf{kekuatan} dan juga \textbf{arah hubungan}, kita juga dapat menganalisis \textbf{pola} hubungan tersebut, yakni persebaran titik-titik data secara \textbf{grafis}.

Dalam praktikum ini kita akan mempraktikkan perhitungan dua jenis koefisien: \textbf{Spearman's rho} (\(\rho\)) dan \textbf{Pearson's r.}

\subsection{\texorpdfstring{Pustaka \emph{(Libraries)} yang Diperlukan dan Memuat Data}{Pustaka (Libraries) yang Diperlukan dan Memuat Data}}\label{pustaka-libraries-yang-diperlukan-dan-memuat-data-1}

Seperti biasa, kita perlu memuat pustaka \emph{(libraries)} yang diperlukan dalam pengolahan data kita. Dalam analisis korelasi variabel metrik kita tidak lagi menggunakan tabel silang, tetapi kita langsung menganalisis kolom-kolom yang ada di dataset kita.

\textbf{Memuat dataset}

Kita akan menggunakan dataset keempat kampus di Kota Bandar Lampung dan sekitarnya sebagai bahan. Tulis ulang dan jalankan baris perintah berikut untuk mengolah data keempat kampus

\textbf{Mengatur Faktor untuk Variabel Kategoris: Opsional}

Karena kita hanya akan berkutat dengan variabel-variabel metrik, maka kita bisa saja melewatkan tahap ini.

\subsection{Pola Hubungan Data}\label{pola-hubungan-data}

Dalam analisis korelasi variabel-variabel metrik kita dapat menganalisis satu lagi sifat hubungan dalam dua variabel yang kita perhitungkan: \textbf{pola hubungan data.} Pola hubungan adalah bentuk sebaran titik-titik yang dapat kita lihat dengan diagram pencar \emph{(scatter plot)}.

\textbf{Membuat Diagram Pencar}

Membuat diagram pencar dapat dilakukan dengan menerapkan perintah \texttt{geom\_point()} dari pustaka \texttt{ggplot2}yang dimuat bersama pustaka \texttt{tidyverse}.

Kita akan melihat pola sebaran data kita dilihat dari variabel jarak dari kampus (\texttt{jarak}) dengan biaya yang dikeluarkan untuk transportasi selama sepekan (\texttt{biaya\ sepekan}). Variabel-variabel yang akan kita lihat hubungannya itu kita masukkan ke \texttt{x} dan \texttt{y} yang adalah dalam perintah \texttt{aes()}.

\pandocbounded{\includegraphics[keepaspectratio]{figures/mod07-create-scatterplot-1.pdf}}

Dari diagram yang dihasilkan kita dapat menarik interpretasi hubungan antara kedua variabel secara visual:

\begin{itemize}
\tightlist
\item
  Tidak ada kecenderungan arah hubungan antara jarak tempat tinggal ke kampus dengan biaya transportasi per pekan
\item
  Terdapat responden yang tinggal dekat dengan kampus (\textless10 km) tetapi biayanya tetap tinggi (\textgreater Rp200 ribu), juga yang tinggal jauh dari kampus (20-40 km) tetapi biayanya rendah (\textless Rp200 ribu)
\end{itemize}

Dari hasil diagram ini kita sudah bisa menduga bahwa hubungan antara jarak tempat tinggal ke kampus dengan biaya yang dikeluarkan tidak terlalu erat dan arahnya tidak beraturan.

Akan tetapi, untuk lebih jelas, kita perlu meninjaunya lewat angka koefisien korelasi.

\subsection{\texorpdfstring{Analisis Korelasi Spearman's \(\rho\)}{Analisis Korelasi Spearman's \textbackslash rho}}\label{analisis-korelasi-spearmans-rho}

Setidaknya ada dua kondisi yang menganjurkan kita menganalisis korelasi suatu pasangan variabel metrik dengan koefisien \(\rho\) Spearman:

\begin{itemize}
\item
  Koefisien \(\rho\) Spearman biasanya digunakan untuk menganalisis korelasi dua variabel peringkat \emph{(rank)}. Dengan kata lain, koefisien ini lebih cocok dikenakan pada variabel-variabel dengan tingkat pengukuran \textbf{interval}, seperti peringkat, \emph{rating} atau data lain yang tidak memiliki titik nol absolut yang bermakna.
\item
  Kita \textbf{tidak menemukan adanya hubungan linear} antara dua variabel metrik yang kita analisis. Jika menurut pola data kita ditemukan hubungan linear, kita dianjurkan menggunakan koefisien \(\rho\) Spearman ini.
\end{itemize}

Kita akan menggunakan koefisien korelasi \(\rho\) Spearman ini untuk menganalisis hubungan antara jarak tempat tinggal dengan biaya yang dikeluarkan per pekan.

Koefisien korelasi untuk variabel metrik di R dapat dianalisis dengan perintah \texttt{cor()} yang mengambil masukan berupa vektor data angka variabel-variabel yang kita analisis. Adapun jenis korelasi dapat kita pilih dengan menambahkan argumen \texttt{method\ =} yang dapat bernilai \texttt{"spearman",} \texttt{"kendall"}, atau \texttt{"pearson"}, sesuai dengan metode yang kita gunakan.

\begin{verbatim}
## [1] 0.08576446
\end{verbatim}

Sebagaimana koefisien-koefisien korelasi lainnya, nilai \(\rho\) Spearman berkisar antara \(-1\) hingga \(+1\) yang menyatakan hubungan berlawanan yang kuat hingga hubungan searah yang kuat. Secara umum, tanda positif pada koefisien tersebut menunjukkan hubungan yang searah antara biaya transportasi sepekan dengan jarak tempuh ke kampus. Akan tetapi, dilihat dari besar nilainya, sulit mengatakan bahwa terdapat hubungan yang kuat antara jarak tempuh ke kampus dengan biaya perjalanan sepekan.

\subsection{\texorpdfstring{Analisis Korelasi Pearson's \emph{r}}{Analisis Korelasi Pearson's r}}\label{analisis-korelasi-pearsons-r}

Untuk analisis menggunakan koefisien korelasi Pearson's r, kita akan memodifikasi sedikit data kita. Kita akan melihat hubungan antara jumlah perjalanan di hari kerja \emph{(weekdays)} dengan jarak tempat tinggal ke kampus \textbf{untuk pengguna transportasi online saja}. Dengan demikian, kita perlu membuat dataset terpisah dari dataset utama kita.

Terlebih dahulu, kita perlu membuat variabel khusus \texttt{Jumlah\ Perjalanan\ Weekdays} yang merupakan penjumlahan dari kolom-kolom \texttt{Jumlah\ Perjalanan\ Senin} hingga \texttt{Jumlah\ Perjalanan\ Jumat}. Perhatikan cara pembuatannya yang menggunakan perintah \texttt{rowSums()} dan \texttt{across()} yang merupakan perintah khusus untuk operasi-operasi antarkolom. Tanda \texttt{:} bermakna ``pilih kolom dari jumlah perjalanan hari senin sampai kolom jumlah perjalanan hari jumat''. Hal ini memungkinkan karena dalam dataset kita kolom-kolom tersebut posisinya berdekatan.

Setelah itu, barulah kita membuat dataset khusus pengguna layanan online saja. Kita menggunakan perintah \texttt{filter()} dengan operator \texttt{==} yang bermakna \emph{saringlah data dengan nilai} \texttt{\textasciigrave{}kendaraan\ utama\textasciigrave{}} \emph{sama dengan} \texttt{"Layanan\ online"}\emph{.}

Kita dapat mengecek hasilnya dengan melakukan perintah \texttt{group\_by()} dan \texttt{summarize()}. Hasilnya akan menampilkan \texttt{kendaraan\ utama} kita hanya bernilai \texttt{"Layanan\ online"}

\begin{verbatim}
## # A tibble: 1 x 2
##   `kendaraan utama` jumlah
##   <chr>              <int>
## 1 Layanan online       188
\end{verbatim}

Kemudian, kita bisa menghitung koefisien korelasi Pearson's \(r\)-nya.

\begin{verbatim}
## [1] -0.2980376
\end{verbatim}

Interpretasi hasil koefisien tersebut sama dengan interpretasi koefisien korelasi variabel ordinal, yakni tanda menunjukkan arah hubungan sementara besar angka menunjukkan kekuatan hubungan.

\textbf{Latihan: Interpretasikan nilai korelasi r tersebut}

Kita akan menelaah hubungan antara dua variabel tersebut berdasarkan tampilan visualnya dari diagram pencar \emph{(scatterplot)}.

\pandocbounded{\includegraphics[keepaspectratio]{figures/mod07-scatterplot-online-1.pdf}}

Perhatikan bentuk sebaran titik-titik dalam diagram pencarnya. Secara sekilas, sebaran titik-titik tersebut membentuk formasi kemiringan ke bawah, akan tetapi kerapatannya kecil.

\textbf{Aktivitas Mandiri 3: Analisis Korelasi Metrik Komprehensif {[}STP-11.2{]}}

\begin{enumerate}
\def\labelenumi{\arabic{enumi}.}
\tightlist
\item
  \textbf{Pilih 3 Pasang Variabel Metrik:}

  \begin{itemize}
  \tightlist
  \item
    Buat scatter plot untuk setiap pasangan
  \item
    Pilih Pearson atau Spearman berdasarkan pola
  \item
    Hitung koefisien korelasi
  \item
    Interpretasikan arah dan kekuatan
  \end{itemize}
\item
  \textbf{Jelaskan Pemilihan Metode:}

  \begin{itemize}
  \tightlist
  \item
    Mengapa Pearson atau Spearman untuk setiap pasangan?
  \item
    Pertimbangkan: pola linear vs monoton, outlier, distribusi data
  \end{itemize}
\end{enumerate}

\begin{center}\rule{0.5\linewidth}{0.5pt}\end{center}

\section{Modul-8: Analisis Hubungan Kausalitas: Regresi Linear Sederhana dan Berganda}\label{modul-8-analisis-hubungan-kausalitas-regresi-linear-sederhana-dan-berganda}

Setelah mempelajari modul ini, Anda diharapkan dapat:

\begin{enumerate}
\def\labelenumi{\arabic{enumi}.}
\tightlist
\item
  mampu menghasilkan regresi linear sederhana dengan perangkat lunak komputer \capaian{STP-13.3}
\item
  mampu menghasilkan persamaan regresi linear berganda dengan perangkat lunak komputer \capaian{STP-14.3}
\end{enumerate}

\begin{center}\rule{0.5\linewidth}{0.5pt}\end{center}

\subsection{Pendahuluan}\label{pendahuluan-4}

Analisis regresi linear adalah analisis statistik untuk menyatakan hubungan sebab-akibat (kausalitas) antar minimal dua variabel. Analisis regresi linear yang melibatkan \emph{dua variabel saja} (\emph{satu variabel dependen} dan \emph{satu variabel independen}), kita sebut sebagai \textbf{regresi linear sederhana}, sedangkan analisis regresi linar yang melibatkan lebih dari dua variabel (\emph{satu variabel dependen} dan \emph{lebih dari satu variabel independen}) kita sebut sebagai \textbf{regresi linear berganda \emph{(multiple linear regression)}}

Analisis regresi linear mewajibkan tingkat pengukuran minimal untuk variabel dependennya adalah \textbf{metrik}. Sementara itu, variabel independennya dapat berupa metrik atau bukan (ordinal/nominal).

Analisis regresi linear menghasilkan persamaan yang disebut \textbf{persamaan regresi linear}. Bentuk umum persamaan regresi linear tersebut adalah sebagai berikut.

\[
y = β_0 + β_1x
\]

dengan

\begin{itemize}
\tightlist
\item
  \(y\) adalah variabel dependen (disebut juga \emph{variabel respons} atau \emph{predicted variable})
\item
  \(x\) adalah variabel independen (disebut juga \emph{variabel eksplanatory} atau \emph{predictor variable})
\item
  \(\beta_0\) adalah konstanta yang menjadi \emph{intercept}, yaitu nilai \(y\) ketika \(x=0\)
\item
  \(\beta_1\) adalah koefisien yang menyatakan seberapa besar perubahan \(y\) ketika satu unit nilai \(x\) berubah.
\end{itemize}

Persamaan regresi linear di atas adalah untuk \textbf{analisis regresi linear sederhana}, yakni analisis regresi linear yang melibatkan dua variabel saja. Untuk \textbf{regresi linear berganda}, kita hanya perlu menambahkan pasangan \(\beta\) dan \(x\) lainnya, sehingga bentuk umum untuk persamaan regresi linear berganda adalah seperti berikut.

\[
y = β_0 + β_1x_1 + β_2x_2 + ... + β_kx_k
\]

dengan

\begin{itemize}
\tightlist
\item
  \(x_k\) adalah variabel independen ke-\(k\)
\item
  \(\beta_k\) adalah koefisien untuk variabel independen ke-\(k\) tersebut
\end{itemize}

\subsection{Penjelasan Kasus}\label{penjelasan-kasus}

Dalam praktikum kali ini, kita akan memodelkan hubungan kausal antara dua variabel metrik: \textbf{jarak tempat tinggal ke kampus} dan \textbf{biaya perjalanan sepekan} untuk mahasiswa-mahasiswi yang menggunakan \textbf{kendaraan layanan online saja}. Harga perjalanan menggunakan layanan \emph{online} tentunya dipengaruhi oleh jarak tempuh kendaraan tersebut. Hubungan kausal ini menjadi kasus untuk \textbf{regresi linear sederhana}.

Untuk \textbf{regresi linear berganda}, kita akan melibatkan satu variabel tambahan sebagai variabel independen, yakni variabel berjenis kategoris , sehingga kita memiliki total variabel sejumlah 3 buah (1 variabel dependen, 2 variabel independen).

\subsection{\texorpdfstring{Memuat Pustaka \emph{(Libraries)} yang Diperlukan}{Memuat Pustaka (Libraries) yang Diperlukan}}\label{memuat-pustaka-libraries-yang-diperlukan}

Seperti biasa, kita perlu memuat pustaka \emph{(libraries)} yang diperlukan dalam pengolahan data kita. Seperti halnya juga analisis korelasi variabel metrik, kita tidak lagi menggunakan tabel silang, tetapi kita langsung menganalisis kolom-kolom yang ada di dataset kita.

\subsection{Memuat Dataset}\label{memuat-dataset}

Kita akan menggunakan dataset keempat kampus di Kota Bandar Lampung dan sekitarnya sebagai bahan. Kemudian kita akan menyaring objek-objek yang memiliki nilai variabel \texttt{kendaraan\ utama} sama dengan \texttt{\textquotesingle{}Layanan\ online\textquotesingle{}}.

\subsection{Pola Hubungan Data}\label{pola-hubungan-data-1}

Sebagai pendahuluan, kita akan mengidentifikasi kekuatan, arah, dan pola hubungan antara variabel dependen dan independen kita (biaya perjalanan sepekan dan jarak tempat tinggal-kampus). Untuk itu kita dapat menghitung koefisien korelasi untuk hubungan variabel metrik juga, yakni koefisien korelasi Pearson's \(r\).

Pola hubungan akan kita analisis dengan membuat diagram pencar \emph{(scatter plot)} antara variabel dependen dengan variabel independennya.

\textbf{Menghitung Koefisien Pearson's} \(r\)

Menghitung koefisien korelasi Pearson's \(r\) dapat kita lakukan dengan perintah \texttt{cor()} dengan atribut \texttt{method\ =\ "pearson"} setelah terlebih dahulu menyimpan nilai masing-masing variabel ke dalam suatu vektor.

\begin{verbatim}
## [1] -0.1813941
\end{verbatim}

\begin{quote}
\textbf{Jawablah:} Tuliskan interpretasi kalian terhadap koefisien korelasi antara variabel jarak tempuh dengan biaya perjalanan sepekan.

\emph{Petunjuk: ulas kekuatan serta arah hubungannya serta maknai secara kontekstual ulasan tersebut}
\end{quote}

\textbf{Membuat Diagram Pencar}

Membuat diagram pencar dapat dilakukan dengan menerapkan perintah \texttt{geom\_point()} dari pustaka \texttt{ggplot2} yang dimuat bersama pustaka \texttt{tidyverse}.

\pandocbounded{\includegraphics[keepaspectratio]{figures/mod08-create-initial-scatterplot-1.pdf}}

Setelah Anda melakukan pembuatan diagram pencar tersebut, Anda akan menyadari bahwa kita memiliki \emph{outlier} (pencilan), yakni objek yang memiliki nilai jarak dari tempat tinggal ke kampus mencapai hampir 25 km.

Keberadaan pencilan ini akan mengganggu hasil analisis kita. Mari kita buktikan dengan menghilangkan data pencilan tersebut.

Di sini kita akan menghitung ulang nilai koefisien korelasi kita.

\begin{verbatim}
## [1] -0.3086803
\end{verbatim}

\begin{quote}
\textbf{Jawablah:} Bagaimana perbedaan nilai koefisiennya?
\end{quote}

Kita pun dapat memeriksa hasil perbaikan dataset kita dari diagram pencar yang baru berikut.

\pandocbounded{\includegraphics[keepaspectratio]{figures/mod08-create-clean-scatterplot-1.pdf}}

\subsection{Model Regresi Linear Sederhana}\label{model-regresi-linear-sederhana}

Dalam bagian ini kita akan mempelajari cara menyusun, menampilkan, menafsirkan/menginterpretasi, dan memprediksi nilai variabel dependen berdasarkan model yang kita susun.

\subsubsection{Penyusunan Model}\label{penyusunan-model}

Untuk membuat persamaan model regresi linear, kita akan menggunakan perintah \texttt{lm} yang sudah disediakan secara bawaan oleh R. Hasil dari perintah ini dapat kita simpan sebagai variabel tertentu.

Argumen yang kita gunakan adalah \texttt{var\_dependen\ \textasciitilde{}\ var\_independen} serta nama dataset yang kita gunakan. Tanda \texttt{\textasciitilde{}} (disebut \emph{tilde}) dapat kita masukkan dengan menekan \texttt{Shift} + \texttt{\textasciigrave{}} yang ada di sebelah kiri tombol angka 1 di \emph{keyboard} kita.

Sebelumnya kita sudah mendefinisikan \texttt{y} dan \texttt{x} sebagai variabel untuk vektor biaya transportasi sepekan dan vektor jarak, sehingga kita bisa langsung menggunakannya di atas. Dataset yang kita gunakan adalah dataset mahasiswa yang menggunakan layanan \emph{online} sebagai moda transportasi utama yang kita simpan dalam \texttt{data\_mahasiswa\_online}.

\subsubsection{Penampilan Model}\label{penampilan-model}

Untuk menampilkan hasil penyusunan model regresi linear, kita akan menggunakan perintah \texttt{summary} yang juga sudah disediakan secara bawaan oleh R.

\begin{verbatim}
## 
## Call:
## lm(formula = y ~ x, data = data_mahasiswa_online)
## 
## Residuals:
##     Min      1Q  Median      3Q     Max 
## -79.255 -30.717  -3.562  26.758 120.787 
## 
## Coefficients:
##             Estimate Std. Error t value Pr(>|t|)    
## (Intercept)   92.557      4.598  20.129  < 2e-16 ***
## x             -5.321      1.205  -4.414 1.72e-05 ***
## ---
## Signif. codes:  0 '***' 0.001 '**' 0.01 '*' 0.05 '.' 0.1 ' ' 1
## 
## Residual standard error: 38.11 on 185 degrees of freedom
## Multiple R-squared:  0.09528,    Adjusted R-squared:  0.09039 
## F-statistic: 19.48 on 1 and 185 DF,  p-value: 1.721e-05
\end{verbatim}

Yang ditampilkan dengan perintah tersebut antara lain adalah sebagai berikut. Penjelasan lebih lanjutnya akan diberikan di subbagian selanjutnya.

\begin{itemize}
\item
  \texttt{Call} : bentuk persamaan model yang sudah kita input sebelumnya
\item
  \texttt{Residuals} : informasi residual model
\item
  \texttt{Coefficients} : tabel yang menunjukkan angka-angka dalam persamaan model yang dihasilkan
\item
  Nilai-nilai uji kualitas model seperti \texttt{Residual\ standard\ error} \emph{(standard error} estimasi), \emph{Multiple R-squared} dan \emph{Adjusted R-squared} (koefisien determinasi), serta \emph{F-statistic} (uji signifikansi model).
\end{itemize}

Selain menggunakan \texttt{summary}, kita juga dapat menggunakan perintah \texttt{tbl\_regression}, perintah yang disediakan oleh paket/\emph{library} \texttt{gtsummary}. Hasil dari perintah ini adalah tabel yang sudah diformat sesuai dengan \emph{templat} artikel ilmiah.

\begin{table}[t]
\fontsize{12.0pt}{14.0pt}\selectfont
\begin{tabular*}{\linewidth}{@{\extracolsep{\fill}}lccc}
\toprule
\textbf{Characteristic} & \textbf{Beta} & \textbf{95\% CI} & \textbf{p-value} \\ 
\midrule\addlinespace[2.5pt]
x & -5.3 & -7.7, -2.9 & <0.001 \\ 
\bottomrule
\end{tabular*}
\begin{minipage}{\linewidth}
Abbreviation: CI = Confidence Interval\\
\end{minipage}
\end{table}

Jika kalian menemukan pertanyaan untuk menginstal sebuah paket di \emph{console}, yakni \texttt{broom.helpers}, pilih saja ``Yes'' untuk mengunduh dan menginstalnya.

Secara bawaan, tabel tidak menampilkan nilai konstanta \emph{(intercept)}, menampilkan nilai \emph{confidence level} 95\%, dan menampilkan nilai dari variabel independen sesuai variabel yang kita tetapkan sebelumnya, dalam hal ini adalah \texttt{x}. Untuk menggantinya, kita dapat melakukan penyesuaian perintah sebagai berikut.

\begin{table}[t]
\fontsize{12.0pt}{14.0pt}\selectfont
\begin{tabular*}{\linewidth}{@{\extracolsep{\fill}}lccc}
\toprule
\textbf{Characteristic} & \textbf{Beta} & \textbf{99\% CI} & \textbf{p-value} \\ 
\midrule\addlinespace[2.5pt]
(Intercept) & 93 & 81, 105 & <0.001 \\ 
Jarak tempuh ke kampus, km & -5.3 & -8.5, -2.2 & <0.001 \\ 
\bottomrule
\end{tabular*}
\begin{minipage}{\linewidth}
Abbreviation: CI = Confidence Interval\\
\end{minipage}
\end{table}

\subsubsection{Penafsiran/interpretasi Model}\label{penafsiraninterpretasi-model}

Yang kita akan tafsirkan dari model kita di antaranya adalah:

\begin{itemize}
\item
  persamaan model
\item
  makna nilai konstanta dan koefisien
\item
  uji kualitas model (ANOVA, \texttt{F-statistic})
\item
  uji kualitas variabel (\texttt{Pr(\textgreater{}\textbar{}t\textbar{})})
\end{itemize}

\paragraph{Tafsiran persamaan model}\label{tafsiran-persamaan-model}

Persamaan model dapat kita tafsirkan berdasarkan keluaran dari \texttt{summary} berupa \texttt{Coefficients} atau hasil dari \texttt{tbl\_regression} milik \texttt{gtsummary}.

Berdasarkan keluaran kedua perintah tersebut, kita dapat mengetahui bagaimana persamaan regresi linearnya dengan mengganti bentuk umum persamaan regresi linear dengan angka-angka yang dihasilkan di \texttt{model}.

Bentuk umum:

\[
y = β_0 + β_1x
\]

Dengan mengambil hasil dari \emph{summary(model)}, kita dapat menuliskan persamaan regresi linear kita menjadi:

\[
y = 93 - 5,3x
\]

Dari persamaan model ini kita dapat menggambar garis yang melewati titik-titik data kita. Ini merupakan persamaan garis terbaik dari seluruh kemungkinan persamaan garis yang ada, yang kita tentukan dengan meminimalkan kuadrat terkecil.

\pandocbounded{\includegraphics[keepaspectratio]{figures/mod08-visualize-regression-line-1-1.pdf}}

\begin{itemize}
\item
  \texttt{coef(model){[}1{]}} mengacu pada keluaran fungsi \texttt{coef} dari variabel \texttt{model} kita yang pertama (\texttt{{[}1{]}}), yakni angka \emph{intercept}/konstantanya. Ini menjadi nilai dari atribut \texttt{intercept} yang terletak dalam fungsi \texttt{geom\_abline}
\item
  \texttt{coef(model){[}2{]}} mengacu pada keluaran fungsi \texttt{coef} dari variabel \texttt{model} kita yang kedua ({[}\texttt{2{]}}), yakni angka koefisien variabel independen kita. Ini menjadi nilai dari atribut \texttt{slope} yang terletak dalam fungsi \texttt{geom\_abline} karena nilai koefisien variabel independen menunjukkan kemiringan garis persamaan regresi linear kita
\end{itemize}

Persamaan model dapat kita tafsirkan berdasarkan keluaran dari \texttt{summary} berupa \texttt{Coefficients} atau hasil dari \texttt{tbl\_regression} milik \texttt{gtsummary}.

Berdasarkan keluaran kedua perintah tersebut, kita dapat mengetahui bagaimana persamaan regresi linearnya dengan mengganti bentuk umum persamaan regresi linear dengan angka-angka yang dihasilkan di \texttt{model}.

Bentuk umum:

\[
y = β_0 + β_1x
\]

Dengan mengambil hasil dari \emph{summary(model)}, kita dapat menuliskan persamaan regresi linear kita menjadi:

\[
y = 93 - 5,3x
\]

Dari persamaan model ini kita dapat menggambar garis yang melewati titik-titik data kita. Ini merupakan persamaan garis terbaik dari seluruh kemungkinan persamaan garis yang ada, yang kita tentukan dengan meminimalkan kuadrat terkecil.

\pandocbounded{\includegraphics[keepaspectratio]{figures/mod08-visualize-regression-line-2-1.pdf}}

\begin{itemize}
\item
  \texttt{coef(model){[}1{]}} mengacu pada keluaran fungsi \texttt{coef} dari variabel \texttt{model} kita yang pertama (\texttt{{[}1{]}}), yakni angka \emph{intercept}/konstantanya. Ini menjadi nilai dari atribut \texttt{intercept} yang terletak dalam fungsi \texttt{geom\_abline}
\item
  \texttt{coef(model){[}2{]}} mengacu pada keluaran fungsi \texttt{coef} dari variabel \texttt{model} kita yang kedua ({[}\texttt{2{]}}), yakni angka koefisien variabel independen kita. Ini menjadi nilai dari atribut \texttt{slope} yang terletak dalam fungsi \texttt{geom\_abline} karena nilai koefisien variabel independen menunjukkan kemiringan garis persamaan regresi linear kita
\end{itemize}

\paragraph{Tafsiran nilai konstanta dan koefisien}\label{tafsiran-nilai-konstanta-dan-koefisien}

Persamaan regresi linear ini bermakna nilai konstanta sebesar 93 berarti bahwa dengan tinggal di kampus (\(x=0\)), mahasiswa masih akan mengeluarkan biaya sebesar 93.000 rupiah (ingat bahwa variabel \texttt{biaya\ sepekan} menggunakan satuan ribuan rupiah) untuk mengakses kampus selama sepekan. Kita bisa pahami hal tersebut sebagai perkiraan harga sewa penginapan di kampus.

Sementara itu, nilai koefisien sebesar \(-5,3\) berarti bahwa penambahan jarak sebesar 1 km akan \textbf{mengurangi} biaya transportasi sepekan sebesar 5.300 rupiah. Artinya, makin jauh 1 km seorang mahasiswa tinggal dari kampus, makin kecil pula biaya transportasi sepekannya sebesar 5.300 rupiah.

\begin{quote}
\textbf{Pertanyaan bonus:} Mengapa mahasiswa yang tinggal lebih jauh dari kampus biaya perjalanan sepekannya lebih kecil? Bukankah biaya transportasi makin jauh malah makin mahal?
\end{quote}

\paragraph{Tafsiran uji kualitas model}\label{tafsiran-uji-kualitas-model}

Uji kualitas model kita terdiri atas \emph{standard error} dari residual, koefisien determinasi atau \(R^2\), dan uji signifikansi persamaan menggunakan statistik \(F\) (\texttt{F-statistic}) yang merupakan hasil ANOVA \emph{(analysis of variance)}.

\begin{itemize}
\item
  \emph{Standard error} dari residual mencerminkan rata-rata galat (residu), yaitu selisih nilai variabel dependen yang dihitung (biaya perjalanan sepekan hasil prediksi dari model) dengan nilai variabel dependen yang kita peroleh dari data.

  \begin{itemize}
  \tightlist
  \item
    Hasil \texttt{summary} menunjukkan angka 38,11 yang artinya rata-rata kesalahan prediksi dari model kita adalah sebesar 38.110 rupiah.
  \end{itemize}
\item
  Koefisien determinasi atau \(R^2\) menyatakan kesesuaian \emph{(fit)} dari model kita, yakni seberapa banyak proporsi variansi data variabel dependen kita (biaya perjalanan sepekan) dijelaskan oleh variabel independennya (jarak dari kampus ke tempat tinggal).

  Sisa dari angka \(R^2\) ini adalah banyak variansi yang tidak dijelaskan oleh variabel independen, artinya dijelaskan oleh variabel lainnya yang tidak ada dalam model dan juga keacakan (misal dalam pengambilan data atau kejadian acak).

  \begin{itemize}
  \item
    Hasil \texttt{summary} menunjukkan angka 0,095 yang berarti hanya 9,5\% variansi biaya perjalanan sepekan dijelaskan oleh variansi jarak perjalanan ke kampus.
  \item
    Makin tinggi nilai \(R^2\) dari model kita, makin besar proporsi variansi variabel dependen kita dijelaskan oleh variabel independennya.
  \item
    Hal ini menunjukkan bahwa pengaruh dari variabel independen kita terhadap variabel dependennya kuat
  \end{itemize}
\item
  Koefisien uji signifikansi (\(F\)) (\texttt{F-statistics}) adalah uji hipotesis ANOVA \emph{(analysis of variance)} yang membuktikan hipotesis nol bahwa tidak ada hubungan antara variabel dependen dengan variabel independen yang kita modelkan. Penerimaan hipotesis nol ini ditandai dari nilai signifikansi (\texttt{p-value}) yang lebih kecil dari 5\% (\texttt{0.05}).

  \begin{itemize}
  \item
    Hasil \texttt{summary} menunjukkan bahwa nilai \texttt{p-value} kita adalah \texttt{1,721e-05}. Angka tersebut menyatakan notasi ilmiah \(1,721\times10^{-5}\) yang berarti bahwa nilai \texttt{p-value} kita sangat-sangat lebih kecil dari \texttt{0.05.}
  \item
    Ini artinya kita tidak dapat menerima hipotesis nol kita yang menyatakan tidak ada hubungan antara variabel biaya perjalanan sepekan (variabel dependen) dengan variabel jarak tempuh ke kampus (variabel independen).
  \end{itemize}
\end{itemize}

\textbf{Simpulan tafsiran kualitas model}: kita sudah melihat bahwa hubungan kausal antara variabel jarak ke kampus terhadap biaya perjalanan sepekan adalah terbalik dengan besar kemiringan 5,3. Hubungan kausal tersebut benar-benar ada berdasarkan hasil dari koefisien uji signifikansi (ANOVA) yang \emph{p-value-}nya \textless0,05. Akan tetapi, hubungan tersebut tidak cukup kuat karena nilai koefisien determinasi yang kecil, yakni hanya 9\% saja.

\paragraph{Tafsiran uji kualitas variabel}\label{tafsiran-uji-kualitas-variabel}

Uji kualitas variabel adalah pengujian terhadap signifikansi atau kepentingan variabel independen kita. Hasil pengujian ini diperlihatkan oleh angka \emph{p-value} dari uji t (\texttt{t\ value})-nya, yakni yang ada di kolom \texttt{Pr(\textgreater{}\textbar{}t\textbar{})}.

Apabila nilai \texttt{Pr(\textgreater{}\textbar{}t\textbar{})} kita \textgreater0,05 maka variabel independen kita dianggap \textbf{tidak signifikan}, karena artinya nilai koefisien yang ada di kolom \texttt{Estimate} \textbf{sebenarnya adalah nol} sehingga \textbf{bisa dikeluarkan dari persamaan regresi linear}.

Sebaliknya, jika nilai \texttt{Pr(\textgreater{}\textbar{}t\textbar{})} kita \textless0,05 maka variabel independen kita dianggap \textbf{signifikan}, artinya nilai koefisien yang ada di kolom \texttt{Estimate} \textbf{sebenarnya adalah nilai koefisien tersebut.}

Nilai \texttt{Pr(\textgreater{}\textbar{}t\textbar{})} variabel \texttt{x} adalah \texttt{1,721e-05} yang berarti koefisien senilai \texttt{-5.321} adalah \textbf{signifikan}.

\subsubsection{Melakukan Prediksi}\label{melakukan-prediksi}

Kita melakukan prediksi dengan menggunakan perintah \texttt{predict}. Misalkan kita ingin melakukan prediksi biaya perjalanan sepekan untuk mahasiswa yang tinggal di jarak 90 dan 120 km dari kampus

\begin{verbatim}
##         1         2 
## -386.3117 -545.9345
\end{verbatim}

\begin{itemize}
\item
  Perintah \texttt{tibble} adalah perintah untuk membuat dataset berisi kolom \texttt{x} yang terdiri atas 2 nilai: 90 km dan 120 km. Kita menyimpan dataset contoh tersebut ke dalam variabel bernama \texttt{bahan\_pred}
\item
  Dalam membuat dataset untuk prediksi kita harus memperhatikan nama variabel/nama kolom yang kita berikan dalam perintah \texttt{tibble}. Nama kolom yang akan kita prediksi harus \textbf{sama persis} dengan nama variabel yang kita nyatakan dalam perintah \texttt{lm}.
\end{itemize}

\subsection{Model Regresi Linear Berganda}\label{model-regresi-linear-berganda}

Dalam bagian ini kita hanya akan mempelajari penggunaan variabel \emph{dummy} untuk meningkatkan kekuatan prediksi kita. Di akhir, Anda akan menafsirkan hal-hal yang sudah Anda pelajari pada bagian regresi linear.

Kita akan menggunakan variabel \texttt{jenis\ tempat\ tinggal} untuk variabel \emph{dummy}. Pertama, kita akan mengatur jenis nilai pada kolom \texttt{jenis\ tempat\ tinggal} menggunakan perintah \texttt{factor}.

Kedua, kita akan membuat model seperti halnya yang kita lakukan dalam analisis regresi linear sederhana. Kita akan menggunakan nama yang lebih representatif.

\begin{verbatim}
## 
## Call:
## lm(formula = biaya ~ jarak + ling, data = data_mahasiswa_online)
## 
## Residuals:
##     Min      1Q  Median      3Q     Max 
## -73.743 -23.421  -2.143  19.424 114.322 
## 
## Coefficients:
##                              Estimate Std. Error t value Pr(>|t|)    
## (Intercept)                    98.103      8.104  12.105  < 2e-16 ***
## jarak                          -4.954      1.118  -4.431 1.63e-05 ***
## lingKos sendiri               -20.779      8.787  -2.365   0.0191 *  
## lingRumah mengontrak bersama  -17.198     10.409  -1.652   0.1002    
## lingRumah mengontrak pribadi   27.744     16.031   1.731   0.0852 .  
## lingRumah pribadi              20.348      9.866   2.062   0.0406 *  
## lingRumah saudara              10.365     12.689   0.817   0.4151    
## ---
## Signif. codes:  0 '***' 0.001 '**' 0.01 '*' 0.05 '.' 0.1 ' ' 1
## 
## Residual standard error: 34.23 on 180 degrees of freedom
## Multiple R-squared:  0.2899, Adjusted R-squared:  0.2663 
## F-statistic: 12.25 on 6 and 180 DF,  p-value: 1.538e-11
\end{verbatim}

Selanjutnya, kita akan menafsirkan hasil dari summary model ini.

\subsubsection{\texorpdfstring{Penjelasan Variabel \emph{Dummy}}{Penjelasan Variabel Dummy}}\label{penjelasan-variabel-dummy}

Dalam persamaan model kita tersebut, variabel \texttt{ling} atau \texttt{jenis\ tempat\ tinggal} diubah menjadi variabel \emph{dummy} :

\begin{enumerate}
\def\labelenumi{\arabic{enumi}.}
\tightlist
\item
  \texttt{lingKos\ sendiri},
\item
  \texttt{lingRumah\ mengontrak\ bersama},
\item
  \texttt{lingRumah\ mengontrak\ pribadi},
\item
  \texttt{lingRumah\ pribadi}, dan
\item
  \texttt{lingRumah\ saudara}.
\end{enumerate}

Nama variabel-variabel tersebut tak lain adalah gabungan kata \texttt{ling} dengan tiap-tiap kategori dalam variabel \texttt{jenis\ tempat\ tinggal}.

Perhatikan bahwa kita jadi memiliki 5 variabel \emph{dummy} dari satu variabel kategoris \texttt{jenis\ tempat\ tinggal} yang terdiri atas 6 kategori nilai: ``Kos bersama-sama'', ``Kos sendiri'', ``Rumah mengontrak bersama'', ``Rumah mengontrak pribadi'', ``Rumah pribadi'', dan ``Rumah saudara.''

Variabel \emph{dummy} adalah variabel yang bernilai 0 (nol) atau 1 (satu) saja. Nol artinya variabel \emph{dummy} tersebut bernilai ``salah'' atau ``tidak'', sedangkan satu artinya variabel \emph{dummy} tersebut bernilai ``benar'' atau ``ya.''

Variabel-variabel \emph{dummy} yang 5 buah ini hanya akan bernilai 1 atau 0 sesuai dengan kategori dari variabel \texttt{jenis\ tempat\ tinggal} suatu objek. Perhatikan bahwa kita tidak punya ``\texttt{lingKos\ sendiri}'' dalam variabel \emph{dummy} kita. Ini artinya apabila kategori \texttt{jenis\ tempat\ tinggal} suatu objek adalah ``Kos sendiri'', \textbf{kelima variabel \emph{dummy} akan bernilai 0.}

Tabel berikut merangkum nilai-nilai variabel \emph{dummy} untuk setiap kategori nilai variabel \texttt{jenis\ tempat\ tinggal}.

\begin{longtable}[]{@{}
  >{\raggedright\arraybackslash}p{(\linewidth - 10\tabcolsep) * \real{0.1727}}
  >{\raggedright\arraybackslash}p{(\linewidth - 10\tabcolsep) * \real{0.1223}}
  >{\raggedright\arraybackslash}p{(\linewidth - 10\tabcolsep) * \real{0.2158}}
  >{\raggedright\arraybackslash}p{(\linewidth - 10\tabcolsep) * \real{0.2158}}
  >{\raggedright\arraybackslash}p{(\linewidth - 10\tabcolsep) * \real{0.1367}}
  >{\raggedright\arraybackslash}p{(\linewidth - 10\tabcolsep) * \real{0.1367}}@{}}
\caption{Nilai variabel dummy untuk setiap kategori}\tabularnewline
\toprule\noalign{}
\begin{minipage}[b]{\linewidth}\raggedright
Kategori
\end{minipage} & \begin{minipage}[b]{\linewidth}\raggedright
\texttt{lingKos\ sendiri}
\end{minipage} & \begin{minipage}[b]{\linewidth}\raggedright
\texttt{lingRumah\ mengontrak\ bersama}
\end{minipage} & \begin{minipage}[b]{\linewidth}\raggedright
\texttt{lingRumah\ mengontrak\ pribadi}
\end{minipage} & \begin{minipage}[b]{\linewidth}\raggedright
\texttt{lingRumah\ pribadi}
\end{minipage} & \begin{minipage}[b]{\linewidth}\raggedright
\texttt{lingRumah\ saudara}
\end{minipage} \\
\midrule\noalign{}
\endfirsthead
\toprule\noalign{}
\begin{minipage}[b]{\linewidth}\raggedright
Kategori
\end{minipage} & \begin{minipage}[b]{\linewidth}\raggedright
\texttt{lingKos\ sendiri}
\end{minipage} & \begin{minipage}[b]{\linewidth}\raggedright
\texttt{lingRumah\ mengontrak\ bersama}
\end{minipage} & \begin{minipage}[b]{\linewidth}\raggedright
\texttt{lingRumah\ mengontrak\ pribadi}
\end{minipage} & \begin{minipage}[b]{\linewidth}\raggedright
\texttt{lingRumah\ pribadi}
\end{minipage} & \begin{minipage}[b]{\linewidth}\raggedright
\texttt{lingRumah\ saudara}
\end{minipage} \\
\midrule\noalign{}
\endhead
\bottomrule\noalign{}
\endlastfoot
Kos sendiri & 1 & 0 & 0 & 0 & 0 \\
Rumah mengontrak bersama & 0 & 1 & 0 & 0 & 0 \\
Rumah mengontrak pribadi & 0 & 0 & 1 & 0 & 0 \\
Rumah pribadi & 0 & 0 & 0 & 1 & 0 \\
Rumah saudara & 0 & 0 & 0 & 0 & 1 \\
Kos bersama & 0 & 0 & 0 & 0 & 0 \\
\end{longtable}

\subsubsection{\texorpdfstring{Interpretasi Variabel \emph{Dummy}}{Interpretasi Variabel Dummy}}\label{interpretasi-variabel-dummy}

Dalam menginterpretasi variable \emph{dummy} kita tidak bisa menggunakan \texttt{Pr(\textgreater{}\textbar{}t\textbar{})} semata, karena variabel \emph{dummy} pada hakikatnya sepaket. Jadi, walaupun nilai \texttt{Pr(\textgreater{}\textbar{}t\textbar{})} sebuah variabel \emph{dummy} adalah \textgreater0,05 ia akan tetap berada dalam persamaan regresi linear.

Interpretasi koefisien variabel \emph{dummy} adalah dengan memahami bahwa nilai variabel dependen berubah sesuai nilai koefisien apabila variabel \emph{dummy} tersebut bernilai \textbf{benar} atau \textbf{1}. Artinya, jika mahasiswa tinggal di kosan sendirian (\texttt{lingKos\ sendiri\ =\ 1}), maka biaya transportasi sepekan berkurang sebesar \texttt{20.779} yang juga berarti sebesar 20,7 ribu rupiah. Jika mahasiswa tinggal di rumah pribadi (\texttt{lingRumah\ pribadi\ =\ 1}) maka biaya transportasi sepekan bertambah sebesar \texttt{20.348} atau 20,3 ribu rupiah.

\subsubsection{Prediksi Model}\label{prediksi-model}

Dari persamaan model regresi linear kita, kita juga dapat memprediksi nilai biaya perjalanan sepekan untuk mahasiswa-mahasiswa dengan kondisi lain. Misalnya kita ingin memprediksi biaya sepekan transportasi mahasiswa yang tinggal di Rumah saudara berjarak 7,9 km dari kampus dan di Rumah kontrakan bersama yang berjarak 10 km dari kampus.

\begin{verbatim}
##        1        2 
## 69.33008 31.36362
\end{verbatim}

\begin{quote}
\textbf{Jawablah:}

\begin{enumerate}
\def\labelenumi{\alph{enumi}.}
\tightlist
\item
  Bagaimana persamaan model regresi linear berganda kita?
\item
  Tafsirlah kualitas model kita, mulai uji kualitas model (\(F\)), \(R^2\), serta uji kualitas setiap variabel
\item
  Apa saja yang meningkatkan dan menurunkan biaya perjalanan sepekan mahasiswa?
\end{enumerate}
\end{quote}

\textbf{Aktivitas Mandiri 4: Analisis Regresi Komprehensif {[}STP-13.3, STP-14.3{]}}

\textbf{A. Regresi Linear Sederhana {[}STP-13.3{]}:}

Model: \texttt{jumlah\ perjalanan\ weekdays} vs \texttt{jarak}
1. Buat model sederhana
2. Tuliskan persamaan regresi
3. Interpretasikan intercept dan slope
4. Uji kualitas model (F-statistic, R²)

\textbf{B. Regresi Linear Berganda {[}STP-14.3{]}:}

Model: \texttt{jumlah\ perjalanan\ weekdays} \textasciitilde{} \texttt{jarak} + \texttt{umur} + \texttt{jenis\ tempat\ tinggal}
1. Buat model berganda
2. Tuliskan persamaan lengkap
3. Identifikasi variabel signifikan
4. Bandingkan R²

\textbf{C. Interpretasi:}
- Faktor yang meningkatkan/menurunkan jumlah perjalanan?
- Proporsi variansi yang dijelaskan?

\begin{center}\rule{0.5\linewidth}{0.5pt}\end{center}

\section{Modul-9: Analisis Hubungan Multivariat Interdependensi - Analisis Faktor dan PCA}\label{modul-9-analisis-hubungan-multivariat-interdependensi---analisis-faktor-dan-pca}

Setelah mempelajari modul ini, Anda diharapkan dapat menghasilkan komponen prinsip menggunakan perangkat lunak komputer \capaian{STP-14.3}

\begin{center}\rule{0.5\linewidth}{0.5pt}\end{center}

\subsection{Pendahuluan}\label{pendahuluan-5}

Analisis Komponen Prinsip (\emph{Principal Component Analysis}, PCA) dan Analisis Faktor (\emph{Common Factor Analysis}) adalah metode analisis multivariat yang digunakan untuk meringkas atau mereduksi jumlah variabel yang banyak menjadi beberapa dimensi baru (disebut komponen atau faktor) yang lebih sedikit, namun tetap merepresentasikan informasi dari variabel asli.

Kedua metode ini termasuk dalam analisis interdependensi, di mana seluruh variabel dianggap setara dan saling berhubungan satu sama lain, tanpa ada pembagian variabel independen dan dependen.

\subsection{Studi Kasus}\label{studi-kasus}

Kita akan menggunakan data \citet{bindar2022faktor}, yaitu penelitian mengenai preferensi masyarakat Kota Bandung dalam mengakses lokasi \emph{Car-Free Day} (CFD). Terdapat 12 variabel yang akan dianalisis:

\begin{enumerate}
\def\labelenumi{\arabic{enumi}.}
\tightlist
\item
  \texttt{ongkos}: Total biaya perjalanan
\item
  \texttt{bparkir}: Biaya parkir
\item
  \texttt{durasi}: Durasi perjalanan
\item
  \texttt{bareng}: Jumlah rombongan dalam perjalanan
\item
  \texttt{toplajur}: Jumlah lajur jalan terbanyak yang dilalui
\item
  \texttt{usia}: Usia pelaku perjalanan
\item
  \texttt{jmlmotor}: Jumlah sepeda motor di rumah tangga
\item
  \texttt{jmlmobil}: Jumlah mobil di rumah tangga
\item
  \texttt{jmlsepeda}: Jumlah sepeda di rumah tangga
\item
  \texttt{jmldewasa}: Jumlah orang dewasa dalam rumah tangga
\item
  \texttt{jmlanak}: Jumlah anak-anak dalam rumah tangga
\item
  \texttt{jarak}: Jarak tempuh dari rumah ke lokasi CFD
\end{enumerate}

\subsection{\texorpdfstring{Memuat Pustaka \emph{(Libraries)}}{Memuat Pustaka (Libraries)}}\label{memuat-pustaka-libraries}

Kita membutuhkan paket \texttt{tidyverse} untuk manipulasi data dan \texttt{psych} untuk melakukan uji KMO, Bartlett, serta fungsi analisis faktor dan PCA.

\subsection{Asumsi Awal}\label{asumsi-awal}

Sebelum melakukan analisis komponen prinsip atau analisis faktor, idealnya kita perlu memeriksa asumsi dasar, yaitu:

\begin{enumerate}
\def\labelenumi{\arabic{enumi}.}
\tightlist
\item
  \textbf{Linearitas}: Adanya hubungan linear antarvariabel.
\item
  \textbf{Normalitas}: Variabel berdistribusi normal multivariat.
\end{enumerate}

Kita dapat melakukan inspeksi visual menggunakan matriks diagram pencar (\emph{scatter plot matrix}). Di sini kita menggunakan fungsi \texttt{pairs.panels} dari paket \texttt{psych} yang memberikan informasi lengkap berupa scatter plot, histogram, dan nilai korelasi.

\pandocbounded{\includegraphics[keepaspectratio]{figures/mod09-check-assumptions-1.pdf}}

\textbf{Penjelasan Kode dan Output:}

\begin{itemize}
\tightlist
\item
  \textbf{Kode}: Kita menggunakan fungsi \texttt{pairs.panels} dari paket \texttt{psych}. Argumen \texttt{method\ =\ "pearson"} digunakan untuk menghitung koefisien korelasi Pearson. \texttt{hist.col} mengatur warna histogram di diagonal utama. \texttt{ellipses\ =\ TRUE} menambahkan elips yang menggambarkan kovarians dan arah hubungan.
\item
  \textbf{Output}:

  \begin{itemize}
  \tightlist
  \item
    \textbf{Diagonal Utama}: Menampilkan histogram distribusi frekuensi untuk setiap variabel. Kita dapat melihat apakah distribusi variabel mendekati normal (bentuk lonceng) atau menceng (\emph{skewed}).
  \item
    \textbf{Bawah Diagonal}: Menampilkan \emph{scatter plot} (diagram pencar) untuk setiap pasangan variabel. Ini berguna untuk mendeteksi pola hubungan (apakah linear) dan adanya pencilan (\emph{outliers}).
  \item
    \textbf{Atas Diagonal}: Menampilkan nilai koefisien korelasi Pearson (\(r\)). Ukuran font angka korelasi menyesuaikan dengan besarnya nilai korelasi (semakin besar nilai, semakin besar font). \textbf{Interpretasi Singkat}:
  \item
    \textbf{Normalitas}: Pada diagonal utama, histogram yang berbentuk lonceng menunjukkan distribusi mendekati normal. Distribusi yang menceng (\emph{skewed}) atau tidak simetris menandakan ketidaknormalan.
  \item
    \textbf{Linearitas}: Perhatikan bagian bawah diagonal (Scatter Plot).

    \begin{itemize}
    \tightlist
    \item
      \textbf{Garis Tren}: Indikasi linearitas dapat dilihat dari garis tren (biasanya berwarna merah) yang terbentuk di antara titik-titik data. Jika garis tersebut cenderung lurus, maka hubungan antar variabel bersifat linear.
    \item
      \textbf{Elips}: Bentuk elips menunjukkan kekuatan korelasi. \textbf{Elips yang pipih (sempit)} menandakan korelasi kuat dan hubungan yang jelas. Sebaliknya, \textbf{elips yang cenderung bulat} menandakan korelasi yang lemah.
    \item
      Perbesar grafik dengan membuka hasil di jendela baru atau klik `Zoom' jika ia muncul di panel `Plots'.
    \end{itemize}
  \end{itemize}
\end{itemize}

\textbf{Latihan 2:}

Berdasarkan \emph{Scatter Plot Matrix} di atas:

\begin{enumerate}
\def\labelenumi{\arabic{enumi}.}
\tightlist
\item
  Sebutkan satu variabel yang menurut Anda memiliki distribusi mendekati normal (lihat histogram diagonal)!
\item
  Sebutkan pasangan variabel yang memiliki hubungan linear cukup kuat (lihat bentuk elips/sebaran data)!
\end{enumerate}

\subsection{Mempersiapkan Data}\label{mempersiapkan-data}

Kita memastikan kembali variabel yang akan digunakan dalam analisis ini.

\begin{verbatim}
## Rows: 319
## Columns: 12
## $ ongkos    <dbl> 464.40, 464.40, 464.40, 0.00, 0.00, 0.00, 0.00, 0.00, 0.00, ~
## $ bparkir   <dbl> 2000, 2000, 3000, 0, 0, 0, 0, 0, 0, 0, 2000, 5000, 0, 0, 0, ~
## $ durasi    <dbl> 20, 10, 15, 30, 5, 30, 5, 30, 5, 9, 30, 30, 10, 30, 2, 30, 6~
## $ bareng    <dbl> 2, 0, 2, 5, 2, 0, 0, 0, 0, 0, 2, 3, 1, 0, 10, 3, 2, 3, 3, 2,~
## $ toplajur  <dbl> 4, 4, 4, 8, 2, 4, 4, 4, 0, 2, 8, 6, 4, 4, 2, 4, 6, 4, 4, 4, ~
## $ usia      <dbl> 45, 46, 50, 25, 22, 63, 22, 60, 20, 19, 16, 46, 22, 28, 50, ~
## $ jmlmotor  <dbl> 2, 2, 3, 0, 2, 1, 1, 2, 1, 0, 2, 1, 1, 4, 5, 1, 3, 1, 1, 1, ~
## $ jmlmobil  <dbl> 0, 1, 0, 0, 3, 1, 1, 1, 0, 0, 2, 1, 0, 0, 1, 0, 0, 0, 0, 0, ~
## $ jmlsepeda <dbl> 2, 4, 0, 2, 3, 2, 1, 1, 0, 0, 0, 2, 0, 2, 6, 1, 4, 1, 1, 0, ~
## $ jmldewasa <dbl> 4, 4, 4, 3, 5, 2, 2, 4, 1, 1, 3, 1, 5, 2, 2, 3, 5, 12, 5, 2,~
## $ jmlanak   <dbl> 1, 1, 0, 0, 0, 0, 1, 0, 0, 0, 2, 2, 0, 0, 0, 0, 3, 3, 0, 1, ~
## $ jarak     <dbl> 3000, 3000, 3000, 7000, 280, 7000, 600, 7200, 0, 500, 5000, ~
\end{verbatim}

\textbf{Latihan 1:}

Berdasarkan output \texttt{glimpse} di atas, jawablah pertanyaan berikut:

\begin{enumerate}
\def\labelenumi{\arabic{enumi}.}
\tightlist
\item
  Berapa jumlah observasi (baris) atau objek dalam data tersebut?
\item
  Berapa jumlah variabel (kolom) yang aktif digunakan dalam analisis?
\end{enumerate}

\subsection{Uji Kelayakan Data (Asumsi)}\label{uji-kelayakan-data-asumsi}

Sebelum melakukan ekstraksi dimensi, kita perlu memastikan data layak untuk dianalisis faktor/PCA. Dua indikator utama adalah \textbf{Uji Bartlett of Sphericity} dan \textbf{Measure of Sampling Adequacy (MSA)} atau \textbf{Kaiser-Meyer-Olkin (KMO)}.

\subsubsection{Uji Bartlett of Sphericity}\label{uji-bartlett-of-sphericity}

Uji ini melihat apakah terdapat korelasi antarvariabel dalam data. Syarat: Nilai \(p < 0,05\).

\begin{verbatim}
## $chisq
## [1] 573.4283
## 
## $p.value
## [1] 5.636504e-82
## 
## $df
## [1] 66
\end{verbatim}

\textbf{Latihan 3:}

Lihat nilai \(p.value\) pada output di atas. Apakah nilai tersebut \(< 0.05\)? Apa kesimpulan Anda mengenai korelasi antar variabel dalam data ini? (Apakah matriks korelasi berbeda secara signifikan dengan matriks identitas?)

\subsubsection{Uji KMO dan MSA}\label{uji-kmo-dan-msa}

Nilai KMO keseluruhan harus \(> 0,5\). Selain itu, nilai MSA per variabel (diagonal pada matriks anti-image correlation) juga harus \(> 0,5\).

\begin{verbatim}
## Kaiser-Meyer-Olkin factor adequacy
## Call: KMO(r = data_analisis)
## Overall MSA =  0.63
## MSA for each item = 
##    ongkos   bparkir    durasi    bareng  toplajur      usia  jmlmotor  jmlmobil 
##      0.66      0.65      0.60      0.63      0.81      0.62      0.61      0.61 
## jmlsepeda jmldewasa   jmlanak     jarak 
##      0.57      0.48      0.44      0.64
\end{verbatim}

Berdasarkan hasil di atas: - Nilai KMO Keseluruhan (\emph{Overall MSA}) = 0,63 (Layak, \textgreater{} 0,5). - Namun, jika dilihat per variabel (\emph{MSA for each item}), terdapat variabel dengan nilai \textless{} 0,5 yaitu \texttt{jmldewasa} (0,48) dan \texttt{jmlanak} (0,44).

Sesuai prosedur, kita harus mengeluarkan variabel yang tidak memenuhi syarat MSA.

\begin{verbatim}
## Kaiser-Meyer-Olkin factor adequacy
## Call: KMO(r = data_analisis_final)
## Overall MSA =  0.65
## MSA for each item = 
##    ongkos   bparkir    durasi    bareng  toplajur      usia  jmlmotor  jmlmobil 
##      0.66      0.65      0.60      0.63      0.82      0.62      0.64      0.62 
## jmlsepeda     jarak 
##      0.58      0.64
\end{verbatim}

Sekarang seluruh variabel memiliki MSA \textgreater{} 0,5 dan KMO keseluruhan naik menjadi 0,68. Data siap diekstraksi.

\subsection{Mengekstrak Dimensi Baru}\label{mengekstrak-dimensi-baru}

Langkah selanjutnya adalah menentukan berapa jumlah dimensi (faktor/komponen) yang akan dibentuk. Kita dapat menggunakan \textbf{Analisis Paralel} atau melihat \textbf{Nilai Eigen} (\emph{Eigenvalues}) dan \textbf{Scree Plot}.

\subsubsection{Nilai Eigen dan Total Variansi}\label{nilai-eigen-dan-total-variansi}

Kita akan melihat berapa banyak variansi yang bisa dijelaskan oleh setiap komponen.

\begin{verbatim}
##  [1] 2.4878770 1.6496907 1.2355641 1.0172985 0.8514614 0.8261586 0.6137405
##  [8] 0.5061041 0.4703792 0.3417258
\end{verbatim}

Kriteria umum penentuan jumlah dimensi: 1. \textbf{Kaiser's Criterion}: Ambil komponen dengan nilai eigen \textgreater{} 1. 2. \textbf{Cumulative Variance}: Ambil jumlah komponen yang menjelaskan total variansi \textgreater{} 60\%.

Dari nilai eigen di atas, terdapat 4 komponen dengan nilai \textgreater{} 1 (2.488, 1.650, 1.236, 1.017).

\subsubsection{\texorpdfstring{\emph{Scree Plot}}{Scree Plot}}\label{scree-plot}

Grafik ini menunjukkan penurunan nilai eigen. Titik di mana grafik mulai melandai (siku) menunjukkan batas jumlah faktor.

\pandocbounded{\includegraphics[keepaspectratio]{figures/mod09-scree-plot-1.pdf}}

\textbf{Latihan 4:}

\begin{enumerate}
\def\labelenumi{\arabic{enumi}.}
\tightlist
\item
  Berdasarkan \textbf{Kaiser's Criterion} (Nilai Eigen \textgreater{} 1) pada sub-bab 9.7.1, berapa komponen yang sebaiknya diekstrak?
\item
  Berdasarkan \textbf{Scree Plot} di atas (lihat titik siku/ \emph{elbow}), berapa komponen yang sebaiknya diekstrak?
\end{enumerate}

Berdasarkan analisis-analisis sebelumnya, diputuskan untuk menggunakan \textbf{4 dimensi}.

\subsection{Melakukan Analisis Faktor dan Rotasi}\label{melakukan-analisis-faktor-dan-rotasi}

Kita akan melakukan ekstraksi 4 faktor menggunakan metode \textbf{Analisis Faktor} (untuk pengelompokan variabel) dan \textbf{PCA} (untuk pembentukan variat), kemudian melakukan \textbf{Rotasi Varimax} agar pengelompokan variabel lebih tegas (nilai \emph{loading} kontras).

\subsubsection{Analisis Faktor (Common Factor Analysis)}\label{analisis-faktor-common-factor-analysis}

Analisis ini bertujuan mengelompokkan variabel-variabel yang mirip (korelasi tinggi) ke dalam faktor laten.

\begin{verbatim}
## 
## Loadings:
##           PA1    PA2    PA3    PA4   
## ongkos                   0.667       
## bparkir                  0.699       
## durasi     0.660                     
## bareng                               
## toplajur   0.364         0.401       
## usia                            0.420
## jmlmotor          0.300              
## jmlmobil          0.391              
## jmlsepeda         1.004              
## jarak      0.855                     
## 
##                  PA1   PA2   PA3   PA4
## SS loadings    1.404 1.329 1.238 0.424
## Proportion Var 0.140 0.133 0.124 0.042
## Cumulative Var 0.140 0.273 0.397 0.440
\end{verbatim}

\emph{Catatan: \texttt{cutoff\ =\ 0.3} digunakan untuk menyembunyikan nilai loading yang kecil agar tabel lebih mudah dibaca.}

\textbf{Cara Membaca Output:} Perhatikan matriks komponen yang dirotasi (\emph{Rotated Factor Matrix}). Setiap kolom (PA1, PA2, dst) merepresentasikan faktor. Nilai angka adalah \emph{factor loading} (korelasi antara variabel dengan faktor).

\textbf{Contoh Identifikasi Faktor 1 (PA1):} Lihat kolom \texttt{PA1}. Variabel dengan nilai \emph{loading} terbesar (dan di atas 0.5) adalah: - \texttt{durasi} (0.98) - \texttt{jarak} (0.96) Maka, Faktor 1 dibentuk oleh \texttt{durasi} dan \texttt{jarak}.

\textbf{Latihan 5:}

Berdasarkan output di atas, tentukan variabel pembentuk faktor lainnya:

\begin{enumerate}
\def\labelenumi{\arabic{enumi}.}
\tightlist
\item
  \textbf{Faktor 2 (PA2)}: Variabel apa saja yang memiliki \emph{loading} tinggi di kolom ini?
\item
  \textbf{Faktor 3 (PA3)}: Variabel apa saja yang memiliki \emph{loading} tinggi di kolom ini?
\item
  \textbf{Faktor 4 (PA4)}: Variabel apa saja yang memiliki \emph{loading} tinggi di kolom ini?
\end{enumerate}

\subsubsection{Analisis Komponen Prinsip (PCA)}\label{analisis-komponen-prinsip-pca}

Jika tujuan kita adalah mereduksi data menjadi skor komponen untuk analisis lanjutan (misal regresi), kita menggunakan PCA.

\begin{verbatim}
## 
## Loadings:
##           RC1    RC3    RC2    RC4   
## ongkos            0.791              
## bparkir           0.823              
## durasi     0.847                     
## bareng                          0.806
## toplajur   0.490  0.425              
## usia                     0.324 -0.601
## jmlmotor   0.344         0.597       
## jmlmobil          0.385  0.638       
## jmlsepeda                0.833       
## jarak      0.825                     
## 
##                  RC1   RC3   RC2   RC4
## SS loadings    1.840 1.753 1.626 1.171
## Proportion Var 0.184 0.175 0.163 0.117
## Cumulative Var 0.184 0.359 0.522 0.639
\end{verbatim}

\begin{verbatim}
##               RC1     RC3     RC2     RC4
## ongkos     0.0254  0.4485 -0.0730  0.0301
## bparkir   -0.1330  0.5089 -0.0205  0.0298
## durasi     0.5237 -0.1667 -0.0855 -0.0144
## bareng    -0.0276 -0.0299  0.1825  0.7256
## toplajur   0.1979  0.1949  0.0396 -0.1924
## usia      -0.0194 -0.0251  0.1391 -0.4847
## jmlmotor   0.1783 -0.2001  0.3905  0.2852
## jmlmobil  -0.2100  0.2495  0.3974 -0.0890
## jmlsepeda -0.0496 -0.0641  0.5290 -0.0047
## jarak      0.4506  0.0303 -0.0532  0.0381
\end{verbatim}

\textbf{Identifikasi Persamaan Komponen}

Komponen Prinsip (RC) adalah kombinasi linear dari variabel asal (yang sudah distandarisasi). Persamaannya dapat ditulis sebagai: \[RC_j = w_{1j}Z_1 + w_{2j}Z_2 + \dots + w_{pj}Z_p\] Dimana \(w\) adalah nilai \textbf{Component Score Coefficients (Weights)}, bukan nilai \emph{loading}. Loading hanya menunjukkan korelasi, sedangkan weights menunjukkan bobot kontribusi setiap variabel dalam pembentukan skor komponen.

\textbf{Contoh:} Misalkan kita ingin membentuk persamaan untuk \textbf{RC1}. Lihat output \texttt{Weights} pada kolom \texttt{RC1}: - \texttt{durasi}: 0.524 - \texttt{jarak}: 0.451 - Variabel lain memiliki bobot yang lebih kecil.

Maka persamaannya: \[RC1 \approx 0.524(Z_{durasi}) + 0.451(Z_{jarak}) + \dots\]

\textbf{Menghasilkan Component Score}

Untuk mendapatkan nilai skor komponen setiap observasi secara otomatis, kita dapat mengakses objek \texttt{\$scores} dari hasil PCA.

\begin{verbatim}
##             RC1        RC3         RC2         RC4
## [1,] -0.1944617 -0.4804209  0.36717340 -0.54187967
## [2,] -0.7752196 -0.1911165  1.50812139 -1.14287093
## [3,] -0.2069996 -0.2693966 -0.04572532 -0.49628033
## [4,]  0.7677605 -0.2647856 -0.21284551  0.01899626
## [5,] -1.6702167 -0.2667413  1.87084197  0.22728535
## [6,]  0.1797747 -0.6385342  0.52415727 -2.00189106
\end{verbatim}

\textbf{Latihan 6:}

Berdasarkan output weights PCA di atas, tuliskan persamaan matematis terbentuknya komponen \textbf{RC2}! (Sebutkan variabel mana saja yang memiliki kontribusi besar beserta koefisien weights-nya).

\subsection{Interpretasi dan Penamaan Dimensi}\label{interpretasi-dan-penamaan-dimensi}

Langkah terakhir adalah memberi nama pada dimensi yang terbentuk berdasarkan variabel-variabel pembentuknya.

\textbf{Latihan 7:} Tuliskan seluruh kelompok dari analisis faktor dan juga seluruh persamaan komponen yang dihasilkan dari PCA!

\textbf{Aktivitas Mandiri Komprehensif: Analisis PCA Mandiri {[}STP-14.3{]}}

Gunakan dataset dengan variabel metrik yang berbeda:

\begin{enumerate}
\def\labelenumi{\arabic{enumi}.}
\tightlist
\item
  \textbf{Persiapan:} Pilih variabel, cek normalitas, uji Bartlett/KMO
\item
  \textbf{Ekstraksi:} Hitung eigenvalues, buat scree plot, tentukan jumlah komponen
\item
  \textbf{Analisis:} Lakukan PCA dengan rotasi Varimax
\item
  \textbf{Interpretasi:} Identifikasi variabel pembentuk, beri nama komponen, tuliskan persamaan
\item
  \textbf{Evaluasi:} Hitung proporsi variansi yang dijelaskan
\end{enumerate}

\begin{center}\rule{0.5\linewidth}{0.5pt}\end{center}

\section{Referensi}\label{referensi}

\renewcommand\refname{Referensi}
\bibliography{referensi.bib}

\end{document}
